\documentclass[11pt,a4paper]{book}
\usepackage{fontspec}
\usepackage{xgreek}
\usepackage{hyperref}
\usepackage{epigraph}
\usepackage{hologo}
\usepackage[margin=2.5cm]{geometry}
\usepackage[Greek,Latin]{ucharclasses}
\usepackage[
  backref=true,
  backend=biber,
  sortcites, ]
  {biblatex}
\DefineBibliographyStrings{english}{%
    backrefpage  = {see p.}, % for single page number
    backrefpages = {see pp.} % for multiple page numbers
}
\usepackage{csquotes}
\author{George Sofianos}
\title{GPGPU Programming}
\setmainfont[Kerning=On,Mapping=tex-text]{GFS Artemisia}
\addbibresource{bibliography.bib}

\begin{document}

\maketitle
 
\tableofcontents

\chapter{Εισαγωγή}
\epigraph{τεστ τεστ δοκιμή}{Γιώργος}
Ιστορία του GPGPU Programming:
Τα αρχικά GPGPU πηγάζουν απο την φράση General Purpose computation on Graphics Processing Units, ή αλλιώς γνωστή ώς GPU Computing, δηλαδή υπολογισμός γενικής χρήσης σε μονάδες επεξεργασίας γραφικών.
Οι GPUs, είναι επεξεργαστές υψηλών επιδόσεων με δυνατότητα πολύ υψηλού υπολογισμού και διεκπεραιωτικότητας δεδομένων. Σχεδιασμένες αρχικά για γραφικά υπολογιστών με αρκετές δυσκολίες στον προγραμματισμό τους, οι σημερινές μονάδες επεξεργασίας γραφικών είναι παράλληλοι επεξεργαστές γενικής χρήσης με υποστήριξη για προσβάσιμες προγραμματιστικές διεπαφές και βιομηχανικά πρότυπα γλωσσών όπως η C. Οι προγραμματιστές που μεταφέρουν τις εφαρμογές τους σε GPUs συνήθως πετυχαίνουν ταχύτητες πολλαπλάσιες από ότι μια αντίστοιχη εφαρμογή ειδικά βελτιστοποιημένη για κεντρική μονάδα επεξεργασίας (CPU).
Ο όρος GPGPU δημιουργήθηκε από τον \href{http://en.wikipedia.org/wiki/Mark_Harris_(programmer)}{Mark Harris} το 2002 όταν συνειδητοποίησε ότι αναπτυσσόταν μια τάση για χρήση των μονάδων επεξεργασίας γραφικών για εφαρμογές που δεν είχαν σχέση με γραφικά. 

Από το 2012, οι GPU έχουν αναπτυχθεί σε συστήματα πολυπύρηνων επεξεργαστών παράλληλου υπολογισμού δίνοντας μας την δυνατότητα για πολύ αποδοτικό χειρισμό μεγάλου όγκου δεδομένων. Αυτός ο σχεδιασμός είναι πιο αποδοτικός από ότι οι κεντρικές μονάδες επεξεργασιάς (CPU) για αλγόριθμους όπου η επεξεργασία μεγάλου όγκου δεδομένων γίνεται παράλληλα, όπως σε αλγορίθμους sort μεγάλων λιστών, μετασχηματισμό κυμάτων δυο διαστάσεων, προσομοίωση βιολογικών δυναμικών.
\section{Εργαλεία που χρησιμοποιήθηκαν}

Για την εκπόνηση της πτυχιακής εργασίας, χρησιμοποιήθηκαν αρκετά εργαλεία για λόγους ευχρηστίας αλλά και για εκπαιδευτικούς σκοπούς, σε μια προσπάθεια να αποκτηθούν γνώσεις για τεχνολογίες που πιστεύω πως χρειάζονται σε έναν απόφοιτο πληροφορικής.\\
1) \hologo{XeTeX} - Texmaker\\
Το \hologo{XeTeX} είναι μια μηχανή τυπογραφίας τύπου TeX η οποία χρησιμοποιεί κωδικοποίηση Unicode και υποστηρίζει σύγχρονες τεχνολογίες γραμματοσειρών όπως οι Opentype, Graphite και Apple Advanced Typography. Έχει σχεδιαστεί από τον Jonathan Kew και διανέμεται κάτω από την ελεύθερη άδεια λογισμικού X11. Ενώ δημιουργήθηκε αποκλειστικά για το Mac OS X, πλέον είναι διαθέσιμο για όλες τις γνωστές πλατφόρμες. Υποστηρίζει κωδικοποίηση Unicode και τα αρχεία κειμένου είναι εξαρχής σε μορφή UTF-8. Το \hologo{XeTeX} μπορεί να χρησιμοποιήσει τις γραμματοσειρές που είναι εγκατεστημένες στο σύστημα, και να κάνει χρήση των ανεπτυγμένων τυπογραφικών δυνατοτήτων τους. Υποστηρίζει και μικρο-τυπογραφία, δηλαδή μια σειρά από μεθόδους που βελτιώνουν την αισθητική του κειμένου και το καθιστούν πιο ευανάγνωστο. Οι μέθοδοι συμπεριλαμβάνουν την μείωση μεγάλων κενών μεταξύ των λέξεων(expansion), την επέκταση των γραμμών όταν τελειώνουν με κάποιο μικρό σύμβολο, όπως η τελεία η ένα στρογγυλό γράμμα όπως το "ο" (protrusion), και ο χωρισμός γραμμών (hyphenation). Αν και το \hologo{XeTeX} είναι υποδεέστερο του \hologo{LaTeX} στην μικρο-τυπογραφία, επιλέχτηκε για αυτήν την πτυχιακή εργασία λόγω των πολλών άλλων πλεονεκτημάτων, όπως η χρήση unicode χαρακτήρων αλλά και η ευκολία διαχείρισης γραμματοσειρών. Το \hologo{XeTeX} 

2) Git - TortoiseGit
3) 
\section{Βιβλιοθήκες}
Το πεδίο του προγραμματισμού γενικής χρήσης έχει αναπτυχθεί με ταχύτατους ρυθμούς τα τελευταία χρόνια, έτσι ώστε τώρα υπάρχουν αρκετές υλοποιήσεις για τον προγραμματισμό των μονάδων επεξεργασίας γραφικών. Πρόσφατα, έχουν γίνει προσπάθειες δημιουργίας προτύπων.

Ο προγραμματισμός των GPUs αναπτύχθηκε όταν το CUDA και το Stream κατέφθασαν στο τέλος του 2006. Αυτές οι διεπαφές και οι γλώσσες, σχεδιάστηκαν απο τις εταιρίες κατασκευής των GPUs σε πολύ κοντινή σχέση με το υλικό, το οποίο αποτέλεσε μεγάλο βήμα προς ένα πιο εύχρηστο, ταιριαστό και μελλοντικά-ασφαλές προγραμματιστικό μοντέλο. Η ανοιχτή γλώσσα προγραμματισμού (OpenCL) δημιουργήθηκε για να παρέχει ένα γενικό API ετερογενή υπολογισμού σε διάφορες μορφές παράλληλων συσκευών, συμπεριλαμβανομένου μονάδων επεξεργασίας γραφικών, πολυπύρηνων κεντρικών μονάδων επεξεργαστών, κ.α


\section{Βιβλιογραφικές αναφορές}
H πρώτη βιβλιογραφική αναφορά είναι: \parencite{doe}, ενώ η δεύτερη είναι \parencite{papadopoulos}.

\printbibliography[title={Βιβλιογραφία}]
\end{document}