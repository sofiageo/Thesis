\documentclass[11pt,a4paper,draft]{book}
\usepackage{fontspec}
\usepackage{xgreek}
\usepackage{hyperref}
\usepackage{epigraph}
\author{George Sofianos}
\title{GPGPU Programming}
\setmainfont{GFS Artemisia}

\begin{document}
\epigraph{τεστ τεστ δοκιμή}{Γιώργος}
Ιστορία του GPGPU Programming:
Τα αρχικά GPGPU πηγάζουν απο την φράση General Purpose computation on Graphics Processing Units, ή αλλιώς γνωστή ώς GPU Computing, δηλαδή υπολογισμός γενικής χρήσης σε μονάδες επεξεργασίας γραφικών.
Οι GPUs, είναι επεξεργαστές υψηλών επιδόσεων με δυνατότητα πολύ υψηλού υπολογισμού και διεκπεραιωτικότητας δεδομένων. Σχεδιασμένες αρχικά για γραφικά υπολογιστών με αρκετές δυσκολίες στον προγραμματισμό τους, οι σημερινές μονάδες επεξεργασίας γραφικών είναι παράλληλοι επεξεργαστές γενικής χρήσης με υποστήριξη για προσβάσιμες προγραμματιστικές διεπαφές και βιομηχανικά πρότυπα γλωσσών όπως η C. Οι προγραμματιστές που μεταφέρουν τις εφαρμογές τους σε GPUs συνήθως πετυχαίνουν ταχύτητες πολλαπλάσιες από ότι μια αντίστοιχη εφαρμογή ειδικά βελτιστοποιημένη για κεντρική μονάδα επεξεργασίας (CPU).
Ο όρος GPGPU δημιουργήθηκε από τον \href{http://en.wikipedia.org/wiki/Mark_Harris_(programmer)}{Mark Harris} το 2002 όταν συνειδητοποίησε ότι αναπτυσσόταν μια τάση για χρήση των μονάδων επεξεργασίας γραφικών για εφαρμογές που δεν είχαν σχέση με γραφικά. 
\section{mpla}
Το πεδίο του προγραμματισμού γενικής χρήσης έχει αναπτυχθεί με ταχύτατους ρυθμούς τα τελευταία χρόνια, έτσι ώστε τώρα υπάρχουν αρκετές υλοποιήσεις για τον προγραμματισμό των μονάδων επεξεργασίας γραφικών. Πρόσφατα, έχουν γίνει προσπάθειες δημιουργίας προτύπων.

Ο προγραμματισμός των GPUs αναπτύχθηκε όταν το CUDA και το Stream κατέφθασαν στο τέλος του 2006. Αυτές οι διεπαφές και οι γλώσσες, σχεδιάστηκαν απο τις εταιρίες κατασκευής των GPUs σε πολύ κοντινή σχέση με το υλικό, το οποίο αποτέλεσε μεγάλο βήμα προς ένα πιο εύχρηστο, ταιριαστό και μελλοντικά-ασφαλές προγραμματιστικό μοντέλο. Η ανοιχτή γλώσσα προγραμματισμού (OpenCL) δημιουργήθηκε για να παρέχει ένα γενικό API ετερογενή υπολογισμού σε διάφορες μορφές παράλληλων συσκευών, συμπεριλαμβανομένου μονάδων επεξεργασίας γραφικών, πολυπύρηνων κεντρικών μονάδων επεξεργαστών, κ.α
\end{document}