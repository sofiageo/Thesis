\documentclass[11pt,a4paper,twoside]{book}
\usepackage{fontspec}
\usepackage{xgreek}
\usepackage{hyperref}
\usepackage{epigraph}
\usepackage{hologo}
\usepackage[top=2.54cm, bottom=2.54cm, left=3.17cm, right=3.17cm]{geometry}
\usepackage[Greek,Latin]{ucharclasses}
\usepackage[
  backref=true,
  backend=biber,
  sortcites, ]
  {biblatex}
\DefineBibliographyStrings{english}{%
    backrefpage  = {see p.}, % for single page number
    backrefpages = {see pp.} % for multiple page numbers
}
\usepackage{titlesec}

\titleformat{\chapter}
  {\Large\bfseries} % format
  {}                % label
  {0pt}             % sep
  {\huge}           % before-code
\usepackage{enumitem}
\usepackage{graphicx}
\graphicspath{{./images/}}
\usepackage{wrapfig}
\usepackage{csquotes}
\author{Γεώργιος Σοφιανός}
\title{GPGPU Programming}
\setmainfont[Kerning=On,Mapping=tex-text]{GFS Artemisia}
%Νέες εντολές
\newcommand{\HRule}{\rule{\linewidth}{0.5mm}}
\addbibresource{bibliography.bib}

\begin{document}
\setlist[itemize]{leftmargin=*}
\setlist[enumerate]{leftmargin=*}
\begin{titlepage}
\begin{center}

\includegraphics[scale=0.25]{Logo_teiath}~\\[1cm]

\textsc{\LARGE ΤΕΙ ΑΘΗΝΑΣ ΤΜΗΜΑ ΠΛΗΡΟΦΟΡΙΚΗΣ}\\[1.5cm]

\textsc{\Large Πτυχιακή Εργασία}\\[0.5cm]

% Τίτλος
\HRule \\[0.4cm]
\textsc{\Large Προγραμματισμός γενικού σκοπού σε μονάδες επεξεργασίας γραφικών\\ (GPGPU)} \\[0.4cm]

\HRule \\[1.5cm]

% Φοιτητής και επόπτης
\begin{minipage}[t]{0.4\textwidth}
\begin{flushleft} \large
\emph{Φοιτητής:}\\
\textsc{Γεώργιος Σοφιανός}\\
\emph{A.M:} 031053
\end{flushleft}
\end{minipage}
\begin{minipage}[t]{0.4\textwidth}
\begin{flushright} \large
\emph{Επόπτης:} \\
\textsc{κ. Γεώργιος Μπαρδής}
\end{flushright}
\end{minipage}

\vfill

% Τέλος σελίδας
{\large Ημερομηνία παράδοσης: \\ \today}

\end{center}
\end{titlepage}
 
\tableofcontents

\chapter{Εισαγωγή}
\epigraph{"Anyone can build a fast CPU. The trick is to build a fast system."}{Seymour Cray}

Τα αρχικά GPGPU πηγάζουν απο την φράση General Purpose computation on Graphics Processing Units, ή αλλιώς γνωστή ώς GPU Computing, δηλαδή υπολογισμός γενικού σκοπού σε μονάδες επεξεργασίας γραφικών.

Οι GPUs, είναι επεξεργαστές υψηλών επιδόσεων με δυνατότητα πολύ υψηλού υπολογισμού και διεκπεραιωτικότητας δεδομένων. Σχεδιασμένες αρχικά για γραφικά υπολογιστών με αρκετές δυσκολίες στον προγραμματισμό τους, οι σημερινές μονάδες επεξεργασίας γραφικών είναι παράλληλοι επεξεργαστές γενικής χρήσης με υποστήριξη για προσβάσιμες προγραμματιστικές διεπαφές και βιομηχανικά πρότυπα γλωσσών όπως η C.

Οι προγραμματιστές που μεταφέρουν τις εφαρμογές τους σε GPUs συνήθως πετυχαίνουν ταχύτητες πολλαπλάσιες από ότι μια αντίστοιχη εφαρμογή ειδικά βελτιστοποιημένη για κεντρική μονάδα επεξεργασίας (CPU).
Ο όρος GPGPU δημιουργήθηκε από τον Mark Harris το 2002 όταν συνειδητοποίησε ότι αναπτυσσόταν μια τάση για χρήση των μονάδων επεξεργασίας γραφικών για εφαρμογές που δεν είχαν σχέση με γραφικά. 

Από το 2012, οι GPU έχουν αναπτυχθεί σε συστήματα πολυπύρηνων επεξεργαστών παράλληλου υπολογισμού δίνοντας μας την δυνατότητα για πολύ αποδοτικό χειρισμό μεγάλου όγκου δεδομένων. Αυτός ο σχεδιασμός είναι πιο αποδοτικός από ότι οι κεντρικές μονάδες επεξεργασιάς (CPU) για αλγόριθμους όπου η επεξεργασία μεγάλου όγκου δεδομένων γίνεται παράλληλα, όπως σε αλγορίθμους sort μεγάλων λιστών, μετασχηματισμό κυμάτων δυο διαστάσεων, προσομοίωση βιολογικών δυναμικών.

Σκοπός αυτής της εργασίας είναι να διερευνήσει και αξιολογήσει πρότυπα και τεχνολογίες για προγραμματισμό γενικού σκοπού με χρήση μονάδων επεξεργασίας γραφικών, ειδικότερα όσον αφορά εφαρμογές υψηλών υπολογιστικών απαιτήσεων οι οποίες εκμεταλλεύονται τις δυνατότητες αυτών των τεχνολογιών για επιτάχυνση και αύξηση των επιδόσεων τους. Το θέμα είναι μεγάλης σημασίας



\section{Παράλληλος υπολογισμός}
Για 30 χρόνια, ένας από τους πιο σημαντικούς τρόπους για να βελτιώσουμε την απόδοση των υπολογιστικών συσκευών των καταναλωτών ήταν η αύξηση της ταχύτητας στην οποία λειτουργεί το ρολόι ενός επεξεργαστή. Ξεκινώντας από περίπου το 1MHZ το 1980, οι περισσότεροι σύγχρονοι επεξεργαστές έχουν ταχύτητες μεταξύ 1GHz και 4GHz, δηλαδή είναι περίπου 1000 φορές πιο γρήγοροι. Αν και δεν είναι ο μόνος τρόπος με τον οποίο έχουν βελτιωθεί οι επεξεργαστές, αποτελεί συνήθως μια αξιόπιστη πηγή για αύξηση της απόδοσης.

Τα τελευταία χρόνια όμως, οι κατασκευαστές έχουν αναγκαστεί να ψάξουν για εναλλακτικούς τρόπους αύξησης της υπολογιστικής δύναμης. Εξ αιτίας διάφορων περιορισμών στην κατασκευή ενσωματωμένων κυκλωμάτων, δεν είναι πλέον εύκολο να αυξάνουμε την ταχύτητα του ρολογιού του επεξεργαστή σαν τρόπο αύξησης της απόδοσης στις υπάρχουσες αρχιτεκτονικές. Στην αναζήτηση για επιπλέον υπολογιστική δύναμη για τους προσωπικούς επεξεργαστές, οι ερευνητές χρησιμοποίησαν τεχνολογίες που ήταν ήδη γνωστές από τους υπερ-υπολογιστές, στους οποίους είναι σύνηθες φαινόμενο να αποτελούνται από δεκάδες ή εκατοντάδες επεξεργαστές, οι οποίοι εκτελούν παράλληλες διεργασίες. Έτσι το 2005, οι κύριοι κατασκευαστές επεξεργαστών άρχισαν να προσφέρουν επεξεργαστές με δύο πυρήνες αντί για έναν. 

Τα επόμενα χρόνια, ακολούθησαν υλοποιήσεις με τρεις,τέσσερις,έξι, ακόμα και οκτώ πυρήνες. Έχει ξεκινήσει ήδη μια μεγάλη στροφή της βιομηχανίας υπολογιστών στον παράλληλο υπολογισμό. Με την κυκλοφορία των διπύρηνων μέχρι και 8 ή 16 πυρήνων επεξεργαστών για σταθμούς εργασίας, ο παράλληλος υπολογισμός δεν είναι πλέον υπόθεση που αφορά μόνο τους εξωτικούς υπερ-υπολογιστές. Επίσης οι φορητές συσκευές όπως κινητά τηλέφωνα και φορητές συσκευές μουσικής έχουν αρχίσει να ενσωματώνουν δυνατότητες παράλληλου υπολογισμού σε μια προσπάθεια να προσφέρουν δυνατότητες πολύ ανώτερες από τους προγόνους τους. Όλο και περισσότερο, οι προγραμματιστές λογισμικού πρέπει να εξοικειωθούν με πλατφόρμες και τεχνολογίες παράλληλου υπολογισμού ώστε να προμηθεύουν με πλούσιες εμπειρίες την βάση των χρηστών τους. Το μέλλον αποτελείται από πολύ-νηματικές εφαρμογές, και από φορητές συσκευές που μπορούν ταυτόχρονα να παίζουν μουσική, να εξερευνούν το διαδίκτυο, και να παρέχουν GPS υπηρεσίες.
\section{GPU Computing}

Η επιτάχυνση υπολογισμού γενικού σκοπού από μονάδες επεξεργασίας γραφικών, είναι η χρήση των GPUs μαζί με χρήση CPUs για να επιταχύνουν επιστημονικές, αναλυτικές, κατασκευαστικές, καταναλωτικές, και εμπορικές εφαρμογές. Από την πρώτη εμφάνιση της τεχνολογίας το 2007 από την NVIDIA, οι επιταχυντές GPU τώρα βρίσκονται σε κέντρα δεδομένων σε εργαστήρια αποδοτικής ενέργειας για λογαριασμό κυβερνήσεων, πανεπιστήμια, και μικρές και μεγάλες επιχειρήσεις σε όλον τον κόσμο. Οι GPUs επιταχύνουν εφαρμογές σε πλατφόρμες που εκτείνονται από αυτοκίνητα, σε κινητά τηλέφωνα, σε drones και ρομπότ.

Η επιτάχυνση μέσω μονάδων επεξεργασίας γραφικών παρέχει πρωτοφανή αποτελέσματα επιδόσεων διαχωρίζοντας και φορτώνοντας τις εντολές εντατικών υπολογισμών στην GPU, ενώ διατηρεί την εκτέλεση του υπόλοιπου κώδικα στην CPU. Από την οπτική γωνία του χρήστη, οι εφαρμογές απλώς τρέχουν πιο γρήγορα.

\begin{figure}[h]
\centering
\includegraphics[scale=0.50]{gpuintro1}
\caption{Εκτέλεση κώδικα σε CPU και GPU}
\end{figure}
\section{Εργαλεία που χρησιμοποιήθηκαν}

Για την εκπόνηση της πτυχιακής εργασίας, χρησιμοποιήθηκαν αρκετά εργαλεία για λόγους ευχρηστίας αλλά και για εκπαιδευτικούς σκοπούς, σε μια προσπάθεια να αποκτηθούν γνώσεις για τεχνολογίες που πιστεύω πως χρειάζονται σε έναν απόφοιτο πληροφορικής.\\
1) \hologo{XeTeX} - Texmaker\\
Το \hologo{XeTeX} είναι μια μηχανή τυπογραφίας τύπου TeX η οποία χρησιμοποιεί κωδικοποίηση Unicode και υποστηρίζει σύγχρονες τεχνολογίες γραμματοσειρών όπως οι Opentype, Graphite και Apple Advanced Typography. Έχει σχεδιαστεί από τον Jonathan Kew και διανέμεται κάτω από την ελεύθερη άδεια λογισμικού X11. Ενώ δημιουργήθηκε αποκλειστικά για το Mac OS X, πλέον είναι διαθέσιμο για όλες τις γνωστές πλατφόρμες. Υποστηρίζει κωδικοποίηση Unicode και τα αρχεία κειμένου είναι εξαρχής σε μορφή UTF-8. Το \hologo{XeTeX} μπορεί να χρησιμοποιήσει τις γραμματοσειρές που είναι εγκατεστημένες στο σύστημα, και να κάνει χρήση των ανεπτυγμένων τυπογραφικών δυνατοτήτων τους. Υποστηρίζει και μικρο-τυπογραφία, δηλαδή μια σειρά από μεθόδους που βελτιώνουν την αισθητική του κειμένου και το καθιστούν πιο ευανάγνωστο. Οι μέθοδοι συμπεριλαμβάνουν την μείωση μεγάλων κενών μεταξύ των λέξεων(expansion), την επέκταση των γραμμών όταν τελειώνουν με κάποιο μικρό σύμβολο, όπως η τελεία η ένα στρογγυλό γράμμα όπως το "ο" (protrusion), και ο χωρισμός γραμμών (hyphenation). Αν και το \hologo{XeTeX} είναι υποδεέστερο του \hologo{LaTeX} στην μικρο-τυπογραφία, επιλέχτηκε για αυτήν την πτυχιακή εργασία λόγω των πολλών άλλων πλεονεκτημάτων, όπως η χρήση unicode χαρακτήρων αλλά και η ευκολία διαχείρισης γραμματοσειρών. Το \hologo{XeTeX} 

2) Git - TortoiseGit
3) 
\section{Βιβλιοθήκες}
Το πεδίο του προγραμματισμού γενικής χρήσης έχει αναπτυχθεί με ταχύτατους ρυθμούς τα τελευταία χρόνια, έτσι ώστε τώρα υπάρχουν αρκετές υλοποιήσεις για τον προγραμματισμό των μονάδων επεξεργασίας γραφικών. Πρόσφατα, έχουν γίνει προσπάθειες δημιουργίας προτύπων.

Ο προγραμματισμός των GPUs αναπτύχθηκε όταν το CUDA και το Stream κατέφθασαν στο τέλος του 2006. Αυτές οι διεπαφές και οι γλώσσες, σχεδιάστηκαν απο τις εταιρίες κατασκευής των GPUs σε πολύ κοντινή σχέση με το υλικό, το οποίο αποτέλεσε μεγάλο βήμα προς ένα πιο εύχρηστο, ταιριαστό και μελλοντικά-ασφαλές προγραμματιστικό μοντέλο. Η ανοιχτή γλώσσα προγραμματισμού (OpenCL) δημιουργήθηκε για να παρέχει ένα γενικό API ετερογενή υπολογισμού σε διάφορες μορφές παράλληλων συσκευών, συμπεριλαμβανομένου μονάδων επεξεργασίας γραφικών, πολυπύρηνων κεντρικών μονάδων επεξεργαστών, κ.α
\chapter{Εφαρμογές}
\section{Εισαγωγή}
Τα τελευταία περίπου 20 χρόνια οι εταιρίες παραγωγής υλικού γραφικών έχουν εστιάσει στην προσπάθεια να παράγουν γρήγορες μονάδες γραφικής επεξεργασίας (GPU), ειδικότερα για την κοινότητα των gamer. Αυτό έχει ως αποτέλεσμα πρόσφατα να δημιουργηθούν συσκευές οι επιδόσεις των οποίων ξεπερνούν τις κεντρικές μονάδες επεξεργασίας (CPU), σε συγκεκριμένες εφαρμογές, ειδικότερα σε μετρήσεις εκατομμυρίων εντολών το δευτερόλεπτο (MIPS). Έτσι, καθιερώθηκε μια κοινότητα για να αξιοποιήσει αυτήν την μεγάλη δύναμη των GPU για υπολογισμούς γενικής χρήσης(GPGPU). Τα τελευταία δύο χρόνια έχουν εξαλειφθεί οι περισσότεροι περιορισμοί που υπήρχαν όσον αφορά το σετ εντολών και την διαχείριση μνήμης, με την ενσωμάτωση ενοποιημένων υπολογιστικών μονάδων στις κάρτες γραφικών, δίνοντας έτσι την δυνατότητα στους προγραμματιστές να δημιουργήσουν ένα πλήθος από προγράμματα με εφαρμογές σε πολλούς τομείς.
\section{Κρυπτογράφηση}
Στο πεδίο της ασύμμετρης κρυπτογράφησης, η ασφάλεια όλων των πρακτικών κρυπτοσυστημάτων βασίζεται στην δυσκολία υπολογισμού προβλημάτων, εξαρτημένη από την επιλογή των παραμέτρων. Με την όποια αύξηση των παραμέτρων όμως (συνήθως στο εύρος 1024-4096 bits), οι υπολογισμοί γίνονται όλο και πιο απαιτητικοί για τον εκάστοτε επεξεργαστή. Σε σύγχρονο υλικό, ο υπολογισμός μιας μονής εντολής κρυπτογράφησης δεν είναι κρίσιμος, όμως σε ένα σύστημα επικοινωνίας πολλών-προς-ένα, για παράδειγμα ένας κεντρικός server στο κέντρο δεδομένων μιας εταιρίας, μπορεί να αντιμετωπίσει ταυτόχρονα εκατοντάδες η και χιλιάδες ταυτόχρονες συνδέσεις και εντολές κρυπτογράφησης. Ως αποτέλεσμα, η πιο συνήθης λύση για ένα τέτοιο σενάριο είναι η χρήση καρτών επιτάχυνσης κρυπτογράφησης. Λόγω της μικρής αγοράς, η τιμή τους φτάνει συνήθως αρκετά χιλιάδες ευρώ η δολάρια.\\
Τελευταία, η ερευνητική κοινότητα έχει αρχίσει να εξερευνά τεχνικές για επιτάχυνση των αλγορίθμων κρυπτογράφησης με χρήση της GPU.  

\section{Βιοπληροφορική}
\subsection{Εισαγωγή}
Η Βιοπληροφορική παίζει μεγάλο ρόλο σε πολλές πτυχές της βιολογίας. Στην πειραματική μοριακή βιολογία, οι τεχνικές βιοπληροφορικής όπως επεξεργασία εικόνας και σήματος, επιτρέπει την εξόρυξη χρήσιμων αποτελεσμάτων από μεγάλο όγκο δεδομένων. Στο πεδίο της γενετικής και γονιδιωματικής, συμβάλλει στην αλληλουχία και υποσημείωση γονιδιωμάτων και την παρατήρηση των μεταλλάξεων τους. Παίζει μεγάλο ρόλο στην εξόρυξη τεχνικών όρων και στην κατασκευή βιολογικών και γονιδιακών οντολογιών για την οργάνωση και αναζήτηση βιολογικών δεδομένων. Έχει επίσης μεγάλο ρόλο στην ανάλυση των γονιδίων και στην ρύθμιση πρωτεϊνών. Τα εργαλεία της βιοπληροφορικής συμβάλουν στην σύγκριση γενετικών και γονιδιακών δεδομένων και γενικότερα στην κατανόηση των αναπτυξιακών πτυχών της μοριακής βιολογίας. Σε πιο εσωτερικό επίπεδο, συμβάλει στην ανάλυση και κατηγοριοποίηση των βιολογικών διαδρόμων και δικτύων τα οποία είναι σημαντικό κομμάτι της συστεμικής βιολογίας. Στην Δομική βιολογία, συμβάλει στην εξομοίωση και μοντελισμό του DNA, RNA, και δομές πρωτεϊνών όπως και μοριακών αλληλεπιδράσεων.
\begin{figure}[h]
\centering
\includegraphics[scale=0.25]{bioinformatics}
\caption{Βιολογία και πληροφορική}
\end{figure}
\section{Αστροφυσική}
Μια μεγάλη εμβέλεια από μεγάλα προβλήματα αστροφυσικής

\section{Βιβλιογραφικές αναφορές}
H πρώτη βιβλιογραφική αναφορά είναι: \parencite{doe}, ενώ η δεύτερη είναι \parencite{papadopoulos}.

\printbibliography[title={Βιβλιογραφία}]
\end{document}