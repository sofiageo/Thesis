\documentclass[11pt,a4paper]{book}
\usepackage{fontspec}
\usepackage{xgreek}
\usepackage{hyperref}
\usepackage{epigraph}
\usepackage{hologo}
\usepackage[margin=2.5cm]{geometry}
\usepackage[Greek,Latin]{ucharclasses}
\usepackage[
  backref=true,
  backend=biber,
  sortcites, ]
  {biblatex}
\DefineBibliographyStrings{english}{%
    backrefpage  = {see p.}, % for single page number
    backrefpages = {see pp.} % for multiple page numbers
}
\usepackage{csquotes}
\author{George Sofianos}
\title{GPGPU Programming}
\setmainfont[Kerning=On,Mapping=tex-text]{GFS Artemisia}
\addbibresource{bibliography.bib}

\begin{document}

\maketitle
 
\tableofcontents

\chapter{Εισαγωγή}
\epigraph{τεστ τεστ δοκιμή}{Γιώργος}
Ιστορία του GPGPU Programming:
Τα αρχικά GPGPU πηγάζουν απο την φράση General Purpose computation on Graphics Processing Units, ή αλλιώς γνωστή ώς GPU Computing, δηλαδή υπολογισμός γενικής χρήσης σε μονάδες επεξεργασίας γραφικών.
Οι GPUs, είναι επεξεργαστές υψηλών επιδόσεων με δυνατότητα πολύ υψηλού υπολογισμού και διεκπεραιωτικότητας δεδομένων. Σχεδιασμένες αρχικά για γραφικά υπολογιστών με αρκετές δυσκολίες στον προγραμματισμό τους, οι σημερινές μονάδες επεξεργασίας γραφικών είναι παράλληλοι επεξεργαστές γενικής χρήσης με υποστήριξη για προσβάσιμες προγραμματιστικές διεπαφές και βιομηχανικά πρότυπα γλωσσών όπως η C. Οι προγραμματιστές που μεταφέρουν τις εφαρμογές τους σε GPUs συνήθως πετυχαίνουν ταχύτητες πολλαπλάσιες από ότι μια αντίστοιχη εφαρμογή ειδικά βελτιστοποιημένη για κεντρική μονάδα επεξεργασίας (CPU).
Ο όρος GPGPU δημιουργήθηκε από τον \href{http://en.wikipedia.org/wiki/Mark_Harris_(programmer)}{Mark Harris} το 2002 όταν συνειδητοποίησε ότι αναπτυσσόταν μια τάση για χρήση των μονάδων επεξεργασίας γραφικών για εφαρμογές που δεν είχαν σχέση με γραφικά. 

Από το 2012, οι GPU έχουν αναπτυχθεί σε συστήματα πολυπύρηνων επεξεργαστών παράλληλου υπολογισμού δίνοντας μας την δυνατότητα για πολύ αποδοτικό χειρισμό μεγάλου όγκου δεδομένων. Αυτός ο σχεδιασμός είναι πιο αποδοτικός από ότι οι κεντρικές μονάδες επεξεργασιάς (CPU) για αλγόριθμους όπου η επεξεργασία μεγάλου όγκου δεδομένων γίνεται παράλληλα, όπως σε αλγορίθμους sort μεγάλων λιστών, μετασχηματισμό κυμάτων δυο διαστάσεων, προσομοίωση βιολογικών δυναμικών.
\section{mpla}
Το πεδίο του προγραμματισμού γενικής χρήσης έχει αναπτυχθεί με ταχύτατους ρυθμούς τα τελευταία χρόνια, έτσι ώστε τώρα υπάρχουν αρκετές υλοποιήσεις για τον προγραμματισμό των μονάδων επεξεργασίας γραφικών. Πρόσφατα, έχουν γίνει προσπάθειες δημιουργίας προτύπων.

Ο προγραμματισμός των GPUs αναπτύχθηκε όταν το CUDA και το Stream κατέφθασαν στο τέλος του 2006. Αυτές οι διεπαφές και οι γλώσσες, σχεδιάστηκαν απο τις εταιρίες κατασκευής των GPUs σε πολύ κοντινή σχέση με το υλικό, το οποίο αποτέλεσε μεγάλο βήμα προς ένα πιο εύχρηστο, ταιριαστό και μελλοντικά-ασφαλές προγραμματιστικό μοντέλο. Η ανοιχτή γλώσσα προγραμματισμού (OpenCL) δημιουργήθηκε για να παρέχει ένα γενικό API ετερογενή υπολογισμού σε διάφορες μορφές παράλληλων συσκευών, συμπεριλαμβανομένου μονάδων επεξεργασίας γραφικών, πολυπύρηνων κεντρικών μονάδων επεξεργαστών, κ.α
\chapter{Υλοποιήσεις}
\section{Ιστορία}
Παραδοσιακά, η σειρά αγωγών των γραφικών, αποτελείται από τις καταστάσεις μετατροπή και φωτισμός, συναρμολόγηση αρχέγονων, μετατροπή σε pixels, και σκίαση. Οι πρώτες GPU είχαν όλες τις λειτουργίες που χρειάζονται για να εκτελεστεί η σειρά αγωγών, αλλά με τον καιρό όλο και περισσότερες καταστάσεις έγιναν δυνατό να προγραμματιστούν με την έλευση ειδικών επεξεργαστών, όπως επεξεργαστές κορυφών και τεμάχια επεξεργαστών, που κατέστησαν κάποιες λειτουργίες πιο ευέλικτες.\\
Όταν οι τιμές συνέχισαν να πέφτουν ενώ η υπολογιστική δύναμη αυξανόταν, η ερευνητική κοινότητα σκέφτηκε τρόπους να αξιοποιηθεί αυτή η δύναμη για τον υπολογισμό δύσκολων λειτουργιών. Όμως, καθώς η δυνατότητα των επεξεργαστών ήταν περιορισμένη και η διεπαφή προγραμματιστικής διεπαφής (API) των οδηγών γραφικών ήταν σχεδιασμένη για να υλοποιεί συγκεκριμένα την σειρά αγωγών, έπρεπε να ληφθούν υπόψιν πολλές παράμετροι. Για παράδειγμα, όλα τα δεδομένα έπρεπε να κωδικοποιηθούν σε υφές ως πίνακες δυο διαστάσεων που αναπαριστούν pixel, με περιεχόμενο τιμές χρωμάτων και κάποιο κανάλι alpha για την διαφάνεια. Επιπλέον, οι υφές είναι αντικείμενα μόνο προσπελάσιμα, και δεν επαναγράφονταν, κάτι που ανάγκαζε τους προγραμματιστές να αποθηκεύουν κάθε φορά καινούρια υφή με τις αλλαγές. Τέλος, οι περισσότερες GPU υποστήριζαν μόνο λειτουργίες μονής κινητής υποδιαστολής, αναγκάζοντας τους προγραμματιστές να προσομοιώνουν λογικές λειτουργίες.\\
Αυτοί οι περιορισμοί, ήταν ο μεγαλύτερος λόγος που ώθησε τους κατασκευαστές GPU (AMD,NVIDIA,INTEL), να δημιουργήσουν προγραμματιστικές διεπαφές ειδικές για την κοινότητα του GPGPU και να εξελίξουν τις συσκευές τους για καλύτερη υποστήριξη.

Το πεδίο του προγραμματισμού γενικής χρήσης έχει αναπτυχθεί με ταχύτατους ρυθμούς τα τελευταία χρόνια, έτσι ώστε τώρα υπάρχουν αρκετές υλοποιήσεις για τον προγραμματισμό των μονάδων επεξεργασίας γραφικών. Πρόσφατα, έχουν γίνει προσπάθειες δημιουργίας προτύπων.
Ο προγραμματισμός των GPUs αναπτύχθηκε όταν το CUDA και το Stream κατέφθασαν στο τέλος του 2006. Αυτές οι διεπαφές και οι γλώσσες, σχεδιάστηκαν από τις εταιρίες κατασκευής των GPUs σε πολύ κοντινή σχέση με το υλικό, το οποίο αποτέλεσε μεγάλο βήμα προς ένα πιο εύχρηστο, ταιριαστό και μελλοντικά-ασφαλές προγραμματιστικό μοντέλο. Η ανοιχτή γλώσσα προγραμματισμού (OpenCL) δημιουργήθηκε για να παρέχει ένα γενικό API ετερογενή υπολογισμού σε διάφορες μορφές παράλληλων συσκευών, συμπεριλαμβανομένου μονάδων επεξεργασίας γραφικών, πολυπύρηνων κεντρικών μονάδων επεξεργασίας, κ.α

\section{CUDA}
\subsection{Εισαγωγή}
To CUDA είναι μια πλατφόρμα παράλληλου υπολογισμού, που δημιουργήθηκε από την NVIDIA και υλοποιήθηκε στις κάρτες γραφικών τις οποίες παράγει η ίδια. Το CUDA δίνει στους προγραμματιστές άμεση πρόσβαση στο σετ εικονικών εντολών και την μνήμη των στοιχείων του παράλληλου υπολογισμού σε κάρτες γραφικών NVIDIA. \\
\\
Αξιοποιώντας το CUDA, οι κάρτες γραφικών(GPU) μπορούν να χρησιμοποιηθούν για υπολογισμό γενικής χρήσης (δηλαδή όχι αποκλειστικά για γραφικά).Οι GPU έχουν μια αρχιτεκτονική παράλληλης εξόδου η οποία δίνει έμφαση στην εκτέλεση πολλών threads με μικρή ταχύτητα, σε αντίθεση με τις CPU όπου εκτελείται ένα thread με μεγάλη ταχύτητα. \\
\\
Η πλατφόρμα CUDA είναι προσβάσιμη στους προγραμματιστές μέσω βιβλιοθηκών, εντολών μεταγλώττισης, και προεκτάσεων σε γλώσσες προγραμματισμού βιομηχανικής κλίμακας, όπως η C, C++ και Fortran. Οι προγραμματιστές της C/C++, χρησιμοποιούν το CUDA C/C++, μεταγλωττισμένο με το nvcc, έναν LLVM βασισμένο μεταγλωττιστή, και οι προγραμματιστές της Fortran χρησιμοποιούν το CUDA Fortran, μεταγλωττισμένο με τον μεταγλωττιστή PGI CUDA Fortran απο το The Portland Group. Εκτώς απο τα παραπάνω, η πλατφόρμα CUDA υποστηρίζει και άλλες διεπαφές υπολογισμού, όπως το OpenCL του Khronos Group, το DirectCompute της Microsoft, και το C++ AMP.\\
Στην βιομηχανία των υπολογιστών, οι GPUs δεν χρησιμοποιούνται μόνο για τα γραφικά αλλά και στους υπολογισμούς φυσικής παιχνιδιών (π.χ καπνός, φωτιά, ροή υγρών). Γνωστά παραδείγματα αποτελούν οι μηχανές PhysX και η Bullet. Το CUDA επίσης χρησιμοποιείται για να επιταχύνει μη-γραφικές εφαρμογές στην βιοπληροφορική, στην κρυπτογραφία, και σε πολλά άλλα πεδία.\\

Γενικότερα, η υπολογιστική δύναμη της GPU, βασίζεται στην παράλληλη αρχιτεκτονική της. Για αυτό, η πλατφόρμα του CUDA παρουσιάζει το νήμα(thread) ως το μικρότερο στοιχείο παραλληλισμού. Όμως, σε σύγκριση με την κεντρική μονάδα επεξεργασίας, τα νήματα της GPU έχουν μικρότερο κόστος χρήσης πόρων και μικρότερο κόστος δημιουργίας και αντικατάστασης. Σημειώνεται ότι οι GPU είναι αποτελεσματικές, μόνο όταν τρέχει μεγάλος αριθμός απο τέτοια νήματα. Μια ομάδα από νήματα, που εκτελούνται παράλληλα, επικοινωνούν και συγχρονίζονται μεταξύ τους ονομάζεται block. Ο μέγιστος αριθμός των νημάτων σε ενα block είναι ένας περιορισμός που υπάρχει στην κάθε μονάδα γραφικής επεξεργασίας. Τέλος, μια ομάδα από blocks τα οποία έχουν την ίδια διάσταση και εκτελούνται απο το ίδιο πρόγραμμα CUDA παράλληλα, ονομάζεται πλέγμα.\\
Για να επιτρέψει βέλτιστη επίδοση για διαφορετικά πρότυπα, το CUDA εκτελεί ένα ιεραρχικό μοντέλο μνήμης, αντίθετα με τα παραδοσιακά μοντέλα που συναντάμε συνήθως στους υπολογιστές. Ο υπολογιστής και η συσκευή, έχουν τις δικές τους περιοχές μνήμης, τις οποίες ονομάζουν host memory και device memory, αντίστοιχα. Το CUDA παρέχει βελτιστοποιημένες λειτουργίες για να μεταφέρει δεδομένα από και προς αυτούς τους ξεχωριστούς χώρους.
Κάθε νήμα κατέχει το δικό του αρχείο καταχώρησης, το οποίο μπορεί να προσπελαστεί και να εγγραφεί. Επιπλέον, μπορεί να προσπελάσει το δικό του αντίγραφο της τοπικής μνήμης. Όλα τα νήματα στο ίδιο πλέγμα μπορούν να προσπελάσουν και να γράψουν στην περιοχή της κοινόχρηστης μνήμης (shared memory). Για να αποφευχθούν κίνδυνοι από ταυτόχρονη προσπέλαση, μηχανισμοί συγχρονισμού νημάτων πρέπει να χρησιμοποιηθούν. Η κοινόχρηστη μνήμη, είναι οργανωμένη σε ομάδες που ονομάζονται τράπεζες, οι οποίες μπορούν να προσπελαστούν παράλληλα. Όλα τα νήματα έχουν επίσης πρόσβαση στον χώρο μνήμης που ονομάζεται καθολική μνήμη (global memory) και στις περιοχές που ονομάζονται μνήμη σταθερών (constant memory) και μνήμη υφής (texture memory).

\subsection{Πλεονεκτήματα}
Το CUDA έχει τα εξής πλεονεκτήματα σε σχέση με τους παραδοσιακούς τρόπους υπολογισμού γενικής χρήσης που εκτελούνται μέσω προγραμματιστικών διεπαφών γραφικών:
 Διασκορπισμένες προσπελάσεις - ο κώδικας μπορεί να διαβαστεί από αυθαίρετες διευθύνσεις στην μνήμη.
 Ενοποιημένη εικονική μνήμη (CUDA 6)
 Κοινόχρηστη μνήμη - το CUDA εκθέτει μια γρήγορη περιοχή κοινόχρηστης μνήμης (μέχρι 48KB για κάθε επεξεργαστή) η οποία μπορεί να μοιραστεί ανάμεσα στα threads. Αυτή μπορεί να χρησιμοποιηθεί σαν κρυφή μνήμη διαχειρίσιμη απο τον χρήστη, επιτρέποντας μεγαλύτερο εύρος δεδομένων απο ότι είναι δυνατό με τις προσπελάσεις υφών.
 Πιο γρήγορες μεταφορτώσεις και προσπελάσεις από και προς την GPU
 Πλήρης υποστήριξη για ακέραιες και bitwise λειτουργίες, για παράδειγμα τις προσπελάσεις υφών.
\subsection{Περιορισμοί}
 Το CUDA δεν υποστηρίζει ολόκληρο το πρότυπο της γλώσσας C, καθώς τρέχει μέσω ενός μεταγλωττιστή C++, ο οποίος εμποδίζει συγκεκριμένα μέρη της γλώσσας C να μεταγλωττιστούν.
 
\begin{table}
\begin{tabular}{ | c | c |}
Feature support (unlisted features are\\ supported for all compute capabilities) & Compute capability (version)\\ \hline
\end{tabular}
\end{table}

\section{OpenCL}
\subsection{Εισαγωγή}
Το OpenCL αποτελεί το πρώτο ανοιχτό, ελεύθερο από τέλη αδειών πρότυπο για cross-platform, παράλληλο προγραμματισμό μοντέρνων επεξεργαστών, που χρησιμοποιούνται συνήθως σε προσωπικούς υπολογιστές, διακομιστές, και φορητές/ενσωματωμένες συσκευές. Το OpenCL (Open Computing Language) βελτιώνει αισθητά την ταχύτητα και την απόκριση μεγάλου εύρους εφαρμογών σε διάφορες κατηγορίες αγορών από παιχνίδια και ψυχαγωγία, μέχρι επιστημονικές εφαρμογές και εφαρμογές υγείας.
\subsection{Φορητός Ετερογενής Υπολογισμός}
\begin{itemize}
\item Ελεύθερο από τέλη έμφυτο, cross-platform, cross-vendor πρότυπο
	\begin{itemize}
	\item Στοχεύει υπέρ-υπολογιστές -> ενσωματωμένα συστήματα -> φορητές συσκευές
	\end{itemize}
\item Επιτρέπει προγραμματισμό με διάφορους υπολογιστικούς πόρους
	\begin{itemize}
	\item CPU, GPU, DSP, FPGA - και τεμάχια υλικού
	\end{itemize}
\item Ένα δέντρο κώδικα μπορεί να εκτελείται σε CPU, GPU, DSP, και υλικό
	\begin{itemize}
	\item Δυναμική αναγνώριση φόρτου συστήματος και ισομοιρασμός στους διαθέσιμους επεξεργαστές
	\end{itemize}
\item Δυνατή, χαμηλού επιπέδου ευελιξία
	\begin{itemize}
	\item Θεμελιώδης πρόσβαση στους υπολογιστικούς πόρους για υψηλού επιπέδου μηχανές, εργαλεία και γλώσσες
	\end{itemize}
\end{itemize}
\subsection{Αρχιτεκτονική}
\begin{itemize}[leftmargin=*]
\item C API Επίπεδο πλατφόρμας
	\begin{itemize}
	\item Αναζήτηση, επιλογή και εκκίνηση συσκευών υπολογισμού
	\end{itemize}
\item Προσδιορισμός Γλώσσας Πυρήνα
	\begin{itemize}
	\item Υποσύνολο του ISO C99 με επεκτάσεις γλώσσας
	\item Πολύ καλά προσδιορισμένη αριθμητική ακρίβεια - IEEE 754 στρογγυλοποίηση με ορισμένη μέγιστη απόκλιση
	\item Πλούσιο σύνολο από ενσωματωμένες λειτουργίες: cross, dot, sin, cos, pow, log...
	\end{itemize}
\item C API Εκτέλεσης
	\begin{itemize}
	\item Μεταγλώττιση πυρήνα σε χρόνο εκτέλεσης ή χρόνο κατασκευής
	\item Εκτέλεση πυρήνων υπολογισμού σε πολλαπλές συσκευές
	\end{itemize}
\item Ενσωμάτωση προφίλ
	\begin{itemize}
	\item Καμία ανάγκη για ξεχωριστό προσδιορισμό "ES"
	\item Μειώνει τις ανάγκες ακρίβειας
	\end{itemize}
\end{itemize}
\subsection{Μοντέλο πλατφόρμας}
\begin{itemize}
\item Ένας οικοδεσπότης είναι συνδεδεμένος με μια η περισσότερες OpenCL συσκευές.
\item Μια OpenCL συσκευή είναι μια συλλογή από μία ή περισσότερες υπολογιστικές μονάδες.
\item Ένα υπολογιστικό στοιχείο συντίθεται από ένα ή περισσότερα στοιχεία επεξεργασίας.
\end{itemize}
%\begin{figure}[b]
%\begin{flushright}
%\includegraphics[scale=1]{images/OpenCL_Logo_RGB_60mm}
%\end{flushright}
%\end{figure}



\section{Βιβλιογραφικές αναφορές}
H πρώτη βιβλιογραφική αναφορά είναι: \parencite{doe}, ενώ η δεύτερη είναι \parencite{papadopoulos}.

\printbibliography[title={Βιβλιογραφία}]
\end{document}