%
%	Copyright 2014 George Sofianos
%
%	Licensed under the EUPL, Version 1.1 or – as soon they
%	will be approved by the European Commission - subsequent
%	versions of the EUPL (the "Licence");
%	You may not use this work except in compliance with the
%	Licence.
% 	You may obtain a copy of the Licence at:
%
%	http://ec.europa.eu/idabc/eupl
%
%	Unless required by applicable law or agreed to in
%	writing, software distributed under the Licence is
%	distributed on an "AS IS" basis,
%	WITHOUT WARRANTIES OR CONDITIONS OF ANY KIND, either
%	express or implied.
%	See the Licence for the specific language governing
%	permissions and limitations under the Licence.
%
\section{Άδεια χρήσης}
\subsection{Εισαγωγή}
Η άδεια χρήσης λογισμικού, είναι ένα νομικό μέσο (συνήθως της μορφής του δικαίου των συμβάσεων) το οποίο ορίζει την χρήση και την διαμοίραση του λογισμικού. Σύμφωνα με τον Ευρωπαϊκό νόμο, όταν κάποιος γράφει ένα λογισμικό, η πνευματική ιδιοκτησία για αυτό το έργο προστατεύεται από τον νόμο, όπως ένα έργο λογοτεχνίας η τέχνης. Ο νόμος της πνευματικής ιδιοκτησίας, δίνει στον ιδιοκτήτη του έργου συγκεκριμένα δικαιώματα, και βάζει όρια στο πώς οι άλλοι μπορούν να χρησιμοποιήσουν το έργο. Η πνευματική ιδιοκτησία στο λογισμικό προέρχεται από την προστασία των γραπτών έργων, και είναι σημαντικό να γνωρίζουμε ότι το λογισμικό του υπολογιστή αντιμετωπίζεται από τον νόμο σαν ένα έργο λογοτεχνικό. Η ιδιοκτησία πνευματικών δικαιωμάτων ενός έργου, είτε βιβλίου είτε λογισμικού, σημαίνει ότι ο ιδιοκτήτης, δηλαδή ο εκδότης ή ο εργοδότης, αποφασίζει για το ποιος μπορεί να το αντιγράψει, να το τροποποιήσει και να το διανέμει. Εξ αρχής, μόνο ο ιδιοκτήτης μπορεί να το κάνει αυτό.\cite{EUPL-guideline} Όποιος αντιγράψει,τροποποιήσει ή διανέμει έργο που ανήκει σε κάποιον άλλον χωρίς την άδεια του, μπορεί να βρεθεί αντιμέτωπος με τον νόμο. \\
Για να αποκτήσει κάποιος άδεια για τα παραπάνω, έχουμε συγκεκριμένες άδειες που θεωρούνται συμβόλαια, μεταξύ του εκδότη του λογισμικού και του χρήστη, ο οποίος μπορεί να το χρησιμοποιήσει όπως αναγράφεται στους όρους της άδειας χρήσης. Σημείωση: σε περίπτωση που δεν συμφωνεί ο χρήστης με τους όρους της άδειας, δεν μπορεί να χρησιμοποιήσει,αντιγράψει,τροποποιήσει, ή διανέμει το λογισμικό. Σε αντίθετη περίπτωση, παραβιάζει τον νόμο περί πνευματικών δικαιωμάτων. \\
Συνήθως οι άδειες χρήσεις λογισμικού εντάσσονται σε μια από τις παρακάτω κατηγορίες, ενώ η διαφορά τους βρίσκεται στους όρους με τους οποίους ο τελικός χρήστης μπορεί στην συνέχεια να διαμοιράσει το λογισμικό.
\subsubsection{Άδειες αποκλειστικής ιδιοκτησίας}
Το σήμα κατατεθέν του λογισμικού αποκλειστικής ιδιοκτησίας είναι ότι ο εκδότης επιτρέπει την χρήση μιας η περισσότερων αντίτυπων του λογισμικού, κάτω από μια συμφωνητικού τύπου άδεια χρήσης (EULA), αλλά η ιδιοκτησία των αντίτυπων παραμένει στον εκδότη. Αυτό το χαρακτηριστικό αυτής της άδειας, σημαίνει ότι συγκεκριμένα δικαιώματα σχετικά με το λογισμικό έχουν παρακρατηθεί από τον εκδότη του λογισμικού. Γιαυτό είναι συνηθισμένο, στις EULA να υπάρχουν όροι που ορίζουν της χρήσεις του λογισμικού, όπως ο αριθμός των εγκαταστάσεων που επιτρέπονται από τους όρους της διανομής.\\
Η μεγαλύτερη επίδραση αυτού του τύπου άδειας, είναι ότι, αν η ιδιοκτησία του λογισμικού παραμένει στον εκδότη, τότε ο τελικός χρήστης πρέπει να αποδεχτεί την άδεια του λογισμικού, ενώ σε περίπτωση που δεν το κάνει, δεν μπορεί να χρησιμοποιήσει το λογισμικό. Σε αυτού του τύπου άδειες περιέχεται συνήθως και μια εκτενής λίστα με λειτουργίες που απαγορεύονται, όπως για παράδειγμα reverse engineering, ταυτόχρονη χρήση λογισμικού από πολλούς χρήστες, κ.α
\subsubsection{Άδειες ελεύθερου και ανοιχτού κώδικα}
Οι άδειες ελεύθερου και ανοιχτού κώδικα, κατατάσσονται σε μια από τις δύο κατηγορίες: 
\begin{itemize}
\item Permissive άδειες - Αυτές οι άδειες έχουν λιγότερες απαιτήσεις για το πώς θα χρησιμοποιηθεί και διανεμηθεί το λογισμικό. Παράδειγμα permissive άδειας είναι η άδεια BSD και η άδεια MIT, που δίνουν απόλυτη ελευθερία για χρήση,μελέτη, και ιδιωτική τροποποίηση του λογισμικού, και περιέχει μόνο ελάχιστες απαιτήσεις για την διανομή. Αυτό δίνει σε κάποιον την δυνατότητα να χρησιμοποιήσει τον κώδικα σαν μέρος ενός λογισμικού κλειστού κώδικα, ή λογισμικού αποκλειστικής ιδιοκτησίας.
\begin{figure}[h]
	\centering
	\includegraphics[scale=0.15]{osi_standard_logo}
	\caption{Λογότυπο Open Source Initiative}
\end{figure}
\item Copyleft άδειες - Αυτές οι άδειες έχουν ως στόχο να διατηρήσουν τις ελευθερίες που δίνονται στον χρήστη. Ένα παράδειγμα copyleft άδειας χρήσης λογισμικού, είναι το GPL. Αυτή η άδεια, στοχεύει στο να δώσει στους χρήστες την ελευθερία για χρήση,μελέτη, και ιδιωτική τροποποίηση του λογισμικού, και αν ο χρήστης εμμείνει στους όρους και προϋποθέσεις του GPL, ελευθερία για διανομή του λογισμικού και κάθε τροποποίησης που έχει γίνει σε αυτό. Για παράδειγμα, κάθε τροποποίηση που γίνεται από τον χρήστη πρέπει να συνοδεύεται από τον πηγαίο κώδικα των αλλαγών όταν το λογισμικό διανεμηθεί, και η άδεια για οποιαδήποτε παράγωγο του έργου να μην έχει τους οποιεσδήποτε περιορισμούς εκτός από αυτούς που το GPL ορίζει.
\begin{figure}[h]
	\centering
	\includegraphics[scale=0.15]{fsf_logo}
	\caption{Λογότυπο Free Software Foundation}
\end{figure}
\end{itemize}

\subsection{Επιλογή}
Μετά από εκτενή αναζήτηση και έρευνα πάνω στις άδειες χρήσεις λογισμικου, αποφασίστηκε να χρησιμοποιηθούν δύο άδειες για την εκπόνηση της πτυχιακής εργασίας. Μία για τον πηγαίο κώδικα και μία για το τελικό αποτέλεσμα. 
\subsubsection{Πηγαίος κώδικας}
Η άδεια χρήσης EUPL 1.1 κρίθηκε η καλύτερη επιλογή για την πτυχιακή εργασία. Η EUPL είναι μια άδεια λογισμικού ελεύθερου, ανοιχτού κώδικα. Έχει εγκριθεί το 2009 απο το Open Source Initiative, καθώς εκπληρώνει τις προϋποθέσεις  του Open Source Definition (OSD)\cite{osi-definition}. Η άδεια εκπληρώνει επίσης τις προϋποθέσεις του Free Software Foundation (FSF) κάτι το οποίο μπορεί να συνοψίσει τέσσερις σημαντικές ελευθερίες στην άδεια
\begin{itemize}
\item Ελευθερία για χρήση ή εκτέλεση για οποιαδήποτε χρήση και για οποιονδήποτε αριθμό χρηστών
\item Ελευθερία απόκτησης του πηγαίου κώδικα (για μελέτη του τρόπου με τον οποίο λειτουργεί το λογισμικό)
\item Ελευθερία αναδιανομής αντιτύπων του λογισμικού
\item Ελευθερία μετατροπής, προσάρμοσης, βελτίωσης του λογισμικού σύμφωνα με συγκεκριμένες ανάγκες, και αναδιανομή αυτών των μετατροπών.
\end{itemize}

\subsubsection{Βιβλίο}
Για την επιλογή της άδειας χρησιμοποιήθηκε ο οδηγός στο site της creativecommons \url{http://creativecommons.org/choose/}
Επέλεξα την άδεια Attribution-ShareAlike 4.0 International \ccbysa γιατί θεωρώ ότι η πτυχιακή εργασία μου μπορεί να βοηθήσει στο μέλλον και άλλους φοιτητές να γνωρίσουν την τεχνολογία του \LaTeX ,αλλά ταυτόχρονα είναι και συμβατή με την άδεια χρήσης του πηγαίου κώδικα.
\vfill
Το περιεχόμενο του παρόντος κειμένου υπάγεται σε Άδεια Χρήσης Creative Commons Attribution 4.0 \url{http://creativecommons.org/licenses/by-sa/4.0/}