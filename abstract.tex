\chapter{Περίληψη}
\subsubsection{Περίληψη}
Σκοπός αυτής της εργασίας είναι η διερεύνηση και αξιολόγηση προτύπων και τεχνολογιών για προγραμματισμό γενικού σκοπού με χρήση μονάδων επεξεργασίας γραφικών, και η παρουσίαση εφαρμογών υψηλών υπολογιστικών απαιτήσεων οι οποίες εκμεταλλεύονται τις δυνατότητες αυτών των τεχνολογιών για επιτάχυνση και αύξηση των επιδόσεων τους. 

Λόγω της μεγάλης έκτασης των δυνατοτήτων της τεχνολογίας GPGPU, επιλέχτηκαν συγκεκριμένα πεδία για έρευνα τα οποία παρουσιάζουν ιδιαίτερο ενδιαφέρον. Αντίστοιχο βάρος δόθηκε στην έρευνα και καταγραφή των εφαρμογών που χρησιμοποιούνται στα συγκεκριμένα πεδία για την επίλυση διάφορων προβλημάτων. Τέλος, επιλέχτηκαν εφαρμογές για παρουσίαση που περιέχουν αρκετές γραφικές αναπαραστάσεις των λύσεων των προβλημάτων, αλλά και εφαρμογές που μελετούν την απόδοση των υλοποιήσεων σε προβλήματα που απασχολούν τους προγραμματιστές λόγω εντατικών υπολογισμών.
\\
\\
\subsubsection{Abstract}
The purpose of this thesis is the research and evaluation of patterns and technologies used by general purpose programming on graphics processing units, and the demonstration of high processing computing applications which exploit the capabilities of these technologies in order to accelerate and increase their performance.

Because of the great range of posibilities for GPGPU technology, specific fields were chosen for research which have a special interest. Additional focus was given for research and document the applications that are being actively used in the specific fields as solutions for various problems. Last, some applications were chosen for demonstration that include many graphical implementations of solutions of the problems, and applications that study the performance of implementations of computational intensive problems.

\newpage