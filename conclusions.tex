\chapter{Συμπεράσματα}
\section{Γενικά}
H χρήση της τεχνολογίας GPGPU είναι πλέον διαδεδομένη και ενισχύεται συνεχώς από προγραμματιστικές προσπάθειες που γίνονται σε όλον τον κόσμο. Η σημερινή μορφή του διαδικτύου βοηθάει στην εξάπλωση των καινοτόμων τεχνολογιών και μέσω μηχανισμών αλληλοβοήθειας (forums, stackoverflow, ειδικές ιστοσελίδες,κ.α) οι προγραμματιστές μπορούν να λύσουν τις απορίες τους και να εισέλθουν δυναμικά σε έναν καινούριο προγραμματιστικό μοντέλο.

Οι πλατφόρμες CUDA και OpenCL δεν αποτελούν εξαίρεση, καθώς υπάρχει τεράστιο υποστηρικτικό υλικό στο διαδίκτυο και στα βιβλιοπωλεία, ενώ τα εργαλεία ανάπτυξης λογισμικού που χρησιμοποιούνται για την ανάπτυξη GPGPU εφαρμογών γίνονται συνεχώς και πιο φιλικά προς τον χρήστη - προγραμματιστή. Ταυτόχρονα, ειδικά σεμινάρια προγραμματισμού για τις τεχνολογίες αυτές γίνονται όλο και πιο διαδεδομένα, με αποκορύφωση την ετήσια διοργάνωση της NVIDIA "GPU Technology Conference", όπου ακόμα και αν ο προγραμματιστής δεν μπορεί να παρευρεθεί στο σεμινάριο, μπορεί να βρει σχεδόν ολόκληρο το υλικό δωρεάν στο διαδίκτυο, μαζί με τις παρουσιάσεις και τα βίντεο της διοργάνωσης.

Οι εφαρμογές με επιτάχυνση μέσω GPUs έχουν αρχίσει να ωριμάζουν όσον αφορά επιστημονικά πεδία έρευνας όπως βιοπληροφορική, χημεία, ιατρική, επεξεργασία εικόνας και βίντεο, αλλά σε άλλα πεδία υπάρχει ακόμα ανάγκη για έρευνα και ανάπτυξη λογισμικού, με σκοπό την επιτάχυνση της επίλυσης διάφορων προβλημάτων. Για παράδειγμα, η επιτάχυνση GPU δεν χρησιμοποιείται αρκετά σε επιτραπέζιες λειτουργίες των λειτουργικών συστημάτων, όπως η αναζήτηση σε ένα ευρετήριο. Θεωρώ ότι υπάρχει αρκετός χώρος και ευκαιρίες για ανάπτυξη εφαρμογών οπτικής απεικόνισης, ειδικότερα όσον αφορά την υλοποίηση σε εφαρμογές διαδικτύου.

Η έρευνα μου στις ήδη υπάρχουσες εφαρμογές μου έδωσε την εντύπωση ότι ενώ υπάρχουν εφαρμογές και υλοποιήσεις που χρησιμοποιούν επιτάχυνση GPU, αυτές είναι μόνο ένα μικρό ποσοστό σε σχέση με τις εφαρμογές που υπάρχουν και αναπτύσσονται οι οποίες χρησιμοποιούν αποκλειστικά εκτέλεση μέσω CPU. Ο λόγος για αυτό είναι σίγουρα ότι η τεχνολογία έχει μόλις αρχίσει να ωριμάζει, αλλά μεγάλο ρόλο παίζει και η διάσπαση των τεχνολογιών σε CUDA και OpenCL, κάτι το οποίο διχάζει τις εταιρίες και αποτελεί πρόβλημα στην επιλογή πλατφόρμας. Το OpenCL, δείχνει να επωφελείται περισσότερο από αυτήν την κατάσταση, αφού εκτελείται στις περισσότερες συσκευές συμπεριλαμβανομένου και στις συσκευές NVIDIA και αυτό φαίνεται από το πλήθος τον εφαρμογών που αναπτύσσονται σε σχέση με τις εφαρμογές CUDA. Πρόσφατα, η τελευταίες κάρτες της NVIDIA ξεπέρασαν σε απόδοση OpenCL τις αντίστοιχες κάρτες της AMD, κάτι που δείχνει την πρόθεση και των δύο εταιριών να συνεχίσουν να υποστηρίζουν την πλατφόρμα OpenCL. 

Τέλος, το πλεονέκτημα της πλατφόρμας DirectCompute με το οποίο ο προγραμματιστής γράφει κατευθείαν εντολές OpenGL και DirectX για εντατικούς υπολογισμούς, μπορεί να αλλάξει την σκηνή μελλοντικά, και να κερδίσει έδαφος από τις ανταγωνιστικές υλοποιήσεις.
\section{Προτάσεις}
Αυτή η εργασία μπορεί να αποτελέσει την βάση για επόμενες εργασίες, όπως για ανάπτυξη εφαρμογών διαδικτύου αλλά και εκπαιδευτικές εφαρμογές επιτάχυνσης μέσω GPU, ή ακόμα και να αναπτυχθεί περισσότερο με καταγραφή και ανάλυση των πεδίων επιστημονικών και όχι μόνο, στα οποία υπάρχουν ελλείψεις εφαρμογών με επιτάχυνση GPU. Για παράδειγμα, οι περισσότεροι κωδικοποιητές βίντεο και ήχου δεν υποστηρίζουν ακόμα επιτάχυνση GPU, και οι μεταγλωττιστές προγραμματιστικών γλωσσών δεν έχουν εκμεταλλευτεί ακόμα τις δυνατότητες των GPUs. Επιπλέον, η χρήση του \hologo{XeTeX} που έγινε για την συγκεκριμένη εργασία, θα μπορούσε να εκμεταλλευτεί τις δυνατότητες της επιτάχυνσης GPU για να μεταγλωττίσει και σε συνεργασία με την CPU να παράγει πιο γρήγορα το τελικό αρχείο παρουσίασης.

Θεωρώ ένα από τα πιο σημαντικά και πρόσφατα ζητήματα, την ανάπτυξη εφαρμογών WebCL, η οποία θα παίξει μεγάλο ρόλο στο προσεχές μέλλον για εφαρμογές διαδικτύου και για αυτό θα πρότεινα την άμεση εκμετάλλευση του για εκπαιδευτικές και όχι μόνο εφαρμογές. Η χρήση της κάρτας γραφικών για επιτάχυνση υπολογισμών μέσα σε browsers που λόγω της φύσης τους είναι δύσκολο να αποδώσουν τα μέγιστα των δυνατοτήτων του υπολογιστή, ανοίγει νέες ευκαιρίες για εφαρμογές γραφικών αλλά και δίνει ένα επιπλέον προβάδισμα στην γλώσσα Javascript και βοηθάει στην μείωση του φόρτου εργασίας των νημάτων του browser.
 
Τέλος, η χρήση του Blender με την μηχανή φυσικής Bullet 3.0 η οποία βρίσκεται ακόμα σε δοκιμαστικό στάδιο και υποστηρίζει επιτάχυνση μέσω GPU, μπορεί να χρησιμοποιηθεί για εργασίες αλλά και για παραδείγματα εκπαιδευτικού τύπου στο μάθημα των γραφικών του ΤΕΙ, ώστε να προετοιμάσει και να εφοδιάσει τους φοιτητές με γνώσεις γραφικών αλλά και προγραμματισμού εφαρμογών ψυχαγωγίας όπως παιχνίδια ηλεκτρονικών υπολογιστών με πραγματικά εντυπωσιακή εξομοίωση φυσικής, κίνησης ρευστών και καπνού, δηλαδή ιδιότητες που μέχρι και πριν λίγα χρόνια ήταν αδύνατο να διδαχθούν λόγω περιορισμών στην υπολογιστική δύναμη αλλά και στην έλλειψη περιβαλλόντων ανάπτυξης ειδικών εφαρμογών.