\chapter{Εισαγωγή}
\epigraph{"Anyone can build a fast CPU. The trick is to build a fast system."}{Seymour Cray}

Τα αρχικά GPGPU πηγάζουν απο την φράση General Purpose computation on Graphics Processing Units, ή αλλιώς γνωστή ώς GPU Computing, δηλαδή υπολογισμός γενικού σκοπού σε μονάδες επεξεργασίας γραφικών.

Οι GPUs, είναι επεξεργαστές υψηλών επιδόσεων με δυνατότητα πολύ υψηλού υπολογισμού και διεκπεραιωτικότητας δεδομένων. Σχεδιασμένες αρχικά για γραφικά υπολογιστών με αρκετές δυσκολίες στον προγραμματισμό τους, οι σημερινές μονάδες επεξεργασίας γραφικών είναι παράλληλοι επεξεργαστές γενικής χρήσης με υποστήριξη για προσβάσιμες προγραμματιστικές διεπαφές και βιομηχανικά πρότυπα γλωσσών όπως η C.

Οι προγραμματιστές που μεταφέρουν τις εφαρμογές τους σε GPUs συνήθως πετυχαίνουν ταχύτητες πολλαπλάσιες από ότι μια αντίστοιχη εφαρμογή ειδικά βελτιστοποιημένη για κεντρική μονάδα επεξεργασίας (CPU).
Ο όρος GPGPU δημιουργήθηκε από τον Mark Harris το 2002 όταν συνειδητοποίησε ότι αναπτυσσόταν μια τάση για χρήση των μονάδων επεξεργασίας γραφικών για εφαρμογές που δεν είχαν σχέση με γραφικά. 

Από το 2012, οι GPU έχουν αναπτυχθεί σε συστήματα πολυπύρηνων επεξεργαστών παράλληλου υπολογισμού δίνοντας μας την δυνατότητα για πολύ αποδοτικό χειρισμό μεγάλου όγκου δεδομένων. Αυτός ο σχεδιασμός είναι πιο αποδοτικός από ότι οι κεντρικές μονάδες επεξεργασιάς (CPU) για αλγόριθμους όπου η επεξεργασία μεγάλου όγκου δεδομένων γίνεται παράλληλα, όπως σε αλγορίθμους sort μεγάλων λιστών, μετασχηματισμό κυμάτων δυο διαστάσεων, προσομοίωση βιολογικών δυναμικών.

Σκοπός αυτής της εργασίας είναι να διερευνήσει και αξιολογήσει πρότυπα και τεχνολογίες για προγραμματισμό γενικού σκοπού με χρήση μονάδων επεξεργασίας γραφικών, ειδικότερα όσον αφορά εφαρμογές υψηλών υπολογιστικών απαιτήσεων οι οποίες εκμεταλλεύονται τις δυνατότητες αυτών των τεχνολογιών για επιτάχυνση και αύξηση των επιδόσεων τους. Το θέμα είναι μεγάλης σημασίας



\section{Παράλληλος υπολογισμός}
Για 30 χρόνια, ένας από τους πιο σημαντικούς τρόπους για να βελτιώσουμε την απόδοση των υπολογιστικών συσκευών των καταναλωτών ήταν η αύξηση της ταχύτητας στην οποία λειτουργεί το ρολόι ενός επεξεργαστή. Ξεκινώντας από περίπου το 1MHZ το 1980, οι περισσότεροι σύγχρονοι επεξεργαστές έχουν ταχύτητες μεταξύ 1GHz και 4GHz, δηλαδή είναι περίπου 1000 φορές πιο γρήγοροι. Αν και δεν είναι ο μόνος τρόπος με τον οποίο έχουν βελτιωθεί οι επεξεργαστές, αποτελεί συνήθως μια αξιόπιστη πηγή για αύξηση της απόδοσης.

Τα τελευταία χρόνια όμως, οι κατασκευαστές έχουν αναγκαστεί να ψάξουν για εναλλακτικούς τρόπους αύξησης της υπολογιστικής δύναμης. Εξ αιτίας διάφορων περιορισμών στην κατασκευή ενσωματωμένων κυκλωμάτων, δεν είναι πλέον εύκολο να αυξάνουμε την ταχύτητα του ρολογιού του επεξεργαστή σαν τρόπο αύξησης της απόδοσης στις υπάρχουσες αρχιτεκτονικές. Στην αναζήτηση για επιπλέον υπολογιστική δύναμη για τους προσωπικούς επεξεργαστές, οι ερευνητές χρησιμοποίησαν τεχνολογίες που ήταν ήδη γνωστές από τους υπερ-υπολογιστές, στους οποίους είναι σύνηθες φαινόμενο να αποτελούνται από δεκάδες ή εκατοντάδες επεξεργαστές, οι οποίοι εκτελούν παράλληλες διεργασίες. Έτσι το 2005, οι κύριοι κατασκευαστές επεξεργαστών άρχισαν να προσφέρουν επεξεργαστές με δύο πυρήνες αντί για έναν. 

Τα επόμενα χρόνια, ακολούθησαν υλοποιήσεις με τρεις,τέσσερις,έξι, ακόμα και οκτώ πυρήνες. Έχει ξεκινήσει ήδη μια μεγάλη στροφή της βιομηχανίας υπολογιστών στον παράλληλο υπολογισμό. Με την κυκλοφορία των διπύρηνων μέχρι και 8 ή 16 πυρήνων επεξεργαστών για σταθμούς εργασίας, ο παράλληλος υπολογισμός δεν είναι πλέον υπόθεση που αφορά μόνο τους εξωτικούς υπερ-υπολογιστές. Επίσης οι φορητές συσκευές όπως κινητά τηλέφωνα και φορητές συσκευές μουσικής έχουν αρχίσει να ενσωματώνουν δυνατότητες παράλληλου υπολογισμού σε μια προσπάθεια να προσφέρουν δυνατότητες πολύ ανώτερες από τους προγόνους τους. Όλο και περισσότερο, οι προγραμματιστές λογισμικού πρέπει να εξοικειωθούν με πλατφόρμες και τεχνολογίες παράλληλου υπολογισμού ώστε να προμηθεύουν με πλούσιες εμπειρίες την βάση των χρηστών τους. Το μέλλον αποτελείται από πολύ-νηματικές εφαρμογές, και από φορητές συσκευές που μπορούν ταυτόχρονα να παίζουν μουσική, να εξερευνούν το διαδίκτυο, και να παρέχουν GPS υπηρεσίες.
\section{GPU Computing}

Η επιτάχυνση υπολογισμού γενικού σκοπού από μονάδες επεξεργασίας γραφικών, είναι η χρήση των GPUs μαζί με χρήση CPUs για να επιταχύνουν επιστημονικές, αναλυτικές, κατασκευαστικές, καταναλωτικές, και εμπορικές εφαρμογές. Από την πρώτη εμφάνιση της τεχνολογίας το 2007 από την NVIDIA, οι επιταχυντές GPU τώρα βρίσκονται σε κέντρα δεδομένων σε εργαστήρια αποδοτικής ενέργειας για λογαριασμό κυβερνήσεων, πανεπιστήμια, και μικρές και μεγάλες επιχειρήσεις σε όλον τον κόσμο. Οι GPUs επιταχύνουν εφαρμογές σε πλατφόρμες που εκτείνονται από αυτοκίνητα, σε κινητά τηλέφωνα, σε drones και ρομπότ.

Η επιτάχυνση μέσω μονάδων επεξεργασίας γραφικών παρέχει πρωτοφανή αποτελέσματα επιδόσεων διαχωρίζοντας και φορτώνοντας τις εντολές εντατικών υπολογισμών στην GPU, ενώ διατηρεί την εκτέλεση του υπόλοιπου κώδικα στην CPU. Από την οπτική γωνία του χρήστη, οι εφαρμογές απλώς τρέχουν πιο γρήγορα.

\begin{figure}[h]
\centering
\includegraphics[scale=0.50]{gpuintro1}
\caption{Εκτέλεση κώδικα σε CPU και GPU}
\end{figure}