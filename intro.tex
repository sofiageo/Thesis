\chapter{Εισαγωγή}
\epigraph{"Anyone can build a fast CPU. The trick is to build a fast system."}{Seymour Cray}

Τα αρχικά GPGPU πηγάζουν απο την φράση General Purpose computation on Graphics Processing Units, ή αλλιώς γνωστή ώς GPU Computing, δηλαδή υπολογισμός γενικού σκοπού σε μονάδες επεξεργασίας γραφικών.

Οι GPUs, είναι επεξεργαστές υψηλών επιδόσεων με δυνατότητα πολύ υψηλού υπολογισμού και διεκπεραιωτικότητας δεδομένων. Σχεδιασμένες αρχικά για γραφικά υπολογιστών με αρκετές δυσκολίες στον προγραμματισμό τους, οι σημερινές μονάδες επεξεργασίας γραφικών είναι παράλληλοι επεξεργαστές γενικής χρήσης με υποστήριξη για προσβάσιμες προγραμματιστικές διεπαφές και βιομηχανικά πρότυπα γλωσσών όπως η C.

Οι προγραμματιστές που μεταφέρουν τις εφαρμογές τους σε GPUs συνήθως πετυχαίνουν ταχύτητες πολλαπλάσιες από ότι μια αντίστοιχη εφαρμογή ειδικά βελτιστοποιημένη για κεντρική μονάδα επεξεργασίας (CPU).
Ο όρος GPGPU δημιουργήθηκε από τον Mark Harris το 2002 όταν συνειδητοποίησε ότι αναπτυσσόταν μια τάση για χρήση των μονάδων επεξεργασίας γραφικών για εφαρμογές που δεν είχαν σχέση με γραφικά. 
Οποιοσδήποτε κοίταγε κάποιον GPGPU κώδικα 5 χρόνια πριν θα πίστευε ότι είναι απλός κώδικας γραφικών. Ο κώδικας φόρτωνε περίεργα μέρη υφών στην GPU, ζωγράφιζε τρίγωνα και τα μετακινούσε, και τέλος διάβαζε το αποτέλεσμα στην μνήμη, και επαναλάμβανε.

Δεν έβγαζε νόημα για κάποιον να τον παρακολουθήσει, αλλά οι προγραμματιστές που γνώριζαν τον εσωτερικό τρόπο λειτουργίας, το θεωρούσαν θαύμα του κόσμου της πληροφορικής. Μπορούσαν να στείλουν δεδομένα σε υφές και να τις εμφανίσουν στην οθόνη. Να εμφανίσουν άλλες υφές πάνω από αυτές, και να αλλάξουν τις ιδιότητες της αντανάκλασης και άλλων λειτουργιών, και να διαβάσουν πίσω τα αποτελέσματα. Υπήρχαν όμως προβλήματα όπως ότι η μεγαλύτερη καθυστέρηση προερχόταν από την φόρτωση των δεδομένων στην GPU, και στην ανάγνωση τους από αυτήν. Επίσης δεν υπήρχε δυνατότητα διακλαδώσεων λειτουργιών και προϋποθέσεων, ή στην καλύτερη ήταν πολύ περιορισμένη. Τέλος, η γλώσσα ήταν πολύ δύσκολη να χρησιμοποιηθεί. 

Τα δύο πρώτα αποτελούν ακόμα πρόβλημα. Η μεταφορά δεδομένων από την GPU προς την κύρια μνήμη είναι η μεγαλύτερη καθυστέρηση που συμβαίνει στα προγράμματα GPGPU. Αυτό σημαίνει ότι οι αλγόριθμοι μας πρέπει να είναι δομημένοι με τέτοιο τρόπο ώστε όλα τα δεδομένα να μπορούν να αποσταλούν στην GPU, να επεξεργαστούν με κάποιον τρόπο, και να μας επιστρέψει την απάντηση. Οι αλγόριθμοι που υπολογίζουν έναν αριθμό και μετά τον χρησιμοποιούν σε άλλον υπολογισμό, με επαναληπτικές διαδικασίες, δεν δουλεύουν καλά σε μια GPU. Αυτή η καθυστέρηση μας ώθησε στο να μετατρέψουμε μεγάλο μέρος κώδικα από αλγορίθμους σε νέα, φιλική μορφή για παράλληλο υπολογισμό. 

Από το 2012, οι GPU έχουν αναπτυχθεί σε συστήματα πολυπύρηνων επεξεργαστών παράλληλου υπολογισμού δίνοντας μας την δυνατότητα για πολύ αποδοτικό χειρισμό μεγάλου όγκου δεδομένων. Αυτός ο σχεδιασμός είναι πιο αποδοτικός από ότι οι κεντρικές μονάδες επεξεργασιάς (CPU) για αλγόριθμους όπου η επεξεργασία μεγάλου όγκου δεδομένων γίνεται παράλληλα, όπως σε αλγορίθμους sort μεγάλων λιστών, μετασχηματισμό κυμάτων δυο διαστάσεων, προσομοίωση βιολογικών δυναμικών. Ένα παράδειγμα αυτής της μετατροπής είναι ο παράλληλος FFT (Fast Fourier Transform), που χρησιμοποιείται σε πολλές ρουτίνες επεξεργασίας ήχου και εικόνας.

Τα προβλήματα που αναφέραμε σημαίνει ότι η επιτάχυνση μέσω GPU σε προγράμματα της αγοράς θα αργήσει να έρθει. Αν και μπορούμε να βρούμε υλοποιήσεις σε εφαρμογές καταναλωτών, η χρήση της είναι περιορισμένη σε περιοχές όπου η έρευνα έχει ολοκληρωθεί, κυρίως σε προγράμματα επεξεργασίας βίντεο/εικόνας. Προγράμματα όπως Adobe Premiere, Nero Move-it, και άλλα, χρησιμοποιούν το CUDA και άλλες GPGPU τεχνολογίες για να μειώσουν τους χρόνους της κωδικοποίησης και αποκωδικοποίησης. Άλλες εφαρμογές, όπως ο έλεγχος ορθογραφικών λαθών ή προγράμματα διαμοιρασμού αρχείων, δεν έχουν βρει ακόμα τρόπους να αυξήσουν την απόδοση μέσω του GPGPU. Οι υπολογιστές ακόμα "πνίγονται" από τους περιορισμούς I/O όταν γράφουν στους δίσκους, και αυτό συνήθως αποτελεί τον λόγο της υπολογιστικής καθυστέρησης καθώς τα αποτελέσματα πρέπει να σώζονται κάπου.

Σκοπός αυτής της εργασίας είναι να διερευνήσει και αξιολογήσει πρότυπα και τεχνολογίες για προγραμματισμό γενικού σκοπού με χρήση μονάδων επεξεργασίας γραφικών, ειδικότερα όσον αφορά εφαρμογές υψηλών υπολογιστικών απαιτήσεων οι οποίες εκμεταλλεύονται τις δυνατότητες αυτών των τεχνολογιών για επιτάχυνση και αύξηση των επιδόσεων τους. Το θέμα είναι μεγάλης σημασίας καθώς είναι δυνατόν να λυθούν ή να υπολογιστούν προβλήματα που σήμερα είναι πολύ δύσκολο να λυθούν, ενώ δημιουργεί καινούριες δυνατότητες και σκέψεις για περαιτέρω ανάπτυξη του τρόπου που οι υπολογιστές μας βοηθάνε να βελτιώσουμε τον τρόπο ζωής μας.

\section*{Παράλληλος υπολογισμός}
Για 30 χρόνια, ένας από τους πιο σημαντικούς τρόπους για να βελτιώσουμε την απόδοση των υπολογιστικών συσκευών των καταναλωτών ήταν η αύξηση της ταχύτητας στην οποία λειτουργεί το ρολόι ενός επεξεργαστή. Ξεκινώντας από περίπου το 1MHZ το 1980, οι περισσότεροι σύγχρονοι επεξεργαστές έχουν ταχύτητες μεταξύ 1GHz και 4GHz, δηλαδή είναι περίπου 1000 φορές πιο γρήγοροι. Αν και δεν είναι ο μόνος τρόπος με τον οποίο έχουν βελτιωθεί οι επεξεργαστές, αποτελεί συνήθως μια αξιόπιστη πηγή για αύξηση της απόδοσης.

Τα τελευταία χρόνια όμως, οι κατασκευαστές έχουν αναγκαστεί να ψάξουν για εναλλακτικούς τρόπους αύξησης της υπολογιστικής δύναμης. Εξ αιτίας διάφορων περιορισμών στην κατασκευή ενσωματωμένων κυκλωμάτων, δεν είναι πλέον εύκολο να αυξάνουμε την ταχύτητα του ρολογιού του επεξεργαστή σαν τρόπο αύξησης της απόδοσης στις υπάρχουσες αρχιτεκτονικές. Στην αναζήτηση για επιπλέον υπολογιστική δύναμη για τους προσωπικούς επεξεργαστές, οι ερευνητές χρησιμοποίησαν τεχνολογίες που ήταν ήδη γνωστές από τους υπερ-υπολογιστές, στους οποίους είναι σύνηθες φαινόμενο να αποτελούνται από δεκάδες ή εκατοντάδες επεξεργαστές, οι οποίοι εκτελούν παράλληλες διεργασίες. Έτσι το 2005, οι κύριοι κατασκευαστές επεξεργαστών άρχισαν να προσφέρουν επεξεργαστές με δύο πυρήνες αντί για έναν. 

Τα επόμενα χρόνια, ακολούθησαν υλοποιήσεις με τρεις,τέσσερις,έξι, ακόμα και οκτώ πυρήνες. Έχει ξεκινήσει ήδη μια μεγάλη στροφή της βιομηχανίας υπολογιστών στον παράλληλο υπολογισμό. Με την κυκλοφορία των διπύρηνων μέχρι και 8 ή 16 πυρήνων επεξεργαστών για σταθμούς εργασίας, ο παράλληλος υπολογισμός δεν είναι πλέον υπόθεση που αφορά μόνο τους εξωτικούς υπερ-υπολογιστές. Επίσης οι φορητές συσκευές όπως κινητά τηλέφωνα και φορητές συσκευές μουσικής έχουν αρχίσει να ενσωματώνουν δυνατότητες παράλληλου υπολογισμού σε μια προσπάθεια να προσφέρουν δυνατότητες πολύ ανώτερες από τους προγόνους τους. Όλο και περισσότερο, οι προγραμματιστές λογισμικού πρέπει να εξοικειωθούν με πλατφόρμες και τεχνολογίες παράλληλου υπολογισμού ώστε να προμηθεύουν με πλούσιες εμπειρίες την βάση των χρηστών τους. Το μέλλον αποτελείται από πολύ-νηματικές εφαρμογές, και από φορητές συσκευές που μπορούν ταυτόχρονα να παίζουν μουσική, να εξερευνούν το διαδίκτυο, και να παρέχουν GPS υπηρεσίες.
\section*{GPU Computing}

Η επιτάχυνση υπολογισμού γενικού σκοπού από μονάδες επεξεργασίας γραφικών, είναι η χρήση των GPUs μαζί με χρήση CPUs για να επιταχύνουν επιστημονικές, αναλυτικές, κατασκευαστικές, καταναλωτικές, και εμπορικές εφαρμογές. Από την πρώτη εμφάνιση της τεχνολογίας το 2007 από την NVIDIA, οι επιταχυντές GPU τώρα βρίσκονται σε κέντρα δεδομένων σε εργαστήρια αποδοτικής ενέργειας για λογαριασμό κυβερνήσεων, πανεπιστήμια, και μικρές και μεγάλες επιχειρήσεις σε όλον τον κόσμο. Οι GPUs επιταχύνουν εφαρμογές σε πλατφόρμες που εκτείνονται από αυτοκίνητα, σε κινητά τηλέφωνα, σε drones και ρομπότ.

Η επιτάχυνση μέσω μονάδων επεξεργασίας γραφικών παρέχει πρωτοφανή αποτελέσματα επιδόσεων διαχωρίζοντας και φορτώνοντας τις εντολές εντατικών υπολογισμών στην GPU, ενώ διατηρεί την εκτέλεση του υπόλοιπου κώδικα στην CPU. Από την οπτική γωνία του χρήστη, οι εφαρμογές απλώς τρέχουν πιο γρήγορα.

\begin{figure}[h]
\centering
\includegraphics[scale=1]{gpuintro1}
\caption{Εκτέλεση κώδικα σε CPU και GPU}
\end{figure}