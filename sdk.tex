\chapter{Υλοποιήσεις}
\section{Ιστορία}
Παραδοσιακά, η σειρά αγωγών των γραφικών, αποτελείται από τις καταστάσεις μετατροπή και φωτισμός, συναρμολόγηση αρχέγονων, μετατροπή σε pixels, και σκίαση. Οι πρώτες GPU είχαν όλες τις λειτουργίες που χρειάζονται για να εκτελεστεί η σειρά αγωγών, αλλά με τον καιρό όλο και περισσότερες καταστάσεις έγιναν δυνατό να προγραμματιστούν με την έλευση ειδικών επεξεργαστών, όπως επεξεργαστές κορυφών και τεμάχια επεξεργαστών, που κατέστησαν κάποιες λειτουργίες πιο ευέλικτες.\\
Όταν οι τιμές 
Το πεδίο του προγραμματισμού γενικής χρήσης έχει αναπτυχθεί με ταχύτατους ρυθμούς τα τελευταία χρόνια, έτσι ώστε τώρα υπάρχουν αρκετές υλοποιήσεις για τον προγραμματισμό των μονάδων επεξεργασίας γραφικών. Πρόσφατα, έχουν γίνει προσπάθειες δημιουργίας προτύπων.

Ο προγραμματισμός των GPUs αναπτύχθηκε όταν το CUDA και το Stream κατέφθασαν στο τέλος του 2006. Αυτές οι διεπαφές και οι γλώσσες, σχεδιάστηκαν απο τις εταιρίες κατασκευής των GPUs σε πολύ κοντινή σχέση με το υλικό, το οποίο αποτέλεσε μεγάλο βήμα προς ένα πιο εύχρηστο, ταιριαστό και μελλοντικά-ασφαλές προγραμματιστικό μοντέλο. Η ανοιχτή γλώσσα προγραμματισμού (OpenCL) δημιουργήθηκε για να παρέχει ένα γενικό API ετερογενή υπολογισμού σε διάφορες μορφές παράλληλων συσκευών, συμπεριλαμβανομένου μονάδων επεξεργασίας γραφικών, πολυπύρηνων κεντρικών μονάδων επεξεργασίας, κ.α

\section{CUDA}
\subsection{Εισαγωγή}
To CUDA είναι μια πλατφόρμα παράλληλου υπολογισμού, που δημιουργήθηκε από την NVIDIA και υλοποιήθηκε στις κάρτες γραφικών τις οποίες παράγει η ίδια. Το CUDA δίνει στους προγραμματιστές άμεση πρόσβαση στο σετ εικονικών εντολών και την μνήμη των στοιχείων του παράλληλου υπολογισμού σε κάρτες γραφικών NVIDIA. \\
\\
Αξιοποιώντας το CUDA, οι κάρτες γραφικών(GPU) μπορούν να χρησιμοποιηθούν για υπολογισμό γενικής χρήσης (δηλαδή όχι αποκλειστικά για γραφικά).Οι GPU έχουν μια αρχιτεκτονική παράλληλης εξόδου η οποία δίνει έμφαση στην εκτέλεση πολλών threads με μικρή ταχύτητα, σε αντίθεση με τις CPU όπου εκτελείται ένα thread με μεγάλη ταχύτητα. \\
\\
Η πλατφόρμα CUDA είναι προσβάσιμη στους προγραμματιστές μέσω βιβλιοθηκών, εντολών μεταγλώττισης, και προεκτάσεων σε γλώσσες προγραμματισμού βιομηχανικής κλίμακας, όπως η C, C++ και Fortran. Οι προγραμματιστές της C/C++, χρησιμοποιούν το CUDA C/C++, μεταγλωττισμένο με το nvcc, έναν LLVM βασισμένο μεταγλωττιστή, και οι προγραμματιστές της Fortran χρησιμοποιούν το CUDA Fortran, μεταγλωττισμένο με τον μεταγλωττιστή PGI CUDA Fortran απο το The Portland Group. Εκτώς απο τα παραπάνω, η πλατφόρμα CUDA υποστηρίζει και άλλες διεπαφές υπολογισμού, όπως το OpenCL του Khronos Group, το DirectCompute της Microsoft, και το C++ AMP.\\
Στην βιομηχανία των υπολογιστών, οι GPUs δεν χρησιμοποιούνται μόνο για τα γραφικά αλλά και στους υπολογισμούς φυσικής παιχνιδιών (π.χ καπνός, φωτιά, ροή υγρών). Γνωστά παραδείγματα αποτελούν οι μηχανές PhysX και η Bullet. Το CUDA επίσης χρησιμοποιείται για να επιταχύνει μη-γραφικές εφαρμογές στην βιοπληροφορική, στην κρυπτογραφία, και σε πολλά άλλα πεδία.\\
\subsection{Πλεονεκτήματα}
Το CUDA έχει τα εξής πλεονεκτήματα σε σχέση με τους παραδοσιακούς τρόπους υπολογισμού γενικής χρήσης που εκτελούνται μέσω προγραμματιστικών διεπαφών γραφικών:
 Διασκορπισμένες προσπελάσεις - ο κώδικας μπορεί να διαβαστεί από αυθαίρετες διευθύνσεις στην μνήμη.
 Ενοποιημένη εικονική μνήμη (CUDA 6)
 Κοινόχρηστη μνήμη - το CUDA εκθέτει μια γρήγορη περιοχή κοινόχρηστης μνήμης (μέχρι 48KB για κάθε επεξεργαστή) η οποία μπορεί να μοιραστεί ανάμεσα στα threads. Αυτή μπορεί να χρησιμοποιηθεί σαν κρυφή μνήμη διαχειρίσιμη απο τον χρήστη, επιτρέποντας μεγαλύτερο εύρος δεδομένων απο ότι είναι δυνατό με τις προσπελάσεις υφών.
 Πιο γρήγορες μεταφορτώσεις και προσπελάσεις από και προς την GPU
 Πλήρης υποστήριξη για ακέραιες και bitwise λειτουργίες, για παράδειγμα τις προσπελάσεις υφών.
\subsection{Περιορισμοί}
 Το CUDA δεν υποστηρίζει ολόκληρο το πρότυπο της γλώσσας C, καθώς τρέχει μέσω ενός μεταγλωττιστή C++, ο οποίος εμποδίζει συγκεκριμένα μέρη της γλώσσας C να μεταγλωττιστούν.
 
\begin{table}
\begin{tabular}{ | c | c |}
Feature support (unlisted features are\\ supported for all compute capabilities) & Compute capability (version)\\ \hline
\end{tabular}
\end{table}