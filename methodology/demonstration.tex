\subsection{Παρουσίαση}
\subsubsection{Blender και Bullet Physics}
Το Blender είναι ένα επαγγελματικό πρόγραμμα τρισδιάστατων γραφικών το οποίο είναι ελεύθερο και ανοιχτού κώδικα. Χρησιμοποιείται για την δημιουργία ταινιών, οπτικών εφέ, τέχνης, τρισδιάστατων μοντέλων εκτύπωσης, διαδραστικές τρισδιάστατες εφαρμογές, και παιχνίδια υπολογιστών. Τα χαρακτηριστικά του Blender περιέχουν τρισδιάστατη μοντελοποίηση, δημιουργία υφών, εξομοίωση ρευστών και καπνού, εξομοίωση μαλακών σωμάτων, ανταπόδοση σκηνών. Περιέχει επίσης και μηχανή κατασκευής παιχνιδιών.
\begin{figure}[h]
\centering
\includegraphics[scale=1]{blender}
\caption{Λογότυπο Blender 3D}
\end{figure}

To Blender επιλέχτηκε για την παρουσίαση της πτυχιακής εργασίας, με εκπαιδευτικό παράδειγμα για το πως μπορεί να επιταχυνθεί η ανταπόδοση γραφικών και εξομοίωση φυσικής σε ένα παράδειγμα γραφικών.

H μηχανή εξομοίωσης φυσικής επιλέχτηκε για την παρουσίαση της πτυχιακής εργασίας. H Bullet είναι μια μηχανή φυσικής που εξομοιώνει ανίχνευση συγκρούσεων, δυναμική άκαμπτων και μαλακών σωμάτων. Χρησιμοποιείται σε παιχνίδια υπολογιστών αλλά και για οπτικά εφέ σε ταινίες. Η βιβλιοθήκη της bullet physics είναι ελεύθερη και ανοιχτού κώδικα κάτω από την άδεια zlib. Η επιλογή της μηχανής φυσικής Bullet έγινε λόγω της μεγάλης δημοτικότητας που έχει και της καταξιωμένης εφαρμογής της σε γνωστές ταινίες και παιχνίδια, συμπεριλαμβανομένου των: Toy Story 3, Grand Theft Auto IV,V, 2012(ταινία), Hancock, Megamind, και λόγω της ενσωμάτωση της στο πρόγραμμα γραφικών Blender.

\begin{figure}[h]
\centering
\includegraphics[width=0.9\linewidth]{bullet}
\caption{Λογότυπο Blender 3D}
\end{figure}

\subsubsection{Netbeans και JOCL}
Το Netbeans είναι ένα περιβάλλον ανάπτυξης λογισμικού που χρησιμοποιείται κυρίως σε συνδυασμό με την γλώσσα προγραμματισμού Java, αλλά και με άλλες γλώσσες, όπως PHP,C/C++, και HTML5. Αποτελεί επίσης μια πλατφόρμα ανάπτυξης για επιτραπέζιες εφαρμογές Java, κ.α. Το Netbeans είναι ανοιχτού κώδικα και υποστηρίζει έλεγχο εκδόσεων.

Ο λόγος της επιλογής του Netbeans είναι ότι υπάρχει υποστήριξη για Java Bindings για προγραμματισμό OpenCL μέσω της βιβλιοθήκης JOCL. To JOCL υποστηρίζει όλες τις υλοποιήσεις OpenCL που υπάρχουν αυτήν την στιγμή.
\begin{itemize}
\item AMD APP SDK με υποστήριξη για OpenCL 1.2
\item NVIDIA οδηγοί με υποστήριξη για OpenCL 1.1
\item OpenCL σε OSX
\item Intel OpenCL SDK
\end{itemize}

Το JOCL χρησιμοποιήθηκε για προγραμματισμό εκπαιδευτικού παραδείγματος με την χρήση OpenCL και μελετήθηκε η απόδοση διαφόρων παραδειγμάτων σε CPU και GPU αντίστοιχα. Καθώς η Java είναι μια γλώσσα που έχουμε διδαχτεί ενδελεχώς κατά την διάρκεια των σπουδών μας, θεωρώ ότι είναι χρήσιμη και ίσως η πιο εύκολη μετάβαση από τον παραδοσιακό προγραμματισμό σε προγραμματισμό GPU.
Αντίστοιχα bindings παρέχονται και για την τεχνολογία CUDA, μέσω της βιβλιοθήκης Jcuda.

\subsubsection{Sfera}
Το Sfera είναι ένα εκπαιδευτικό παιχνίδι με μεγάλη επιστημονική σημασία, ανεπτυγμένο από τον προγραμματιστή David Dade Bucciarelli. Το Sfera χρησιμοποιεί κυρίως OpenCL και OpenGL, και εκτελεί ανταπόδοση γραφικών με ιχνογράφηση ακτίνας πραγματικού χρόνου. Χρησιμοποιεί την μηχανή φυσικής Bullet. Το sfera βασίζεται στο πρόγραμμα SmallLiuxGPU v2.0 Μερικά από τα χαρακτηριστικά του είναι τα εξής:
\begin{itemize}
\item Πολλαπλές μηχανές ανταπόδοσης: μονο-νηματική, πολυ-νηματική, μία ή περισσότερες GPUs
\item Πολλαπλά υλικά που εκπέμπουν φως: καθρεύτες, γυαλί, μέταλλο, κράμα,ματ
\item Σχεδιασμός Υφών
\item Βάθος πεδίου
\item Φορητό: υλοποιήσεις σε Windows, Linux, MacOS, κ.α
\end{itemize}

\subsubsection{Compubench}
Το Compubench είναι ένα πρόγραμμα μετρήσεων απόδοσης OpenCL για τον έλεγχο και την σύγκριση της παράλληλης υπολογιστικής απόδοσης των CPUs,GPUs και επιταχυντές επιτραπέζιων και φορητών συσκευών. Περιλαμβάνει πολλές δοκιμές απόδοσης όπως αναγνώριση προσώπων, εξομοίωση επιφάνειας ωκεανών, εξομοίωση σωματιδίων, σύνθεση και μετατροπή βίντεο, εξόρυξη κρυπτο-νομισμάτων, κ.α.
\subsubsection{GPU Caps Viewer}
Το GPU Caps Viewer είναι ένα ελεύθερο και πλούσιο σε χαρακτηριστικά πρόγραμμα το οποίο περιγράφει τις δυνατότητες της GPU, συμπεριλαμβανομένου του τύπου της GPU, του μεγέθους της μνήμης, την έκδοση της υποστηριζόμενης έκδοσης OpenGL, ενώ περιέχει και διαδικασίες ελέγχου σταθερότητας και έντασης. Το GPU Caps Viewer περιέχει αρκετά παραδείγματα της χρήσης της τεχνολογίας OpenCL με χρήση γραφικών, ενώ βρίσκεται ανάμεσα στα πρώτα προγράμματα που υποστήριξαν την τεχνολογία OpenCL.
\subsubsection{Artifacts και demoscene}
Το παράδειγμα παρουσίασης Artifacts είναι μια παρουσίαση οπτικών εφέ και μουσικής, που κέρδισε την πρώτη θέση στον αντίστοιχο διαγωνισμό της demoscene στο TokyoDemoFest 2013. Είναι ένα καλό παράδειγμα των δυνατοτήτων της GPU τόσο από την οπτική γωνία των γραφικών όσο και από την οπτική γωνία της παρουσίασης των δυνατοτήτων της υπολογιστικής δύναμης των GPU.
