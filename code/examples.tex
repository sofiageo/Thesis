\section{Ανάλυση διεπαφής προγραμματισμού εφαρμογών}
\subsection{Υλοποίηση OpenGL}
\subsubsection{Αποστολή}
Ένα αντικείμενο προγράμματος μπορεί να έχει shader υπολογισμού μέσα του. Ο shader υπολογισμού συνδέεται με καταστάσεις shader μέσω κάποιων λειτουργιών απόδοσης. Υπάρχουν δύο λειτουργίες για να ξεκινήσουν οι διαδικασίες υπολογισμού. Χρησιμοποιούν οποιονδήποτε shader υπολογισμού είναι ενεργός. Οι λειτουργίες είναι οι εξής:
\begin{itemize}
\item void glDispatchCompute(GLuint num\_groups\_x, GLuint num\_groups\_u, GLuint num\_groups\_z); - Οι παράμετροι num\_groups\_* ορίζουν τον αριθμό των ομάδων εργασίας, σε τρεις διαστάσεις. Αυτοί οι αριθμοί δεν μπορούν να είναι μηδέν. Υπάρχουν όρια στον αριθμό των ομάδων εργασίας που μπορούν να αποσταλούν.
\item void glDispatchComputeIndirect(GLintptr indirect); - H παράμετρος indirect είναι το αντιστάθμισμα του buffer GL\_DISPATCH\_INDIRECT\_BUFFER. Ισχύουν τα ίδια όρια του αριθμού ομάδων εργασίας, όμως η αποστολή indirect παρακάμπτει τον έλεγχο λαθών του OpenGL. Έτσι, η αποστολή με εκτός ορίων μεγέθους ομάδας εργασίας, μπορεί να προκαλέσει προβλήματα ακόμα και πάγωμα του συστήματος.
\end{itemize}
\subsubsection{Είσοδοι}
Τα shader υπολογισμού  δεν μπορούν να έχουν μεταβλητές καθορισμένες απο τον χρήστη. Τα shader υπολογισμού έχουν τις παρακάτω ενσωματωμένες μεταβλητές εξόδου:
\begin{itemize}
\item in uvec3 gl\_NumWorkGroups; - Αυτή η μεταβλητή περιέχει τον αριθμό των ομάδων εργασίας για την λειτουργία αποστολής
\item in uvec3 gl\_WorkGroupID; - Αυτή η μεταβλητή περιέχει την ισχύουσα ομάδα εργασίας για την επίκληση του shader.
\item in uvec3 gl\_LocalInvocationID; - Αυτή η μεταβλητή περιέχει την ισχύουσα επίκληση του shader μέσα στην ομάδα εργασίας.
\item in uvec3 gl\_GlobalInvocationID; - Αυτή η μεταβλητή αναγνωρίζει μοναδικά την συγκεκριμένη επίκληση του shader υπολογισμού  ανάμεσα σε όλες τις επικλήσεις της κλήσης αποστολής υπολογισμού. Είναι μια συντόμευση για τον μαθηματικό υπολογισμό gl\_WorkGroupID * gl\_WorkGroupSize + gl\_LocalInvocationID;
\item in uint  gl\_LocalInvocationIndex;
\end{itemize}
\subsubsection{Τοπικό μέγεθος}
Το τοπικό μέγεθος ενός shader υπολογισμού ορίζεται απο τον shader, χρησιμοποιώντας μια ειδική δήλωση εισόδου: 
layout(local\_size\_x = X, local\_size\_y = Y, local\_size\_z = Z) in;
Αρχικά, τα τοπικά μεγέθη είναι 1, οπότε αν θέλουμε μονοδιάστατο ή δισδιάστατο χώρο ομάδων εργασίας, μπορούμε να ορίσουμε μόνο το Χ ή το Χ και το Υ. Πρέπει να είναι σταθερές εκφράσεις τιμής μεγαλύτερης του 0. Οι τιμές πρέπει να ορίζονται σε σχέση με τους περιορισμούς που υπάρχουν παρακάτω. Σε αντίθετη περίπτωση προκύπτουν λάθη. Το τοπικό μέγεθος είναι διαθέσιμο στον shader σαν σταθερά, οπότε δεν χρειάζεται να την ορίζουμε εμείς.
\begin{itemize}
\item const uvec3 gl\_WorkGroupSize;
\end{itemize}
\subsubsection{Περιορισμοί}
Ο αριθμός των ομάδων εργασίας που μπορούν να αποσταλούν, ορίζεται από μια ειδική σταθερά. Αυτή η σταθερά πρέπει να διαβαστεί απο την glGetIntegeri\_v, με τιμές ανάμεσα στο κλειστό όριο [0,2]. 

Προσπάθεια να καλέσουμε την glDispatchCompute με τιμές που ξεπερνούν το όριο είναι λάθος. Προσπάθεια κλήσης της glDispatchComputeIndirect είναι χειρότερα, μπορεί να διακόψει την λειτουργία του προγράμματος ακόμα και να παγώσει το σύστημα. Σημείωση: ο μικρότερος αριθμός αυτών των τιμών πρέπει να είναι 65535 σε όλους τους άξονες. Αυτό δίνει αρκετό χώρο για εργασία. 

Υπάρχουν όρια στο τοπικό μέγεθος επίσης. Συγκεκριμένα, υπάρχουν δύο τύποι περιορισμών. 
\begin{itemize}
\item Ο γενικός περιορισμός των διαστάσεων τοπικού μεγέθους, σε συνδυασμό με την GL\_MAX\_COMPUTE\_WORK\_GROUP\_SIZE, όπως και παραπάνω. Η διαφορά είναι οτι ο μικρότερος αριθμός των τιμών είναι πολύ μικρότερος. 1024 για τον Χ και τον Υ, και μόνο 64 για τον Ζ.
\item Ο αριθμός των επικλήσεων μέσα σε μια ομάδα εργασίας. Δηλαδή, το προϊόν των στοιχείων Χ,Υ,Ζ του τοπικού μεγέθους πρέπει να είναι μικρότερο απο GL\_MAX\_WORK \_GROUP\_INVOCATIONS. Η μικρότερη τιμή είναι 1024.
\end{itemize}
Υπάρχει ακόμα ο περιορισμός του ολικού μεγέθους αποθήκευσης για όλες τις κοινές μεταβλητές ενός shader υπολογισμού. Ορίζεται απο την GL\_MAX\_COMPUTE \_SHARED\_MEMORY\_SIZE, που αναφέρεται σε bytes. Η μικρότερη τιμή για το OpenGL είναι 32KB.
\subsection{Υλοποίηση DirectX}
Ένας shader υπολογισμού είναι μια κατάσταση shader υπολογισμού που επεκτείνει την χρήση του το Microsoft Direct3D 11 πέρα απο τον προγραμματισμό γραφικών. Η τεχνολογία αυτή είναι γνωστή και ως τεχνολογία DirectCompute\cite{computeshaders-2}

Όπως όλα τα προγραμματιστικά shader (για παράδειγμα shader γεωμετρίας και κορυφών), ένα shader υπολογισμού είναι σχεδιασμένο να χρησιμοποιεί μια Γλώσσα Υψηλού Προγραμματισμού Shader(HLSL) για το DirectX. Η HLSL, χρησιμοποιείται για το DirectX και μας δίνει την δυνατότητα να δημιουργήσουμε C like shaders για την γραμμή σωλήνων Direct3D. 

Η HLSL δημιουργήθηκε ξεκινώντας απο το DirectX 9 για την κατασκευή προγραμματιζόμενων τρισδιάστατων γραμμής σωλήνα. Μας δίνει την δυνατότητα να προγραμματίσουμε την γραμμή σωλήνα με τον συνδυασμό οδηγιών assembly, οδηγιών HLSL, και δηλώσεις καθορισμένων λειτουργιών.

Ένα shader υπολογισμού προμηθεύει υψηλής ταχύτητας υπολογισμούς γενικού προγραμματισμού, και εκμεταλλεύεται τον μεγάλο αριθμό παράλληλων επεξεργαστών που βρίσκονται στην μονάδα επεξεργασίας γραφικών (GPU). Τα shader υπολογισμού προμηθεύουν διαμοιρασμό μνήμης και συγχρονισμό νημάτων, για να επιτρέψει καλύτερες μεθόδους παράλληλου προγραμματισμού. 

Με την κλήση των μεθόδων ID3D11DeviceContext ::Dispatch ή ID3D11DeviceContext ::DispatchIndirect γίνεται η εκτέλεση εντολών σε ένα shader υπολογισμού, οι οποίες μπορούν να εκτελεστούν παράλληλα σε πολλά νήματα.