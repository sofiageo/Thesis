\chapter{Υλοποιήσεις}
\section{Ιστορία}
Παραδοσιακά, η σειρά αγωγών των γραφικών, αποτελείται από τις καταστάσεις μετατροπή και φωτισμό, συναρμολόγηση αρχέγονων, μετατροπή σε pixels, και σκίαση. Οι πρώτες GPU είχαν όλες τις λειτουργίες που χρειάζονται για να εκτελεστεί η σειρά αγωγών, αλλά με τον καιρό όλο και περισσότερες καταστάσεις έγιναν δυνατό να προγραμματιστούν με την έλευση ειδικών επεξεργαστών, όπως επεξεργαστές κορυφών και τεμάχια επεξεργαστών, που κατέστησαν κάποιες λειτουργίες πιο ευέλικτες.

Όταν οι τιμές συνέχισαν να πέφτουν ενώ η υπολογιστική δύναμη αυξανόταν, η ερευνητική κοινότητα σκέφτηκε τρόπους να αξιοποιηθεί αυτή η δύναμη για τον υπολογισμό δύσκολων λειτουργιών. Όμως, καθώς η δυνατότητα των επεξεργαστών ήταν περιορισμένη και η διεπαφή προγραμματιστικής διεπαφής (API) των οδηγών γραφικών ήταν σχεδιασμένη για να υλοποιεί συγκεκριμένα την σειρά αγωγών, έπρεπε να ληφθούν υπόψιν πολλές παράμετροι.

Για παράδειγμα, όλα τα δεδομένα έπρεπε να κωδικοποιηθούν σε υφές ως πίνακες δυο διαστάσεων που αναπαριστούν pixel, με περιεχόμενο τιμές χρωμάτων και κάποιο κανάλι alpha για την διαφάνεια. Επιπλέον, οι υφές είναι αντικείμενα μόνο προσπελάσιμα, και δεν επαναγράφονταν, κάτι που ανάγκαζε τους προγραμματιστές να αποθηκεύουν κάθε φορά καινούρια υφή με τις αλλαγές. Τέλος, οι περισσότερες GPU υποστήριζαν μόνο λειτουργίες μονής κινητής υποδιαστολής, αναγκάζοντας τους προγραμματιστές να προσομοιώνουν λογικές λειτουργίες.

Αυτοί οι περιορισμοί, ήταν ο μεγαλύτερος λόγος που ώθησε τους κατασκευαστές GPU (AMD,NVIDIA,INTEL), να δημιουργήσουν προγραμματιστικές διεπαφές ειδικές για την κοινότητα του GPGPU και να εξελίξουν τις συσκευές τους για καλύτερη υποστήριξη.

Το πεδίο του προγραμματισμού γενικής χρήσης έχει αναπτυχθεί με ταχύτατους ρυθμούς τα τελευταία χρόνια, έτσι ώστε τώρα υπάρχουν αρκετές υλοποιήσεις για τον προγραμματισμό των μονάδων επεξεργασίας γραφικών. Πρόσφατα, έχουν γίνει προσπάθειες δημιουργίας προτύπων.
Ο προγραμματισμός των GPUs αναπτύχθηκε όταν το CUDA και το Stream κατέφθασαν στο τέλος του 2006. Αυτές οι διεπαφές και οι γλώσσες, σχεδιάστηκαν από τις εταιρίες κατασκευής των GPUs σε πολύ κοντινή σχέση με το υλικό, το οποίο αποτέλεσε μεγάλο βήμα προς ένα πιο εύχρηστο, ταιριαστό και μελλοντικά-ασφαλές προγραμματιστικό μοντέλο.

Η ανοιχτή γλώσσα προγραμματισμού (OpenCL) δημιουργήθηκε για να παρέχει ένα γενικό API ετερογενή υπολογισμού σε διάφορες μορφές παράλληλων συσκευών, συμπεριλαμβανομένου μονάδων επεξεργασίας γραφικών, πολυπύρηνων κεντρικών μονάδων επεξεργασίας, κ.α
\subsection{Μέλλον}
Οι πρόσφατες δραστηριότητες των μεγάλων κατασκευαστών μας δείχνουν ότι τα μελλοντικά σχέδια των μικροεπεξεργαστών και μεγάλων HPC συστημάτων θα είναι υβριδικά/ετερογενούς φύσης. Αυτά τα συστήματα θα βασίζονται στην ενσωμάτωση δύο τύπων εξαρτημάτων:
\begin{itemize}
\item Τεχνολογία πολυπύρηνων CPU: ο αριθμός των πυρήνων θα συνεχίσει να αυξάνεται λόγω της επιθυμίας να ενσωματώσουμε περισσότερα εξαρτήματα σε ένα τσιπ.
\item Ειδικού τύπου υλικό και μαζικά παράλληλους επιταχυντές: Για παράδειγμα, οι GPUs υπερτερούν των CPUs σε απόδοση κινητής υποδιαστολής, τα τελευταία χρόνια. Επίσης ο προγραμματισμός σε αυτές έχει γίνει εύκολος, αν όχι ευκολότερος, από ότι στις CPUs
\end{itemize}
Η σχετική ισορροπία στα μελλοντικά σχέδια δεν είναι ξεκάθαρη και μπορεί να αλλάξει με την πάροδο του χρόνου. Δεν υπάρχει καμία αμφιβολία ότι οι μελλοντικές γενιές των υπολογιστικών συστημάτων, από τους φορητούς υπολογιστές μέχρι και τους υπερ-υπολογιστές θα αποτελείται από μια σύσταση ετερογενών συστημάτων. 
\subsection{Προβλήματα}
Τα προβλήματα και οι προκλήσεις για τους προγραμματιστές στο καινούριο περιβάλλον των υβριδικών συστημάτων, είναι υπαρκτά. Κρίσιμα τμήματα του λογισμικού ήδη δυσκολεύονται να προλάβουν τον ρυθμό των αλλαγών. Σε μερικές περιπτώσεις, η απόδοση δεν είναι ανάλογη του αριθμού των πυρήνων, γιατί ένα μεγάλο μέρος του χρόνου ξοδεύεται στην μετακίνηση των δεδομένων παρά στους υπολογισμούς. Σε άλλες περιπτώσεις, το βελτιστοποιημένο λογισμικό για το συγκεκριμένο υλικό, παραδίδεται χρόνια μετά από την παράδοση του υλικού, και έτσι είναι απαρχαιωμένο όταν παραδοθεί. Και σε άλλες περιπτώσεις, όπως σε μερικές πρόσφατες υλοποιήσεις GPU, το λογισμικό δεν εκτελείται καθόλου γιατί το προγραμματιστικό περιβάλλον έχει αλλάξει υπερβολικά.
\section{CUDA}
\subsection{Εισαγωγή}
To CUDA είναι μια πλατφόρμα παράλληλου υπολογισμού, που δημιουργήθηκε από την NVIDIA και υλοποιήθηκε στις κάρτες γραφικών τις οποίες παράγει η ίδια. Το CUDA δίνει στους προγραμματιστές άμεση πρόσβαση στο σετ εικονικών εντολών και την μνήμη των στοιχείων του παράλληλου υπολογισμού σε κάρτες γραφικών NVIDIA. 

Αξιοποιώντας το CUDA, οι κάρτες γραφικών(GPU) μπορούν να χρησιμοποιηθούν για υπολογισμό γενικής χρήσης (δηλαδή όχι αποκλειστικά για γραφικά).Οι GPU έχουν μια αρχιτεκτονική παράλληλης εξόδου η οποία δίνει έμφαση στην εκτέλεση πολλών threads με μικρή ταχύτητα, σε αντίθεση με τις CPU όπου εκτελείται ένα thread με μεγάλη ταχύτητα. 

\begin{figure}[h]
\centering
\includegraphics[width=\linewidth]{nvidia-cuda}
\caption{Λογότυπο NVIDIA CUDA\cite{figure-3}}
\end{figure}

Η πλατφόρμα CUDA είναι προσβάσιμη στους προγραμματιστές μέσω βιβλιοθηκών, εντολών μεταγλώττισης, και προεκτάσεων σε γλώσσες προγραμματισμού βιομηχανικής κλίμακας, όπως η C, C++ και Fortran.

Οι προγραμματιστές της C/C++, χρησιμοποιούν το CUDA C/C++, μεταγλωττισμένο με το nvcc, έναν LLVM βασισμένο μεταγλωττιστή, και οι προγραμματιστές της Fortran χρησιμοποιούν το CUDA Fortran, μεταγλωττισμένο με τον μεταγλωττιστή PGI CUDA Fortran απο το The Portland Group. Εκτός από τα παραπάνω, η πλατφόρμα CUDA υποστηρίζει και άλλες διεπαφές υπολογισμού, όπως το OpenCL του Khronos Group, το DirectCompute της Microsoft, και το C++ AMP.

Στην βιομηχανία των υπολογιστών, οι GPUs δεν χρησιμοποιούνται μόνο για τα γραφικά αλλά και στους υπολογισμούς φυσικής παιχνιδιών (π.χ καπνός, φωτιά, ροή υγρών). Γνωστά παραδείγματα αποτελούν οι μηχανές PhysX και η Bullet. Το CUDA επίσης χρησιμοποιείται για να επιταχύνει μη-γραφικές εφαρμογές στην βιοπληροφορική, στην κρυπτογραφία, και σε πολλά άλλα πεδία.\cite{cuda-1}

Γενικότερα, η υπολογιστική δύναμη της GPU, βασίζεται στην παράλληλη αρχιτεκτονική της. Για αυτό, η πλατφόρμα του CUDA παρουσιάζει το νήμα(thread) ως το μικρότερο στοιχείο παραλληλισμού. Όμως, σε σύγκριση με την κεντρική μονάδα επεξεργασίας, τα νήματα της GPU έχουν μικρότερο κόστος χρήσης πόρων και μικρότερο κόστος δημιουργίας και αντικατάστασης.

Σημειώνεται ότι οι GPU είναι αποτελεσματικές, μόνο όταν τρέχει μεγάλος αριθμός απο τέτοια νήματα. Μια ομάδα από νήματα, που εκτελούνται παράλληλα, επικοινωνούν και συγχρονίζονται μεταξύ τους ονομάζεται block. Ο μέγιστος αριθμός των νημάτων σε ενα block είναι ένας περιορισμός που υπάρχει στην κάθε μονάδα γραφικής επεξεργασίας. Τέλος, μια ομάδα από blocks τα οποία έχουν την ίδια διάσταση και εκτελούνται απο το ίδιο πρόγραμμα CUDA παράλληλα, ονομάζεται πλέγμα.

Για να επιτρέψει βέλτιστη επίδοση για διαφορετικά πρότυπα, το CUDA εκτελεί ένα ιεραρχικό μοντέλο μνήμης, αντίθετα με τα παραδοσιακά μοντέλα που συναντάμε συνήθως στους υπολογιστές. Ο υπολογιστής και η συσκευή, έχουν τις δικές τους περιοχές μνήμης, τις οποίες ονομάζουν host memory και device memory, αντίστοιχα. Το CUDA παρέχει βελτιστοποιημένες λειτουργίες για να μεταφέρει δεδομένα από και προς αυτούς τους ξεχωριστούς χώρους.
Κάθε νήμα κατέχει το δικό του αρχείο καταχώρησης, το οποίο μπορεί να προσπελαστεί και να εγγραφεί.

Επιπλέον, μπορεί να προσπελάσει το δικό του αντίγραφο της τοπικής μνήμης. Όλα τα νήματα στο ίδιο πλέγμα μπορούν να προσπελάσουν και να γράψουν στην περιοχή της κοινόχρηστης μνήμης (shared memory). Για να αποφευχθούν κίνδυνοι από ταυτόχρονη προσπέλαση, μηχανισμοί συγχρονισμού νημάτων πρέπει να χρησιμοποιηθούν. Η κοινόχρηστη μνήμη, είναι οργανωμένη σε ομάδες που ονομάζονται τράπεζες, οι οποίες μπορούν να προσπελαστούν παράλληλα. Όλα τα νήματα έχουν επίσης πρόσβαση στον χώρο μνήμης που ονομάζεται καθολική μνήμη (global memory) και στις περιοχές που ονομάζονται μνήμη σταθερών (constant memory) και μνήμη υφής (texture memory).\cite{cuda-2}

\subsection{Πλεονεκτήματα}
Το CUDA έχει τα εξής πλεονεκτήματα σε σχέση με τους παραδοσιακούς τρόπους υπολογισμού γενικής χρήσης που εκτελούνται μέσω προγραμματιστικών διεπαφών γραφικών:
\begin{itemize}
\item Διασκορπισμένες προσπελάσεις - ο κώδικας μπορεί να διαβαστεί από αυθαίρετες διευθύνσεις στην μνήμη.
\item Ενοποιημένη εικονική μνήμη (CUDA 6)
\item Κοινόχρηστη μνήμη - το CUDA εκθέτει μια γρήγορη περιοχή κοινόχρηστης μνήμης (μέχρι 48KB για κάθε επεξεργαστή) η οποία μπορεί να μοιραστεί ανάμεσα στα threads. Αυτή μπορεί να χρησιμοποιηθεί σαν κρυφή μνήμη διαχειρίσιμη απο τον χρήστη, επιτρέποντας μεγαλύτερο εύρος δεδομένων απο ότι είναι δυνατό με τις προσπελάσεις υφών.
\item Πιο γρήγορες μεταφορτώσεις και προσπελάσεις από και προς την GPU
\item Πλήρης υποστήριξη για ακέραιες και bitwise λειτουργίες, για παράδειγμα τις προσπελάσεις υφών.
\end{itemize}
\subsection{Περιορισμοί}
 Το CUDA δεν υποστηρίζει ολόκληρο το πρότυπο της γλώσσας C, καθώς τρέχει μέσω ενός μεταγλωττιστή C++, ο οποίος εμποδίζει συγκεκριμένα μέρη της γλώσσας C να μεταγλωττιστούν.

\subsection{Μνήμη}
\subsubsection{Κοινή μνήμη και συγχρονισμός}
Ο μεταγλωττιστής CUDA C μεταχειρίζεται τις μεταβλητές στην κοινή μνήμη με διαφορετικό τρόπο απο ότι τις τυπικές μεταβλητές. Δημιουργεί ένα αντίγραφο για κάθε block που εκτελείται στην CPU. Κάθε νήμα σε αυτό το block μοιράζεται την μνήμη, αλλά τα νήματα δεν μπορούν να προσπελάσουν και να επεξεργαστούν το αντίγραφο της μεταβλητής που φαίνεται στα άλλα blocks.

Αυτό παρέχει ένα καλό τρόπο με τον οποίο τα νήματα μέσα σε ένα block μπορούν να επικοινωνούν και να συνεργάζονται στους υπολογισμούς. Επιπλέον, τα buffers κοινής μνήμης βρίσκονται πάνω στην GPU, με αυτον τον τρόπο, η καθυστέρηση στην προσπέλαση της κοινής μνήμης είναι πολύ μικρότερη από ότι στα τυπικά buffers, καθιστώντας την κοινή μνήμη πολύ αποδοτική.\cite{cuda-3}

Η επικοινωνία μεταξύ των νημάτων, αν και πολύ ενδιαφέρουσα, χρειάζεται έναν μηχανισμό για τον συγχρονισμό της. Για παράδειγμα, αν το νήμα Α γράψει μια τιμή στην κοινή μνήμη και θέλουμε το νήμα Β να κάνει κάτι με αυτήν την τιμή, δεν μπορούμε να ξεκινήσουμε το νήμα Β έως ότου γνωρίζουμε ότι η εγγραφή από το νήμα Α ολοκληρώθηκε. Χωρίς τον συγχρονισμό, θα είχαμε έναν αγώνα δρόμου όπου το σωστό αποτέλεσμα της εκτέλεσης θα εξαρτάται από μη ντετερμινιστικά χαρακτηριστικά του υλικού.
\subsubsection{Σταθερή μνήμη}
Έχουμε αναλύσει το πως οι μοντέρνες GPUs είναι εφοδιασμένες με τεράστιες δυνατότητες υπολογιστικής δύναμης. Το υπολογιστικό πλεονέκτημα που έχουν οι μονάδες επεξεργασίας γραφικών βοήθησε στην ανάπτυξη του προγραμματισμού γενικού σκοπού. Με εκατοντάδες αριθμητικές μονάδες στην GPU, συνήθως ο περιορισμός δεν είναι η αριθμητική απόδοση, αλλά το εύρος ζώνης της μνήμης. Υπάρχουν τόσες πολλές ALUs στους επεξεργαστές γραφικών, όπου πολλές φορές δεν προλαβαίνουμε να μεταφέρουμε τα δεδομένα αρκετά γρήγορα ώστε να διατηρήσουμε έναν υψηλό ρυθμό υπολογισμού. 

Η γλώσσα CUDA παρέχει άλλον έναν τύπο μνήμης γνωστή ως σταθερή μνήμη. Όπως φαίνεται από το όνομα, χρησιμοποιούμε την σταθερή μνήμη για δεδομένα που δεν αλλάζουν κατά την διάρκεια μιας εκτέλεσης πυρήνα. Το υλικό NVIDIA παρέχει 64KB σταθερής μνήμης που χρησιμοποιεί με διαφορετικό τρόπο από ότι την γενική μνήμη. Σε μερικές περιπτώσεις, η χρήση της σταθερής μνήμης αντί της γενικής μνήμης μειώνει το εύρος ζώνης της μνήμης. 

Δηλώνοντας την μνήμη σαν σταθερή, περιορίζουμε την χρήση της σε ανάγνωσης μόνο. Λόγω αυτού του περιορισμού, περιμένουμε να κερδίσουμε κάτι από αυτήν την διαδικασία. Η χρήση σταθερής μνήμης μπορεί να μας διαφυλάξει εύρος μνήμης σε σχέση με την ανάγνωση των δεδομένων από την γενική μνήμη. Υπάρχουν δύο λόγοι γιατί η ανάγνωση από την σταθερή μνήμη των 64KB μπορεί να διαφυλάξει εύρος μνήμης:
\begin{itemize}
\item Μια ανάγνωση από την σταθερή μνήμη μπορεί να αναμεταδοθεί σε κοντινά νήματα, σώζοντας μας έως και 15 αναγνώσεις.
\item Η σταθερή μνήμη είναι cached, έτσι επόμενες αναγνώσεις της ίδιας διεύθυνσης δεν θα δημιουργήσουν επιπλέον κίνηση στην μνήμη.
\end{itemize}
Για να εξηγήσουμε αυτήν την δήλωση και το τι είναι τα κοντινά νήματα, σκεφτόμαστε την περίπτωση της ύφανσης. Στην ύφανση, το στημόνι (warp) αναφέρεται σε ένα σύνολο από νήματα, που υφαίνονται μαζί σε ένα ύφασμα. Στην αρχιτεκτονική CUDA, το στημόνι αναφέρεται σε μια συλλογή από 32 νήματα τα οποία υφαίνονται μεταξύ τους, και εκτελούνται αμφίδρομα. Σε κάθε γραμμή του κώδικα, κάθε νήμα από το στημόνι εκτελεί την ίδια διαδικασία, σε διαφορετικά δεδομένα.
Όσον αφορά τον χειρισμό της σταθερής μνήμης, το υλικό NVIDIA μπορεί να αναμεταδώσει μια απλή ανάγνωση σε μια ομάδα απο 16 νήματα: τα μισά από τα 32 νήματα που βρίσκονται στο στημόνι. Αν κάθε νήμα ζητάει δεδομένα απο την ίδια διεύθυνση της σταθερής μνήμης, η μονάδα επεξεργασίας γραφικών θα εκτελέσει μόνο ένα αίτημα ανάγνωσης και θα αναμεταδώσει τα δεδομένα σε όλα τα νήματα. Αν η ανάγνωση γίνεται απο την σταθερή μνήμη για μεγάλο όγκο πληροφοριών, θα δημιουργηθεί κίνηση που αντιστοιχεί μόνο στο 1/16 της κίνησης μνήμης που θα χρειαζόταν για την γενική μνήμη.

Τα πλεονεκτήματα όμως δεν σταματάνε εκεί. Επειδή έχουμε αποφασίσει να αφήσουμε άθικτη την μνήμη, το υλικό μπορεί να αποθηκεύσει τα σταθερά δεδομένα στην GPU. Έτσι μετά από την πρώτη ανάγνωση από την διεύθυνση της σταθερής μνήμης, όλες οι επόμενες αναγνώσεις δεν θα δημιουργήσουν επιπλέον κίνηση στην μνήμη. Δυστυχώς, μπορεί να υπάρξουν στιγμές που η απόδοση να μειώνεται λόγω της σταθερής μνήμης. Η αναμετάδοση μέρους από το στημόνι (warp) είναι ένα δίκοπο μαχαίρι. Αν και μπορεί να επιταχύνει την απόδοση όταν τα 16 νήματα διαβάζουν την ίδια διεύθυνση, μπορεί αντίστοιχα να μειώσει την απόδοση όταν τα 16 νήματα διαβάζουν διαφορετικές διευθύνσεις. Για παράδειγμα, αν τα 16 νήματα χρειάζονται διαφορετικά δεδομένα απο την σταθερή μνήμη, οι 16 αναγνώσεις θα γίνονταν με την σειρά, και θα χρειάζονταν 16 φορές περισσότερο χρόνο για να εκτελέσουν την εντολή. Αν διάβαζαν απο την γενική μνήμη, η εντολή θα δινόταν άμεσα. Σε αυτήν την περίπτωση, η ανάγνωση απο την σταθερή μνήμη θα ήταν πιο αργή από το να γινόταν ανάγνωση της γενικής μνήμης.\cite{cuda-4}


\subsection{CUDA streams}
Εξηγήσαμε το πως ο παράλληλος προγραμματισμός δεδομένων σε μια GPU μπορεί να δώσει εντυπωσιακά αποτελέσματα σε σχέση με την εκτέλεση σε CPU. Όμως υπάρχει ακόμα ένας τύπος παράλληλου υπολογισμού που μπορούμε να εκμεταλλευτούμε σε μια μονάδα επεξεργασίας γραφικών NVIDIA. Ο παραλληλισμός αυτός μοιάζει με αυτόν που συμβαίνει στα πολυ-νηματικά προγράμματα CPU. Αντί να εκτελούμε την ίδια διαδικασία σε πολλά στοιχεία δεδομένων όπως στον παραλληλισμό δεδομένων, ο παραλληλισμός έργων περιλαμβάνει παράλληλη εκτέλεση δύο η περισσότερων έργων. 

Σαν έργο μπορούμε να θεωρήσουμε μεγάλο αριθμό πραγμάτων. Για παράδειγμα, μια εφαρμογή μπορεί να εκτελεί δύο έργα: επανασχεδίαση του γραφικού περιβάλλοντος με ένα νήμα, και μεταφόρτωση μιας αναβάθμισης μέσω δικτύου με κάποιο άλλο νήμα. Αυτά τα έργα εκτελούνται παράλληλα, και ας μην έχουν τίποτε κοινό. Αν και ο παραλληλισμός έργων δεν είναι τόσο ευέλικτος όσο στις CPUs, συνεχίζει να μας προσφέρει ευκαιρίες για να αποκτήσουμε περισσότερη ταχύτητα από τις βασισμένες σε GPU εφαρμογές μας.

Τα CUDA streams, παίζουν μεγάλο ρόλο στην επιτάχυνση των εφαρμογών μας. Ένα CUDA stream αντιπροσωπεύει μια λίστα από GPU διεργασίες που θα εκτελεσθούν με συγκεκριμένη σειρά. Οι διεργασίες μπορούν να περιλαμβάνουν εκτελέσεις πυρήνων, αντιγραφές μνήμης, και εκκινήσεις/τερματισμούς συμβάντων ενός stream. H σειρά εισαγωγής των διεργασιών ορίζει την σειρά εκτέλεσης τους, με ευκαιρίες για αυτά τα έργα να εκτελεστούν παράλληλα.

%TODO: STUFF MISSING here 
\section{OpenCL}
\subsection{Εισαγωγή}
Το πρώτο GPGPU framework δημιουργήθηκε απο την NVIDIA και ήταν το CUDA. Το CUDA παρείχε στους χρήστες ένα προγραμματιστικό περιβάλλον σε C like γλώσσα για την GPU. Όμως ήταν κλειστού κώδικα και μπορεί να τρέχει μόνο σε NVIDIA κάρτες γραφικών. Λόγω της μεγάλης δημοτικότητας του CUDA, η ανάγκη για ένα ανοιχτό πρότυπο αρχιτεκτονικής που θα υποστηρίζει διάφορα είδη συσκευών απο διάφορους κατασκευαστές γινόταν όλο και πιο σημαντική. Έτσι τον Ιούνιο του 2008 το Khronos Group δημιούργησε το OpenCL 1.0. Αρκετοί κατασκευαστές σταδιακά παρείχαν εργαλεία για προγραμματισμό σε OpenCL συμπεριλαμβανομένων των: Nvidia OpenCL Drivers and Tools, AMD APP SDK,Intel SDK for OpenCL applications,IBM Server with OpenCL development Kit, κ.α. Σήμερα το OpenCL επιτρέπει πολυπύρηνο προγραμματισμό, προγραμματισμό GPU, κ.α.\cite{opencl-1}   

Το OpenCL είναι ένα πλαίσιο για κατασκευή εφαρμογών που εκτελούνται σε ετερογενή συστήματα που αποτελούνται από κεντρικές μονάδες επεξεργασίας(CPU), μονάδες επεξεργασίας γραφικών(GPU), επεξεργαστές ψηφιακών σημάτων(DPS), συστοιχίες προγραμματιζόμενων θυρίδων(FPGA), και άλλους επεξεργαστές. Το OpenCL περιέχει μια γλώσσα, υποσύνολο του ISO C99 με επεκτάσεις, για τον προγραμματισμό αυτών των συσκευών, προγραμματιστικές διεπαφές εφαρμογών(API) για τον έλεγχο της πλατφόρμας και την εκτέλεση προγραμμάτων στις υπολογιστικές συσκευές. Το OpenCL παρέχει παράλληλο υπολογισμό χρησιμοποιώντας παραλληλισμό διεργασιών και δεδομένων.\\
Αποτελεί το πρώτο ανοιχτό, ελεύθερο από τέλη αδειών πρότυπο για cross-platform, παράλληλο προγραμματισμό μοντέρνων επεξεργαστών, που χρησιμοποιούνται συνήθως σε προσωπικούς υπολογιστές, διακομιστές, και φορητές/ενσωματωμένες συσκευές. Το OpenCL (Open Computing Language) βελτιώνει αισθητά την ταχύτητα και την απόκριση μεγάλου εύρους εφαρμογών σε διάφορες κατηγορίες αγορών από παιχνίδια και ψυχαγωγία, μέχρι επιστημονικές εφαρμογές και εφαρμογές υγείας. Συντηρείται απο το μή-κερδοσκοπικό τεχνολογικό συνεταιρισμό Khronos Group. Έχει υιοθετηθεί από πολλές μεγάλες εταιρίες όπως η Apple, Intel, Qualcomm, AMD, Nvidia,Samsung,ARM Holdings.
\begin{figure}[h]
	\includegraphics[width=0.5\linewidth]{opencl_logo}
	\centering
	\caption{Λογότυπο OpenCL}
\end{figure}
\subsection{Επισκόπηση}
Το OpenCL ορίζει μια διεπαφή προγραμματισμού εφαρμογών με την οποία επιτρέπει στα προγράμματα που τρέχουν στον οικοδεσπότη να εκτελέσουν πυρήνες στην συσκευή υπολογισμού, και να διαχειριστούν την μνήμη της συσκευής, που είναι ξεχωριστή από την μνήμη του οικοδεσπότη. Τα προγράμματα του OpenCL είναι σχεδιασμένα ώστε να μεταγλωττίζονται την ώρα της εκτέλεσης, και με αυτόν τον τρόπο γίνεται δυνατό να εκτελεστούν σε διάφορες συσκευές. Το πρότυπο του OpenCL ορίζει διεπαφές προγραμματισμού για γλώσσα C και C++. Διεπαφές επίσης υπάρχουν και για άλλες γλώσσες, όπως Python, Julia, και Java. Μια εφαρμογή του OpenCL προτύπου αποτελείται απο μια βιβλιοθήκη που υλοποιεί την διεπαφή για C και C++, και έναν μεταγλωττιστή OpenCL για τις συσκευές υπολογισμού.\\
\subsubsection{Ιεραρχία μνήμης}
Το OpenCL ορίζει ιεραρχία τεσσάρων επιπέδων για την μνήμη των συσκευών υπολογισμού:
\begin{itemize}
\item Καθολική μνήμη: διαμοιράζεται σε όλες τις συσκευές υπολογισμού, αλλά έχει μεγάλη καθυστέρηση απόκρισης
\item Μνήμη προσπέλασης: μικρότερη, χαμηλή καθυστέρηση απόκρισης, εγγράψιμη από την κεντρική μονάδα επεξεργασίας του οικοδεσπότη, αλλά όχι των συσκευών υπολογισμού.
\item Τοπική μνήμη: διαμοιράζεται σε πολλά στοιχεία υπολογισμού μιας συσκευής
\item Ιδιωτική μνήμη στοιχείου (καταχωρητές) 
\end{itemize}   
Δεν είναι απαραίτητο για όλες τις συσκευές να υλοποιήσουν την ιεραρχία της μνήμης στο υλικό. Η συνέπεια στα διάφορα επίπεδα της ιεραρχίας είναι χαλαρή, και επιβάλλεται μόνο από κατηγορηματικά στοιχεία συγχρονισμού, όπως τα εμπόδια.
\subsection{Ιστορία}
Το OpenCL δημιουργήθηκε αρχικά απο την Apple Inc., η οποία κατέχει τα πνευματικά δικαιώματα, και εξευγενίστηκε σε αρχική πρόταση σε συνεργασία με τεχνικές ομάδες της AMD,IBM,Qualcomm,Intel, και Nvidia. Η Apple καταχώρησε την πρόταση στο Khronos Group, και τον Ιούνιο του 2008 διαμορφώθηκε το Khronos Compute Working Group με αντιπρόσωπους από εταιρίες επεξεργαστών, μονάδων υλικού γραφικών, ενσωματωμένων-επεξεργαστών, και λογισμικού. Αυτό το group εργάστηκε για 5 μήνες ώστε να φέρει σε πέρας τον πρώτο προσδιορισμό για το OpenCL 1.0, ο οποίος κυκλοφόρησε τον Δεκέμβριο του 2008. 
\begin{table}[h]
	\begin{tabular}{|l|p{.9\linewidth}|}
	\hline
OpenCL 1.0 & Η AMD, αν και αρχικά εργαζόταν πάνω στο πρότυπο Close to Metal, αποφάσισε να στραφεί και να υποστηρίξει το OpenCL. Η Nvidia ανακοίνωσε πλήρης υποστήριξη στην εργαλειοθήκη υπολογισμού GPU. Το 2009, η IBM κυκλοφόρησε την πρώτη έκδοση του μεταγλωττιστή της με υποστήριξη για OpenCL\\ \hline
OpenCL 1.1 & Το OpenCL 1.1 επικυρώθηκε από το Khronos Group τον Ιούνιο του 2010, και προσθέτει σημαντικές λειτουργίες για βελτιωμένη ευελιξία παράλληλου προγραμματισμού, λειτουργικότητα, και επιδόσεις.\\ \hline
OpenCL 1.2 & Το OpenCL 1.2 ανακοινώθηκε τον Νοέμβριο του 2011 απο το Khronos Group, το οποίο προσθέτει αρκετές λειτουργίες σε σχέση με τις προηγούμενες εκδόσεις όσον αφορά τις επιδόσεις και χαρακτηριστικά για παράλληλο προγραμματισμό.\\ \hline
OpenCL 2.0 & Το OpenCL 2.0 επικυρώθηκε και κυκλοφόρησε τον Νοέμβριο του 2013 απο το Khronos Group και αποτελεί την τελευταία έκδοση του OpenCL. \\ \hline
	\end{tabular}
\end{table}

\subsection{Στόχοι}
Ο στόχος του OpenCL είναι να κάνει ορισμένους τύπους παράλληλου προγραμματισμού πιο εύκολους, και να παρέχει ανεξαρτήτου κατασκευαστή παράλληλη εκτέλεση κώδικα μέσω επιτάχυνσης υλικού. Το OpenCL είναι το πρώτο ανοιχτό, ελεύθερο πρότυπο για παράλληλο προγραμματισμό γενικού σκοπού ετερογενών συστημάτων. Παρέχει ένα προγραμματιστικό περιβάλλον που βοηθάει τους προγραμματιστές να γράψουν αποδοτικό, φορητό κώδικα για συστήματα υψηλής απόδοσης, προσωπικούς υπολογιστές, και κινητές συσκευές χρησιμοποιώντας ένα μείγμα πολυπύρηνων CPUs,GPUs, και DSPs.\\
Το OpenCL παρέχει στους προγραμματιστές ένα κοινό σετ εργαλείων εύκολης χρήσης, ώστε αυτοί να εκμεταλλευτούν οποιαδήποτε συσκευή που περιέχει οδηγό OpenCL για την εκτέλεση παράλληλου κώδικα. Το OpenCL framework ορίζει μια γλώσσα C like για την δημιουργία των πυρήνων, και ένα σετ απο APIs για την δημιουργία και την διαχείριση αυτών των πυρήνων. Οι πυρήνες είναι διαδικασίες οι οποίες μπορούν να εκτελούνται σε διαφορετικές συσκευές. Οι πυρήνες μεταγλωττίζονται απο έναν μεταγλωτιστή runtime, μέσω κατάλληλου προγράμματος. Αυτό επιτρέπει στα προγράμματα να εκμεταλλεύονται όλες τις συσκευές ενός συστήματος με ένα σετ φορητών υπολογιστικών πυρήνων.
\subsection{Μοντέλο εκτέλεσης}
Τα κύρια μέρη εκτέλεσης ενός προγράμματος OpenCL είναι ο πυρήνας και το πρόγραμμα ξενιστή. Οι πυρήνες εκτελούνται στην συσκευή OpenCL και το πρόγραμμα ξενιστή, στον υπολογιστή που εκτελείται το πρόγραμμα. Ο σκοπός του προγράμματος ξενιστή είναι να δημιουργήσει και να ζητήσει την πλατφόρμα και τις ιδιότητες της συσκευής, να ορίσει το περιεχόμενο, να κατασκευάσει τον πυρήνα, και να διαχειριστεί την εκτέλεση των πυρήνων. Όταν καταχωρηθεί ο πυρήνας απο τον ξενιστή στην συσκευή, δημιουργείται ένα Ν διαστάσεων χώρος ευρετηρίου, με το Ν να είναι από 1 έως 3. Κάθε περιστατικό πυρήνα δημιουργείται στις συντεταγμένες του χώρου ευρετηρίου. Αυτό το περιστατικό ονομάζεται αντικείμενο εργασίας και ο χώρος ευρετηρίου καλείται NDRange.

\subsection{WebCL}
Με την δημοτικότητα των εφαρμογών web να αυξάνεται, υπάρχει ανάγκη για αύξηση της απόδοσης παράλληλης υπολογιστικής επεξεργασίας για να επιταχύνουμε τις διεργασίες εντατικού υπολογισμού σε εφαρμογές ιστού. Αυτές οι εφαρμογές περιλαμβάνουν για παράδειγμα, παρουσίαση απεικόνισης δεδομένων, επεξεργασία εικόνας και βίντεο, παιχνίδια τριών διαστάσεων, υπολογιστική φωτογραφία, κρυπτογραφία. Παρέχοντας στους προγραμματιστές με ένα πρότυπο Javascript API και μια γλώσσα φορητού προγραμματισμού, το WebCL επιτρέπει παράλληλο υπολογισμό σε ετερογενή πολυ-πύρηνα συστήματα σε μια σωρεία συσκευών, συμπεριλαμβανομένου φορητών, σταθερών, και διακομιστών. 

Το WebCL 1.0 ορίζει ένα Javascript binding στο πρότυπο OpenCL για ετερογενή παράλληλο υπολογισμό, ενώ επιτρέπει σε εφαρμογές ιστού να εκμεταλλευτούν τις δυνατότητες της GPU και τον παράλληλο υπολογισμό πολυπύρηνων CPU, μέσα απο έναν Web Browser, ενεργοποιώντας σημαντική επιτάχυνση των εφαρμογών όπως επεξεργασία βίντεο και εικόνας, και ανώτερης εξομοίωσης φυσικής για παιχνίδια WebGL. Το WebCL έχει αναπτυχτεί σε στενή συνεργασία με την κοινότητα του web, και παρέχει την δυνατότητα να επεκταθούν οι δυνατότητες των HTML5 browsers ώστε να επιταχύνουν τις εφαρμογές υψηλών απαιτήσεων υπολογισμού και πλούσιου οπτικού υπολογισμού. Το WebCL έχει σχεδιαστεί και συντηρείται από το μη κερδοσκοπικό ίδρυμα Khronos Group. Οι ολοκληρωμένες προδιαγραφές της πρώτης έκδοσης του WebCL ανακοινώθηκαν τον Μάρτιο του 2014.

Ανάλογα την υλοποίηση, Το WebCL επιτρέπει στις διεργασίες να εκτελούνται ταυτόχρονα με την Javascript. Συγκεκριμένα, είναι δυνατόν για την εφαρμογή να επεξεργαστεί ένα buffer καθώς αντιγράφεται από ένα αντικείμενο WebCLMemoryObject. Για να αποτρέψουμε την φθορά των δεδομένων, οι εφαρμογές θα πρέπει να μην επεξεργάζονται τα buffers που έχουν σημαδευτεί για ασύγχρονη ανάγνωση/εγγραφή, έως ότου η σχετική ουρά εντολών WebCL έχει τελειώσει την εκτέλεση της.

\begin{figure}[h]
	\includegraphics[scale=1]{webcl-logo}
	\centering
	\caption{Λογότυπο WebCL}
\end{figure}
\begin{itemize}
\item Khronos Launching new WebCL initiative
	\begin{itemize}
	\item Ανακοινώθηκε τον Μάρτιο του 2011
	\item API definitions already underway
	\end{itemize}
\item Javascript binding για OpenCL
	\begin{itemize}
	\item Η ασφάλεια πρώτη προτεραιότητα
	\end{itemize}
\item Πολλές περιπτώσεις χρήσης
	\begin{itemize}
	\item Μηχανές φυσικής για συμπλήρωση του WebGL
	\item Επεξεργασία εικόνας και βίντεο σε browser
	\end{itemize}
\item Πολύ στενή σχέση με το πρότυπο OpenCL
	\begin{itemize}
	\item Maximum flexibility
	\item Foundation for higher-level middleware
	\end{itemize}
\end{itemize}


\subsubsection{Υλοποιήσεις}
Αυτήν την στιγμή, μόνο μια δοκιμαστική έκδοση του Chromium browser υποστηρίζει το WebCL, γιατί η τεχνολογία είναι σχετικά καινούρια. Όμως το webCL μπορεί να εκτελεστεί και ως ένα πρόσθετο. Για παράδειγμα, η Nokia έχει δημιουργήσει ένα πρόσθετο WebCL για τον Mozilla Firefox. Η Mozilla έχει ανακοινώσει ότι δεν θα υποστηρίξει το WebCL, λόγω των Compute Shader του OpenGL ES 3.1. Σαν παράδειγμα αναφέρουμε υλοποιήσεις που προσπαθούν να ακολουθήσουν τις προδιαγραφές WebCL 1.0
\begin{itemize}
\item Samsung (WebKit) - https://github.com/SRA-SiliconValley/webkit-webcl
\item Nokia (Firefox) - http://webcl.nokiaresearch.com/
\item Motorola Mobility node-webcl (node.js) - https://github.com/Motorola-Mobility/node-webcl
\end{itemize}


\subsubsection{Ασφάλεια}
Το WebCL έχει δημιουργηθεί δίνοντας μεγάλη έμφαση στην ασφάλεια. Μερικά από τα φαινόμενα που μπορεί να προκύψουν κατά την μεταγλώττιση ή και την εκτέλεση ενός πυρήνα WebCL είναι τα παρακάτω
\begin{itemize}
\item Πρόσβαση OOB(Out of Bounds) - Οι πυρήνες του WebCL πρέπει να μην επιτρέπουν πρόσβαση σε κρίσιμα μέρη της μνήμης, χωρίς να γίνονται διακρίσεις ανάλογα με τον τύπο της μεταβλητής.(private, global, constant). Αν ανιχνευθεί την ώρα της μεταγλώττισης, η πρόσβαση OOB πρέπει να παράγει ένα λάθος μεταγλώττισης. Την ώρα της εκτέλεσης, η προσπέλαση OOB πρέπει να επιστρέψει μηδέν, και οι εγγραφές να αγνοηθούν. Για λόγους ελέγχου του ορίου, η υλοποίηση μπορεί να μεταχειρίζεται τις μεταβλητές σαν ένα συνεχόμενο block μνήμης. Για τον έλεγχο των προσπελάσεων OOB, το Khronos Group ανέπτυξε τον WebCL Validator, ο οποίος αναγκάζει την αρχικοποίηση τοπικής μνήμης.
\item Αρχικοποίηση μνήμης - Για να σιγουρέψουμε ότι οι εφαρμογές δεν μπορούν να προσπελάσουν παλαιότερα δεδομένα που έχουν μείνει στην μνήμη απο προηγούμενες εφαρμογές, η υλοποίηση WebCL πρέπει να αρχικοποιεί όλα τα buffers και μεταβλητές στο μηδέν πριν επιτρέψει την πρόσβαση στην εφαρμογή μας. Αυτή η ανάγκη υπάρχει ανεξάρτητα από τον τύπο της μεταβλητής και ανεξάρτητα την συσκευή στην οποία εκτελείται ο κώδικας. Όπου είναι δυνατόν, το πρόσθετο OpenCL 1.2 "cl{\_}khr{\_}initialize{\_}memory" επιτρέπει στις υλοποιήσεις WebCL να αρχικοποιήσουν την τοπική μνήμη αυτόματα πριν την εκτέλεση ενός πυρήνα.
\item Άρνηση υπηρεσίας - Πυρήνες που εκτελούνται για μεγάλο χρονικό διάστημα ή πυρήνες εντατικών υπολογισμών (ή άλλες εντολές στην ουρά εντολών) μπορεί να προκαλέσουν αστάθεια στο σύστημα λόγω της υπερκατανάλωσης πόρων. Δεν είναι εύκολο να γίνει έλεγχος του συγκεκριμένου προβλήματος στο επίπεδο του WebCL. Οι απαραίτητοι μηχανισμοί, όπως μετρητές watchdog και προληπτικός πολυ-νηματικός προγραμματισμός, πρέπει να παρέχονται από τον οδηγό OpenCL και το λειτουργικό σύστημα. Σε συστήματα όπου οι παραπάνω μηχανισμοί είναι διαθέσιμοι, οι υλοποιήσεις WebCL πρέπει να τους χρησιμοποιούν για να:
	\begin{enumerate}
		\item Ανιχνεύουν προβληματικούς πυρήνες και άλλες εντολές. Μια εντολή θεωρείται προβληματική όταν εκτελείται για πολύ μεγάλο χρονικό διάστημα, ή καταναλώνει υπερβολικά πολλούς πόρους συστήματος
		\item Τερματίζει τα περιεχόμενα που σχετίζονται με τις προβληματικές εντολές, πριν αυτές καταστήσουν την συσκευή OpenCL απαθής, και την αναγκάσουν να χρειαστεί επανεκκίνηση.
	\end{enumerate}
\end{itemize}
Όπου αυτό είναι δυνατόν, το πρόσθετο OpenCL 1.2 "cl{\_}kh{\_}-terminate{\_}context" μπορεί να χρησιμοποιηθεί για γρήγορο τερματισμό ενός περιεχομένου, αν για παράδειγμα κάποιος πυρήνας εκτελείται για πολύ μεγάλο χρονικό διάστημα ή ένα από τα προγράμματα τερματιστεί λόγω εξαιρέσεων.

\section{Compute Shaders}
Τα shaders υπολογισμού είναι μια κατάσταση shader που χρησιμοποιείται σχεδόν αποκλειστικά για υπολογισμούς αυθαίρετης πληροφορίας.Αν και μπορεί να χρησιμοποιηθεί για απόδοση, συνήθως χρησιμοποιείται για διεργασίες που δεν σχετίζονται άμεσα με σχεδιασμό τριγώνων και pixel.\cite{computeshaders-1}
\subsection{Shaders}
\subsubsection{Εισαγωγή}
Στον τομέα των γραφικών υπολογιστών, ένα shader είναι ένα πρόγραμμα που χρησιμοποιείται για να εκτελέσει την λεγόμενη σκίαση: την παραγωγή συγκεκριμένου επιπέδου χρώματος μέσα σε μια εικόνα, ή την παραγωγή ειδικών εφέ ή μετατροπές βίντεο. Ένας όρος που περιγράφει την σκίαση είναι "ένα πρόγραμμα που μαθαίνει τον υπολογιστή πώς να ζωγραφίσει κάτι με έναν ειδικό και μοναδικό τρόπο".\\
Τα shader υπολογίζουν αποδόσεις εφέ σε υλικό γραφικών, με ένα μεγάλο βαθμό ευλυγισίας. Τα περισσότερα shader είναι σχεδιασμένα για χρήση σε μονάδα επεξεργασίας γραφικών (GPU), όμως αυτό δεν είναι αποκλειστική ανάγκη. Οι γλώσσες σκίασης, χρησιμοποιούνται συνήθως για να προγραμματίσουν την γραμμή σωλήνα απόδοσης της GPU. Η θέση, η απόχρωση, ο κορεσμός, η φωτεινότητα, και η αντίθεση όλων των στοιχείων, κορυφών, ή υφών, χρησιμοποιούνται για να αποδώσουν μια τελική εικόνα που μπορούμε να επεξεργαστούμε απευθείας με χρήση αλγορίθμων ορισμένων στα shader, είτε με αλλαγές απο εξωτερικές μεταβλητές που εισάγει το πρόγραμμα το οποίο καλεί τον shader.
Τα shader χρησιμοποιούνται πολύ στην κινηματογραφική επεξεργασία, στις εικόνες που αποδίδονται απο τον υπολογιστή, αλλα και σε παιχνίδια υπολογιστών, για να παράγουν ένα μεγάλο αριθμό απο εφέ. Εκτός απο τα απλά μοντέλα φωτισμού, μερικά απο τα πολύπλοκα εφέ επεξεργάζονται την εικόνα και προσθέτουν blur, light bloom, volumetric lightning, normal mapping, bokeh, cel shading,posterization,bump mapping,distortion,chroma keying, edge detection, motion detection, κ.α\\
Η σύγχρονη χρήση των shader ξεκίνησε απο την Pixar, τον Μάιο του 1988. Όσο οι μονάδες επεξεργασίας γραφικών εξελίσσονταν, οι γνωστές βιβλιοθήκες γραφικών ξεκίνησαν να υποστηρίζουν τα shader. Οι πρώτες κάρτες γραφικών υποστήριζαν μόνο pixel shader, αλλά σύντομα ακολούθησε η εισαγωγή των vertex shader όταν οι προγραμματιστές κατάλαβαν τις δυνατότητες τους. Τα shader γεωμετρίας εισήχθηκαν μόλις με το Direct3D 10 και το OpenGL 3.2
\subsubsection{Τύποι}
Υπάρχουν διάφοροι τύποι shader που χρησιμοποιούνται γενικά. Ενώ οι παλιές κάρτες γραφικών είχαν ξεχωριστό τρόπο επεξεργασίας στοιχείων για κάθε τύπο shader, οι καινούριες έχουν ενωμένους shader που έχουν την δυνατότητα να εκτελούν οποιονδήποτε τύπο shader. Αυτό επιτρέπει στις κάρτες γραφικών να έχουν πιο αποδοτική χρήση της επεξεργαστικής τους δύναμης.
\begin{itemize}
\item Vertex shaders - Μετατρέπουν κάθε θέση τρισδιάστατη στον εικονικό χώρο σε δισδιάστατη. Μπορούν να επεξεργαστούν ιδιότητες όπως η θέση, το χρώμα, και συντεταγμένες υφής, αλλα δεν μπορούν να δημιουργήσουν καινούρια vertices.
\item Pixel shaders - Υπολογίζουν το χρώμα και άλλες ιδιότητες ενός τεμαχίου. Οι απλές μορφές τους αποδίδουν ένα pixel εικόνας, ενώ οι πιο πολύπλοκες μορφές αποδίδουν πολλά. Στα τρισδιάστατα γραφικά, ένας pixel shader δεν μπορεί να παράγει πολύπλοκα εφέ, γιατί επεξεργάζεται μόνο ένα τεμάχιο, χωρίς κάποια γνώση της γεωμετρίας της οθόνης. Μπορούν όμως να εφαρμοστούν σε εφέ δύο διαστάσεων και χρησιμοποιούνται για επεξεργασία υφών. Για παράδειγμα είναι ο μόνος τύπος shader που μπορεί να λειτουργήσει σαν φίλτρο για μια ροή βίντεο.
\item Geometry shaders - Τα shader γεωμετρίας είναι σχετικά καινούριος τύπος shader. Μπορεί να δημιουργήσει γραφικά αρχικά στοιχεία όπως γραμμές, τρίγωνα. Τα shader γεωμετρίας εκτελούνται μετά απο τα vertex shader. Τυπικές χρήσεις των shader γεωμετρίας συμπεριλαμβάνουν γεωμετρική ψηφίδωση, εξώθηση σκιώδους όγκου, και απόδοση σε χάρτη κύβου. Για παράδειγμα, σε μια καμπυλωτή γραμμή, τα δεδομένα των στοιχείων εισάγονται σαν είσοδος στον shader, και αυτός αναλαμβάνει να δημιουργήσει αυτόματα επιπλέον γραμμές που δίνουν πιο εύστοχη απεικόνιση της καμπύλης.
\item Tesselation shaders - Με την κυκλοφορία του OpenGL 4.0 και του Direct3D 11, ένας καινούριος τύπος shader έχει προστεθεί που ονομάζεται shader ψηφίδωσης. Επιτρέπει στα αντικείμενα κοντά στην κάμερα να έχουν καλύτερη λεπτομέρεια, ενώ τα αντικείμενα που είναι πιο μακριά να έχουν μικρότερη λεπτομέρεια αλλά να μοιάζουν όμοια στην ποιότητα.
\item Compute shaders - Είναι ένας τύπος shader που χρησιμοποιείται αποκλειστικά για υπολογισμό αυθαίρετης πληροφορίας. Αν και μπορεί να χρησιμοποιηθεί για απόδοση, συνήθως χρησιμοποιείται για διεργασίες που δεν σχετίζονται άμεσα με σχεδιασμό τριγώνων και pixel.
\end{itemize}
\subsection{Εισαγωγή}
Τα shaders υπολογισμού λειτουργούν διαφορετικά από τις άλλες καταστάσεις shader. Όλες οι καταστάσεις shader έχουν προκαθορισμένου τύπου τιμές εισόδου, μερικές ενσωματωμένες και μερικές καθορισμένες από τον χρήστη. Η συχότητα στην οποία εκτελείται μια κατάσταση shader εξαρτάται από την φύση της κατάστασης. Για παράδειγμα, τα shader κορυφών εκτελούνται μια φορά για κάθε κορυφή. \\
Τα shader υπολογισμού λειτουργούν πολύ διαφορετικά. Ο "χώρος" στον οποίο ένα shader υπολογισμού λειτουργεί είναι αφηρημένος. Είναι στην κρίση του κάθε shader υπολογισμού να αποφασίσει τι σημαίνει αυτός ο "χώρος". Ο αριθμός των εκτελέσεων των shader υπολογισμού ορίζεται από την διεργασία που χρησιμοποιείται για να εκτελεστεί η υπολογιστική λειτουργία. Πιο σημαντικό από όλα, τα shader υπολογισμού δεν έχουν εισόδους καθορισμένες απο τον χρήστη και ούτε καμία έξοδο. Οι ενσωματωμένες είσοδοι ορίζουν μόνο το πού στον "χώρο" της εκτέλεσης βρίσκεται ένας συγκεκριμένος shader υπολογισμού.\\
Έτσι, αν κάποιος shader υπολογισμού πρέπει να πάρει κάποιες τιμές σαν είσοδο, είναι στην ευθύνη του shader να αποκτήσει τα δεδομένα, μέσω πρόσβασης υφών, αυθαίρετης φόρτωσης εικόνας, ή άλλες μορφές διεπαφής. Παρομοίως, αν ένας shader υπολογισμού υπολογίζει κάτι, θα πρέπει να το αποθηκεύσει σε μια εικόνα ή σε ένα block αποθήκευσης shader.
\subsection{Χώρος υπολογισμού}
Ο χώρος στον οποίο λειτουργεί ένα shader υπολογισμού είναι αφηρημένος. Υπάρχει η έννοια της ομάδας εργασίας. Είναι ο μικρότερος αριθμός από λειτουργίες υπολογισμού τις οποίες μπορεί να εκτελέσει ο χρήστης.
Ο αριθμός των ομάδων εργασίας με τον οποίο μια λειτουργία υπολογισμού εκτελείται, ορίζεται απο τον χρήστη όταν επικαλείται την λειτουργία υπολογισμού. Ο χώρος αυτών των ομάδων είναι τρισδιάστατος, οπότε έχει ένα αριθμό απο ομάδες "Χ","Υ","Ζ". Κάθε ένας απο αυτούς μπορεί να είναι 1, οπότε είναι δυνατή η εκτέλεση λειτουργιών υπολογισμού δυο ή και μίας διάστασης αντί για τρεις διαστάσεις. Αυτό είναι χρήσιμο για την επεξεργασία δεδομένων εικόνας ή γραμμικών πινάκων ενός συστήματος.\\
Η κάθε ομάδα εργασίας μπορεί να αποτελείται απο πολλούς shader υπολογισμού. Αυτό ονομάζεται τοπικό μέγεθος της ομάδας εργασίας. Κάθε shader υπολογισμού έχει τρισδιάστατο τοπικό μέγεθος (το οποίο μπορεί να είναι 1 επιτρέποντας δισδιάστατη ή μονοδιάστατη επεξεργασία). Αυτό ορίζει τον αριθμό των επικλήσεων ενός shader που θα εκτελεστούν σε κάθε ομάδα εργασίας. Για παράδειγμα, αν το τοπικό μέγεθος ενός shader υπολογισμού είναι (128, 1, 1) και εκτελεστεί με έναν αριθμό ομάδων εργασίας (16, 8 , 64) τότε θα έχουμε 1,048,576 ξεχωριστές επικλήσεις shader. Κάθε επίκληση θα έχει ένα σετ από εισόδους που αναγνωρίζουν μοναδικά την κάθε επίκληση. Αυτός ο διαχωρισμός είναι χρήσιμος για διάφορες μορφές συμπίεσης και αποσυμπίεσης εικόνας. Το τοπικό μέγεθος θα είναι το μέγεθος ενός block δεδομένων εικόνας (8x8 για παράδειγμα), ενώ ο αριθμός των ομάδων θα είναι το μέγεθος της εικόνας διαιρούμενο με το μέγεθος του block. Κάθε block κατεργάζεται σαν μια μοναδική ομάδα εργασίας.
\subsection{Υλοποίηση OpenGL}
\subsubsection{Αποστολή}
Ένα αντικείμενο προγράμματος μπορεί να έχει shader υπολογισμού μέσα του. Ο shader υπολογισμού συνδέεται με καταστάσεις shader μέσω κάποιων λειτουργιών απόδοσης. Υπάρχουν δύο λειτουργίες για να ξεκινήσουν οι διαδικασίες υπολογισμού. Χρησιμοποιούν οποιονδήποτε shader υπολογισμού είναι ενεργός. Οι λειτουργίες είναι οι εξής:
\begin{itemize}
\item void glDispatchCompute(GLuint num\_groups\_x, GLuint num\_groups\_u, GLuint num\_groups\_z); - Οι παράμετροι num\_groups\_* ορίζουν τον αριθμό των ομάδων εργασίας, σε τρεις διαστάσεις. Αυτοί οι αριθμοί δεν μπορούν να είναι μηδέν. Υπάρχουν όρια στον αριθμό των ομάδων εργασίας που μπορούν να αποσταλούν.
\item void glDispatchComputeIndirect(GLintptr indirect); - H παράμετρος indirect είναι το αντιστάθμισμα του buffer GL\_DISPATCH\_INDIRECT\_BUFFER. Ισχύουν τα ίδια όρια του αριθμού ομάδων εργασίας, όμως η αποστολή indirect παρακάμπτει τον έλεγχο λαθών του OpenGL. Έτσι, η αποστολή με εκτός ορίων μεγέθους ομάδας εργασίας, μπορεί να προκαλέσει προβλήματα ακόμα και πάγωμα του συστήματος.
\end{itemize}
\subsubsection{Είδοδοι}
Τα shader υπολογισμού  δεν μπορούν να έχουν μεταβλητές καθορισμένες απο τον χρήστη. Τα shader υπολογισμού έχουν τις παρακάτω ενσωματωμένες μεταβλητές εξόδου:
\begin{itemize}
\item in uvec3 gl\_NumWorkGroups; - Αυτή η μεταβλητή περιέχει τον αριθμό των ομάδων εργασίας για την λειτουργία αποστολής
\item in uvec3 gl\_WorkGroupID; - Αυτή η μεταβλητή περιέχει την ισχύουσα ομάδα εργασίας για την επίκληση του shader.
\item in uvec3 gl\_LocalInvocationID; - Αυτή η μεταβλητή περιέχει την ισχύουσα επίκληση του shader μέσα στην ομάδα εργασίας.
\item in uvec3 gl\_GlobalInvocationID; - Αυτή η μεταβλητή αναγνωρίζει μοναδικά την συγκεκριμένη επίκληση του shader υπολογισμού  ανάμεσα σε όλες τις επικλήσεις της κλήσης αποστολής υπολογισμού. Είναι μια συντόμευση για τον μαθηματικό υπολογισμό gl\_WorkGroupID * gl\_WorkGroupSize + gl\_LocalInvocationID;
\item in uint  gl\_LocalInvocationIndex;
\end{itemize}
\subsubsection{Τοπικό μέγεθος}
Το τοπικό μέγεθος ενός shader υπολογισμού ορίζεται απο τον shader, χρησιμοποιώντας μια ειδική δήλωση εισόδου: 
layout(local\_size\_x = X, local\_size\_y = Y, local\_size\_z = Z) in;
Αρχικά, τα τοπικά μεγέθη είναι 1, οπότε αν θέλουμε μονοδιάστατο ή δισδιάστατο χώρο ομάδων εργασίας, μπορούμε να ορίσουμε μόνο το Χ ή το Χ και το Υ. Πρέπει να είναι σταθερές εκφράσεις τιμής μεγαλύτερης του 0. Οι τιμές πρέπει να ορίζονται σε σχέση με τους περιορισμούς που υπάρχουν παρακάτω. Σε αντίθετη περίπτωση προκύπτουν λάθη. Το τοπικό μέγεθος είναι διαθέσιμο στον shader σαν σταθερά, οπότε δεν χρειάζεται να την ορίζουμε εμείς.
\begin{itemize}
\item const uvec3 gl\_WorkGroupSize;
\end{itemize}
\subsubsection{Περιορισμοί}
Ο αριθμός των ομάδων εργασίας που μπορούν να αποσταλούν, ορίζεται από την GL\_MAX\_COMPUTE\_WORK\_GROUP\_COUNT. Αυτή η σταθερά πρέπει να διαβαστεί απο την glGetIntegeri\_v, με τιμές ανάμεσα στο κλειστό όριο [0,2]. Προσπάθεια να καλέσουμε την glDispatchCompute με τιμές που ξεπερνούν το όριο είναι λάθος. Προσπάθεια κλήσης της glDispatchComputeIndirect είναι χειρότερα, μπορεί να διακόψει την λειτουργία του προγράμματος ακόμα και να παγώσει το σύστημα. Σημείωση: ο μικρότερος αριθμός αυτών των τιμών πρέπει να είναι 65535 σε όλους τους άξονες. Αυτό δίνει αρκετό χώρο για εργασία. Υπάρχουν όρια στο τοπικό μέγεθος επίσης. Συγκεκριμένα, υπάρχουν δύο τύποι περιορισμών. 
\begin{itemize}
\item Ο γενικός περιορισμός των διαστάσεων τοπικού μεγέθους, σε συνδυασμό με την GL\_MAX\_COMPUTE\_WORK\_GROUP\_SIZE, όπως και παραπάνω. Η διαφορά είναι οτι ο μικρότερος αριθμός των τιμών είναι πολύ μικρότερος. 1024 για τον Χ και τον Υ, και μόνο 64 για τον Ζ.
\item Ο αριθμός των επικλήσεων μέσα σε μια ομάδα εργασίας. Δηλαδή, το προϊόν των στοιχείων Χ,Υ,Ζ του τοπικού μεγέθους πρέπει να είναι μικρότερο απο GL\_MAX\_WORK\_GROUP\_INVOCATIONS. Η μικρότερη τιμή είναι 1024.
\end{itemize}
Υπάρχει ακόμα ο περιορισμός του ολικού μεγέθους αποθήκευσης για όλες τις κοινές μεταβλητές ενός shader υπολογισμού. Ορίζεται απο την GL\_MAX\_COMPUTE\_SHARED\_MEMORY\_SIZE, που αναφέρεται σε bytes. Η μικρότερη τιμή για το OpenGL είναι 32KB.
\subsection{Υλοποίηση DirectX}
Ένας shader υπολογισμού είναι μια κατάσταση shader υπολογισμού που εξαπλώνει το Microsoft Direct3D 11 πέρα απο τον προγραμματισμό γραφικών. Η τεχνολογία αυτή είναι γνωστή και ως τεχνολογία DirectCompute\cite{computeshaders-4}
\\Όπως όλα τα προγραμματιστικά shader (για παράδειγμα shader γεωμετρίας και κορυφών), ένα shader υπολογισμού είναι σχεδιασμένο να χρησιμοποιεί μια Γλώσσα Υψηλού Προγραμματισμού Shader(HLSL) για το DirectX. Η HLSL, χρησιμοποιείται για το DirectX και μας δίνει την δυνατότητα να δημιουργήσουμε C like shaders για την γραμμή σωλήνων Direct3D. Η HLSL δημιουργήθηκε ξεκινώντας απο το DirectX 9 για την κατασκευή προγραμματιζόμενων τρισδιάστατων γραμμής σωλήνα. Μας δίνει την δυνατότητα να προγραμματίσουμε την γραμμή σωλήνα με τον συνδυασμό οδηγιών assembly, οδηγιών HLSL, και δηλώσεις καθορισμένων λειτουργιών.\\
Ένα shader υπολογισμού προμηθεύει υψηλής ταχύτητας υπολογισμούς γενικού προγραμματισμού, και εκμεταλλεύεται τον μεγάλο αριθμό παράλληλων επεξεργαστών που βρίσκονται στην μονάδα επεξεργασίας γραφικών (GPU). Τα shader υπολογισμού προμηθεύει διαμοιρασμό μνήμης και συγχρονισμό νημάτων, για να επιτρέψει καλύτερες μεθόδους παράλληλου προγραμματισμού. Με την κλήση των μεθόδων ID3D11DeviceContext::Dispatch ή ID3D11DeviceContext::DispatchIndirect γίνεται η εκτέλεση εντολών σε ένα shader υπολογισμού, οι οποίες μπορούν να εκτελεστούν παράλληλα σε πολλά νήματα.
\section{PathScale Enzo}