\section{PathScale Enzo}
\subsection{Εισαγωγή}
Η σουίτα PathScale ENZO™ συνδιάζει το ελεύθερο πρότυπο υβριδικού πολυπύρηνου παράλληλου προγραμματισμού (HMPP), με την δυνατότητα άμεσης παραγωγής κώδικα για τις NVIDIA Tesla GPUs. Αυτή η προσέγγιση χρησιμοποιεί την δύναμη της GPU ως επιταχυντή υλικού (HWA) για να αντικαταστήσει τις παραδοσιακές μονάδες υπολογισμού SIMD. Χρησιμοποιώντας οδηγίες HMPP με το PathScale ENZO™ επιτρέπει στον προγραμματιστή να γράψει εφαρμογές ανεξάρτητες από το υλικό όπου ο κώδικας για το συγκεκριμένο υλικό διαχωρίζεται από τον παραδοσιακό κώδικα. Οι εφαρμογές δεν πρέπει να γραφτούν ξανά κάθε φορά που στοχεύουμε μια διαφορετική αρχιτεκτονική. 

Το PathScale ENZO™ υποστηρίζει προς το παρόν την HMPP Fortran όπου, όταν συνδυαστεί με το εκτελέσιμο του ENZO™, επιτρέπει άμεση εκτέλεση εφαρμογών ENZO™ GPGPU. Οι μελλοντικές εφαρμογές του ENZO™ θα περιέχουν υποστήριξη για HMPP C,C++ και ENZO™ C++ περιγράμματα. Για να βελτιώσει το πόσο γρήγορα τρέχει η εφαρμογή μας, το ENZO™ πρώτα αναγνωρίζει τις περιοχές του πηγαίου κώδικα της εφαρμογής που είναι κατάλληλες για τον στόχο HWA. Αυτές οι περιοχές καθίστανται περιοχές ή διεργασίες που ονομάζονται "HMPP codelets", με την χρήση οδηγιών HMPP. Οι εκδόσεις των επιταχυνομένων από υλικό των περιοχών ή codelets ορίζονται στην ίδια πηγαία γλώσσα όπως και το υπόλοιπο πρόγραμμα, π.χ η Fortran, με την χρήση του HMPP προγραμματιστικού μοντέλου.\cite{pathscale-1}

\begin{figure}[h]
	\includegraphics[width=\linewidth]{pathscale}
	\centering
	\caption{PathScale ENZO™ με χρήση επιτάχυνσης GPU NVIDIA Tesla}
\end{figure}


Ο HMPP πηγαίος κώδικας αναλύεται από το ειδικό μέρος της PathScale Fotran που μεταφράζει τις HMPP οδηγίες σε κλήσεις του ENZO™ API εκτέλεσης. Το ENZO™ API εκτέλεσης έχει την ευθύνη για την διαχείριση της ταυτόχρονης εκτέλεσης όλων των περιοχών και codelets. Οι οδηγίες HMPP επιτρέπουν επίσης να ομαδοποιήσουμε codelets. Βασιζόμενοι στην προσέγγιση codelet, αυτές οι ομάδες επιτρέπουν στον προγραμματιστή να χρησιμοποιεί δεδομένα που είναι ήδη διαθέσιμα σε έναν επιταχυντή υλικού, ώστε αυτά τα δεδομένα να μπορούν να διαμοιραστούν μεταξύ διαφορετικών codelet που εκτελούνται σε διαφορετικές στιγμές, χωρίς να χρειάζονται επιπλέον μεταφορές δεδομένων μεταξύ της μνήμης του ξενιστή και του HWA.

Όσον αφορά το μοντέλο μνήμης του ENZO™, οι διευθύνσεις μνήμης διαχειρίζονται από το επίπεδο του ξενιστή και είναι διαφορετικές στο επίπεδο HWA. Η εφαρμογή και το API εκτέλεσης του ENZO™ έχουν την δικιά τους τοπική μνήμη. Το ENZO™ το αντιμετωπίζει με τρόπο διαφανή προς τον χρήστη. Το ENZO™ είναι η προγραμματιστική κόλλα μεταξύ των προγραμματιστικών περιβαλλόντων και του προγραμματισμού γενικού σκοπού.

