\section{Compute Shaders}
Τα shaders υπολογισμού είναι μια κατάσταση shader που χρησιμοποιείται σχεδόν αποκλειστικά για υπολογισμούς αυθαίρετης πληροφορίας.Αν και μπορεί να χρησιμοποιηθεί για απόδοση, συνήθως χρησιμοποιείται για διεργασίες που δεν σχετίζονται άμεσα με σχεδιασμό τριγώνων και pixel.\cite{computeshaders-1}
\subsection{Shaders}
\subsubsection{Εισαγωγή}
Στον τομέα των γραφικών υπολογιστών, ένα shader είναι ένα πρόγραμμα που χρησιμοποιείται για να εκτελέσει την λεγόμενη σκίαση: την παραγωγή συγκεκριμένου επιπέδου χρώματος μέσα σε μια εικόνα, ή την παραγωγή ειδικών εφέ ή μετατροπές βίντεο. Ένας όρος που περιγράφει την σκίαση είναι "ένα πρόγραμμα που μαθαίνει τον υπολογιστή πώς να ζωγραφίσει κάτι με έναν ειδικό και μοναδικό τρόπο".\\
Τα shader υπολογίζουν αποδόσεις εφέ σε υλικό γραφικών, με ένα μεγάλο βαθμό ευλυγισίας. Τα περισσότερα shader είναι σχεδιασμένα για χρήση σε μονάδα επεξεργασίας γραφικών (GPU), όμως αυτό δεν είναι αποκλειστική ανάγκη. Οι γλώσσες σκίασης, χρησιμοποιούνται συνήθως για να προγραμματίσουν την γραμμή σωλήνα απόδοσης της GPU. Η θέση, η απόχρωση, ο κορεσμός, η φωτεινότητα, και η αντίθεση όλων των στοιχείων, κορυφών, ή υφών, χρησιμοποιούνται για να αποδώσουν μια τελική εικόνα που μπορούμε να επεξεργαστούμε απευθείας με χρήση αλγορίθμων ορισμένων στα shader, είτε με αλλαγές απο εξωτερικές μεταβλητές που εισάγει το πρόγραμμα το οποίο καλεί τον shader.
Τα shader χρησιμοποιούνται πολύ στην κινηματογραφική επεξεργασία, στις εικόνες που αποδίδονται απο τον υπολογιστή, αλλα και σε παιχνίδια υπολογιστών, για να παράγουν ένα μεγάλο αριθμό απο εφέ. Εκτός απο τα απλά μοντέλα φωτισμού, μερικά απο τα πολύπλοκα εφέ επεξεργάζονται την εικόνα και προσθέτουν blur, light bloom, volumetric lightning, normal mapping, bokeh, cel shading,posterization,bump mapping,distortion,chroma keying, edge detection, motion detection, κ.α\\
Η σύγχρονη χρήση των shader ξεκίνησε απο την Pixar, τον Μάιο του 1988. Όσο οι μονάδες επεξεργασίας γραφικών εξελίσσονταν, οι γνωστές βιβλιοθήκες γραφικών ξεκίνησαν να υποστηρίζουν τα shader. Οι πρώτες κάρτες γραφικών υποστήριζαν μόνο pixel shader, αλλά σύντομα ακολούθησε η εισαγωγή των vertex shader όταν οι προγραμματιστές κατάλαβαν τις δυνατότητες τους. Τα shader γεωμετρίας εισήχθηκαν μόλις με το Direct3D 10 και το OpenGL 3.2
\subsubsection{Τύποι}
Υπάρχουν διάφοροι τύποι shader που χρησιμοποιούνται γενικά. Ενώ οι παλιές κάρτες γραφικών είχαν ξεχωριστό τρόπο επεξεργασίας στοιχείων για κάθε τύπο shader, οι καινούριες έχουν ενωμένους shader που έχουν την δυνατότητα να εκτελούν οποιονδήποτε τύπο shader. Αυτό επιτρέπει στις κάρτες γραφικών να έχουν πιο αποδοτική χρήση της επεξεργαστικής τους δύναμης.\cite{computeshaders-2}
\begin{itemize}
\item Vertex shaders - Μετατρέπουν κάθε θέση τρισδιάστατη στον εικονικό χώρο σε δισδιάστατη. Μπορούν να επεξεργαστούν ιδιότητες όπως η θέση, το χρώμα, και συντεταγμένες υφής, αλλα δεν μπορούν να δημιουργήσουν καινούρια vertices.
\item Pixel shaders - Υπολογίζουν το χρώμα και άλλες ιδιότητες ενός τεμαχίου. Οι απλές μορφές τους αποδίδουν ένα pixel εικόνας, ενώ οι πιο πολύπλοκες μορφές αποδίδουν πολλά. Στα τρισδιάστατα γραφικά, ένας pixel shader δεν μπορεί να παράγει πολύπλοκα εφέ, γιατί επεξεργάζεται μόνο ένα τεμάχιο, χωρίς κάποια γνώση της γεωμετρίας της οθόνης. Μπορούν όμως να εφαρμοστούν σε εφέ δύο διαστάσεων και χρησιμοποιούνται για επεξεργασία υφών. Για παράδειγμα είναι ο μόνος τύπος shader που μπορεί να λειτουργήσει σαν φίλτρο για μια ροή βίντεο.
\item Geometry shaders - Τα shader γεωμετρίας είναι σχετικά καινούριος τύπος shader. Μπορεί να δημιουργήσει γραφικά αρχικά στοιχεία όπως γραμμές, τρίγωνα. Τα shader γεωμετρίας εκτελούνται μετά απο τα vertex shader. Τυπικές χρήσεις των shader γεωμετρίας συμπεριλαμβάνουν γεωμετρική ψηφίδωση, εξώθηση σκιώδους όγκου, και απόδοση σε χάρτη κύβου. Για παράδειγμα, σε μια καμπυλωτή γραμμή, τα δεδομένα των στοιχείων εισάγονται σαν είσοδος στον shader, και αυτός αναλαμβάνει να δημιουργήσει αυτόματα επιπλέον γραμμές που δίνουν πιο εύστοχη απεικόνιση της καμπύλης.
\item Tesselation shaders - Με την κυκλοφορία του OpenGL 4.0 και του Direct3D 11, ένας καινούριος τύπος shader έχει προστεθεί που ονομάζεται shader ψηφίδωσης. Επιτρέπει στα αντικείμενα κοντά στην κάμερα να έχουν καλύτερη λεπτομέρεια, ενώ τα αντικείμενα που είναι πιο μακριά να έχουν μικρότερη λεπτομέρεια αλλά να μοιάζουν όμοια στην ποιότητα.
\item Compute shaders - Είναι ένας τύπος shader που χρησιμοποιείται αποκλειστικά για υπολογισμό αυθαίρετης πληροφορίας. Αν και μπορεί να χρησιμοποιηθεί για απόδοση, συνήθως χρησιμοποιείται για διεργασίες που δεν σχετίζονται άμεσα με σχεδιασμό τριγώνων και pixel.
\end{itemize}
\subsection{Εισαγωγή}
Τα shaders υπολογισμού λειτουργούν διαφορετικά από τις άλλες καταστάσεις shader. Όλες οι καταστάσεις shader έχουν προκαθορισμένου τύπου τιμές εισόδου, μερικές ενσωματωμένες και μερικές καθορισμένες από τον χρήστη. Η συχότητα στην οποία εκτελείται μια κατάσταση shader εξαρτάται από την φύση της κατάστασης. Για παράδειγμα, τα shader κορυφών εκτελούνται μια φορά για κάθε κορυφή. \\
Τα shader υπολογισμού λειτουργούν πολύ διαφορετικά. Ο "χώρος" στον οποίο ένα shader υπολογισμού λειτουργεί είναι αφηρημένος. Είναι στην κρίση του κάθε shader υπολογισμού να αποφασίσει τι σημαίνει αυτός ο "χώρος". Ο αριθμός των εκτελέσεων των shader υπολογισμού ορίζεται από την διεργασία που χρησιμοποιείται για να εκτελεστεί η υπολογιστική λειτουργία. Πιο σημαντικό από όλα, τα shader υπολογισμού δεν έχουν εισόδους καθορισμένες απο τον χρήστη και ούτε καμία έξοδο. Οι ενσωματωμένες είσοδοι ορίζουν μόνο το πού στον "χώρο" της εκτέλεσης βρίσκεται ένας συγκεκριμένος shader υπολογισμού.\\
Έτσι, αν κάποιος shader υπολογισμού πρέπει να πάρει κάποιες τιμές σαν είσοδο, είναι στην ευθύνη του shader να αποκτήσει τα δεδομένα, μέσω πρόσβασης υφών, αυθαίρετης φόρτωσης εικόνας, ή άλλες μορφές διεπαφής. Παρομοίως, αν ένας shader υπολογισμού υπολογίζει κάτι, θα πρέπει να το αποθηκεύσει σε μια εικόνα ή σε ένα block αποθήκευσης shader.
\subsection{Χώρος υπολογισμού}
Ο χώρος στον οποίο λειτουργεί ένα shader υπολογισμού είναι αφηρημένος. Υπάρχει η έννοια της ομάδας εργασίας. Είναι ο μικρότερος αριθμός από λειτουργίες υπολογισμού τις οποίες μπορεί να εκτελέσει ο χρήστης.
Ο αριθμός των ομάδων εργασίας με τον οποίο μια λειτουργία υπολογισμού εκτελείται, ορίζεται απο τον χρήστη όταν επικαλείται την λειτουργία υπολογισμού. Ο χώρος αυτών των ομάδων είναι τρισδιάστατος, οπότε έχει ένα αριθμό απο ομάδες "Χ","Υ","Ζ". Κάθε ένας απο αυτούς μπορεί να είναι 1, οπότε είναι δυνατή η εκτέλεση λειτουργιών υπολογισμού δυο ή και μίας διάστασης αντί για τρεις διαστάσεις. Αυτό είναι χρήσιμο για την επεξεργασία δεδομένων εικόνας ή γραμμικών πινάκων ενός συστήματος.\\
Η κάθε ομάδα εργασίας μπορεί να αποτελείται απο πολλούς shader υπολογισμού. Αυτό ονομάζεται τοπικό μέγεθος της ομάδας εργασίας. Κάθε shader υπολογισμού έχει τρισδιάστατο τοπικό μέγεθος (το οποίο μπορεί να είναι 1 επιτρέποντας δισδιάστατη ή μονοδιάστατη επεξεργασία). Αυτό ορίζει τον αριθμό των επικλήσεων ενός shader που θα εκτελεστούν σε κάθε ομάδα εργασίας. Για παράδειγμα, αν το τοπικό μέγεθος ενός shader υπολογισμού είναι (128, 1, 1) και εκτελεστεί με έναν αριθμό ομάδων εργασίας (16, 8 , 64) τότε θα έχουμε 1,048,576 ξεχωριστές επικλήσεις shader. Κάθε επίκληση θα έχει ένα σετ από εισόδους που αναγνωρίζουν μοναδικά την κάθε επίκληση. Αυτός ο διαχωρισμός είναι χρήσιμος για διάφορες μορφές συμπίεσης και αποσυμπίεσης εικόνας. Το τοπικό μέγεθος θα είναι το μέγεθος ενός block δεδομένων εικόνας (8x8 για παράδειγμα), ενώ ο αριθμός των ομάδων θα είναι το μέγεθος της εικόνας διαιρούμενο με το μέγεθος του block. Κάθε block κατεργάζεται σαν μια μοναδική ομάδα εργασίας.
\subsection{Υλοποίηση OpenGL}
\subsubsection{Αποστολή}
Ένα αντικείμενο προγράμματος μπορεί να έχει shader υπολογισμού μέσα του. Ο shader υπολογισμού συνδέεται με καταστάσεις shader μέσω κάποιων λειτουργιών απόδοσης. Υπάρχουν δύο λειτουργίες για να ξεκινήσουν οι διαδικασίες υπολογισμού. Χρησιμοποιούν οποιονδήποτε shader υπολογισμού είναι ενεργός. Οι λειτουργίες είναι οι εξής:
\begin{itemize}
\item void glDispatchCompute(GLuint num\_groups\_x, GLuint num\_groups\_u, GLuint num\_groups\_z); - Οι παράμετροι num\_groups\_* ορίζουν τον αριθμό των ομάδων εργασίας, σε τρεις διαστάσεις. Αυτοί οι αριθμοί δεν μπορούν να είναι μηδέν. Υπάρχουν όρια στον αριθμό των ομάδων εργασίας που μπορούν να αποσταλούν.
\item void glDispatchComputeIndirect(GLintptr indirect); - H παράμετρος indirect είναι το αντιστάθμισμα του buffer GL\_DISPATCH\_INDIRECT\_BUFFER. Ισχύουν τα ίδια όρια του αριθμού ομάδων εργασίας, όμως η αποστολή indirect παρακάμπτει τον έλεγχο λαθών του OpenGL. Έτσι, η αποστολή με εκτός ορίων μεγέθους ομάδας εργασίας, μπορεί να προκαλέσει προβλήματα ακόμα και πάγωμα του συστήματος.
\end{itemize}
\subsubsection{Είδοδοι}
Τα shader υπολογισμού  δεν μπορούν να έχουν μεταβλητές καθορισμένες απο τον χρήστη. Τα shader υπολογισμού έχουν τις παρακάτω ενσωματωμένες μεταβλητές εξόδου:
\begin{itemize}
\item in uvec3 gl\_NumWorkGroups; - Αυτή η μεταβλητή περιέχει τον αριθμό των ομάδων εργασίας για την λειτουργία αποστολής
\item in uvec3 gl\_WorkGroupID; - Αυτή η μεταβλητή περιέχει την ισχύουσα ομάδα εργασίας για την επίκληση του shader.
\item in uvec3 gl\_LocalInvocationID; - Αυτή η μεταβλητή περιέχει την ισχύουσα επίκληση του shader μέσα στην ομάδα εργασίας.
\item in uvec3 gl\_GlobalInvocationID; - Αυτή η μεταβλητή αναγνωρίζει μοναδικά την συγκεκριμένη επίκληση του shader υπολογισμού  ανάμεσα σε όλες τις επικλήσεις της κλήσης αποστολής υπολογισμού. Είναι μια συντόμευση για τον μαθηματικό υπολογισμό gl\_WorkGroupID * gl\_WorkGroupSize + gl\_LocalInvocationID;
\item in uint  gl\_LocalInvocationIndex;
\end{itemize}
\subsubsection{Τοπικό μέγεθος}
Το τοπικό μέγεθος ενός shader υπολογισμού ορίζεται απο τον shader, χρησιμοποιώντας μια ειδική δήλωση εισόδου: 
layout(local\_size\_x = X, local\_size\_y = Y, local\_size\_z = Z) in;
Αρχικά, τα τοπικά μεγέθη είναι 1, οπότε αν θέλουμε μονοδιάστατο ή δισδιάστατο χώρο ομάδων εργασίας, μπορούμε να ορίσουμε μόνο το Χ ή το Χ και το Υ. Πρέπει να είναι σταθερές εκφράσεις τιμής μεγαλύτερης του 0. Οι τιμές πρέπει να ορίζονται σε σχέση με τους περιορισμούς που υπάρχουν παρακάτω. Σε αντίθετη περίπτωση προκύπτουν λάθη. Το τοπικό μέγεθος είναι διαθέσιμο στον shader σαν σταθερά, οπότε δεν χρειάζεται να την ορίζουμε εμείς.
\begin{itemize}
\item const uvec3 gl\_WorkGroupSize;
\end{itemize}
\subsubsection{Περιορισμοί}
Ο αριθμός των ομάδων εργασίας που μπορούν να αποσταλούν, ορίζεται από την GL\_MAX\_COMPUTE\_WORK\_GROUP\_COUNT. Αυτή η σταθερά πρέπει να διαβαστεί απο την glGetIntegeri\_v, με τιμές ανάμεσα στο κλειστό όριο [0,2]. Προσπάθεια να καλέσουμε την glDispatchCompute με τιμές που ξεπερνούν το όριο είναι λάθος. Προσπάθεια κλήσης της glDispatchComputeIndirect είναι χειρότερα, μπορεί να διακόψει την λειτουργία του προγράμματος ακόμα και να παγώσει το σύστημα. Σημείωση: ο μικρότερος αριθμός αυτών των τιμών πρέπει να είναι 65535 σε όλους τους άξονες. Αυτό δίνει αρκετό χώρο για εργασία. Υπάρχουν όρια στο τοπικό μέγεθος επίσης. Συγκεκριμένα, υπάρχουν δύο τύποι περιορισμών. 
\begin{itemize}
\item Ο γενικός περιορισμός των διαστάσεων τοπικού μεγέθους, σε συνδυασμό με την GL\_MAX\_COMPUTE\_WORK\_GROUP\_SIZE, όπως και παραπάνω. Η διαφορά είναι οτι ο μικρότερος αριθμός των τιμών είναι πολύ μικρότερος. 1024 για τον Χ και τον Υ, και μόνο 64 για τον Ζ.
\item Ο αριθμός των επικλήσεων μέσα σε μια ομάδα εργασίας. Δηλαδή, το προϊόν των στοιχείων Χ,Υ,Ζ του τοπικού μεγέθους πρέπει να είναι μικρότερο απο GL\_MAX\_WORK\_GROUP\_INVOCATIONS. Η μικρότερη τιμή είναι 1024.
\end{itemize}
Υπάρχει ακόμα ο περιορισμός του ολικού μεγέθους αποθήκευσης για όλες τις κοινές μεταβλητές ενός shader υπολογισμού. Ορίζεται απο την GL\_MAX\_COMPUTE\_SHARED\_MEMORY\_SIZE, που αναφέρεται σε bytes. Η μικρότερη τιμή για το OpenGL είναι 32KB.
\subsection{Υλοποίηση DirectX}
Ένας shader υπολογισμού είναι μια κατάσταση shader υπολογισμού που εξαπλώνει το Microsoft Direct3D 11 πέρα απο τον προγραμματισμό γραφικών. Η τεχνολογία αυτή είναι γνωστή και ως τεχνολογία DirectCompute\cite{computeshaders-2}
\\Όπως όλα τα προγραμματιστικά shader (για παράδειγμα shader γεωμετρίας και κορυφών), ένα shader υπολογισμού είναι σχεδιασμένο να χρησιμοποιεί μια Γλώσσα Υψηλού Προγραμματισμού Shader(HLSL) για το DirectX. Η HLSL, χρησιμοποιείται για το DirectX και μας δίνει την δυνατότητα να δημιουργήσουμε C like shaders για την γραμμή σωλήνων Direct3D. Η HLSL δημιουργήθηκε ξεκινώντας απο το DirectX 9 για την κατασκευή προγραμματιζόμενων τρισδιάστατων γραμμής σωλήνα. Μας δίνει την δυνατότητα να προγραμματίσουμε την γραμμή σωλήνα με τον συνδυασμό οδηγιών assembly, οδηγιών HLSL, και δηλώσεις καθορισμένων λειτουργιών.\\
Ένα shader υπολογισμού προμηθεύει υψηλής ταχύτητας υπολογισμούς γενικού προγραμματισμού, και εκμεταλλεύεται τον μεγάλο αριθμό παράλληλων επεξεργαστών που βρίσκονται στην μονάδα επεξεργασίας γραφικών (GPU). Τα shader υπολογισμού προμηθεύει διαμοιρασμό μνήμης και συγχρονισμό νημάτων, για να επιτρέψει καλύτερες μεθόδους παράλληλου προγραμματισμού. Με την κλήση των μεθόδων ID3D11DeviceContext::Dispatch ή ID3D11DeviceContext::DispatchIndirect γίνεται η εκτέλεση εντολών σε ένα shader υπολογισμού, οι οποίες μπορούν να εκτελεστούν παράλληλα σε πολλά νήματα.