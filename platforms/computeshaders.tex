\section{Compute Shaders}
Τα shaders υπολογισμού είναι μια κατάσταση shader που χρησιμοποιείται σχεδόν αποκλειστικά για υπολογισμούς αυθαίρετης πληροφορίας. Αν και μπορεί να κάνει rendering, συνήθως χρησιμοποιείται για διεργασίες που δεν σχετίζονται άμεσα με σχεδιασμό τριγώνων και pixel.\cite{computeshaders-1}
\subsection{Εισαγωγή}
Τα shaders υπολογισμού λειτουργούν διαφορετικά από τις άλλες καταστάσεις shader. Όλες οι καταστάσεις shader έχουν προκαθορισμένου τύπου τιμές εισόδου, μερικές ενσωματωμένες και μερικές καθορισμένες από τον χρήστη. Η συχότητα στην οποία εκτελείται μια κατάσταση shader εξαρτάται απο την φύση της κατάστασης. Για παράδειγμα, τα shader κορυφών εκτελούνται μια φορά για κάθε κορυφή. \\
Τα shader υπολογισμού λειτουργούν πολύ διαφορετικά. Ο "χώρος" στον οποίο ένα shader υπολογισμού λειτουργεί είναι αφηρημένος. Είναι στην κρίση του κάθε shader υπολογισμού να αποφασίσει τι σημαίνει αυτός ο "χώρος". Ο αριθμός των εκτελέσεων των shader υπολογισμού ορίζεται απο την διεργασία που χρησιμοποιείται για να εκτελεστεί η υπολογιστική λειτουργία. Πιο σημαντικό απο όλα, τα shader υπολογισμού δεν έχουν εισόδους καθορισμένες απο τον χρήστη και ούτε καμία έξοδο. Οι ενσωματωμένες είσοδοι ορίζουν μόνο το πού στον "χώρο" της εκτέλεσης βρίσκεται ένας συγκεκριμένος shader υπολογισμού.\\
Έτσι, αν κάποιος shader υπολογισμού πρέπει να πάρει κάποιες τιμές σαν είσοδο, είναι στην ευθύνη του shader να αποκτήσει τα δεδομένα, μέσω πρόσβασης υφών, αυθαίρετης φόρτωσης εικόνας, ή άλλες μορφές διεπαφής. Παρομοίως, αν ένας shader υπολογισμού υπολογίζει κάτι, θα πρέπει να το αποθηκεύσει σε μια εικόνα ή σε ένα block αποθήκευσης shader.
\subsection{Χώρος υπολογισμού}
Ο χώρος στον οποίο λειτουργεί ένα shader υπολογισμού είναι αφηρημένος. Υπάρχει η έννοια της ομάδας εργασίας. Είναι ο μικρότερος αριθμός απο λειτουργίες υπολογισμού τις οποίες μπορεί να εκτελέσει ο χρήστης.
Ο αριθμός των ομάδων εργασίας με τον οποίο μια λειτουργία υπολογισμού εκτελείται, ορίζεται απο τον χρήστη όταν επικαλείται την λειτουργία υπολογισμού. Ο χώρος αυτών των ομάδων είναι τρισδιάστατος, οπότε έχει ένα αριθμό απο ομάδες "Χ","Υ","Ζ". Κάθε ένας απο αυτά μπορεί να είναι 1, οπότε είναι δυνατή η εκτέλεση λειτουργιών υπολογισμού δυο ή και μίας διάστασης αντί για τρεις διαστάσεις. Αυτό είναι χρήσιμο για την επεξεργασία δεδομένων εικόνας ή γραμμικών πινάκων ενός συστήματος.\\



\subsection{Υλοποίηση OpenGL}

\subsection{Υλοποίηση DirectX}
Ένας shader υπολογισμού είναι μια κατάσταση shader υπολογισμού που εξαπλώνει το Microsoft Direct3D 11 πέρα απο τον προγραμματισμό γραφικών. Η τεχνολογία αυτή είναι γνωστή και ως τεχνολογία DirectCompute\cite{computeshaders-4}
\\Όπως όλα τα προγραμματιστικά shader (για παράδειγμα shader γεωμετρίας και κορυφών), ένα shader υπολογισμού είναι σχεδιασμένο να χρησιμοποιεί μια Γλώσσα Υψηλού Προγραμματισμού Shader(HLSL) για το DirectX. Με το HLSL, μπορούμε να δημιουργούμε προγραμματιστικά shader για την γραμμή σωλήνων Direct3D. Ένα shader υπολογισμού προμηθεύει υψηλής ταχύτητας υπολογισμούς γενικού προγραμματισμού, και εκμεταλλεύεται τον μεγάλο αριθμό παράλληλων επεξεργαστών που βρίσκονται στην μονάδα επεξεργασίας γραφικών (GPU). Τα shader υπολογισμού προμηθεύει διαμοιρασμό μνήμης και συγχρονισμό νημάτων, για να επιτρέψει καλύτερες μεθόδους παράλληλου προγραμματισμού. Με την κλήση των μεθόδων ID3D11DeviceContext::Dispatch ή ID3D11DeviceContext::DispatchIndirect γίνεται η εκτέλεση εντολών σε ένα shader υπολογισμού, οι οποίες μπορούν να εκτελεστούν παράλληλα σε πολλά νήματα.