\chapter{Ιστορία}
Παραδοσιακά, ο αγωγός απόδοσης των γραφικών (3D pipeline), αποτελείται από τις καταστάσεις μετατροπή και φωτισμό (transform and lighting), συναρμολόγηση αρχέγονων (primitive assembly), pixel transformation, και σκίαση (shading). Οι πρώτες GPU είχαν όλες τις λειτουργίες που χρειάζονται για να εκτελεστεί ο αγωγός απόδοσης, αλλά με τον καιρό όλο και περισσότερες καταστάσεις έγιναν δυνατό να προγραμματιστούν με την έλευση ειδικών επεξεργαστών, όπως επεξεργαστές κορυφών (vertex processors) και τεμάχια επεξεργαστών(fragment processors), που κατέστησαν κάποιες λειτουργίες πιο ευέλικτες.

Όταν οι τιμές συνέχισαν να πέφτουν ενώ η υπολογιστική δύναμη αυξανόταν, η ερευνητική κοινότητα σκέφτηκε τρόπους να αξιοποιηθεί αυτή η δύναμη για τον υπολογισμό δύσκολων λειτουργιών. Όμως, καθώς η δυνατότητα των επεξεργαστών ήταν περιορισμένη και η διεπαφή προγραμματισμού εφαρμογών (API) των οδηγών γραφικών ήταν σχεδιασμένη για να υλοποιεί συγκεκριμένα τον αγωγό απόδοσης, έπρεπε να ληφθούν υπόψιν πολλές παράμετροι.

Για παράδειγμα, όλα τα δεδομένα έπρεπε να κωδικοποιηθούν σε υφές ως πίνακες δυο διαστάσεων που αναπαριστούν pixel, με περιεχόμενο τιμές χρωμάτων και κάποιο κανάλι alpha για την διαφάνεια. Επιπλέον, οι υφές είναι αντικείμενα μόνο προσπελάσιμα, και δεν επαναγράφονταν, κάτι που ανάγκαζε τους προγραμματιστές να αποθηκεύουν κάθε φορά καινούρια υφή με τις αλλαγές. Τέλος, οι περισσότερες GPU υποστήριζαν μόνο λειτουργίες κινητής υποδιαστολής απλής ακρίβειας, αναγκάζοντας τους προγραμματιστές να προσομοιώνουν λογικές λειτουργίες.

Αυτοί οι περιορισμοί, ήταν ο μεγαλύτερος λόγος που ώθησε τους κατασκευαστές GPU (AMD,NVIDIA,INTEL), να δημιουργήσουν προγραμματιστικές διεπαφές ειδικές για την κοινότητα του GPGPU και να εξελίξουν τις συσκευές τους για καλύτερη υποστήριξη.

Το πεδίο του προγραμματισμού γενικής χρήσης έχει αναπτυχθεί με ταχύτατους ρυθμούς τα τελευταία χρόνια, έτσι ώστε τώρα υπάρχουν αρκετές υλοποιήσεις για τον προγραμματισμό των μονάδων επεξεργασίας γραφικών. Πρόσφατα, έχουν γίνει προσπάθειες δημιουργίας προτύπων.
Ο προγραμματισμός των GPUs αναπτύχθηκε όταν το CUDA (από τα αρχικά: Compute Unified Device Architecture) και το AMD FireStream κατέφθασαν στο τέλος του 2006. Αυτές οι διεπαφές και οι γλώσσες, σχεδιάστηκαν από τις εταιρίες κατασκευής των GPUs σε πολύ κοντινή σχέση με το υλικό, το οποίο αποτέλεσε μεγάλο βήμα προς ένα πιο εύχρηστο, ταιριαστό και μελλοντικά-ασφαλές προγραμματιστικό μοντέλο.

Η ανοιχτή γλώσσα προγραμματισμού (OpenCL) δημιουργήθηκε για να παρέχει ένα γενικό API ετερογενή υπολογισμού σε διάφορες μορφές παράλληλων συσκευών, συμπεριλαμβανομένου μονάδων επεξεργασίας γραφικών, πολυπύρηνων κεντρικών μονάδων επεξεργασίας, κ.α
\section{Μέλλον}
Οι πρόσφατες δραστηριότητες των μεγάλων κατασκευαστών μας δείχνουν ότι τα μελλοντικά σχέδια των μικροεπεξεργαστών και μεγάλων HPC συστημάτων θα είναι υβριδικά/ετερογενούς φύσης. Αυτά τα συστήματα θα βασίζονται στην ενσωμάτωση δύο τύπων εξαρτημάτων:
\begin{itemize}
\item Τεχνολογία πολυπύρηνων CPU: ο αριθμός των πυρήνων θα συνεχίσει να αυξάνεται λόγω της επιθυμίας να ενσωματώσουμε περισσότερα εξαρτήματα σε ένα τσιπ.
\item Ειδικού τύπου υλικό και μαζικά παράλληλους επιταχυντές: Για παράδειγμα, οι GPUs υπερτερούν των CPUs σε απόδοση κινητής υποδιαστολής, τα τελευταία χρόνια. Επίσης ο προγραμματισμός σε αυτές έχει γίνει εύκολος, αν όχι ευκολότερος, από ότι στις CPUs
\end{itemize}
Η σχετική ισορροπία στα μελλοντικά σχέδια δεν είναι ξεκάθαρη και μπορεί να αλλάξει με την πάροδο του χρόνου. Δεν υπάρχει καμία αμφιβολία ότι οι μελλοντικές γενιές των υπολογιστικών συστημάτων, από τους φορητούς υπολογιστές μέχρι και τους υπερ-υπολογιστές θα αποτελείται από μια σύσταση ετερογενών συστημάτων. 
\section{Προβλήματα}
Τα προβλήματα και οι προκλήσεις για τους προγραμματιστές στο καινούριο περιβάλλον των υβριδικών συστημάτων, είναι υπαρκτά. Κρίσιμα τμήματα του λογισμικού ήδη δυσκολεύονται να προλάβουν τον ρυθμό των αλλαγών. Σε μερικές περιπτώσεις, η απόδοση δεν είναι ανάλογη του αριθμού των πυρήνων, γιατί ένα μεγάλο μέρος του χρόνου ξοδεύεται στην μετακίνηση των δεδομένων παρά στους υπολογισμούς. Σε άλλες περιπτώσεις, το βελτιστοποιημένο λογισμικό για το συγκεκριμένο υλικό, παραδίδεται χρόνια μετά από την παράδοση του υλικού, και έτσι είναι απαρχαιωμένο όταν παραδοθεί. Και σε άλλες περιπτώσεις, όπως σε μερικές πρόσφατες υλοποιήσεις GPU, το λογισμικό δεν εκτελείται καθόλου γιατί το προγραμματιστικό περιβάλλον έχει αλλάξει υπερβολικά.
