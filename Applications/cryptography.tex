\section{Κρυπτογράφηση}
Στο πεδίο της ασύμμετρης κρυπτογράφησης, η ασφάλεια όλων των πρακτικών κρυπτοσυστημάτων βασίζεται στην δυσκολία υπολογισμού προβλημάτων, εξαρτημένη από την επιλογή των παραμέτρων. Με την όποια αύξηση των παραμέτρων όμως (συνήθως στο εύρος 1024-4096 bits), οι υπολογισμοί γίνονται όλο και πιο απαιτητικοί για τον εκάστοτε επεξεργαστή. Σε σύγχρονο υλικό, ο υπολογισμός μιας μονής εντολής κρυπτογράφησης δεν είναι κρίσιμος, όμως σε ένα σύστημα επικοινωνίας πολλών-προς-ένα, για παράδειγμα ένας κεντρικός server στο κέντρο δεδομένων μιας εταιρίας, μπορεί να αντιμετωπίσει ταυτόχρονα εκατοντάδες η και χιλιάδες ταυτόχρονες συνδέσεις και εντολές κρυπτογράφησης.

Ως αποτέλεσμα, η πιο συνήθης λύση για ένα τέτοιο σενάριο είναι η χρήση καρτών επιτάχυνσης κρυπτογράφησης. Λόγω της μικρής αγοράς, η τιμή τους φτάνει συνήθως αρκετά χιλιάδες ευρώ η δολάρια.
Τελευταία, η ερευνητική κοινότητα έχει αρχίσει να εξερευνά τεχνικές για επιτάχυνση των αλγορίθμων κρυπτογράφησης με χρήση της GPU.  

\begin{figure}[h]
\centering
\includegraphics[scale=0.50]{crypto-1}
\caption{Μηχανή κρυπτογράφησης German Lorenz, χρησιμοποιήθηκε στον δεύτερο παγκόσμιο πόλεμο για να κρυπτογραφεί μηνύματα για προσωπικό πολύ υψηλής σημασίας}
\end{figure}

\subsection{Κρυπτογραφία συμμετρικού κλειδιού}

\subsection{Κρυπτογραφία δημοσίου κλειδιού}
Τα κρυπτοσυστήματα συμμετρικών κλειδιών χρησιμοποιούν το ίδιο κλειδί για την κρυπτογράφηση και την αποκρυπτογράφηση ενός μηνύματος, αν και ένα μήνυμα ή ομάδα μηνυμάτων μπορεί να έχουν διαφορετικό κλειδί από τα άλλα. Ένα σημαντικό μειονέκτημα των συμμετρικών κρυπτογραφημάτων είναι η διαχείριση κλειδιών, ώστε αυτά να χρησιμοποιηθούν με ασφάλεια. 

\begin{figure}[h]
\centering
\includegraphics[scale=0.50]{crypto-2}
\caption{Γαλλική μηχανή κρυπτογράφησης σε σχήμα βιβλίου του 16ου-αιώνα}
\end{figure}

\begin{figure}[h]
\centering
\includegraphics[scale=0.50]{crypto-3}
\caption{Η μηχανή κρυπτογράφησης Enigma, χρησιμοποιήθηκε από τον Γερμανικό στρατό και τις πολιτικές αρχές από τα τέλη του 1920 μέχρι και τον δεύτερο παγκόσμιο πόλεμο, παρείχε ένα πολύπλοκο ηλεκτρο-μηχανικό πολύ-αλφαβητικό κρυπτογράφημα. H αποκρυπτογράφηση του αλγόριθμου αποδείχτηκε μεγάλης σημασίας για την νίκη των συμμάχων.}
\end{figure}



\subsection{Hashcat}
Για λόγους έρευνας της εργασίας επιλέχτηκε το πρόγραμμα oclHashcat ώστε να μελετήσουμε 



\subsection{Bitcoin Mining}
