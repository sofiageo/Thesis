\section{Κρυπτογράφηση}
Στο πεδίο της ασύμμετρης κρυπτογράφησης, η ασφάλεια όλων των πρακτικών κρυπτοσυστημάτων βασίζεται στην δυσκολία υπολογισμού προβλημάτων, εξαρτημένη από την επιλογή των παραμέτρων. Με την όποια αύξηση των παραμέτρων όμως (συνήθως στο εύρος 1024-4096 bits), οι υπολογισμοί γίνονται όλο και πιο απαιτητικοί για τον εκάστοτε επεξεργαστή. Σε σύγχρονο υλικό, ο υπολογισμός μιας μονής εντολής κρυπτογράφησης δεν είναι κρίσιμος, όμως σε ένα σύστημα επικοινωνίας πολλών-προς-ένα, για παράδειγμα ένας κεντρικός server στο κέντρο δεδομένων μιας εταιρίας, μπορεί να αντιμετωπίσει ταυτόχρονα εκατοντάδες η και χιλιάδες ταυτόχρονες συνδέσεις και εντολές κρυπτογράφησης.

Ως αποτέλεσμα, η πιο συνήθης λύση για ένα τέτοιο σενάριο είναι η χρήση καρτών επιτάχυνσης κρυπτογράφησης. Λόγω της μικρής αγοράς, η τιμή τους φτάνει συνήθως αρκετά χιλιάδες ευρώ η δολάρια.
Τελευταία, η ερευνητική κοινότητα έχει αρχίσει να εξερευνά τεχνικές για επιτάχυνση των αλγορίθμων κρυπτογράφησης με χρήση της GPU. \cite{cryptography-1}

\begin{figure}[h]
\centering
\includegraphics[scale=0.50]{crypto-1}
\caption{Μηχανή κρυπτογράφησης German Lorenz, χρησιμοποιήθηκε στον δεύτερο παγκόσμιο πόλεμο για να κρυπτογραφεί μηνύματα για προσωπικό πολύ υψηλής σημασίας\cite{figure-7}}
\end{figure}

\subsection{Κρυπτογράφηση συμμετρικού κλειδιού}
Η κρυπτογραφία συμμετρικού κλειδιού αναφέρεται στις κρυπτογραφικές μεθόδους στις οποίες ο αποστολέας και ο αποδέκτης μοιράζονται το ίδιο κλειδί (ή σε πιο σπάνια περίπτωση, όταν τα κλειδιά είναι διαφορετικά, αλλά βρίσκονται εύκολα μέσω απλού υπολογισμού). Αυτή ήταν η μόνη μορφή κρυπτογραφίας γνωστή μέχρι το 1976.
Οι κρυπτογράφοι συμμετρικών κλειδιών υλοποιούνται σαν block ή κρυπτογράφοι ροής. Ένας κρυπτογράφος block κρυπτογραφεί τα δεδομένα σε blocks αντί για τον κάθε χαρακτήρα ξεχωριστά, δηλαδή με τον τρόπο που λειτουργεί ο κρυπτογράφος ροής.\cite{cryptography-2}


\begin{figure}[h]
\centering
\includegraphics[scale=0.50]{crypto-2}
\caption{Γαλλική μηχανή κρυπτογράφησης σε σχήμα βιβλίου του 16ου-αιώνα\cite{figure-8}}
\end{figure}

Το πρότυπο κρυπτογράφησης δεδομένων (DES) και το ανεπτυγμένο πρότυπο κρυπτογράφησης (AES) είναι κρυπτογράφοι block σχεδιασμένοι από την Αμερικάνικη κυβέρνηση. Αν και έχει σταματήσει να υποστηρίζεται σαν επίσημο πρότυπο, η χρήση του DES είναι ακόμα αρκετά δημοφιλής. Χρησιμοποιείται για μια μεγάλη σειρά από εφαρμογές, όπως κρυπτογράφηση ATM, ασφαλής απομακρυσμένη πρόσβαση, ασφαλής επικοινωνία ηλεκτρονικού ταχυδρομείου, κ.α.

Οι κρυπτογράφοι ροής, σε αντίθεση με τους block, δημιουργούν μια μεγάλη σειρά από υλικά κλειδιών, που συνδυάζονται με το απλό κείμενο χαρακτήρα προς χαρακτήρα. Σε έναν κρυπτογράφο ροής, το αποτέλεσμα παράγεται βασισμένο σε μια κρυμμένη κατάσταση που αλλάζει όσο λειτουργεί ο κρυπτογράφος. Αυτή η εσωτερική κατάσταση αρχικοποιείται με την χρήση ενός κρυμμένου κλειδιού. Παράδειγμα γνωστού κρυπτογράφου ροής είναι ο RC4.\cite{cryptography-3}

Οι κρυπτογραφικές διαδικασίες hash είναι ένας τρίτος τύπος αλγόριθμου κρυπτογράφησης. Δέχονται σαν είσοδο οποιαδήποτε μήκος κειμένου, και δίνουν αποτέλεσμα ένα μικρό κείμενο συγκεκριμένου μήκους που μπορεί να χρησιμοποιηθεί σαν ψηφιακή ταυτότητα. Όσον αφορά τις καλές διαδικασίες hash, ένας εισβολέας δεν μπορεί να βρει δύο μηνύματα που παράγουν το ίδιο hash. \cite{cryptography-4}To MD5 για παράδειγμα, ενώ χρησιμοποιείται εδώ και πολλά χρόνια, έχει αποδειχτεί ότι είναι εύκολο να παραβιαστεί. Το SHA-1 είναι περισσότερο διαδεδομένο ως ασφαλής διαδικασία hash κρυπτογράφησης, αλλά αναλυτές έχουν ανακαλύψει επιθέσεις για την παραβίαση του. Το 2012, μια καινούρια διαδικασία κρυπτογράφησης hash ανακαλύφθηκε, η SHA-3, η οποία χρησιμοποιεί τον αλγόριθμο Keccak για την υλοποίηση της. Σε παρακάτω υπο-ενότητα θα μελετήσουμε την χρήση προγράμματος για παραβίαση των διαδικασιών κρυπτογράφησης hash.\cite{cryptography-5}
\subsection{Κρυπτογράφηση δημοσίου κλειδιού}
Τα κρυπτοσυστήματα συμμετρικών κλειδιών χρησιμοποιούν το ίδιο κλειδί για την κρυπτογράφηση και την αποκρυπτογράφηση ενός μηνύματος, αν και ένα μήνυμα ή ομάδα μηνυμάτων μπορεί να έχουν διαφορετικό κλειδί από τα άλλα. Ένα σημαντικό μειονέκτημα των συμμετρικών κρυπτογραφημάτων είναι η διαχείριση κλειδιών, ώστε αυτά να χρησιμοποιηθούν με ασφάλεια. 

\begin{figure}[h]
\centering
\includegraphics[width=\linewidth]{crypto-3}
\caption{Η μηχανή κρυπτογράφησης Enigma, χρησιμοποιήθηκε από τον Γερμανικό στρατό και τις πολιτικές αρχές από τα τέλη του 1920 μέχρι και τον δεύτερο παγκόσμιο πόλεμο, παρείχε ένα πολύπλοκο ηλεκτρο-μηχανικό πολύ-αλφαβητικό κρυπτογράφημα. H αποκρυπτογράφηση του αλγόριθμου αποδείχτηκε μεγάλης σημασίας για την νίκη των συμμάχων.\cite{figure-9}}
\end{figure}

\subsection{Hashcat}
Για λόγους έρευνας της εργασίας επιλέχτηκε το πρόγραμμα oclHashcat ώστε να μελετήσουμε την απόδοση των κρυπτογραφικών προγραμμάτων με χρήση GPGPU. Το Hashcat είναι το πιο γρήγορο πρόγραμμα επαναφοράς κωδικών. Είναι ελεύθερο, αν και κλειστού κώδικα. Υπάρχουν εκδόσεις για Linux,OSX,και Windows, και υλοποιήσεις για CPU ή GPU. Το hashcat υποστηρίζει μεγάλο αριθμό από hashing αλγορίθμους, συμπεριλαμβανομένου MD4,MD5,SHA, Unix Crypt, κ.α
\subsubsection{Υλοποιήσεις}
Το Hashcat διανέμεται σε δύο εκδόσεις
\begin{itemize}
\item Hashcat - Ένα εργαλείο επαναφοράς κωδικών για CPU
\item oclHashcat - Επαναφορά κωδικών με χρήση GPGPU
\end{itemize}

Πολλοί από τους αλγόριθμους που υποστηρίζει το hashcat μπορούν να βρεθούν σε μικρότερο χρόνο με την χρήση της GPU-επιτάχυνσης του oclHashcat(όπως MD5,SHA1, κ.α) Όμως, δεν επιταχύνονται όλοι οι αλγόριθμοι απο την χρήση GPU. Το Bcrypt είναι ένα τέτοιο παράδειγμα. Λόγω παραγόντων όπως διακλαδώσεις δεδομένων, σειριακοποίηση, μνήμη, κ.α, το oclHashcat δεν είναι απόλυτος αντικαταστάτης του Hashcat.

\subsubsection{Επιθέσεις}
Το Hashcat υποστηρίζει πολλούς τύπους επιθέσεων για την επαναφορά πολύπλοκων κωδικών από τον χώρο κλειδιού ενός hash. Αυτοί οι τύποι περιλαμβάνουν:
\begin{itemize}
\item Επίθεση Brute-Force
\item Επίθεση Combinator
\item Επίθεση λεξικού
\item Επίθεση αποτυπώματος
\item Υβριδική επίθεση
\item Επίθεση μάσκας
\item Επίθεση Permutation
\item Επίθεση βασισμένη σε κανόνες
\item Επίθεση με χρήση πίνακα
\item Επίθεση Toggle-Case
\end{itemize}


\subsection{Κρυπτο-νομίσματα}
Τα κρυπτονομίσματα είναι μια μορφή συναλλαγής με χρήση κρυπτογραφίας για ασφάλεια και τον έλεγχο της δημιουργίας καινούριων νομισμάτων. Το πρώτο κρυπτο-νόμισμα που δημιουργήθηκε ήταν το Bitcoin το 2009. Από τότε, πολλά κρυπτονομίσματα έχουν δημιουργηθεί. Ένα χαρακτηριστικό τους είναι ότι δεν υπάρχει κεντρικός έλεγχος, σε αντίθεση με άλλα συστήματα ηλεκτρονικού χρήματος όπως το Paypal. Ακόμα ένα χαρακτηριστικό είναι ότι οι συναλλαγές καταγράφονται δημόσια, για παράδειγμα στο Bitcoin, οι συναλλαγές καταγράφονται στην block αλυσίδα. \cite{cryptography-6}

\begin{figure}[h]
\centering
\includegraphics[width=0.3\linewidth]{bitcoin}
\caption{Λογότυπο Bitcoin, το πρώτο και το πιο επιτυχημένο κρυπτο-νόμισμα\cite{figure-10}}
\end{figure}

Η δημιουργία κρυπτονομισμάτων γίνεται μέσω ειδικών προγραμμάτων που υπολογίζουν συγκεκριμένα hashes. Τα κρυπτονομίσματα χρησιμοποιούν έναν αλγόριθμο "Proof of work" ώστε να γνωρίζουν ότι ο χρήστης έχει εξορύξει τα νομίσματα που αυτός αναφέρει. (π.χ Scrypt, SHA-256, κ.α). Η εξόρυξη ενός block είναι μια διαδικασία που μοιάζει πολύ με την επαναφορά κωδικών. Για αυτό οι GPUs, είναι πιο αποδοτικές απο τις CPUs - όταν μια τυπική CPU έχει μέχρι 8 πυρήνες, μια GPU μπορεί να έχει εκαντοντάδες, όπου η καθεμία απο αυτές υπολογίζει ένα διαφορετικό "hash".

Υπάρχουν δύο τρόποι για εξόρυξη εικονικών νομισμάτων.
\begin{itemize}
\item Ατομική εξόρυξη - Η ατομική προσπάθεια για εξόρυξη νομισμάτων με χρήση μιας η περισσότερων GPU
\item Εξόρυξη κοινοπραξίας - Η προσπάθεια από πολλούς ανθρώπους να εξορύξουν το ίδιο block, διαιρώντας τα κέρδη.
\end{itemize}
Οι ταχύτητες της εξόρυξης μετρώνται σε KH/s (kilohashes ανα δευτερόλεπτο) και MH/s (megahashes ανα δευτερόλεπτο). Για παράδειγμα, μια Radeon 4870 εξορύσσει ένα νόμισμα SHA-256 με ταχύτητα 80-100 MH/s, ενώ ένα νόμισμα Scrypt με ταχύτητα μόλις 130-140 KH/s.

H χρήση των GPUs για εξόρυξη εικονικών νομισμάτων, οδήγησε την αγορά των καρτών γραφικών σε αύξηση των τιμών, αφού για πολύ καιρό ήταν μια αποδοτική λύση σε αντίθεση με την χρήση CPUs. Πρόσφατα όμως, με την παρουσίαση εξειδικευμένων καρτών (ASICS - Application Specific Integrated Circuit), η χρήση των GPUs για εξόρυξη κρυπτο-νομισμάτων άρχισε να μειώνεται, αφού η απόδοση τους είναι έως και 1000 φορές μικρότερη. Αυτό ήταν επίσης ένας λόγος για τον απλό χρήστη να απομακρυνθεί από την εξόρυξη νομισμάτων, αφού δημιουργήθηκαν μικρότερες κοινοπραξίες που απαιτούν μεγαλύτερη υπολογιστική δύναμη.