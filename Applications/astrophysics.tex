\section{Αστροφυσική}
Οι πρώτες καταγραφές της θεωρητικής αστρονομίας χρονολογούνται το 1550-1292 π.Χ Οι υπολογισμοί που βρέθηκαν σχηματισμένοι σε αιγυπτιακούς τάφους δείχνουν ότι υπήρχαν τεχνικές για την αναγνώριση και την καταγραφή προτύπων στον ουρανό, ένα πεδίο της αστρονομίας που είναι χρήσιμο ακόμα και σήμερα για την καταγραφή και χαρτογράφηση. Μπορεί οι σημερινοί ερευνητές να μην χρησιμοποιούν τα αρχαία εργαλεία, αλλά η ανάπτυξη τους έγινε σταδιακά. \\
Το μέγεθος των τηλεσκοπίων χρειάστηκε 400 χρόνια για να μεγαλώσει από 1 τετραγωνικό μέτρο σε 110 τετραγωνικά μέτρα. Οι ψηφιακοί υπολογιστές εμφανίστηκαν στο τέλος του 1940 με υπολογιστική ταχύτητα περίπου 100 floating point λειτουργιών το δευτερόλεπτο (FLOPS), και εξελίχθηκαν σε περίπου $3*10^{16}$ FLOPS σε λιγότερο από 65 χρόνια. Αυτή η επανάσταση των υπολογιστών συνεχίζεται ακόμα και σήμερα, και έχει οδηγήσει σε ένα καινούριο κομμάτι έρευνας στο οποίο οι εγκαταστάσεις δεν βρίσκονται στο ψηλότερο βουνό του κόσμου, αλλά στο διπλανό δωμάτιο. Οι Αστρονόμοι κατανόησαν γρήγορα ότι μπορούν να χρησιμοποιήσουν τους υπολογιστές για να καταγράψουν, αναλύσουν, αρχειοθετήσουν, τις τεράστιες ποσότητες πληροφοριών που καταγράφονται από τις εκστρατείες παρατηρητών. \\
\begin{figure}[h]
\centering
\includegraphics[width=\linewidth]{astrophysics}
\caption{Αστροφυσική}
\end{figure}
Την μεγαλύτερη όμως επίδραση στον τρόπο με τον οποίο αντιμετωπίζουν οι αστρονόμοι τα αναπάντητα ερωτήματα τους, έχει το πεδίο της εξομοίωσης. Με την χρήση των υπολογιστών, είναι δυνατόν να μελετήσουμε την λειτουργία του Διαγαλαξιακούς κενού, την φυσική των μαύρων τρυπών, κ.α