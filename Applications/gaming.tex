\section{Ψυχαγωγία}
\subsection{Εισαγωγή}
Μια μηχανή φυσικής είναι ένα πρόγραμμα υπολογιστή που παρέχει εξομοίωση συγκεκριμένων συστημάτων φυσικής, όπως δυναμική άκαμπτων σωμάτων, ανίχνευση σύγκρουσης, δυναμική υγρών, για χρήση σε πεδία όπως τα γραφικά υπολογιστών, παιχνίδια, κινούμενα σχέδια, ταινίες. Μια από τις κύριες χρήσεις τους είναι στα παιχνίδια υπολογιστών, στην οποία περίπτωση η εξομοίωση γίνεται σε πραγματικό χρόνο. O όρος χρησιμοποιείται γενικότερα για να περιγράψει οποιοδήποτε σύστημα λογισμικού που εξομοιώνει φυσικά φαινόμενα, όπως επιστημονικές εξομοιώσεις υψηλής απόδοσης.

Οι μηχανές φυσικής έχουν χρησιμοποιηθεί αρκετά στους υπερ-υπολογιστές από την δεκαετία του '80 για να εκτελέσουν μοντελοποίηση δυναμικών υγρών, όπου αναθέτουμε διανύσματα ισχύος σε σωματίδια, για να δείξουμε την κυκλοφορία. Λόγω των υψηλών απαιτήσεων σε ταχύτητα και ακρίβεια, ειδικοί επεξεργαστές δημιουργήθηκαν που είναι γνωστοί ως επεξεργαστές διανυσμάτων για να επιταχύνουν τους υπολογισμούς. Οι τεχνικές μπορούν να χρησιμοποιηθούν για να μοντελοποιήσουν πρότυπα καιρού για την πρόβλεψη καιρού, δεδομένα σήραγγας αέρα για σχεδιασμό αεροπλάνων και υποβρυχίων, και ανάλυση θερμικής απόδοσης για καλύτερο σχεδιασμό ψηκτρών για επεξεργαστές. Φυσικά μεγάλο ρόλο παίζει η ακρίβεια των υπολογισμών, αφού μικρές αποκλίσεις μπορούν να αλλάξουν δραστικά τα αποτελέσματα των υπολογισμών. Οι κατασκευαστές ελαστικών χρησιμοποιούν εξομοιώσεις φυσικής για να μελετήσουν πώς οι καινούριοι τύποι ελαστικών θα αποδίδουν σε συνθήκες βρεγμένου και στεγνού οδοστρώματος, χρησιμοποιώντας καινούρια υλικά και κάτω από διαφορετικές συνθήκες βάρους.\cite{physics-1}

Υπάρχουν γενικά δύο τύποι μηχανών φυσικής. Οι πραγματικού χρόνου, και οι υψηλής ακρίβειας. Οι υψηλής ακρίβειας απαιτούν περισσότερη υπολογιστική δύναμη για να υπολογίσουν φυσικά φαινόμενα με ακρίβεια και χρησιμοποιούνται συνήθως από επιστήμονες αλλά και σε κινούμενα σχέδια. Οι πραγματικού χρόνου - χρησιμοποιούνται σε παιχνίδια υπολογιστών και σε άλλες μορφές διαδραστικού υπολογισμού - χρησιμοποιούν απλοποιημένους υπολογισμούς με μειωμένη ακρίβεια ώστε να επιτρέπουν στο παιχνίδι να αντιδράει σε αποδεκτό ρυθμό για την εμπειρία χρήσης.\cite{physics-2}


\subsection{Παιχνίδια}
\subsubsection{Γενικά}
Στα περισσότερα παιχνίδια, η ταχύτητα των επεξεργαστών και η εμπειρία χρήσης είναι πιο σημαντικά από την ακρίβεια της εξομοίωσης. Αυτό μας οδηγεί σε σχεδιασμούς μηχανών φυσικής που παράγουν αποτελέσματα σε πραγματικό χρόνο αλλά αντιγράφουν φυσικά φαινόμενα μόνο για απλές περιπτώσεις. Τις περισσότερες φορές, η εξομοίωση είναι σχεδιασμένη να παρέχει μια φαινομενικά σωστή εκτίμηση, παρά απόλυτη ακρίβεια. Όμως μερικές μηχανές, απαιτούν μεγαλύτερη ακρίβεια σε σκηνές μάχης ή σε παιχνίδια τύπου παζλ. Οι κινήσεις χαρακτήρων στο παρελθόν χρησιμοποιούσαν φυσική άκαμπτων σωμάτων γιατί είναι γρηγορότερο και πιο εύκολο να υπολογιστεί, όμως τα τελευταία χρόνια τα παιχνίδια και οι ταινίες έχουν αρχίσει να χρησιμοποιούν φυσική μαλακών σωμάτων. Αυτού του τύπου η εξομοίωση χρησιμοποιείται επίσης για εφέ σωματιδίων, κίνηση υγρών και υφασμάτων. Μια μορφή εξομοίωσης δυναμικής υγρών χρησιμοποιείται για να εξομοιώσει νερό και άλλα υγρά αλλά και την ροή της φωτιάς και του καπνού στον αέρα.\cite{physics-3}

\begin{figure}[h]
\centering
\includegraphics[width=0.5\linewidth]{havok-logo}
\caption{Λογότυπο της μηχανή φυσικής havok}
\end{figure}

\subsubsection{Ανίχνευση σύγκρουσης}
Η ανίχνευση σύγκρουσης συνήθως αναφέρεται στο υπολογιστικό πρόβλημα της ανίχνευσης της διασταύρωσης δύο η περισσότερων αντικειμένων. Αν και το θέμα έχει σχέση περισσότερο με την χρήση του στα παιχνίδια και σε άλλες εξομοιώσεις φυσικής, έχει και χρήσεις στην ρομποτική. Εκτός από την ανίχνευση του αν δύο αντικείμενα έχουν συγκρουστεί, τα συστήματα ανίχνευσης μπορούν να υπολογίσουν τον χρόνο της σύγκρουσης(Time Of Impact), και να αναφέρουν ένα σύνολο από σημεία διασταύρωσης. Η αντίδραση σύγκρουσης είναι η εξομοίωση του τι συμβαίνει όταν ανιχνευθεί μια σύγκρουση.
\subsection{Υλοποιήσεις}
\subsubsection{Μονάδα Επεξεργασίας Φυσικής}
Η μονάδα επεξεργασίας φυσικής (PPU) είναι ένας μικροεπεξεργαστής αποκλειστικά σχεδιασμένος για να χειρίζεται τους υπολογισμούς φυσικής, ειδικά σε μηχανές φυσικής των παιχνιδιών υπολογιστών. Η ιδέα είναι ότι αυτοί οι ειδικοί επεξεργαστές ελαφρύνουν το φόρτο εργασίας των CPUs, όπως μια κάρτα γραφικών εκτελεί υπολογισμούς γραφικών. Ο όρος αρχικά δημιουργήθηκε από την εταιρία Ageia για να περιγράψει τους επεξεργαστές PhysX στους καταναλωτές.
H NVIDIA απέκτησε την Ageia Technologies το 2008 και συνεχίζει να αναπτύσσει την πλατφόρμα PhysX και στο υλικό αλλά και στο λογισμικό. Από την έκδοση 2.8.3, η υποστήριξη για κάρτες PPU σταμάτησε, και δεν κατασκευάζονται πλέον.  
\subsubsection{Υπολογισμοί γενικής χρήσης σε GPUs}
Η επιτάχυνση υλικού για υπολογισμούς φυσικής χρησιμοποιείται πλέον από τις GPUs που υποστηρίζουν υπολογισμό γενικής χρήσης. Η εκτέλεση φυσικών υπολογισμών σε GPUs είναι συνήθως αρκετά πιο γρήγορη απο ότι σε μια CPU, έτσι η απόδοση των παιχνιδιών βελτιώνεται και ή ροή εικόνας μπορεί να είναι πολύ πιο γρήγορη. Όμως η χρήση υπολογισμών φυσικής σε ένα παιχνίδι δημιουργεί επιπλέον φόρτο στην GPU. Έτσι, η χρήση ξεχωριστής μονάδας επεξεργασίας γραφικών για εκτελέσεις υπολογισμών φυσικής μπορεί να αποδώσει τα βέλτιστα αποτελέσματα. Το PhysX εκτελείται γρήγορα και αποδίδει μεγαλύτερο ρεαλισμό όταν εκτελείται στην GPU, αποφέροντας 10-20 φορές περισσότερα εφέ και οπτική πιστότητα απο ότι οι υπολογισμοί φυσικής που εκτελούνται σε μια κεντρική μονάδα επεξεργασίας τελευταίας τεχνολογίας. Το PhysX χρησιμοποιεί ετερογενή υπολογισμό για να αποδώσει την καλύτερη εμπειρία χρήσης. Καθώς το παιχνίδι εκτελείται, το σύστημα PhysX εκτελεί μέρη της τεχνολογίας στην CPU αλλά και άλλα μέρη στην GPU. Αυτό γίνεται ώστε να χρησιμοποιείται αποδοτικά το υλικό του υπολογιστή ώστε να παρέχουν την καλύτερη δυνατή εμπειρία στον χρήστη. Το πιο σημαντικό, είναι οτι η τεχνολογία PhysX μπορεί να κλιμακώνεται με την χρήση GPU, σε αντίθεση με άλλες ανταγωνιστικές υλοποιήσεις φυσικής.
\subsubsection{Σύγκριση}
Οι πιο σημαντικές μηχανές γραφικών που χρησιμοποιούνται σήμερα είναι οι παρακάτω:
\begin{itemize}
\item PhysX - Είναι μια μηχανή φυσικής πραγματικού χρόνου, από την NVIDIA. Είναι κλειστού κώδικα και χρησιμοποιείται σε πολλά παιχνίδια υπολογιστών και κονσολών. Υποστηρίζει μεγάλο αριθμό συσκευών.
\item Havok - Είναι μια μηχανή φυσικής που χρησιμοποιείται σε πολλά παιχνίδια υπολογιστών. 
\item ODE - Είναι μια μηχανή φυσικής που υποστηρίζει κυρίως ανίχνευση συγκρούσεων, δυναμικές άκαμπτων σωμάτων. Αποτελεί ανοιχτό και ελεύθερο λογισμικό. Έχει χρησιμοποιηθεί σε πολλά παιχνίδια και εφαρμογές. Αποτελεί δημοφιλής επιλογή για εφαρμογές εξομοίωσης ρομποτικής.
\item Newton Game Dynamics - Είναι μια μηχανή φυσικής ανοιχτού κώδικα που εξομοιώνει άκαμπτα σώματα σε παιχνίδια και άλλες εφαρμογές πραγματικού χρόνου. 
\item Bullet - Είναι μια μηχανή φυσικής που εξομοιώνει ανίχνευση συγκρούσεων, δυναμική άκαμπτων και μαλακών σωμάτων. Χρησιμοποιείται σε παιχνίδια υπολογιστών αλλά και για οπτικά εφέ σε ταινίες. Η βιβλιοθήκη της bullet physics είναι ελεύθερη και ανοιχτού κώδικα κάτω από την άδεια zlib.
\end{itemize}