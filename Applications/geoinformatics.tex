\section{Γεωπληροφορική}
\subsection{Γεωγραφικά πληροφοριακά συστήματα}



\subsection{Ωκεανοί}

\subsection{Σεισμική δραστηριότητα}
Η ενσωματωμένη εξομοίωση σεισμικής δραστηριότητας είναι μια ανώτερη τεχνολογία για προσδιορισμό και απεικόνιση της δομικής βλάβης που δημιουργείται σε ένα σενάριο σεισμικής δραστηριότητας. Στο IES, όλα τα κτήρια που βρίσκονται στην περιοχή δοκιμής απεικονίζονται σαν δομικά μοντέλα, και η δομική ζημιά μπορεί να προσδιοριστεί με γραμμική και μη-γραμμική δυναμική δομική ανάλυση. Επειδή υπάρχουν πολλά κτήρια σε μια περιοχή δοκιμής, και κάθε κτήριο έχει διαφορετικά χαρακτηριστικά, ο υπολογισμός υψηλής ανάλυσης είναι απαραίτητος ώστε να εκτελεστεί η ανάλυση σε σχετικά μικρό χρόνο.

\begin{figure}[h]
\centering
\includegraphics[scale=0.50]{ies}
\caption{Ενσωματωμένη εξομοίωση σεισμικής δραστηριότητας}
\end{figure}

Ένας καλός υποψήφιος για τον IES, είναι το OBASAN (Object Based Structural Analysis). Σε μια εξομοίωση σεισμικής δραστηριότητας, κάθε κτήριο σε μια περιοχή δοκιμών μπορεί να παρασταθεί ως δομικό μοντέλο για ανάλυση του κάθε αντικειμένου. Το γεωγραφικό πληροφοριακό σύστημα παρέχει δεδομένα δομικών σχημάτων στην μορφή των στοιχείων και πολυγώνων, και άλλα συστήματα βάσεων δεδομένων παρέχουν τύπους κτηρίων (π.χ δομή οπλισμένου σκυροδέματος, δομή χάλυβα, ξύλινη δομή, κ.α) Βασισμένο στον τρόπο σχεδιασμού της Ιαπωνίας, η ανάλυση μοντέλων των κτηρίων στον IES δημιουργούνται αυτόματα από τις διαστάσεις και τον τύπο του κάθε κτηρίου, και τότε το OBASAN μπορεί να χρησιμοποιηθεί για να αναλύσει αυτα τα μοντέλα δοκιμής ανάλυσης. Η δοκιμή ζημιά σε όλα τα κτήρια μιας περιοχής δοκιμής μπορεί να προβλεφθεί και η ολική ζημιά σε μια περιοχή πόλης μπορεί να απεικονιστεί από αυτήν την εξομοίωση σεισμού.

Τα πιο πολύπλοκα δομικά μοντέλα μπορούν να αυξήσουν την αξιοπιστία του προσδιορισμού δομικής βλάβης, π.χ τα μοντέλα μπορούν να χρησιμοποιηθούν για δυναμική ανάλυση δομικής απόκρισης. Μοντέλα δομικών αντικειμένων, μπορούν να χρησιμοποιηθούν για να δώσουν ένα πιο αξιόπιστο αποτέλεσμα. 

To OBASAN, είναι μια τεχνολογία που δημιουργήθηκε για την ανάλυση των δομικών μοντέλων ώστε να δέχεται τα αποτελέσματα των αντικειμένων (αξονική δύναμη, παραμόρφωση, καταπόνηση) και τα αποτελέσματα των κόμβων (π.χ ταχύτητα, επιτάχυνση, ισχύς) κάθε κτηρίου σε μια εξομοίωση σεισμικής δραστηριότητας, και έτσι το OBASAN μπορεί να παρέχει διάφορους τύπους αποτελεσμάτων δομικής ανάλυσης. Υπάρχουν δυνατότητες για χρήση υπολογισμού υψηλής απόδοσης ώστε να αυξηθεί η αποδοτικότητα και η δυνατότητα του συστήματος εξομοίωσης σεισμών (IES). 

\begin{figure}[h]
\centering
\includegraphics[scale=0.50]{obasan}
\caption{OBASAN}
\end{figure}

Το OBASAN αναπτύχθηκε με αντικειμενοστραφή προγραμματισμό σε γλώσσα C++. Η γλώσσα C++ μπορεί να επεκταθεί ώστε να χρησιμοποιεί παράλληλο προγραμματισμό στην GPU. Όπως φαίνεται στο διάγραμμα, η αρχιτεκτονική του OBASAN έχει οργανωθεί συστηματικά και είναι εύκολο να κατανοηθεί. Λόγω του ότι υπάρχουν πολλά κτήρια σε μια εξομοίωση σεισμικής δραστηριότητας σε μια περιοχή δοκιμών, ο IES χρησιμοποιεί την αρχιτεκτονική OpenMPI για να ενεργοποιήσει παράλληλο υπολογισμό στην κεντρική μονάδα επεξεργασίας (CPU). Στην CPU, ένα υποσύνολο (π.χ 10) κτηρίων μπορεί να εκτελεστεί με τον μέγιστο αριθμό νημάτων σε έναν παράλληλο υπολογισμό. Όπως και στον IES όπου η αρχιτεκτονική OpenMPI χρησιμοποιείται για τον παράλληλο υπολογισμό των κτηρίων σε μια περιοχή δοκιμών, στο OBASAN χρησιμοποιείται η αρχιτεκτονική CUDA για την παράλληλη επεξεργασία της δυναμικής ανάλυσης δομικής απόκρισης(DSRA).

Λόγω του ότι το DSRA έχει να κάνει με πιο πολύπλοκα μοντέλα, η δομική ανάλυση των μοντέλων πρέπει να παρέχει μεγάλο αριθμό παραμέτρων, που εκφράζονται με μαθηματικές εξισώσεις που περιέχουν μεγάλους πίνακες. Για ένα τόσο προχωρημένο υπολογιστικό έργο, η τεχνική HPC είναι η συνηθέστερη προσέγγιση για την επίλυση των μαθηματικών εξισώσεων σε μικρό χρόνο. Στο HPC, η τεχνολογία GPGPU είναι σχεδιασμένη ώστε να αντιμετωπίζει μεγάλο αριθμό δεδομένων και μεγάλο αριθμό υπολογισμών σε μαθηματικές εξισώσεις. Στο GPGPU, η αρχιτεκτονική CUDA μπορεί να επιλεχτεί για να λύσει αυτές τις μαθηματικές εξισώσεις του DSRA. Το κάθε βήμα της ανάλυσης του DSRA, απαιτεί πολλαπλασιασμούς πινάκων για την δομική ανάλυση.

Στο παρακάτω σχήμα μπορούμε να δούμε ένα παράδειγμα χρόνου εκτέλεσης πολλαπλασιασμού πινάκων. Οι πίνακες Α,Β και C υποθέτουμε ότι είναι τετράγωνοι πίνακες και το μέγεθος τους είναι Ν * Ν. Ο κώδικας C++ του πολλαπλασιασμού πινάκων εκτελείται σε μια CPU (Intel Xeon), ενώ οι εντολές CUDA σε μια GPU(Tesla M2050). 

\begin{figure}[h]
\centering
\includegraphics[scale=0.50]{matrix-multi}
\caption{OBASAN}
\end{figure}

