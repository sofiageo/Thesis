\section{Γεωπληροφορική}

\subsection{Σεισμική δραστηριότητα}
\subsubsection{Εισαγωγή}
Η ενσωματωμένη εξομοίωση σεισμικής δραστηριότητας είναι μια ανώτερη τεχνολογία για προσδιορισμό και απεικόνιση της δομικής βλάβης που δημιουργείται σε ένα σενάριο σεισμικής δραστηριότητας. Στο IES, όλα τα κτήρια που βρίσκονται στην περιοχή δοκιμής απεικονίζονται σαν δομικά μοντέλα, και η δομική ζημιά μπορεί να προσδιοριστεί με γραμμική και μη-γραμμική δυναμική δομική ανάλυση. Επειδή υπάρχουν πολλά κτήρια σε μια περιοχή δοκιμής, και κάθε κτήριο έχει διαφορετικά χαρακτηριστικά, ο υπολογισμός υψηλής ανάλυσης είναι απαραίτητος ώστε να εκτελεστεί η ανάλυση σε σχετικά μικρό χρόνο.\cite{geoinformatics-1}

\begin{figure}[h]
\centering
\includegraphics[scale=1]{ies}
\caption{Ενσωματωμένη εξομοίωση σεισμικής δραστηριότητας}
\end{figure}

Ένας καλός υποψήφιος για τον IES, είναι το OBASAN (Object Based Structural Analysis). Σε μια εξομοίωση σεισμικής δραστηριότητας, κάθε κτήριο σε μια περιοχή δοκιμών μπορεί να παρασταθεί ως δομικό μοντέλο για ανάλυση του κάθε αντικειμένου. Το γεωγραφικό πληροφοριακό σύστημα παρέχει δεδομένα δομικών σχημάτων στην μορφή των στοιχείων και πολυγώνων, και άλλα συστήματα βάσεων δεδομένων παρέχουν τύπους κτηρίων (π.χ δομή οπλισμένου σκυροδέματος, δομή χάλυβα, ξύλινη δομή, κ.α) Βασισμένο στον τρόπο σχεδιασμού της Ιαπωνίας, η ανάλυση μοντέλων των κτηρίων στον IES δημιουργούνται αυτόματα από τις διαστάσεις και τον τύπο του κάθε κτηρίου, και τότε το OBASAN μπορεί να χρησιμοποιηθεί για να αναλύσει αυτα τα μοντέλα δοκιμής ανάλυσης. Η δοκιμή ζημιά σε όλα τα κτήρια μιας περιοχής δοκιμής μπορεί να προβλεφθεί και η ολική ζημιά σε μια περιοχή πόλης μπορεί να απεικονιστεί από αυτήν την εξομοίωση σεισμού.\cite{geoinformatics-2}\cite{geoinformatics-3}

Τα πιο πολύπλοκα δομικά μοντέλα μπορούν να αυξήσουν την αξιοπιστία του προσδιορισμού δομικής βλάβης, π.χ τα μοντέλα μπορούν να χρησιμοποιηθούν για δυναμική ανάλυση δομικής απόκρισης. Μοντέλα δομικών αντικειμένων, μπορούν να χρησιμοποιηθούν για να δώσουν ένα πιο αξιόπιστο αποτέλεσμα. 

\subsubsection{OBASAN}
To OBASAN, είναι μια τεχνολογία που δημιουργήθηκε για την ανάλυση των δομικών μοντέλων ώστε να δέχεται τα αποτελέσματα των αντικειμένων (αξονική δύναμη, παραμόρφωση, καταπόνηση) και τα αποτελέσματα των κόμβων (π.χ ταχύτητα, επιτάχυνση, ισχύς) κάθε κτηρίου σε μια εξομοίωση σεισμικής δραστηριότητας, και έτσι το OBASAN μπορεί να παρέχει διάφορους τύπους αποτελεσμάτων δομικής ανάλυσης. Υπάρχουν δυνατότητες για χρήση υπολογισμού υψηλής απόδοσης ώστε να αυξηθεί η αποδοτικότητα και η δυνατότητα του συστήματος εξομοίωσης σεισμών (IES). 

\begin{figure}[h]
\centering
\hspace*{-1.0in}
\includegraphics[scale=0.50]{obasan}
\caption{OBASAN}
\end{figure}

Το OBASAN αναπτύχθηκε με αντικειμενοστραφή προγραμματισμό σε γλώσσα C++. Η γλώσσα C++ μπορεί να επεκταθεί ώστε να χρησιμοποιεί παράλληλο προγραμματισμό στην GPU. Όπως φαίνεται στο διάγραμμα, η αρχιτεκτονική του OBASAN έχει οργανωθεί συστηματικά και είναι εύκολο να κατανοηθεί. Λόγω του ότι υπάρχουν πολλά κτήρια σε μια εξομοίωση σεισμικής δραστηριότητας σε μια περιοχή δοκιμών, ο IES χρησιμοποιεί την αρχιτεκτονική OpenMPI για να ενεργοποιήσει παράλληλο υπολογισμό στην κεντρική μονάδα επεξεργασίας (CPU). Στην CPU, ένα υποσύνολο (π.χ 10) κτηρίων μπορεί να εκτελεστεί με τον μέγιστο αριθμό νημάτων σε έναν παράλληλο υπολογισμό. Όπως και στον IES όπου η αρχιτεκτονική OpenMPI χρησιμοποιείται για τον παράλληλο υπολογισμό των κτηρίων σε μια περιοχή δοκιμών, στο OBASAN χρησιμοποιείται η αρχιτεκτονική CUDA για την παράλληλη επεξεργασία της δυναμικής ανάλυσης δομικής απόκρισης(DSRA).

Λόγω του ότι η DSRA έχει να κάνει με πιο πολύπλοκα μοντέλα, η δομική ανάλυση των μοντέλων πρέπει να παρέχει μεγάλο αριθμό παραμέτρων, που εκφράζονται με μαθηματικές εξισώσεις που περιέχουν μεγάλους πίνακες. Για ένα τόσο προχωρημένο υπολογιστικό έργο, η τεχνική HPC είναι η συνηθέστερη προσέγγιση για την επίλυση των μαθηματικών εξισώσεων σε μικρό χρόνο. Στο HPC, η τεχνολογία GPGPU είναι σχεδιασμένη ώστε να αντιμετωπίζει μεγάλο αριθμό δεδομένων και μεγάλο αριθμό υπολογισμών σε μαθηματικές εξισώσεις. Στο GPGPU, η αρχιτεκτονική CUDA μπορεί να επιλεχτεί για να λύσει αυτές τις μαθηματικές εξισώσεις της DSRA. Το κάθε βήμα της ανάλυσης της DSRA, απαιτεί πολλαπλασιασμούς πινάκων για την δομική ανάλυση.

Στο παρακάτω σχήμα μπορούμε να δούμε ένα παράδειγμα χρόνου εκτέλεσης πολλαπλασιασμού πινάκων. Οι πίνακες Α,Β και C υποθέτουμε ότι είναι τετράγωνοι πίνακες και το μέγεθος τους είναι Ν * Ν. Ο κώδικας C++ του πολλαπλασιασμού πινάκων εκτελείται σε μια CPU (Intel Xeon), ενώ οι εντολές CUDA σε μια GPU(Tesla M2050). 

\begin{figure}[h]
\centering
\includegraphics[width=\linewidth]{matrix-multi}
\caption{GPU and CPU matrix multiplication times}
\end{figure}

Στην DRSA, το μέγεθος πίνακα ορίζεται απο τον συνολικό αριθμό των μοιρών της ελευθερίας (DOF) σε ένα κτήριο, οπότε το μέγεθος του πίνακα εξαρτάται απο τις διαστάσεις του κάθε κτηρίου. Τα μικρότερα κτήρια όπως τα σπίτια αναπαρίστανται απο μικρότερους πίνακες και τα μεγαλύτερα κτήρια αναπαρίστανται απο μεγαλύτερους πίνακες. Όπως μπορούμε να δούμε στον πίνακα, οι GPUs είναι πολύ πιο γρήγορες απο τις CPUs σε κάθε περίπτωση πολλαπλασιασμού πινάκων. Από τα δεδομένα του πίνακα, μπορούμε να απεικονίσουμε τα αποτελέσματα σε γράφημα. Στο παρακάτω γράφημα βλέπουμε την σχέση που έχει ο χρόνος εκτέλεσης με το μέγεθος του πίνακα, τόσο στην CPU όσο και στην GPU. Οι μεγαλύτεροι πίνακες χρειάζονται περισσότερο χρόνο για να εκτελέσουν τον υπολογισμό πινάκων. Όσο μεγαλύτερο το μέγεθος του πίνακα, τόσο μεγαλύτερο το πλεονέκτημα της GPU έναντι της CPU. Για ένα πίνακα 2000 * 2000 στοιχείων, οι GPUs είναι περίπου 1000 φορές γρηγορότερες απο τις CPUs. Για αυτό ο προγραμματισμός γενικού σκοπού είναι καλή επιλογή για την δυναμική ανάλυση δομικής απόκρισης.

\begin{figure}[h]
\centering
\includegraphics[width=\linewidth]{graph-multi}
\caption{GPU and CPU matrix multiplication times}
\end{figure}

Τα παραπάνω αποτελέσματα, είναι μια απλή τεχνική CUDA. Ο προγραμματισμός υψηλής απόδοσης της DSRA μπορεί να αυξηθεί με την χρήση βιβλιοθηκών CUDA. Οι CUDA βιβλιοθήκες, π.χ CUBLAS(Γραμμική Άλγεβρα),CUFFT(Μετασχηματισμός Fast Fourier), μπορούν να λύσουν τις μαθηματικές εξισώσεις της DSRA. Γενικότερα, όλες οι βιβλιοθήκες CUDA είναι βελτιστοποιημένες για εκτέλεση υψηλής απόδοσης. Επιπλέον, το OBASAN με την χρήση CUDA μπορεί να επιτύχει καλή κλιμάκωση στον IES. 
\subsubsection{Συμπεράσματα}
Για να μειώσουν την κοινωνική και οικονομική επίπτωση των σεισμών, οι ερευνητές προσπαθούν να εφαρμόσουν καινούριες τεχνολογίες και τα τελευταία δεδομένα στον IES. Η αριθμητική εξομοίωση χρησιμοποιείται για να προσδιορίσει πιθανές ζημιές σε σενάρια σεισμών. Σε μια εξομοίωση σεισμών, η σωστή πρόβλεψη της ζημιάς στην κοινωνία είναι πολύ σημαντική στην διαχείριση κρίσεων. Αυτή η προσέγγιση στο μαθηματικό πρόβλημα με την χρήση τεχνολογιών προγραμματισμού υψηλής απόδοσης και CUDA, μπορεί να εφαρμοστεί σε διάφορα πεδία της μηχανικής. \cite{geoinformatics-4}\cite{geoinformatics-5}

Με την χρήση των προσδιορισμών, ένα πιο σταθερό πλάνο μπορεί να χρησιμοποιηθεί για κατασκευή καινούριων κτηρίων για να εμποδίσει δομικές ζημιές σε σενάρια σεισμών. Επιπλέον, σχέδια για εκκένωση κτηρίων και μεταφορά ατόμων μπορούν να δημιουργηθούν για να αναλύσουν τις εξόδους κινδύνου, κάτι που μπορεί να προστατέψει και να σώσει ανθρώπινες ζωές.

\subsection{Προγράμματα}

\begin{apptable}{Γεωπληροφορική}{geoinformatics}
DigitalGlobe Advanced Ortho Series & Γεωδιαστημική Οπτικοποίηση & 50x & ΝΑΙ \\ \hline
Eternix Blaze Terra & Γεωδιαστημική Οπτικοποίηση & 50x & ΝΑΙ \\ \hline
Exelis (ITT) ENVI & Γεωδιαστημική Οπτικοποίηση & 70x & ΝΑΙ \\ \hline
GAIA & ανεπτυγμένη γεωδιαστημική δυνατότητα, παραγωγή χάρτη θερμότητας, και κατανεμημένες υπηρεσίες ανταπόδοσης & - & ΝΑΙ \\ \hline
GeoWeb3d Desktop & Γεωδιαστημική Οπτικοποίηση & - & ΝΑΙ \\ \hline
Incogna GIS  & Γεωδιαστημική Οπτικοποίηση & 50x & ΝΑΙ \\ \hline
LuciadLightspeed & Γεωδιαστημική Οπτικοποίηση και ανάλυση & & ΝΑΙ \\ \hline
Manifold Systems & Γεωγραφικό πληροφοριακό σύστημα, επεξεργασία διανυσμάτων \& ανάλυση & - & ΝΑΙ \\ \hline
MrGeo & Γεωδιαστημική Οπτικοποίηση  & - & ΝΑΙ \\ \hline
OpCoast SNEAK  & Ηλεκτρομαγνητικά σήματα μοντελοποίησης για πολύπλοκα αστικά περιβάλλοντα & 100x & ΝΑΙ \\ \hline
PCI Geomatics GXL  & Γεωδιαστημική Οπτικοποίηση & 20-60x & ΝΑΙ \\ \hline
\end{apptable}

\begin{apptable}{Σεισμική δραστηριότητα}
Acceleware AxRTM AxKTM & Σεισμική επεξεργασία & & ΝΑΙ \\ \hline
CGGV GeoVation & Σεισμική επεξεργασία & & ΝΑΙ \\ \hline
CGG-Veritas Inside Earth & Σεισμική ερμηνεία & & ΝΑΙ \\ \hline
ffA Geoteric & Σεισμική ερμηνεία  & & ΝΑΙ \\ \hline
ffA SEA3D Pro  & Σεισμική ερμηνεία  & & ΝΑΙ \\ \hline
ffA SVI Pro & Σεισμική ερμηνεία  & & ΝΑΙ \\ \hline
GeoMage Multifocusing & Σεισμική απεικόνιση  & & ΝΑΙ \\ \hline
GeoStar Seismic Suite & Σεισμική επεξεργασία  & & ΝΑΙ \\ \hline
HUE Headwave Suite & Σεισμική ερμηνεία  & & ΝΑΙ \\ \hline
HUE HUEspace & Σεισμική ερμηνεία  & & ΝΑΙ \\ \hline
OpenGeo Solutions OpenSeis & Σεισμική επεξεργασία  & & ΝΑΙ \\ \hline
Paradigm Echos RTM & Σεισμική επεξεργασία  & & ΝΑΙ \\ \hline
Paradigm SKUA & Μοντελοποίηση δεξαμενών  & & ΝΑΙ \\ \hline
Panorama Tech & Σεισμική επεξεργασία, μοντελοποίηση & & ΝΑΙ \\ \hline
Roxar RMS & Μοντελοποίηση δεξαμενών & & ΝΑΙ \\ \hline
Schlumberger Omega2 RTM & Σεισμική επεξεργασία & & ΝΑΙ \\ \hline
Seismic City Prestack Interpretation & Σεισμική επεξεργασία & & ΝΑΙ \\ \hline
SpectraSeis & Σεισμική επεξεργασία & & ΝΑΙ \\ \hline
Stoneridge Technologies GAMPACK & Εξομοίωση δεξαμενών & & ΝΑΙ \\ \hline
Tsunami A2011 & Σεισμική επεξεργασία / πακέτο απεικόνισης & & ΝΑΙ \\ \hline
Tsunami RTM  & Σεισμική επεξεργασία & & ΝΑΙ \\ \hline
\end{apptable}

\newpage