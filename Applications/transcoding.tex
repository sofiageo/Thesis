\section{Μετατροπή}
\subsection{Εισαγωγή}
Η μετατροπή είναι η άμεση μετατροπή ψηφιακής κωδικοποίησης, όπως για παράδειγμα σε αρχεία μιας ταινίας(π.χ .mp4, avi), αρχεία ήχου(.wav .mp3), ή κωδικοποίηση χαρακτήρων. (π.χ UTF-8, ISO/IEC 8859). Αυτή γίνεται σε περιπτώσεις που η συσκευή στόχος δεν υποστηρίζει την μορφή του αρχείου ή χρειαζόμαστε μικρότερο μέγεθος αρχείου, ή για να μετατρέψουμε αρχεία παλαιότερης μορφής σε μια πιο σύγχρονη για καλύτερη υποστήριξη σε μελλοντικές εφαρμογές. 

Η μετατροπή χρησιμοποιείται συχνά στα λογισμικά προβολής βίντεο για να ελαττώσουμε το μέγεθος του αρχείου βίντεο. Μια διαδικασία που γίνεται συχνά είναι η μετατροπή από αρχεία MPEG-2(DVD) σε αρχεία μορφής MPEG-4, που ενσωματώνει σύγχρονους αλγόριθμους για καλύτερη ποιότητα εικόνας σε συνδυασμό με μικρότερο μέγεθος αρχείου.

\begin{figure}[h]
\centering
\includegraphics[width=0.5\linewidth]{h264-logo}
\caption{H264, το πιο δημοφιλές πρότυπο συμπίεσης βίντεο}
\end{figure}
\subsection{Μειονεκτήματα} 
Το μεγαλύτερο μειονέκτημα της μετατροπής σε απωλεστικές μορφές αρχείου είναι η μειωμένη ποιότητα. Τα τεχνουργήματα συμπίεσης συσσωρεύονται, οπότε κάθε διαδικασία μετατροπής δημιουργεί μια βαθμιαία απώλεια ποιότητας, που είναι γνωστή ως ψηφιακή απώλεια. Για αυτόν τον λόγο, η μετατροπή συνήθως δεν συνίσταται εκτός και αν δεν μπορούμε να την αποφύγουμε.

\subsection{Βίντεο}
H συμπίεση βίντεο χρησιμοποιεί σύγχρονες τεχνικές για να μειώσει πλεονασμούς στα δεδομένα βίντεο. Οι περισσότεροι αλγόριθμοι συμπίεσης βίντεο συνδυάζουν συμπίεση εικόνας και προσωρινή αποζημίωση κίνησης. Ο ήχος κωδικοποιείται παράλληλα με διαφορετικούς αλγόριθμους συμπίεσης άλλα συνήθως συνδυάζεται σε ένα πακέτο όταν ολοκληρωθεί η διαδικασία.

Οι περισσότεροι αλγόριθμοι συμπίεσης βίντεο χρησιμοποιούν απωλεστική συμπίεση, καθώς το ασυμπίεστο βίντεο απαιτεί πολύ μεγάλους ρυθμούς μετάδοσης. Αν και οι περισσότεροι αλγόριθμοι έχουν ένα παράγοντα συμπίεσης 3, μια τυπική συμπίεση βίντεο MPEG-4 μπορεί να έχει παράγοντα συμπίεσης από 20 έως και 200. Όπως σε όλες τις απωλεστικές διαδικασίες συμπίεσης, υπάρχει ένα δίλημμα μεταξύ της ποιότητας του βίντεο, το κόστος της επεξεργασίας της συμπίεσης και της αποσυμπίεσης, και των απαιτήσεων του συστήματος. Υπερβολικά συμπιεσμένο βίντεο μπορεί να δημιουργήσει οπτικά τεχνουργήματα.

Συνήθως ο τρόπος με τον οποίο λειτουργεί ένας αλγόριθμος συμπίεσης βίντεο είναι με ομάδες γειτονικών εικονοστοιχείων(pixel), που ονομάζονται blocks. Αυτές οι ομάδες από pixels, συγκρίνονται μεταξύ τους από μια εικόνα στην επόμενη, και ο αλγόριθμος αποστέλλει μόνο τις αλλαγές μεταξύ αυτών των block. Σε περιοχές του βίντεο που υπάρχει μεγαλύτερη κίνηση, ο αλγόριθμος πρέπει να συμπιέσει περισσότερα δεδομένα για να προλάβει τον μεγαλύτερο αριθμό εικονοστοιχείων που αλλάζουν. Συνήθως σε σκηνές με φωτιά, εκρήξεις, καπνούς, το αποτέλεσμα της συμπίεσης έχει μεγαλύτερη απώλεια ποιότητας, η αύξηση του ρυθμού μετάδοσης.

Μερικοί από τους σύγχρονους αλγόριθμους μετατροπής είναι οι παρακάτω
\begin{itemize}
\item Lagarith - Είναι ένας μη απωλεστικός αλγόριθμος ανοιχτού κώδικα που δίνει έμφαση στην ταχύτητα, στην υποστήριξη διαφόρων χώρου χρωμάτων(YV12,RGB,YUY2). Είναι ιδανικός για επεξεργασία και αποθήκευση αρχείων. Αν και υπάρχουν κάποιοι καλύτεροι μη απωλεστικοί αλγόριθμοι συμπίεσης, ο lagarith είναι ο πιο γρήγορος και έτσι έχει κερδίσει την υποστήριξη της κοινότητας.
\item VP9 - Είναι ένα ελεύθερο πρότυπο ανοιχτού κώδικα που αναπτύσσεται από την Google. Είναι ο διάδοχος του VP8. Σκοπός του είναι να μειώσει περισσότερο τον χώρο που απαιτείται από το βίντεο διατηρώντας την ίδια ποιότητα. 
\item H.264 - Είναι ένα πρότυπο συμπίεσης βίντεο, που είναι ίσως το πιο επιτυχημένο για την καταγραφή, συμπίεση, και αναμετάδοση περιεχομένου βίντεο. Χρησιμοποιείται για αναμετάδοση ψηφιακού σήματος τηλεόρασης, δορυφόρων, και από υπηρεσίες αναμετάδοσης στο διαδίκτυο.
\item H.265 - Είναι ένα πρότυπο συμπίεσης βίντεο, διάδοχος του επιτυχημένου H.264. Σκοπός του είναι ο διπλασιασμός της συμπίεσης διατηρώντας την ίδια ποιότητα. Υποστηρίζει αναλύσεις έως και 8192x4320.
\item Daala - Είναι μια τεχνολογία συμπίεσης από το ίδρυμα Xiph.Org. Χρησιμοποιεί περιτύλιξη μεταμορφώσεων για να μειώσει τα οπτικά τεχνουργήματα. Ο στόχος είναι η απόδοση του να ξεπεράσει τις δυνατότητες του VP9 και του H.265. 
\end{itemize}

\subsection{Συμπίεση δεδομένων γενετικής}
Οι γενετικοί αλγόριθμοι συμπίεσης είναι η τελευταία γενιά μη απωλεστικών αλγορίθμων για συμπίεση δεδομένων (συνήθως αλληλουχίες nucleotides) χρησιμοποιώντας συμβατικούς αλγορίθμους συμπίεσης αλλά και γενετικούς αλγόριθμους βελτιστοποιημένους στο συγκεκριμένο τύπο δεδομένων. Το 2012, μια ομάδα επιστημόνων απο το John Kopkins University ανακοίνωσε έναν αλγόριθμο συμπίεσης γενετικής(HapZipper), ο οποίος χρησιμοποιεί HapMap δεδομένα για να πετύχει συμπίεση 95\% μείωση στο μέγεθος αρχείου, πετυχαίνοντας καλύτερη συμπίεση σε πολύ καλύτερο χρόνο από ότι οι γνωστοί μη απωλεστικοί αλγόριθμοι. Άλλοι γενετικοί αλγόριθμοι συμπίεσης π.χ GenomeZip πετυχαίνουν μεγαλύτερη συμπίεση καταφέρνοντας αποθήκευση 6 δισεκατομμυρίων ανθρώπινων ζευγαριών γονιδιώματος σε 2.5 megabyte.