\section{Κβαντική χημεία}
\subsection{Εισαγωγή}
Η κβαντική χημεία είναι μια διακλάδωση της χημείας όπου η κύρια εστίαση είναι η εφαρμογή κβαντικής μηχανικής σε μοντέλα φυσικής και σε πειράματα  χημικών συστημάτων.
Η ιστορία της κβαντικής χημείας ουσιαστικά ξεκινάει με την ανακάλυψη των καθοδικών ακτίνων από τον Michael Faraday, την υπόδειξη του 1877 απο τον Ludwig Boltzmann ότι τα επίπεδα ενέργειας σε ένα σύστημα φυσικής μπορεί να είναι διακριτά, και την κβαντική υπόθεση του 1900 από τον Max Planck ότι κάθε ατομικό σύστημα ακτινοβολίας ενέργειας μπορεί να διαιρεθεί σε έναν αριθμό διακριτών στοιχείων έτσι ώστε κάθε από αυτά τα στοιχεία ενέργειας να είναι αναλογικό στην συχνότητα με την οποία ακτινοβολούν ενέργεια και μια αριθμητική τιμή που ονομάζεται σταθερά Planck.

Η Κβαντική χημεία περιλαμβάνει βαθιά χρήση πειραματικών και θεωρητικών μεθόδων:
\begin{itemize}
\item Οι πειραματικοί κβαντικοί χημικοί βασίζονται στην φασματοσκοπία, μέσω της οποίας η πληροφορία σχετικά με την κβάντιση της ενέργειας σε μοριακή κλίμακα μπορεί να αποκτηθεί. Τυπικές μέθοδοι είναι ή υπέρυθρη φασματοσκοπία και η πυρηνική μαγνητική αντήχηση (NMR) φασματοσκοπίας
\item Θεωρητική κβαντική χημεία, το έργο της οποίας βρίσκεται κάτω απο την κατηγορία της υπολογιστικής χημείας, στοχεύει να υπολογίσει τις προβλέψεις της κβαντικής θεωρίας καθώς τα άτομα και τα μόρια μπορούν να έχουν μόνο διακριτές ενέργειες. Λόγω του ότι αυτή η διαδικασία, όταν εφαρμοστεί σε πολύ-ατομικά είδη δημιουργεί υπολογιστικά προβλήματα, αυτοί οι υπολογισμοί γίνονται με την χρήση υπολογιστών αντί για την αναλυτική μέθοδο, και άλλες παραδοσιακές μεθόδους.
\end{itemize}
Κατά αυτούς τους τρόπους, η κβαντική χημεία μελετάει τα χημικά φαινόμενα. Στις αντιδράσεις, η κβαντική χημεία μελετάει την σταθερή κατάσταση των ατόμων και των μορίων, τις μη σταθερές καταστάσεις, και την μετάβαση των καταστάσεων ανάμεσα στις χημικές αντιδράσεις. Στους υπολογισμούς, η μελέτη της κβαντικής χημείας περιλαμβάνει χρήση ημι-εμπειρικών και άλλων μεθόδων που βασίζονται στις αρχές της κβαντικής μηχανικής, και αντιμετωπίζουν προβλήματα σχετικά με τον χρόνο. Πολλές μελέτες κβαντικής χημείας υποθέτουν ότι οι πυρήνες βρίσκονται σε κατάσταση ηρεμίας και πολλοί υπολογισμοί περιέχουν επαναληπτικές μεθόδους. Οι στόχοι της κβαντικής χημείας περιλαμβάνουν την αύξηση της ακρίβειας των αποτελεσμάτων για μικρά μοριακά συστήματα, και αύξηση του μεγέθους των μορίων που μπορούν να επεξεργαστούν. Ο χρόνος υπολογισμού αυξάνεται όσο αυξάνεται ο αριθμός των ατόμων.


\subsection{Προγράμματα}

\begin{apptable}{Κβαντική Χημεία}{chemistry}
Abinit & Επιτρέπει τον υπολογισμό της ολικής ενέργειας, πυκνότητα φόρτισης και ηλεκτρονική δομή του συστήματος αποτελούμενο από ηλεκτρόνια και πυρήνες μέσα στο DFT  & 1.3-2.7x & ΝΑΙ \\ \hline
ADF & Λογισμικό Θεωρίας Λειτουργικής Πυκνότητας (DFT) που επιτρέπει υπολογισμούς ηλεκτρονικών δομών πρώτης αρχής & 1.5-2x & ΝΑΙ \\ \hline
BigDFT & Εκτελεί θεωρία  λειτουργικής πυκνότητας λύνοντας τις εξισώσεις Kohn-Sham περιγράφοντας τα ηλεκτρόνια σε ένα υλικό & 2-25x & ΝΑΙ \\ \hline
CP2K & Πρόγραμμα που εκτελεί ατομιστικές και μοριακές εξομοιώσεις στερεών καταστάσεων, υγρών, μοριακών και βιολογικών συστημάτων & 2-7x & ΝΑΙ \\ \hline
GAMESS-UK & The general purpose ab initio molecular electronic structure program for performing SCF-, DFT- and MCSCF-gradient calculations & 8x & ΝΑΙ \\ \hline
GAMESS-US & Computational chemistry suite used to simulate atomic and molecular electronic structure & 1.3-2.9x & ΝΑΙ \\ \hline
Gaussian (In Development) & Predicts energies, molecular structures,and vibrational frequencies of molecular systems & - & ΝΑΙ \\ \hline 
GPAW & Real-space grid DFT code written in C and Python & 8x  & ΝΑΙ \\ \hline
LATTE & Density matrix computations & - & ΝΑΙ \\ \hline
MOLCAS & Methods for calculating general electronic structures in molecular systems in both ground and excited states & 1.1x & ΝΑΙ \\ \hline
MOPAC2013 & Semiempirical Quantum Chemistry & 2x & ΟΧΙ \\ \hline
NWChem & Calculations & 3-10x & ΝΑΙ \\ \hline
Octopus & Used for ab initio virtual experimentation and quantum chemistry calculations & 1.5-8x & - \\ \hline
Q-CHEM & Computational chemistry package designed for HPC clusters & 8-14x & - \\ \hline
QUICK & QUICK is a GPU-enabled ab intio quantum chemistry software package & 10-100x  & ΝΑΙ \\ \hline
TeraChem & Quantum chemistry software designed to run on NVIDIA GPU & 44-650x & ΝΑΙ \\ \hline
\end{apptable}
