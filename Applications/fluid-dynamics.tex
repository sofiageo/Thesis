\section{Δυναμική Ρευστών}
Στην φυσική, η δυναμική ρευστών είναι ένα υποσύνολο της μηχανικής ρευστών που ασχολείται με την ροή ρευστών - την φυσική επιστήμη των ρευστών(υγρών και αερίων) σε κίνηση. Έχει ακόμα περισσότερα υποσύνολα, όπως η αεροδυναμική (η μελέτη του αέρα και άλλων αερίων σε κίνηση), υδροδυναμική (η μελέτη των υγρών σε κίνηση). Η δυναμική ρευστών έχει μεγάλο εύρος εφαρμογών, συμπεριλαμβανομένου υπολογισμό ισχύων και στιγμών σε ένα αεροπλάνο, έρευνα της ροής μάζας του πετρελαίου μέσα σε σωλήνες μεταφοράς, πρότυπα πρόβλεψης καιρικών φαινομένων, κατανόηση ενός nebulae στο διαγαλαξιακό διάστημα, και μοντελοποίηση εκρήξεων. Μερικές από τις αρχές της δυναμικής ρευστών χρησιμοποιούνται ακόμα και στον σχεδιασμό έργων κυκλοφορίας, όπου η κυκλοφορία θεωρείται ένα συνεχόμενο ρευστό.\cite{fluiddynamics-1} \cite{fluiddynamics-5} 

Οι δυναμικές ρευστών παρέχουν μια συστηματική σχεδίαση, που περιλαμβάνει εμπειρικούς νόμους από μετρήσεις ροής και χρησιμοποιείται για να λύσει πρακτικά προβλήματα. Η λύση στα προβλήματα δυναμικής ρευστών συνήθως περιλαμβάνει υπολογισμούς διάφορων ιδιοτήτων των ρευστών, όπως η ταχύτητα, πίεση, πυκνότητα, και θερμοκρασία, όπως αυτά αντιδρούν μέσα στον χώρο και στον χρόνο.
\subsection{Τυρβώδης ροή}
Η τυρβώδης ροή είναι μια ροή που χαρακτηρίζεται από επανάληψη, και τυχαία συμπεριφορά. Η ροή στην οποία η ταραχή δεν εμφανίζεται ονομάζεται ελασματώδης. Μερικά από τα φαινόμενα μπορεί να εμφανίζονται και στην ελασματώδη ροή. Μαθηματικά, η τυρβώδης ροή εκπροσωπείται μέσω μιας αποσύνθεσης Reynolds, στην οποία η ροή αναλύεται στο άθροισμα ενός μέσου στοιχείου, και ενός στοιχείου άναρχης κατάστασης.

Πιστεύεται ότι η τυρβώδης ροή μπορεί να περιγραφεί μέσω εξισώσεων Navier-Stokes, εξομοίωση άμεσης αρίθμησης (DNS), και περιορίζεται από την ισχύ του υπολογιστή που χρησιμοποιείται για την εξομοίωση, όσο και από την αποδοτικότητα του αλγορίθμου. Τα αποτελέσματα του DNS πολλές φορές συμφωνούν με τα πειραματικά δεδομένα για κάποιες ροές.\cite{fluiddynamics-2} \cite{fluiddynamics-4} 

Οι περισσότερες ροές, έχουν αριθμούς Reynolds πολύ υψηλούς για την χρήση του DNS (έχοντας υπόψην την υπολογιστική δύναμη για τις επόμενες δεκαετίες). Το οποιοδήποτε αεροσκάφος που μεταφέρει ανθρώπους, με ταχύτητα πάνω απο 72 χιλιόμετρα την ώρα, είναι αρκετά εκτός ορίων για την χρήση εξομοίωσης DNS (Re = 4 εκατομμύρια). Τα φτερά του αεροσκάφους έχουν αριθμούς Reynolds πάνω από 40 εκατομμύρια. Για να λυθούν αυτά τα προβλήματα, οι τυρβώδεις ροές είναι αναγκαιότητα για το μεσοπρόθεσμο μέλλον. Οι Navier-Stokes εξισώσεις σε συνδυασμό με τα μοντέλα ταραχών, παρέχουν ένα μοντέλο των αποτελεσμάτων της τυρβώδης ροής. Άλλη μια υποσχόμενη μεθοδολογία είναι ή μεγάλη εξομοίωση eddy (LES).

\subsection{GPU}
Αρκετά σχέδια υπάρχουν τα τελευταία χρόνια σε μοντέλα Navier-Stokes και οι μέθοδοι Lattice Boltzman έχουν αποδείξει πολύ μεγάλες επιταχύνσεις με χρήση υπολογισμό GPGPU. Μερικά από τα αποτελέσματα φαίνονται στο παρακάτω διάγραμμα. Υπάρχουν επίσης πολλά έργα σε μοντελοποίηση καιρικών φαινομένων και μοντελοποίηση ωκεανών με χρήση GPU.

\begin{figure}[h]
\centering
\hspace*{-0.5in}
\includegraphics[scale=1]{fluid-dynamics1}
\caption{Υπολογισμός δυναμικής ρευστών σε GPU\cite{figure-13}}
\end{figure}
 
\subsection{Tsunami}
Το tsunami, γνωστό και ως σεισμικό θαλάσσιο κύμα, είναι μια σειρά από κύματα νερού που προκαλούνται από το εκτόπισμα μεγάλου όγκου νερού, συνήθως ενός ωκεανού ή μιας μεγάλης λίμνης. Οι σεισμοί, οι ηφαιστειακές δραστηριότητες, και άλλες υποθαλάσσιες εκρήξεις (συμπεριλαμβανομένων πυροδοτήσεων πυρηνικών συσκευών), συγκρούσεις από μετεωρίτες, ή και άλλες διαταράξεις πάνω ή κάτω από το νερό έχουν την δυνατότητα να προκαλέσουν ένα tsunami.

Τα κύματα tsunami δεν μοιάζουν με τα συνηθισμένα θαλάσσια κύματα, γιατί το μήκος κύματος είναι πολύ μεγαλύτερο. Αντί να εμφανίζεται σαν μια θραύση κυμάτων, το tsunami αρχικά μπορεί να μοιάζει με έναν τύπο παλιρροιακού κύματος,  κύματα μεγάλου ύψους που δημιουργούνται ειδικά από σεισμικές δραστηριότητες, και για αυτόν τον λόγο ονομάζεται σεισμικό θαλάσσιο κύμα. Ο όρος αυτός προτιμάται από τους γεωλόγους και ωκεανογράφους αν και δεν έχει ακριβώς την σημασία της λέξης tsunami.Ο όρος tsunami προέρχεται από τα δύο γιαπωνέζικα γράμματα {\large\kanji{津}}(tsu) που σημαίνει λιμάνι, και {\large\kanji{波}}(nami) που σημαίνει κύμα. Ο όρος πήρε το όνομα του από τους ψαράδες, όπου ενώ βρίσκονταν στην θάλασσα και δεν συναντούσαν κάποιο φαινόμενο, όταν γυρνούσαν στο λιμάνι βρίσκανε το χωρίο τους κατεστραμμένο από το tsunami. Ακόμα και ο όρος σεισμικό θαλάσσιο κύμα δεν είναι πολύ ακριβής, καθώς άλλες δυνάμεις εκτός από τους σεισμούς μπορούν να δημιουργήσουν τέτοια κύματα, μεταξύ τους οι ηφαιστειακές δραστηριότητες, καθίζηση γης στον ωκεανό, ακόμα και η απότομη αλλαγή πίεσης. \cite{fluiddynamics-3}
 
Τα tsunami γενικά αποτελούνται από μια σειρά κυμάτων με περιόδους που εκτείνονται από λεπτά μέχρι ώρες, και φτάνουν σε μια μορφή τρένου κυμάτων. Κύματα ύψους δεκάδων μέτρων μπορούν να δημιουργηθούν από μεγάλα γεγονότα. Αν και η πρόσκρουση των tsunami περιορίζεται στις ακτές, η καταστροφική τους δύναμη μπορεί να είναι τεράστια και επηρεάζουν ολοκληρωτικά κόλπους οκεανών. Το 2004 το tsunami στον Ινδικό ωκεανό ήταν ανάμεσα στα πιο καταστροφικά φυσικά φαινόμενα στην ιστορία του ανθρώπου, με τουλάχιστον 290.000 νεκρούς και αγνοούμενους σε 14 χώρες που συνορεύουν με τον Ινδικό ωκεανό.

\subsubsection{Χαρακτηριστικά}
Τα tsunami μπορούν να προκαλέσουν ζημιές με δύο μηχανισμούς: την ισχύ ενός τοίχου νερού που ταξιδεύει σε μεγάλη ταχύτητα, και την ισχύ του μεγάλου όγκου νερού όπου μεταφέρει συντρίμμια στην ξηρά, ακόμα και με κύματα που δεν φαίνονται να είναι μεγάλα.

Ενώ τα καθημερινά κύματα έχουν μήκος κύματος περίπου 100 μέτρα και ύψος περίπου 2 μέτρα, ένα tsunami στον βαθύ ωκεανό έχει πολύ μεγαλύτερο μήκος κύματος έως και 200 χιλιόμετρα. Ένα τέτοιο κύμα ταξιδεύει με πολύ περισσότερο από 800 χιλιόμετρα την ώρα, άλλα λόγω του μεγάλου μήκους κύματος χρειάζεται 20-30 λεπτά για να ολοκληρώσει ένα κύκλο, και έχει πλάτος περίπου 1 μέτρο. Αυτό κάνει τα tsunami δύσκολο να εντοπιστούν σε μεγάλο βάθος.

Όταν το tsunami πλησιάζει την ακτή και τα νερά γίνονται πιο ρηχά, το κύμα συμπιέζεται και η ταχύτητα του μειώνεται σε περίπου 80 χιλιόμετρα την ώρα. Το μήκος κύματος του μειώνεται σε λιγότερο από 20 χιλιόμετρα, αλλά το πλάτος του αυξάνεται εξαιρετικά. Λόγω του ότι το κύμα έχει ακόμα την ίδια μεγάλη περίοδο, το tsunami μπορεί να χρειαστεί ολόκληρα λεπτά για να φτάσει το μέγιστο ύψος. Με εξαίρεση τα πολύ μεγάλα tsunami, το κύμα δεν σπάει, αλλά εμφανίζεται σαν ένα γρήγορα αυξανόμενο παλιρροιακό κύμα. Το 80\% των tsunami συναντώνται στον Ειρηνικό ωκεανό, αλλά μπορούν να εμφανιστούν σε όποια περιοχή έχει μεγάλο όγκο νερού, συμπεριλαμβανομένου των μεγάλων λιμνών.

\subsubsection{Προειδοποιήσεις και προβλέψεις}
Κάθε κύμα, έχει μια θετική και μια αρνητική κορυφή. Στην περίπτωση του tsunami, μπορεί να φτάσει οποιαδήποτε από τις δύο. Αν η θετική κορυφή φτάσει πρώτη στην στεριά, τότε μια ξαφνική πλημμύρα θα εμφανιστεί στην στεριά. Αν όμως η αρνητική κορυφή φτάσει πρώτη στην στεριά, τότε θα δημιουργηθεί μια υποχώρηση του νερού, που θα εμφανίσει την βυθισμένη επιφάνεια. Αυτή η επιφάνεια μπορεί να φτάσει εκατοντάδες μέτρα. Ένα τυπικό κύμα tsunami έχει περίοδο περίπου 12 λεπτών. Αυτό σημαίνει ότι αν η αρνητική κορυφή φτάσει πρώτη στην στεριά, η θάλασσα θα υποχωρήσει, εμφανίζοντας την βυθισμένη επιφάνεια μέσα σε 3 λεπτά. Μετά από 6 λεπτά τα κύματα του tsunami μετατρέπονται σε θετική κορυφή, όπου η θάλασσα επανέρχεται και δημιουργεί καταστροφές στην στεριά. Η διαδικασία αυτή επαναλαμβάνεται μόλις φτάσει το επόμενο κύμα.

Το tsunami δεν μπορεί να προβλεφθεί με ακρίβεια, ακόμα και αν το μέγεθος και η περιοχή ενός σεισμού είναι γνωστή. Οι γεωλόγοι, ωκεανογράφοι, και σεισμολόγοι αναλύουν τους σεισμούς και βασισμένοι σε πολλούς παράγοντες προειδοποιούν ή όχι για περιπτώσεις tsunami. Όμως, υπάρχουν προειδοποιητικές πινακίδες σε περιοχές που είναι ευπαθείς σε tsunami, και τα αυτοματοποιημένα συστήματα παρέχουν προειδοποιήσεις αμέσως μετά από ένα σεισμό, ώστε να υπάρχει αρκετός χρόνος για να σωθούν ζωές. Ένα από τα πιο πετυχημένα συστήματα χρησιμοποιεί αισθητήρες χαμηλής πίεσης, που συνεχώς ελέγχουν την πίεση του νερού.

\subsection{Προγράμματα}
\begin{apptable}{Δυναμικές Ρευστών}{fluiddynamics}
Altair AcuSolve & Λογισμικό CFG γενικού σκοπού & 2x & ΝΑΙ \\ \hline
ANSYS Fluent  & Λογισμικό CFG γενικού σκοπού & 2-2.5x & ΝΑΙ \\ \hline
Autodesk Moldflow  & Λογισμικό ψεκασμού με πλαστικό καλούπι & 1.5x & ΝΑΙ \\ \hline
Barracuda & Λογισμικό προσομοίωσης υγροποιημένου στρώματος & - & ΝΑΙ \\ \hline
FluiDyna Culises for OpenFOAM & Βιβλιοθήκη για λογισμικά CFG γενικού σκοπού & 3x & ΝΑΙ \\ \hline
FluiDyna LBultra & Λογισμικό CFG γενικού σκοπού  & 3x & ΝΑΙ \\ \hline
Prometech Particleworks & Βασισμένο σε σωματίδια λογισμικό CFD & 4x - 9x & ΝΑΙ \\ \hline
Turbostream Ltd.  & Λογισμικό CFD για ροές μηχανών τουρμπίνας  & 19x & ΝΑΙ \\ \hline
Vratis ARAEL  & Λογισμικό CFG γενικού σκοπού βασισμένο στο FVM με υποστήριξη OpenFOAM & 3x & ΝΑΙ \\ \hline
Vratis SpeedIT extreme for OpenFOAM  & Βιβλιοθήκη για λογισμικά CFG γενικού σκοπού & 3x & ΝΑΙ \\ \hline
DualSPHysics & Λογισμικό CFD βασισμένο σε SPH & - & ΝΑΙ \\ \hline
FEFLO (GMU-Lohner)  & Λογισμικό CFG γενικού σκοπού για συμπιεζόμενες και αποσυμπιεζόμενες ροές & 2x-10x & ΝΑΙ \\ \hline
NASA FUN3D & Λογισμικό CFG γενικού σκοπού & - & ΝΑΙ \\ \hline
S3D(Sandia and OakRidge NL) & Άμεσος αριθμητικός λύτης (DNS) για τυρβώδη καύσεις & 5x & ΝΑΙ \\ \hline
SD++ (Stanford-Jameson) & Λογισμικό CFG γενικού σκοπού για συμπιεζόμενες ροές. & 15x & ΝΑΙ \\ \hline
\end{apptable}

\newpage