\section{Βιοπληροφορική}
\subsection{Εισαγωγή}
Η συνεχής αύξηση της ποσότητας βιολογικών δεδομένων, η ανάγκη για ανάλυση τους και το συνεχές ενδιαφέρον από την επιστημονική κοινότητα για την κατανόηση των δομικών λειτουργιών των βιολογικών μορίων, αποτέλεσαν τους κύριους λόγους για την ανάπτυξη της βιοπληροφορικής. Για να κατανοήσουμε τις κυτταρικές και βιομοριακές λειτουργίες, τα βιολογικά δεδομένα πρέπει να συνενωθούν για να σχηματίσουν μια ακριβής εικόνα. Οι ερευνητές της βιοπληροφορικής, έχουν αναπτύξει υπολογιστικές τεχνικές για την επεξεργασία των βιολογικών δεδομένων, όπως νουκλεοτιδικές αλληλουχιών, αλληλουχίες αμινο οξέων, τρισδιάστατων δομών, όπως επίσης βιολογικών σημάτων και εικόνων. Μεγάλες ερευνητικές προσπάθειες του πεδίου συμπεριλαμβάνουν αναγνώριση προτύπων, ευθυγράμμιση αλληλουχιών, ανάλυση πρωτεϊνικών δομών, φυλογενητική ανάλυση, μοριακή δυναμική, ανάλυση γονιδιώματος, σχεδιασμός φαρμάκων και ανάπτυξη φαρμάκων. Επίσης, υπάρχουν προφητικές τεχνικές ειδικές για τις εκφράσεις γονιδίων, και την αλληλεπίδραση πρωτεϊνών.\\
\begin{figure}[h]
\centering
\includegraphics[scale=0.25]{bioinformatics}
\caption{Βιολογία και πληροφορική}
\end{figure}
Η Βιοπληροφορική παίζει μεγάλο ρόλο σε πολλές πτυχές της βιολογίας. Στην πειραματική μοριακή βιολογία, οι τεχνικές βιοπληροφορικής όπως επεξεργασία εικόνας και σήματος, επιτρέπει την εξόρυξη χρήσιμων αποτελεσμάτων από μεγάλο όγκο δεδομένων. Στο πεδίο της γενετικής και γονιδιωματικής, συμβάλλει στην αλληλουχία και υποσημείωση γονιδιωμάτων και την παρατήρηση των μεταλλάξεων τους.\cite{bioinformatics-1} Παίζει μεγάλο ρόλο στην εξόρυξη τεχνικών όρων και στην κατασκευή βιολογικών και γονιδιακών οντολογιών για την οργάνωση και αναζήτηση βιολογικών δεδομένων. Έχει επίσης μεγάλο ρόλο στην ανάλυση των γονιδίων και στην ρύθμιση πρωτεϊνών. Τα εργαλεία της Βιοπληροφορικής συμβάλουν στην σύγκριση γενετικών και γονιδιακών δεδομένων και γενικότερα στην κατανόηση των αναπτυξιακών πτυχών της μοριακής βιολογίας. Σε πιο εσωτερικό επίπεδο, συμβάλει στην ανάλυση και κατηγοριοποίηση των βιολογικών διαδρόμων και δικτύων τα οποία είναι σημαντικό κομμάτι της συστεμικής βιολογίας. Στην Δομική βιολογία, συμβάλει στην εξομοίωση και μοντελισμό του DNA, RNA, και δομές πρωτεϊνών όπως και μοριακών αλληλεπιδράσεων.\\

\subsection{Μοριακή δυναμική}
Η Βιοπληροφορική είναι ένα επιστημονικό πεδίο που εστιάζει στην εφαρμογή της τεχνολογίας υπολογιστών στην διαχείριση βιολογικών δεδομένων. Με το πέρασμα του χρόνου, οι εφαρμογές βιοπληροφορικής έχουν χρησιμοποιηθεί για να αποθηκεύσουν, αναλύσουν και να ενσωματώσουν βιολογικές και γενετικές πληροφορίες, χρησιμοποιώντας ένα μεγάλο εύρος μεθοδολογιών. Μια από τις πλέον γνωστές τεχνικές για την κατανόηση των φυσικών κινήσεων των ατόμων και των μορίων, είναι η μοριακή δυναμική. Η μοριακή δυναμική είναι μια μέθοδος εξομοίωσης των φυσικών κινήσεων των ατόμων και των μορίων κάτω από συγκεκριμένες συνθήκες. Έχει ρόλο κλειδί σε επιστήμες όπως η βιολογία, η χημεία, η φυσική, ιατρική. Λόγω της πολυπλοκότητας τους, οι υπολογισμοί της μοριακής δυναμικής χρειάζονται μεγάλες ποσότητες μνήμης και υπολογιστικής δύναμης, και για αυτό η εκτέλεση τους είναι συχνά μεγάλο πρόβλημα.\cite{bioinformatics-3} \\
Οι εξομοιώσεις της μοριακής δυναμικής χρησιμοποιούν πολύπλοκους αριθμητικούς υπολογισμούς, που πολλές φορές οδηγούν σε αριθμητικά λάθη. Πριν την ανακάλυψη του προγραμματισμού γενικής χρήσης, οι GPU χρησιμοποιούνταν μόνο για διαδικασίες απεικόνισης των μοριακών δομών, και η εκτέλεση των αλγορίθμων μοριακής δυναμικής μπορούσε να διαρκέσει από ώρες, έως και μέρες. Η λύση προήλθε από το GPGPU, καθώς οι GPU έχουν πολλές αριθμητικές μονάδες που μπορούν να εκτελεστούν παράλληλα. \\
Στο πεδίο της μοριακής δυναμικής, έχουν αναπτυχθεί πολλές εφαρμογές εφαρμογές βασισμένα στο GPGPU, που υποστηρίζουν εξομοιώσεις σε πολλαπλές μονάδες. Αυτή η καινοτομία δημιουργεί ευκαιρίες για το μέλλον, ειδικά για μικρότερες ερευνητικές ομάδες. Μειώνει τον χρόνο που απαιτείται για διαδικασίες και τα απαραίτητα κονδύλια για έρευνα-ανάπτυξη, προάγει την ανάπτυξη καινούριων εφαρμογών και την επιστημονική πρόοδο.\cite{bioinformatics-4}
\subsubsection{Μετάβαση από CPU σε GPU}
Η διαφορά στην αρχιτεκτονική μεταξύ CPU και GPU, είναι ότι στην τελευταία είναι δυνατή η εκτέλεση πολλαπλών παράλληλων διεργασιών, κάτι που επιτρέπει την καλύτερη εκτέλεση πολύπλοκων αλγορίθμων και καλύτερη διαχείριση μεγάλου όγκου δεδομένων. Επίσης, μια μονάδα επεξεργασίας γραφικών έχει λιγότερες ενεργειακές απαιτήσεις, έτσι η δημιουργία υπερ-υπολογιστών με χρήση GPU εξαλείφει την ανάγκη για τεράστιους χώρους γεμάτους με υπολογιστές. Η εγκατάσταση μιας επιπλέον μονάδας, αντιγράφει τον παραλληλισμό του προγραμματισμού, χωρίς καμία επιπλέον ενέργεια. Επιπλέον, η GPU έχει εντυπωσιακές δυνατότητες υπολογισμού floating point και μεγάλο εύρος ζώνης μνήμης, δίνοντας την δυνατότητα για βελτιστοποιημένη πρόσβαση στην μνήμη, ελεγχόμενη εκτέλεση επιλογών, και διαχείριση πόρων, με χρήση λίγων γραμμών κώδικα. Συγκεκριμένα, οι εφαρμογές μοριακής δυναμικής, κβαντικής χημείας, η απεικόνιση των αποτελεσμάτων τους, τρέχουν μέχρι και 5 φορές πιο γρήγορα. \\
Από την αρχή του GPU προγραμματισμού μέχρι και σήμερα, η προγραμματιστική ανάπτυξη συνεχίζεται αδιάκοπα. Ο αριθμός των εφαρμογών βασισμένων σε αρχιτεκτονικές GPU, φτάνουν τις 200, το οποίο είναι αύξηση της τάξεως πάνω από 60\% μέσα σε δύο χρόνια. Οι καλύτερες εφαρμογές βασισμένες σε GPGPU έχουν σχεδιαστεί για την μοριακή δυναμική, τον σχεδιασμό φαρμάκων, κβαντική χημεία, το κλίμα, την φυσική σύμφωνα με την Nvidia\cite{bioinformatics-2}
\subsection{GPUGRID.net}
\epigraph{"I hope mankind will acknowledge people like you, its real heroes."}{Grzegorz Granowski, Volunteer \& Donor}
Το GPUGRID είναι ένα εθελοντικό κατανεμημένο σύστημα, το οποίο στοχεύει στην βιοϊατρική έρευνα από το πανεπιστήμιο Universitat Pompeu Fabra της Ισπανίας. Το GPUGRID αποτελείται από πολλές μονάδες επεξεργασίας γραφικών, που συνεργάζονται μεταξύ τους για να παραδώσουν υψηλών επιδόσεων εξομοιώσεις βιομορίων. Οι μοριακές εξομοιώσεις πού εκτελούνται από τους εθελοντές του, αποτελούν μερικούς απο τους πιο συνήθης τύπους εξομοιώσεων που εκτελούνται απο τους επιστήμονες του πεδίου, αλλά ταυτόχρονα είναι από τους πιο απαιτητικούς σε υπολογιστική δύναμη και συνήθως απαιτούν υπερ-υπολογιστές.\\
\begin{figure}[h]
\centering
\includegraphics[scale=0.75]{gpugrid}
\caption{Βιολογία και πληροφορική}
\end{figure}\\
Το σύστημα ερευνά μεταξύ άλλων τα παρακάτω προβλήματα
\begin{itemize}
\item Εξομοίωση της ωρίμανσης πρωτεολυτικών του HIV - Μια απο τις πιο σημαντικές πτυχές της ωρίμανσης του HIV είναι το πώς η πρωτεΐνη "ψαλιδιών", δημιουργείται. Η απάντηση σε αυτό το ερώτημα χρειάζεται εξομοιώσεις μοριακής δυναμικής στο όριο των μοντέρνων υπολογιστικών δυνατοτήτων. Το GPUGRID μας επιτρέπει να λύσουμε αυτο το πρόβλημα και έχουμε καταφέρει να δείξουμε οτι τα πρώτα "ψαλίδια" κόβονται απο το "σκοινί" που είναι δεμένα. Αυτό το γεγονός συμβαίνει στην αρχή της ωρίμανσης, και αν σταματήσουμε την ωρίμανση των πρωτεολυτικών, τότε θα σταματήσουμε και την ωρίμανση του HIV σαν σύνολο.\cite{gpugrid-1}
\item Ανακάλυψη του ρόλου των μεμβρανών λιπιδίων στην δραστηριότητα ενζύμων.
\item Μοριακή εξομοίωση αισθητήρων ντοπαμίνης κάτω απο φυσιολογικές ιονικές δυνάμεις.
\item Αποκάλυψη των μηχανισμών αντίδρασης φαρμάκων καρκίνου παχέος εντέρου - Ο καρκίνος είναι βασικά ή ανεξέλεγκτη ανάπτυξη ιστών και εισβολή από μεταλλαγμένα κύτταρα σε έναν οργανισμό. Σε αντίθεση με τις παραδοσιακές χημειοθεραπείες ή ραδιοθεραπείες, οι νεότερες θεραπείες στοχεύουν σε συγκεκριμένους στόχους κακοήθων κυττάρων. Αυτό επιτυγχάνεται με τον εντοπισμό ορισμένων πρωτεϊνών που εκφράζονται διαφορικά σε ογκογεννητικά κύτταρα. Με την βοήθεια του GPUGRID, επιτυγχάνεται η επεξήγηση των μοριακών μηχανισμών που συμβαίνουν στα μεταλλαγμένα μόρια των κυττάρων.
\end{itemize}
Η εκτέλεση του GPUGRID στις GPUs, καινοτομεί στον εθελοντικό υπολογισμό, παραδίδοντας εφαρμογές υπερ-υπολογιστών, σε υποδομές χαμηλού κόστους. Η απόδοση των μονάδων γραφικής επεξεργασίας, καταγράφεται και συγκρίνεται σε σχέση με άλλους χρήστες, ανάλογα με την διάρκεια ολοκλήρωσης των WU (Work Units)\cite{gpugrid-2}.\\
\begin{figure}[h]
\centering
\includegraphics[scale=0.75]{gpugrid-charts}
\caption{Βιολογία και πληροφορική}
\end{figure}

\subsection{Προγράμματα}
\begin{apptable}{Βιοπληροφορική}{bioinformatics}
BarraCUDA & Λογισμικό χαρτογράφησης ακολουθίας & 6-10x & ΝΑΙ \\ \hline
CUDASW++ & Λογισμικό ανοιχτού κώδικα για αναζητήσεις Smith-Waterman σε πρωτεϊνικές βάσεις δεδομένων με χρήση GPUs & 10-50x & ΝΑΙ \\ \hline
CUSHAW & Ευθυγραμμιστής παράλληλων μικρών προσπελάσεων  & 10x & ΝΑΙ \\ \hline
G-BLASTN & Επιταχυνόμενο από GPU εργαλείο ευθυγράμμισης νουκλεοτιδίων βασισμένο στο ευρέως διαδεδομένο NCBI-BLAST & 4-15x & \\ \hline
GPU-BLAST & Τοπική αναζήτηση με γρήγορους ευρετικούς  αλγόριθμους k-tuple & 3-4x & \\ \hline
mCUDA-MEME & Πολύ γρήγορη κλιμακωτή ανακάλυψη μοτίβων βασισμένη στο MEME & 4-10x & ΝΑΙ \\ \hline
MUMmer GPU & Πρόγραμμα υψηλής απόδοσης τοπικής ευθυγράμμισης ακολουθίας & 3-10x & \\ \hline
NVBIO & Βιβλιοθήκη ανοιχτού κώδικα C++ αποτελούμενη από στοιχεία επαναχρησιμοποιήσιμα σχεδιασμένα για να επιταχύνουν εφαρμογές βιοπληροφορικής με χρήση CUDA. & 4-5x & ΝΑΙ \\ \hline
NVBowtie & Μια σε μεγάλο βαθμό ολοκληρωμένη εφαρμογή του ευθυγραμμιστή Bowtie2 πάνω από το NVBIO & 2.75x-8.35x & ΝΑΙ \\ \hline
PEANUT & Καθορισμός ανάγνωσης για ακολουθίες DNA ή RNA σε γνωστή αναφορά γονιδιώματος. & 10x & \\ \hline
REACTA & Ο σκοπός του REACTA είναι ο ποσοτικός προσδιορισμός της συμβολής της γενετικής διακύμανσης στην φαινοτυπική διακύμανση σε πολύπλοκα χαρακτηριστικά. & 2-4x & ΝΑΙ \\ \hline
SeqNFind & Ακολουθία επόμενης γενιάς και συγκρίσεις γονιδιωμάτων & 400x & ΝΑΙ \\ \hline
SOAP3 & Λογισμικό βασισμένο σε GPU για ευθυγράμμιση μικρών προσπελάσεων με αναφορά ακολουθίας. Μπορεί να βρει όλες τις ευθυγραμμίσεις με k ασυμφωνίες, όπου το k είναι ένας αριθμός απο το 0 έως το 3 & 10x & ΝΑΙ \\ \hline
SOAP3-dp & Πολύ γρήγορο εργαλείο βασισμένο σε GPU για ευθυγραμμίσεις μικρών προσπελάσεων μέσω δυναμικού προγραμματισμού υποβοηθούμενου από ευρετήριο & 28-64x & ΝΑΙ \\ \hline
UGENE & Λογισμικό ανοιχτού κώδικα Smith-Waterman για SSE/CUDA, επαναλήψεις βασισμένες σε πίνακα δεικτών & 6-8x & ΝΑΙ \\ \hline
WideLM & Ταιριάζει πολυάριθμα γραμμικά μοντέλα σε μια σταθερή σχεδίαση και απάντηση & 150x & ΝΑΙ \\ \hline
\end{apptable}


%\begin{apptable}{Μοριακή δυναμική}{molecular}
%ACEMD & GPU simulation of molecular mechanics force fields, implicit and explicit solvent & & \\ \hline
%AMBER & Suite of programs to simulate molecular dynamics on biomolecule & & \\ \hline
%CHARMM & MD package to simulate molecular dynamics on biomolecule & & \\ \hline
%DESMOND & High-speed molecular dynamics simulations of biological systems & & \\ \hline
%DL-POLY & Simulate macromolecules, polymers, ionic systems, etc on a distributed memory parallel computer & & \\ \hline
%ESPResSo & Highly versatile software package for performing and analyzing scientific Molecular Dynamics. & & \\ \hline
%Folding@Home & A distributed computing project that studies protein folding, misfolding, aggregation, and related diseases & & \\ \hline
%GPUGrid.net & A distributed computing project that uses GPUs for molecular simulations & & \\ \hline
%GROMACS & Simulation of biochemical molecules with complicated bond interactions & & \\ \hline
%HALMD & Large-scale simulations of simple and complex liquids & & \\ \hline
%HOOMD-Blue & Particle dynamics package written grounds up for GPUs & & \\ \hline
%LAMMPS & Classical molecular dynamics package & & \\ \hline
%NAMD & Designed for high-performance simulation of large molecular systems & & \\ \hline
%OpenMM & Library and application for molecular dynamics for HPC with GPUs & & \\ \hline
%\end{apptable}