\section{Εικόνα και Βίντεο}

\subsection{Ιχνογράφηση ακτίνας}
\subsubsection{Πραγματικότητα}
Στην φύση, μια πηγή φωτός εκπέμπει μια ακτίνα φωτός όταν ταξιδεύει, προς μια επιφάνεια που εμποδίζει την πρόοδο της. Μπορούμε να σκεφτούμε την "ακτίνα" σαν μια ροή φωτονίων που ταξιδεύουν προς το ίδιο μονοπάτι. Στο απόλυτο κενό, αυτή η ακτίνα θα είναι μια ευθεία γραμμή, αν αγνοήσουμε την δράση της σχετικότητας. Ένας συνδυασμός της απορρόφησης, ανάκλασης, διάθλασης και φθορισμού μπορεί να σχηματιστεί από αυτήν την ακτίνα. Μια επιφάνεια μπορεί να απορροφήσει μέρος της ακτίνας, που θα έχει ως αποτέλεσμα την μείωση της έντασης της ανακλώμενης ακτίνας. Μπορεί επίσης να δημιουργήσει αντανάκλαση όλης η μέρος της ακτίνας, προς μία ή παραπάνω κατευθύνσεις. Αν η επιφάνεια έχει διάφανες ιδιότητες, θα διαθλάσει την ακτίνα στην ίδια και σε άλλη μια κατεύθυνση, και θα απορροφήσει μόνο μέρος της ακτίνας, ίσως αλλάζοντας και το χρώμα της. Έπειτα, οι ακτίνες μπορεί να φτάσουν σε επιπλέον επιφάνειες, όπου οι ιδιότητες τους θα επηρεάσουν ξανά την πρόοδο τους. Πολλές από αυτές τις ακτίνες, ταξιδεύουν με τρόπο που φτάνουν στο μάτι μας, επιτρέποντας μας να δούμε την σκηνή και να συνεισφέρουν στην τελική εικόνα. 


\subsubsection{Γραφικά υπολογιστών}
Στα γραφικά των υπολογιστών, η ιχνογράφηση ακτίνας (ray tracing) είναι μια τεχνική για παραγωγή μιας εικόνας που χρησιμοποιεί ιχνογράφηση στο μονοπάτι του φωτός μέσα στον χώρο μιας εικόνας και εξομοιώνει τα εφέ τις επίδρασης του στα εικονικά αντικείμενα. Η τεχνική είναι δυνατόν να παράγει πολύ υψηλό βαθμό εικονικού ρεαλισμού, συνήθως μεγαλύτερο από τις τυπικές μεθόδους, αλλά με πολύ μεγαλύτερο υπολογιστικό κόστος. Αυτό κάνει την ιχνογράφηση ακτίνας πιο χρήσιμη για εφαρμογές όπου η εικόνα μπορεί να δημιουργηθεί σε βάθος του χρόνου, όπως οι σταθερές εικόνες και τα εφέ ταινιών, ενώ είναι συνήθως λιγότερο επιθυμητή σε εφαρμογές πραγματικού χρόνου, όπως τα παιχνίδια όπου η ταχύτητα είναι σημαντική. Η ιχνογράφηση ακτίνας μπορεί να εξομοιώσει ένα μεγάλο αριθμό από οπτικά εφέ, όπως η αντανάκλαση και η διάθλαση, διασκόρπιση και φαινόμενα διασποράς.

\begin{figure}[h]
\centering
\includegraphics[width=\linewidth]{ray-tracing-1}
\caption{Κάθε ακτίνα χρησιμοποιεί αντανάκλαση έως και 16 φορές\cite{figure-15}}
\end{figure}

\subsubsection{Ιστορία}
Ο πρώτος αλγόριθμος ιχνογράφησης ακτίνας παρουσιάστηκε απο τον Arthur Appel το 1968. Ο αλγόριθμος ονομάστηκε από τότε διάχυση ακτίνας. Ένα σημαντικό πλεονέκτημα της ιχνογράφησης ακτίνας προς τους παραδοσιακούς scanline αλγορίθμους είναι η δυνατότητα να αντιμετωπίζει επιφάνειες όπως οι κώνοι και οι σφαίρες. Η επόμενη μεγάλη ανακάλυψη ήρθε απο τον Turner Whitted το 1979. Ενώ οι παλιοί αλγόριθμοι υπολόγιζαν μόνο την ακτίνα έως την στιγμή που θα έφτανε στο πρώτο εμπόδιο, ο Whittman συνέχισε την διαδικασία. Όταν μια ακτίνα φτάνει σε μια επιφάνεια, μπορεί να παράγει ως και τρεις τύπους ακτίνων: αντανάκλαση, διάθλαση, σκίαση. Αυτή η επαναληπτική ιχνογράφηση, έδωσε περισσότερο ρεαλισμό στις εικόνες.

\subsubsection{Μειονεκτήματα}
Ένα σοβαρό μειονέκτημα της ιχνογράφησης ακτίνας είναι η απόδοση. Οι τυπικοί αλγόριθμοι (scanline,κ.α) χρησιμοποιούν στοιχεία συνοχής για να διαμοιράσουν τους υπολογισμούς μεταξύ των εικονοστοιχείων, ενώ η ιχνογράφηση ακτίνας συνήθως ξεκινάει μια καινούρια διαδικασία, θεωρώντας την κάθε ακτίνα σαν ξεχωριστή οντότητα. Αυτός ο διαχωρισμός όμως δίνει και πλεονεκτήματα, όπως η ικανότητα να έχουμε περισσότερες ακτίνες για να δημιουργήσουμε anti-aliasing διαστήματος και να βελτιώσουμε την ποιότητα της εικόνας όπου χρειάζεται.

Αν και χειρίζεται τα οπτικά εφέ όπως η διασκόρπιση με ακρίβεια, η παραδοσιακή ιχνογράφηση ακτίνας δεν είναι πάντα φωτορεαλιστική. Ο πραγματικός φωτορεαλισμός επιτυγχάνεται όταν η εξίσωση απόδοσης προσεγγίζεται πολύ κοντά ή εκτελείται ολοκληρωτικά. Η εκτέλεση της εξίσωσης απόδοσης δίνει αληθινό φωτορεαλισμό, καθώς η εξίσωση περιγράφει κάθε φυσικό εφέ ή ροή φωτός. Συνήθως όμως, λόγω υπολογιστικών περιορισμών δεν είναι δυνατόν να συμβεί.

Ο ρεαλισμός των μεθόδων απόδοσης μπορεί να αξιολογηθεί ως προσέγγιση της εξίσωσης. Η ιχνογράφηση ακτίνας, αν περιοριστεί στον αλγόριθμο του Whitted δεν είναι απαραίτητα η πιο ρεαλιστική. Οι μέθοδοι που ιχνογραφούν ακτίνες, αλλά περιέχουν επιπλέον τεχνικές (καταγραφή φωτονίων, ιχνογράφηση μονοπατιού), δίνουν μεγαλύτερη ακρίβεια εξομοίωσης του φωτισμού στον πραγματικό κόσμο.

Είναι επίσης δυνατό να προσεγγίσουμε την εξίσωση χρησιμοποιώντας διάχυση ακτίνας με διαφορετικό τρόπο από ότι χρησιμοποιείται συνήθως στην ιχνογράφηση ακτίνας. Για λόγους απόδοσης, οι ακτίνες μπορούν να συμπλεχθούν σε σχέση με την κατεύθυνση τους, με ειδικό υλικό να υπολογίζει τις ακτίνες.

\subsubsection{Προγράμματα}
Η μαζική παράλληλη υπολογιστική δύναμη των GPUs ταιριάζει τέλεια με την παράλληλη φύση της ιχνογράφησης ακτίνας, και βελτιώνει αισθητά τον χρόνο απόδοσης για διάφορες βιομηχανίες, χρησιμοποιώντας διάφορους τύπους τεχνικών. Οι GPUs είναι ένα απαραίτητο εργαλείο για τις εταιρίες απόδοσης. Παρακάτω βλέπουμε μια λίστα από προγράμματα που χρησιμοποιούν αυτήν την υπολογιστική δύναμη για να επιταχύνουν αισθητά τις υλοποιήσεις τους και να ορίσουν νέες διαδραστικές δυνατότητες.

\begin{figure}[h]
\centering
\includegraphics[width=\linewidth]{ray-tracing-2}
\caption{Ανταπόδοση φωτορεαλιστικής εικόνας σε 3,5 λεπτά με την χρήση του Furryball\cite{figure-16}}
\end{figure}


\begin{itemize}
\item Nvidia Optix - Είναι ένα εργαλείο ιχνογράφησης ακτίνας για προγραμματιστές υπολογιστών που δημιουργούν εφαρμογές που δίνουν γρήγορα αποτελέσματα. Σε αντίθεση με υλοποιήσεις με υπαγορευμένη αισθητική, ή που περιορίζουν τις δομές δεδομένων τους ή και της γλώσσας απόδοσης τους, η μηχανή Optix είναι υπερβολικά γενικευμένη, δίνοντας την δυνατότητα στους προγραμματιστές να επιταχύνουν ότι διεργασία ιχνογράφησης ακτίνας θέλουν και να την εκτελέσουν σε μονάδες επεξεργασίας γραφικών
\item Arion -  Eίναι ένα προϊόν επόμενης γενιάς που χρησιμοποιείται για εξομοίωση φωτός με τεχνολογία CUDA για να επιταχύνει την ανταπόδοση σε μια ή πολλές μονάδες επεξεργασίας γραφικών. Το Arion είναι ένας εξομοιωτής φωτός βασισμένος σε φυσικά φαινόμενα, ενώ μπορεί να χρησιμοποιήσει CPUs και GPUs για παραγωγή απόδοσης υψηλών απαιτήσεων.
\item Furryball - Είναι ένα πρόγραμμα πραγματικού χρόνου απόδοσης τελικής εικόνας με ανεπτυγμένες δυνατότητες απόδοσης, το οποίο εκτελείται σαν πρόσθετο στις υλοποιήσεις Maya και Autodesk 3ds Max, με ανταπόδοση μέσω δικτύου και υποστήριξη πολλαπλών GPUs. Το Furryball είναι βασισμένο στο Nvidia Optix, ενώ συνδυάζει την ταχύτητα των GPU με την ποιότητα και τις δυνατότητες της απόδοσης CPU. Η δύναμη του έχει αποδειχθεί σε πραγματικές ταινίες και παραγωγές παιχνιδιών από πολλές εταιρίες.
\item Lightworks - Εκμεταλλεύεται το Nvidia Optix για να δημιουργήσει μια καινούρια γενιά πολύ γρήγορων μηχανών ιχνογράφησης ακτίνας για αρχιτεκτονικές, βιομηχανικές ανάγκες, και ανάγκες εσωτερικής διακόσμησης χώρων.
\item Octane Renderer - Είναι μια διαδραστική μηχανή απόδοσης βασισμένη σε GPU, που παράγει φωτορεαλιστικές ανταποδόσεις με μεγάλη ταχύτητα. Αυτό επιτρέπει στους χρήστες να δημιουργήσουν έργα υψηλής αισθητικής, σε ένα κλάσμα του χρόνου που χρειάζεται από τις μηχανές απόδοσης σε CPU. Η εταιρία που δημιουργεί το Octane Renderer είναι μια απο τις πρώτες που ασχολήθηκαν με τις μηχανές φωτορεαλιστικής απόδοσης και έπαιξε μεγάλο ρόλο στην εξέλιξη τους.
\item V-ray - Είναι μια μηχανή απόδοσης εικόνας που χρησιμοποιεί τις διαθέσιμες GPUs που βρίσκονται στο σύστημα, αντί για την CPU. Χρησιμοποιεί OpenCL, και βρίσκεται ως πρόσθετο στο πρόγραμμα Autodesk 3ds Max.
\end{itemize}



\begin{apptable}{Μοντελοποίηση}{modeling}
Autodesk 3ds Max + NVIDIA iray & Τρισδιάστατη μοντελοποίηση, κινούμενα σχέδια, και ανταπόδοση & - & ΝΑΙ \\ \hline
Autodesk Maya & Τρισδιάστατη μοντελοποίηση, κινούμενα σχέδια, και ανταπόδοση & - & ΝΑΙ \\ \hline
Autodesk Motion Builder & Κίνηση χαρακτήρων και καταγραφή κίνησης & - & ΝΑΙ \\ \hline
Autodesk Mudbox & Τρισδιάστατη απεικόνιση αγαλμάτων & - & ΝΑΙ \\ \hline
Cebas finalRender & Ανταπόδοση με χρήση GPU  & - & ΝΑΙ \\ \hline
CentiLeo GPU Render & Ανταπόδοση με χρήση GPU & - & ΝΑΙ \\ \hline
Chaos V-Ray RT  & Ανταπόδοση με χρήση GPU  & - & ΝΑΙ \\ \hline
Jawset TurbulenceFD & Πρόσθετο για εξομοίωση φυσικής & - & ΝΑΙ \\ \hline
Maxon Cinema 4D & Τρισδιάστατη μοντελοποίηση, κινούμενα σχέδια, και ανταπόδοση & - & ΝΑΙ \\ \hline
NewTek Lightwave & Τρισδιάστατη μοντελοποίηση, κινούμενα σχέδια, και ανταπόδοση & - & ΝΑΙ \\ \hline
Otoy Octane Render & Ανταπόδοση με χρήση GPU  & - & ΝΑΙ \\ \hline
Pixologic Sculptris & Τρισδιάστατη απεικόνιση αγαλμάτων  & - & ΝΑΙ \\ \hline
Redshift Renderer & Επιταχυνόμενο από GPU πρόγραμμα απόδοσης & - & ΝΑΙ \\ \hline
Side Effects Houdini & Τρισδιάστατη μοντελοποίηση, κινούμενα σχέδια, και ανταπόδοση & - & ΝΑΙ \\ \hline
The Foundry Mari & Τρισδιάστατη ζωγραφική & - & ΝΑΙ \\ \hline
\end{apptable}

\subsection{Μετατροπή}
Η μετατροπή είναι η άμεση μετατροπή ψηφιακής κωδικοποίησης, όπως για παράδειγμα σε αρχεία μιας ταινίας(π.χ .mp4, avi), αρχεία ήχου(.wav .mp3), ή κωδικοποίηση χαρακτήρων. (π.χ UTF-8, ISO/IEC 8859). Αυτή γίνεται σε περιπτώσεις που η συσκευή στόχος δεν υποστηρίζει την μορφή του αρχείου ή χρειαζόμαστε μικρότερο μέγεθος αρχείου, ή για να μετατρέψουμε αρχεία παλαιότερης μορφής σε μια πιο σύγχρονη για καλύτερη υποστήριξη σε μελλοντικές εφαρμογές. 

Η μετατροπή χρησιμοποιείται συχνά στα λογισμικά προβολής βίντεο για να ελαττώσουμε το μέγεθος του αρχείου βίντεο. Μια διαδικασία που γίνεται συχνά είναι η μετατροπή από αρχεία MPEG-2(DVD) σε αρχεία μορφής MPEG-4, που ενσωματώνει σύγχρονους αλγόριθμους για καλύτερη ποιότητα εικόνας σε συνδυασμό με μικρότερο μέγεθος αρχείου.\cite{videoediting-2}

\begin{figure}[h]
\centering
\includegraphics[width=0.5\linewidth]{h264-logo}
\caption{H264, το πιο δημοφιλές πρότυπο συμπίεσης βίντεο\cite{figure-17}}
\end{figure}
\subsection{Μειονεκτήματα} 
Το μεγαλύτερο μειονέκτημα της μετατροπής σε απωλεστικές μορφές αρχείου είναι η μειωμένη ποιότητα. Τα τεχνουργήματα συμπίεσης συσσωρεύονται, οπότε κάθε διαδικασία μετατροπής δημιουργεί μια βαθμιαία απώλεια ποιότητας, που είναι γνωστή ως ψηφιακή απώλεια. Για αυτόν τον λόγο, η μετατροπή συνήθως δεν συνίσταται εκτός και αν δεν μπορούμε να την αποφύγουμε.

\subsection{Βίντεο}
H συμπίεση βίντεο χρησιμοποιεί σύγχρονες τεχνικές για να μειώσει πλεονασμούς στα δεδομένα βίντεο. Οι περισσότεροι αλγόριθμοι συμπίεσης βίντεο συνδυάζουν συμπίεση εικόνας και προσωρινή αποζημίωση κίνησης. Ο ήχος κωδικοποιείται παράλληλα με διαφορετικούς αλγόριθμους συμπίεσης άλλα συνήθως συνδυάζεται σε ένα πακέτο όταν ολοκληρωθεί η διαδικασία.

Οι περισσότεροι αλγόριθμοι συμπίεσης βίντεο χρησιμοποιούν απωλεστική συμπίεση, καθώς το ασυμπίεστο βίντεο απαιτεί πολύ μεγάλους ρυθμούς μετάδοσης. Αν και οι περισσότεροι αλγόριθμοι έχουν ένα παράγοντα συμπίεσης 3, μια τυπική συμπίεση βίντεο MPEG-4 μπορεί να έχει παράγοντα συμπίεσης από 20 έως και 200. Όπως σε όλες τις απωλεστικές διαδικασίες συμπίεσης, υπάρχει ένα δίλημμα μεταξύ της ποιότητας του βίντεο, το κόστος της επεξεργασίας της συμπίεσης και της αποσυμπίεσης, και των απαιτήσεων του συστήματος. Υπερβολικά συμπιεσμένο βίντεο μπορεί να δημιουργήσει οπτικά τεχνουργήματα.\cite{videoediting-1}

Συνήθως ο τρόπος με τον οποίο λειτουργεί ένας αλγόριθμος συμπίεσης βίντεο είναι με ομάδες γειτονικών εικονοστοιχείων(pixel), που ονομάζονται blocks. Αυτές οι ομάδες από pixels, συγκρίνονται μεταξύ τους από μια εικόνα στην επόμενη, και ο αλγόριθμος αποστέλλει μόνο τις αλλαγές μεταξύ αυτών των block. Σε περιοχές του βίντεο που υπάρχει μεγαλύτερη κίνηση, ο αλγόριθμος πρέπει να συμπιέσει περισσότερα δεδομένα για να προλάβει τον μεγαλύτερο αριθμό εικονοστοιχείων που αλλάζουν. Συνήθως σε σκηνές με φωτιά, εκρήξεις, καπνούς, το αποτέλεσμα της συμπίεσης έχει μεγαλύτερη απώλεια ποιότητας, η αύξηση του ρυθμού μετάδοσης.

Μερικοί από τους σύγχρονους αλγόριθμους μετατροπής είναι οι παρακάτω
\begin{itemize}
\item Lagarith - Είναι ένας μη απωλεστικός αλγόριθμος ανοιχτού κώδικα που δίνει έμφαση στην ταχύτητα, στην υποστήριξη διαφόρων χώρου χρωμάτων(YV12,RGB,YUY2). Είναι ιδανικός για επεξεργασία και αποθήκευση αρχείων. Αν και υπάρχουν κάποιοι καλύτεροι μη απωλεστικοί αλγόριθμοι συμπίεσης, ο lagarith είναι ο πιο γρήγορος και έτσι έχει κερδίσει την υποστήριξη της κοινότητας.
\item VP9 - Είναι ένα ελεύθερο πρότυπο ανοιχτού κώδικα που αναπτύσσεται από την Google. Είναι ο διάδοχος του VP8. Σκοπός του είναι να μειώσει περισσότερο τον χώρο που απαιτείται από το βίντεο διατηρώντας την ίδια ποιότητα. 
\item H.264 - Είναι ένα πρότυπο συμπίεσης βίντεο, που είναι ίσως το πιο επιτυχημένο για την καταγραφή, συμπίεση, και αναμετάδοση περιεχομένου βίντεο. Χρησιμοποιείται για αναμετάδοση ψηφιακού σήματος τηλεόρασης, δορυφόρων, και από υπηρεσίες αναμετάδοσης στο διαδίκτυο.
\item H.265 - Είναι ένα πρότυπο συμπίεσης βίντεο, διάδοχος του επιτυχημένου H.264. Σκοπός του είναι ο διπλασιασμός της συμπίεσης διατηρώντας την ίδια ποιότητα. Υποστηρίζει αναλύσεις έως και 8192x4320.
\item Daala - Είναι μια τεχνολογία συμπίεσης από το ίδρυμα Xiph.Org. Χρησιμοποιεί περιτύλιξη μεταμορφώσεων για να μειώσει τα οπτικά τεχνουργήματα. Ο στόχος είναι η απόδοση του να ξεπεράσει τις δυνατότητες του VP9 και του H.265. 
\end{itemize}

\subsection{Συμπίεση δεδομένων γενετικής}
Οι γενετικοί αλγόριθμοι συμπίεσης είναι η τελευταία γενιά μη απωλεστικών αλγορίθμων για συμπίεση δεδομένων (συνήθως αλληλουχίες nucleotides) χρησιμοποιώντας συμβατικούς αλγορίθμους συμπίεσης αλλά και γενετικούς αλγόριθμους βελτιστοποιημένους στο συγκεκριμένο τύπο δεδομένων. Το 2012, μια ομάδα επιστημόνων απο το John Kopkins University ανακοίνωσε έναν αλγόριθμο συμπίεσης γενετικής(HapZipper), ο οποίος χρησιμοποιεί HapMap δεδομένα για να πετύχει συμπίεση 95\% μείωση στο μέγεθος αρχείου, πετυχαίνοντας καλύτερη συμπίεση σε πολύ καλύτερο χρόνο από ότι οι γνωστοί μη απωλεστικοί αλγόριθμοι. Άλλοι γενετικοί αλγόριθμοι συμπίεσης π.χ GenomeZip πετυχαίνουν μεγαλύτερη συμπίεση καταφέρνοντας αποθήκευση 6 δισεκατομμυρίων ανθρώπινων ζευγαριών γονιδιώματος σε 2.5 megabyte.

\subsection{Προγράμματα}
\begin{apptable}{Κωδικοποίηση βίντεο}{videoencode}
ArcVideo Core & Εξαιρετικά γρήγορο και υψηλής απόδοσης σύστημα επεξεργασίας βίντεο και κωδικοποίησης & - & ΝΑΙ \\ \hline
ArcVideo Live & Υψηλής πυκνότητας σύστημα επεξεργασίας βίντεο πραγματικού χρόνου και κωδικοποίησης & - & ΝΑΙ \\ \hline
Cinnafilm Tachyon & Μετατροπή προτύπων & - & ΝΑΙ \\ \hline
Digimetrics Aurora & Δοκιμές αυτοματοποιημένου βίντεο και μετρήσεις ήχου & - & ΝΑΙ \\ \hline
Elemental Live & Ζωντανή μετάδοση επεξεργασίας βίντεο και κωδικοποίησης & - & ΝΑΙ \\ \hline
Elemental Server & Επεξεργασία και κωδικοποίηση βίντεο βασισμένο σε αρχεία & - & ΝΑΙ \\ \hline
isovideo Viarte  & Μετατροπή προτύπων βίντεο & - & ΝΑΙ \\ \hline
MainConcept CUDA H.264/AVC Encoder SDK & Κωδικοποίηση βίντεο με χρήση αλγορίθμου H.264 & - & ΝΑΙ \\ \hline
Snell Alchemist on Demand & Μετατροπή προτύπων βίντεο & - & ΝΑΙ \\ \hline
Sorenson Squeeze & Εφαρμογή μετατροπής κωδικοποίησης βίντεο και plug-In & - & ΝΑΙ \\ \hline
Telestream Vantage & Μετατροπή κωδικοποίησης βίντεο και επεξεργασία  & - & ΝΑΙ \\ \hline
\end{apptable}

\begin{apptable}{Επεξεργασία βίντεο}{videoediting}
Adobe Photoshop CC & Επεξεργασία εικόνας & - & ΝΑΙ \\ \hline
Adobe Premiere Pro CC & Επεξεργασία Βίντεο & - & ΝΑΙ \\ \hline
Apple Final Cut Pro & Επεξεργασία Βίντεο  & - & ΝΑΙ \\ \hline
Avid Media Composer & Επεξεργασία Βίντεο  & - & ΝΑΙ \\ \hline
Grass Valley Edius & Επεξεργασία Βίντεο  & - & ΝΑΙ \\ \hline
Harris Velocity & Επεξεργασία Βίντεο  & - & ΝΑΙ \\ \hline
Quantel Qube & Αναμετάδοση επεξεργασίας βίντεο & - & ΝΑΙ \\ \hline
Sony Vegas Pro & Επεξεργασία Βίντεο & - & ΝΑΙ \\ \hline
\end{apptable}


\begin{apptable}{Σύνθεση, Τελειοποίηση, εφέ}{videoeffects}
ABSoft Neat Video  & Πρόσθετο μείωσης θορύβου βίντεο & - & ΝΑΙ \\ \hline
Adobe After Effects CC  & Γραφικά και εφέ κίνησης & - & ΝΑΙ \\ \hline
Autodesk Flame Premium & Τελείωμα και βελτίωση χρώματος & - & ΝΑΙ \\ \hline
Autodesk Smoke  & Τελείωμα και επεξεργασία & - & ΝΑΙ \\ \hline
Boris FX Continuum Complete & Πρόσθετο οπτικών εφέ & - & ΝΑΙ \\ \hline
Cinnafilm Dark Energy Plug-in & Διαχείριση χρωμάτων & - & ΝΑΙ \\ \hline
CoreMelt complete & Πρόσθετο οπτικών εφέ & - & ΝΑΙ \\ \hline
eyeon Fusion  & Εφέ και σύνθεση & - & ΝΑΙ \\ \hline
GenArts Monsters GT & Πρόσθετο οπτικών εφέ & - & ΝΑΙ \\ \hline
GenArts Sapphire & Πρόσθετο οπτικών εφέ & - & ΝΑΙ \\ \hline
HS-ART DIAMANT-Film Restoration & Βελτίωση εικόνας και αφαίρεση αντικειμένων & - & ΝΑΙ \\ \hline
Neat Video Open FX  & Πρόσθετο μείωσης θορύβου βίντεο  & - & ΝΑΙ \\ \hline
NewBlueFX Video Essentials & Πρόσθετο οπτικών εφέ βίντεο & - & ΝΑΙ \\ \hline
NewBlue Titler Pro  & Πρόσθετο για υποστήριξη υποτίτλων βίντεο & - & ΝΑΙ \\ \hline
Pixelan AnyFX  & Πρόσθετο οπτικών εφέ βίντεο  & - & ΝΑΙ \\ \hline
Re:Vision Effects  & Πρόσθετο οπτικών εφέ βίντεο & - & ΝΑΙ \\ \hline
Red Giant Effects Suite  & Πρόσθετα οπτικών εφέ βίντεο  & - & ΝΑΙ \\ \hline
Red Giant Magic Bullet Looks & Εργαλεία χρωμάτων και τελειώματος  & - & ΝΑΙ \\ \hline
The Foundry NUKE and NUKEX & Εργαλεία σύνθεσης με τρισδιάστατο ανιχνευτή & - & ΝΑΙ \\ \hline
Video Copilot Software Element 3D & Σύστημα στοιχείων τρισδιάστατων αντικειμένων & - & ΝΑΙ \\ \hline
Video Copilot Optical Flares & Πρόσθετο οπτικών εφέ για το After Effects  & - & ΝΑΙ \\ \hline
Video Copilot Twitch  & Πρόσθετο οπτικών εφέ για το After Effects  & - & ΝΑΙ \\ \hline
\end{apptable}



\subsection{Επεξεργασία εικόνας}
Οι εικόνες αποθηκεύονται στον υπολογιστή με την μορφή πλέγματος εικονοστοιχείων. Αυτά τα εικονοστοιχεία περιέχουν πληροφορία για το χρώμα και την φωτεινότητα. Οι επεξεργαστές εικόνας μπορούν να αλλάξουν τα εικονοστοιχεία για να βελτιώσουν την εικόνα με πολλούς τρόπους. Τα εικονοστοιχεία μπορούν να αλλάξουν σαν ομάδα, ή ξεχωριστά, από τους ανεπτυγμένους αλγορίθμους των επεξεργαστών εικόνας. Αυτό ισχύει περισσότερο για τις εικόνες τύπου bitmap, ενώ τα προγράμματα διανυσματικών γραφικών όπως Adobe Illustrator ή Inkscape, χρησιμοποιούνται για να δημιουργήσουν και να επεξεργαστούν διανυσματικές εικόνες, που αποθηκεύονται με γραμμές, καμπύλες, και κείμενο, αντί για εικονοστοιχεία. Οι διανυσματικές εικόνες είναι πιο εύκολο να επεξεργαστούν καθώς περιέχουν πληροφορίες για κάθε στοιχείο ξεχωριστά, ενώ μπορούμε να τις απεικονίσουμε κλιμακωτά, σε όποια ανάλυση θέλουμε.\cite{imageediting-1}

Τα προγράμματα επεξεργασίας εικόνας συνήθως προσφέρουν δυνατότητες αυτόματης βελτίωσης εικόνας οι οποίες διορθώνουν την απόχρωση και την φωτεινότητα της εικόνας καθώς και άλλα χαρακτηριστικά επεξεργασίας, όπως αφαίρεση κόκκινων ματιών, βελτίωση οξύτητας, χαρακτηριστικά μεγέθυνσης, και αυτόματη περικοπή. Αυτές οι διαδικασίες ονομάζονται αυτόματες γιατί συνήθως εκτελούνται χωρίς την διαδραστικότητα του χρήστη ή προσφέρονται με ένα πάτημα ενός πλήκτρου ή επιλέγοντας μια επιλογή από ένα μενού.

\begin{figure}[h]
\centering
\includegraphics[width=0.75\linewidth]{image_editing}
\caption{Παράδειγμα οπτικών εφέ ενός επεξεργαστή εικόνας\cite{figure-18}}
\end{figure}

\subsubsection{Συμπίεση}
Πολλές μορφές αρχείων εικόνας χρησιμοποιούν συμπίεση για να μειώσουν το μέγεθος του αρχείου και να μειώσουν τον απαιτούμενο χώρο. Η ψηφιακή συμπίεση εικόνων μπορεί να εκτελεστεί στην κάμερα, ή στον υπολογιστή με την χρήση ενός επεξεργαστή εικόνας. Όταν για παράδειγμα αποθηκεύουμε εικόνες σε μορφή JPEG, η συμπίεση έχει ήδη συμβεί. Οι κάμερες και οι υπολογιστές έχουν δυνατότητες για να επεξεργαστούν το ποσοστό της συμπίεσης.

Κάποιοι άλλοι αλγόριθμοι συμπίεσης, όπως των αρχείων PNG, είναι μη απωλεστικοί, που σημαίνει ότι δεν χάνεται πληροφορία όταν το αρχείο αποθηκεύεται. Σε αντίθεση ο αλγόριθμος JPEG είναι απωλεστικός, και όσο μεγαλύτερη είναι η συμπίεση, τόσο μεγαλύτερη πληροφορία χάνεται, μειώνοντας τελικά την ποιότητα της εικόνας σε σημείο που δεν μπορεί να την επαναφέρουμε. Το JPEG χρησιμοποιεί γνώσεις για τον τρόπο που λειτουργεί ο εγκέφαλος μας στην αναγνώριση εικόνων για να κάνει αυτήν την απώλεια πληροφορίας δυσκολότερη να παρατηρηθεί.

\subsubsection{Βελτίωση εικόνας}
Στα γραφικά υπολογιστών, η διαδικασία για βελτίωση της ποιότητας της ψηφιακής εικόνας που είναι αποθηκευμένη στον υπολογιστή είναι μια από τις πιο σημαντικές λειτουργίες ενός επεξεργαστή εικόνας. Πολλά προγράμματα έχουν την δυνατότητα να προσθέτουν φίλτρα για να μετατρέπουν τις εικόνες με διάφορους τρόπους. Αυτά τα προγράμματα περιέχουν διαδικασίες για εξομάλυνση της οξύτητας της εικόνας και θόλωση εικόνας με διάφορους τρόπους. Οι επεξεργαστές εικόνας έχουν επίσης την δυνατότητα να προσθέτουν ειδικά οπτικά εφέ που δημιουργούν περίεργα αποτελέσματα. Οι εικόνες μπορούν να μετακινηθούν και να επεξεργαστούν με διάφορους τρόπους. Μερικά από τα οπτικά εφέ είναι γεωμετρικές μετατροπές, καλλιτεχνικά εφέ, εφέ υφών, και συνδυασμοί αυτών.\cite{imageediting-2}

\subsubsection{Προγράμματα}

\begin{apptable}{Βελτίωση εικόνας}{imageimprovement}
Adobe SpeedGrade CC & Βαθμολόγηση χρώματος & - & ΝΑΙ \\ \hline
Assimilate Scratch & Βαθμολόγηση χρώματος και τελειώματος & - & ΝΑΙ \\ \hline
Blackmagic DaVinci & Βαθμολόγηση χρώματος & - & ΝΑΙ \\ \hline
Cinnafilm Dark Energy & Εφαρμογή και πρόσθετα για βελτίωση εικόνας & - & ΝΑΙ \\ \hline
Digital Vision Nucoda  & Βαθμολόγηση χρώματος & - & ΝΑΙ \\ \hline
HS-ART DIAMANT-Film & Επαναφορά εικόνας \& βελτίωση & - & ΝΑΙ \\ \hline
Restoration & Επαναφορά εικόνας & - & ΝΑΙ \\ \hline
Marquise Technologies Rain & Βαθμολόγηση χρώματος & - & ΝΑΙ \\ \hline
Red Digital Cinema & Βαθμολόγηση χρώματος & - & ΝΑΙ \\ \hline
SGO Mistika & Βαθμολόγηση χρώματος και τελειώματος & - & ΝΑΙ \\ \hline
The Pixel Farm PFClean & Επαναφορά εικόνας \& βελτίωση & - & ΝΑΙ \\ \hline
\end{apptable}

