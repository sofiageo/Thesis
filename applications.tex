\chapter{Εφαρμογές}
\section{Εισαγωγή}
Τα τελευταία περίπου 20 χρόνια οι εταιρίες παραγωγής υλικού γραφικών έχουν εστιάσει στην προσπάθεια να παράγουν γρήγορες μονάδες γραφικής επεξεργασίας (GPU), ειδικότερα για την κοινότητα των gamer. Αυτό έχει ως αποτέλεσμα πρόσφατα να δημιουργηθούν συσκευές οι επιδόσεις των οποίων ξεπερνούν τις κεντρικές μονάδες επεξεργασίας (CPU), σε συγκεκριμένες εφαρμογές, ειδικότερα σε μετρήσεις εκατομμυρίων εντολών το δευτερόλεπτο (MIPS). Έτσι, καθιερώθηκε μια κοινότητα για να αξιοποιήσει αυτήν την μεγάλη δύναμη των GPU για υπολογισμούς γενικής χρήσης(GPGPU). Τα τελευταία δύο χρόνια έχουν εξαλειφθεί οι περισσότεροι περιορισμοί που υπήρχαν όσον αφορά το σετ εντολών και την διαχείριση μνήμης, με την ενσωμάτωση ενοποιημένων υπολογιστικών μονάδων στις κάρτες γραφικών, δίνοντας έτσι την δυνατότητα στους προγραμματιστές να δημιουργήσουν ένα πλήθος από προγράμματα με εφαρμογές σε πολλούς τομείς.
\section{Κρυπτογράφηση}
Στο πεδίο της ασύμμετρης κρυπτογράφησης, η ασφάλεια όλων των πρακτικών κρυπτοσυστημάτων βασίζεται στην δυσκολία υπολογισμού προβλημάτων, εξαρτημένη από την επιλογή των παραμέτρων. Με την όποια αύξηση των παραμέτρων όμως (συνήθως στο εύρος 1024-4096 bits), οι υπολογισμοί γίνονται όλο και πιο απαιτητικοί για τον εκάστοτε επεξεργαστή. Σε σύγχρονο υλικό, ο υπολογισμός μιας μονής εντολής κρυπτογράφησης δεν είναι κρίσιμος, όμως σε ένα σύστημα επικοινωνίας πολλών-προς-ένα, για παράδειγμα ένας κεντρικός server στο κέντρο δεδομένων μιας εταιρίας, μπορεί να αντιμετωπίσει ταυτόχρονα εκατοντάδες η και χιλιάδες ταυτόχρονες συνδέσεις και εντολές κρυπτογράφησης. Ως αποτέλεσμα, η πιο συνήθης λύση για ένα τέτοιο σενάριο είναι η χρήση καρτών επιτάχυνσης κρυπτογράφησης. Λόγω της μικρής αγοράς, η τιμή τους φτάνει συνήθως αρκετά χιλιάδες ευρώ η δολάρια.\\
Τελευταία, η ερευνητική κοινότητα έχει αρχίσει να εξερευνά τεχνικές για επιτάχυνση των αλγορίθμων κρυπτογράφησης με χρήση της GPU.  