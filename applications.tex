\chapter{Εφαρμογές}
\section{Εισαγωγή}
Τα τελευταία περίπου 20 χρόνια οι εταιρίες παραγωγής υλικού γραφικών έχουν εστιάσει στην προσπάθεια να παράγουν γρήγορες μονάδες γραφικής επεξεργασίας (GPU), ειδικότερα για την κοινότητα των gamer. Αυτό έχει ως αποτέλεσμα πρόσφατα να δημιουργηθούν συσκευές οι επιδόσεις των οποίων ξεπερνούν τις κεντρικές μονάδες επεξεργασίας (CPU), σε συγκεκριμένες εφαρμογές, ειδικότερα σε μετρήσεις εκατομμυρίων εντολών το δευτερόλεπτο (MIPS). Έτσι, καθιερώθηκε μια κοινότητα για να αξιοποιήσει αυτήν την μεγάλη δύναμη των GPU για υπολογισμούς γενικής χρήσης(GPGPU). Τα τελευταία δύο χρόνια έχουν εξαλειφθεί οι περισσότεροι περιορισμοί που υπήρχαν όσον αφορά το σετ εντολών και την διαχείριση μνήμης, με την ενσωμάτωση ενοποιημένων υπολογιστικών μονάδων στις κάρτες γραφικών, δίνοντας έτσι την δυνατότητα στους προγραμματιστές να δημιουργήσουν ένα πλήθος από προγράμματα με εφαρμογές σε πολλούς τομείς.
\section{Κρυπτογράφηση}
Στο πεδίο της ασύμμετρης κρυπτογράφησης, η ασφάλεια όλων των πρακτικών κρυπτοσυστημάτων βασίζεται στην δυσκολία υπολογισμού προβλημάτων, εξαρτημένη από την επιλογή των παραμέτρων. Με την όποια αύξηση των παραμέτρων όμως (συνήθως στο εύρος 1024-4096 bits), οι υπολογισμοί γίνονται όλο και πιο απαιτητικοί για τον εκάστοτε επεξεργαστή. Σε σύγχρονο υλικό, ο υπολογισμός μιας μονής εντολής κρυπτογράφησης δεν είναι κρίσιμος, όμως σε ένα σύστημα επικοινωνίας πολλών-προς-ένα, για παράδειγμα ένας κεντρικός server στο κέντρο δεδομένων μιας εταιρίας, μπορεί να αντιμετωπίσει ταυτόχρονα εκατοντάδες η και χιλιάδες ταυτόχρονες συνδέσεις και εντολές κρυπτογράφησης. Ως αποτέλεσμα, η πιο συνήθης λύση για ένα τέτοιο σενάριο είναι η χρήση καρτών επιτάχυνσης κρυπτογράφησης. Λόγω της μικρής αγοράς, η τιμή τους φτάνει συνήθως αρκετά χιλιάδες ευρώ η δολάρια.\\
Τελευταία, η ερευνητική κοινότητα έχει αρχίσει να εξερευνά τεχνικές για επιτάχυνση των αλγορίθμων κρυπτογράφησης με χρήση της GPU.  

\section{Βιοπληροφορική}
\subsection{Εισαγωγή}
Η συνεχής αύξηση της ποσότητας βιολογικών δεδομένων, η ανάγκη για ανάλυση τους και το συνεχές ενδιαφέρον από την επιστημονική κοινότητα για την κατανόηση των δομικών λειτουργιών των βιολογικών μορίων, αποτέλεσαν τους κύριους λόγους για την ανάπτυξη της βιοπληροφορικής. Για να κατανοήσουμε τις κυτταρικές και βιομοριακές λειτουργίες, τα βιολογικά δεδομένα πρέπει να συνενωθούν για να σχηματίσουν μια ακριβής εικόνα. Οι ερευνητές της βιοπληροφορικής, έχουν αναπτύξει υπολογιστικές τεχνικές για την επεξεργασία των βιολογικών δεδομένων, όπως νουκλεοτιδικές αλληλουχιών, αλληλουχίες αμινο οξέων, τρισδιάστατων δομών, όπως επίσης βιολογικών σημάτων και εικόνων. Μεγάλες ερευνητικές προσπάθειες του πεδίου συμπεριλαμβάνουν αναγνώριση προτύπων, ευθυγράμμιση αλληλουχιών, ανάλυση πρωτεϊνικών δομών, φυλογενητική ανάλυση, μοριακή δυναμική, ανάλυση γονιδιώματος, σχεδιασμός φαρμάκων και ανάπτυξη φαρμάκων. Επίσης, υπάρχουν προφητικές τεχνικές ειδικές για τις εκφράσεις γονιδίων, και την αλληλεπίδραση πρωτεϊνών.\\
\begin{figure}[h]
\centering
\includegraphics[scale=0.25]{bioinformatics}
\caption{Βιολογία και πληροφορική}
\end{figure}
Η Βιοπληροφορική παίζει μεγάλο ρόλο σε πολλές πτυχές της βιολογίας. Στην πειραματική μοριακή βιολογία, οι τεχνικές βιοπληροφορικής όπως επεξεργασία εικόνας και σήματος, επιτρέπει την εξόρυξη χρήσιμων αποτελεσμάτων από μεγάλο όγκο δεδομένων. Στο πεδίο της γενετικής και γονιδιωματικής, συμβάλλει στην αλληλουχία και υποσημείωση γονιδιωμάτων και την παρατήρηση των μεταλλάξεων τους.\cite{bioinformatics-1} Παίζει μεγάλο ρόλο στην εξόρυξη τεχνικών όρων και στην κατασκευή βιολογικών και γονιδιακών οντολογιών για την οργάνωση και αναζήτηση βιολογικών δεδομένων. Έχει επίσης μεγάλο ρόλο στην ανάλυση των γονιδίων και στην ρύθμιση πρωτεϊνών. Τα εργαλεία της Βιοπληροφορικής συμβάλουν στην σύγκριση γενετικών και γονιδιακών δεδομένων και γενικότερα στην κατανόηση των αναπτυξιακών πτυχών της μοριακής βιολογίας. Σε πιο εσωτερικό επίπεδο, συμβάλει στην ανάλυση και κατηγοριοποίηση των βιολογικών διαδρόμων και δικτύων τα οποία είναι σημαντικό κομμάτι της συστεμικής βιολογίας. Στην Δομική βιολογία, συμβάλει στην εξομοίωση και μοντελισμό του DNA, RNA, και δομές πρωτεϊνών όπως και μοριακών αλληλεπιδράσεων.\\

\subsection{Μοριακή δυναμική}
Η Βιοπληροφορική είναι ένα επιστημονικό πεδίο που εστιάζει στην εφαρμογή της τεχνολογίας υπολογιστών στην διαχείριση βιολογικών δεδομένων. Με το πέρασμα του χρόνου, οι εφαρμογές βιοπληροφορικής έχουν χρησιμοποιηθεί για να αποθηκεύσουν, αναλύσουν και να ενσωματώσουν βιολογικές και γενετικές πληροφορίες, χρησιμοποιώντας ένα μεγάλο εύρος μεθοδολογιών. Μια από τις πλέον γνωστές τεχνικές για την κατανόηση των φυσικών κινήσεων των ατόμων και των μορίων, είναι η μοριακή δυναμική. Η μοριακή δυναμική είναι μια μέθοδος εξομοίωσης των φυσικών κινήσεων των ατόμων και των μορίων κάτω από συγκεκριμένες συνθήκες. Έχει ρόλο κλειδί σε επιστήμες όπως η βιολογία, η χημεία, η φυσική, ιατρική. Λόγω της πολυπλοκότητας τους, οι υπολογισμοί της μοριακής δυναμικής χρειάζονται μεγάλες ποσότητες μνήμης και υπολογιστικής δύναμης, και για αυτό η εκτέλεση τους είναι συχνά μεγάλο πρόβλημα.\cite{bioinformatics-3} \\
Οι εξομοιώσεις της μοριακής δυναμικής χρησιμοποιούν πολύπλοκους αριθμητικούς υπολογισμούς, που πολλές φορές οδηγούν σε αριθμητικά λάθη. Πριν την ανακάλυψη του προγραμματισμού γενικής χρήσης, οι GPU χρησιμοποιούνταν μόνο για διαδικασίες απεικόνισης των μοριακών δομών, και η εκτέλεση των αλγορίθμων μοριακής δυναμικής μπορούσε να διαρκέσει από ώρες, έως και μέρες. Η λύση προήλθε από το GPGPU, καθώς οι GPU έχουν πολλές αριθμητικές μονάδες που μπορούν να εκτελεστούν παράλληλα. \\
Στο πεδίο της μοριακής δυναμικής, έχουν αναπτυχθεί πολλές εφαρμογές εφαρμογές βασισμένα στο GPGPU, που υποστηρίζουν εξομοιώσεις σε πολλαπλές μονάδες. Αυτή η καινοτομία δημιουργεί ευκαιρίες για το μέλλον, ειδικά για μικρότερες ερευνητικές ομάδες. Μειώνει τον χρόνο που απαιτείται για διαδικασίες και τα απαραίτητα κονδύλια για έρευνα-ανάπτυξη, προάγει την ανάπτυξη καινούριων εφαρμογών και την επιστημονική πρόοδο.\cite{bioinformatics-4}
\subsubsection{Μετάβαση από CPU σε GPU}
Η διαφορά στην αρχιτεκτονική μεταξύ CPU και GPU, είναι ότι στην τελευταία είναι δυνατή η εκτέλεση πολλαπλών παράλληλων διεργασιών, κάτι που επιτρέπει την καλύτερη εκτέλεση πολύπλοκων αλγορίθμων και καλύτερη διαχείριση μεγάλου όγκου δεδομένων. Επίσης, μια μονάδα επεξεργασίας γραφικών έχει λιγότερες ενεργειακές απαιτήσεις, έτσι η δημιουργία υπερ-υπολογιστών με χρήση GPU εξαλείφει την ανάγκη για τεράστιους χώρους γεμάτους με υπολογιστές. Η εγκατάσταση μιας επιπλέον μονάδας, αντιγράφει τον παραλληλισμό του προγραμματισμού, χωρίς καμία επιπλέον ενέργεια. Επιπλέον, η GPU έχει εντυπωσιακές δυνατότητες υπολογισμού floating point και μεγάλο εύρος ζώνης μνήμης, δίνοντας την δυνατότητα για βελτιστοποιημένη πρόσβαση στην μνήμη, ελεγχόμενη εκτέλεση επιλογών, και διαχείριση πόρων, με χρήση λίγων γραμμών κώδικα. Συγκεκριμένα, οι εφαρμογές μοριακής δυναμικής, κβαντικής χημείας, η απεικόνιση των αποτελεσμάτων τους, τρέχουν μέχρι και 5 φορές πιο γρήγορα. \\
Από την αρχή του GPU προγραμματισμού μέχρι και σήμερα, η προγραμματιστική ανάπτυξη συνεχίζεται αδιάκοπα. Ο αριθμός των εφαρμογών βασισμένων σε αρχιτεκτονικές GPU, φτάνουν τις 200, το οποίο είναι αύξηση της τάξεως πάνω από 60\% μέσα σε δύο χρόνια. Οι καλύτερες εφαρμογές βασισμένες σε GPGPU έχουν σχεδιαστεί για την μοριακή δυναμική, τον σχεδιασμό φαρμάκων, κβαντική χημεία, το κλίμα, την φυσική σύμφωνα με την Nvidia\cite{bioinformatics-2}
\subsection{GPUGRID.net}
\epigraph{"I hope mankind will acknowledge people like you, its real heroes."}{Grzegorz Granowski, Volunteer \& Donor}
Το GPUGRID είναι ένα εθελοντικό κατανεμημένο σύστημα, το οποίο στοχεύει στην βιοϊατρική έρευνα από το πανεπιστήμιο Universitat Pompeu Fabra της Ισπανίας. Το GPUGRID αποτελείται από πολλές μονάδες επεξεργασίας γραφικών, που συνεργάζονται μεταξύ τους για να παραδώσουν υψηλών επιδόσεων εξομοιώσεις βιομορίων. Οι μοριακές εξομοιώσεις πού εκτελούνται από τους εθελοντές του, αποτελούν μερικούς απο τους πιο συνήθης τύπους εξομοιώσεων που εκτελούνται απο τους επιστήμονες του πεδίου, αλλά ταυτόχρονα είναι από τους πιο απαιτητικούς σε υπολογιστική δύναμη και συνήθως απαιτούν υπερ-υπολογιστές.\\
\begin{figure}[h]
\centering
\includegraphics[scale=0.75]{gpugrid}
\caption{Βιολογία και πληροφορική}
\end{figure}\\
Το σύστημα ερευνά μεταξύ άλλων τα παρακάτω προβλήματα
\begin{itemize}
\item Εξομοίωση της ωρίμανσης πρωτεολυτικών του HIV - Μια απο τις πιο σημαντικές πτυχές της ωρίμανσης του HIV είναι το πώς η πρωτεΐνη "ψαλιδιών", δημιουργείται. Η απάντηση σε αυτό το ερώτημα χρειάζεται εξομοιώσεις μοριακής δυναμικής στο όριο των μοντέρνων υπολογιστικών δυνατοτήτων. Το GPUGRID μας επιτρέπει να λύσουμε αυτο το πρόβλημα και έχουμε καταφέρει να δείξουμε οτι τα πρώτα "ψαλίδια" κόβονται απο το "σκοινί" που είναι δεμένα. Αυτό το γεγονός συμβαίνει στην αρχή της ωρίμανσης, και αν σταματήσουμε την ωρίμανση των πρωτεολυτικών, τότε θα σταματήσουμε και την ωρίμανση του HIV σαν σύνολο.\cite{gpugrid-1}
\item Ανακάλυψη του ρόλου των μεμβρανών λιπιδίων στην δραστηριότητα ενζύμων.
\item Μοριακή εξομοίωση αισθητήρων ντοπαμίνης κάτω απο φυσιολογικές ιονικές δυνάμεις.
\item Αποκάλυψη των μηχανισμών αντίδρασης φαρμάκων καρκίνου παχέος εντέρου - Ο καρκίνος είναι βασικά ή ανεξέλεγκτη ανάπτυξη ιστών και εισβολή από μεταλλαγμένα κύτταρα σε έναν οργανισμό. Σε αντίθεση με τις παραδοσιακές χημειοθεραπείες ή ραδιοθεραπείες, οι νεότερες θεραπείες στοχεύουν σε συγκεκριμένους στόχους κακοήθων κυττάρων. Αυτό επιτυγχάνεται με τον εντοπισμό ορισμένων πρωτεϊνών που εκφράζονται διαφορικά σε ογκογεννητικά κύτταρα. Με την βοήθεια του GPUGRID, επιτυγχάνεται η επεξήγηση των μοριακών μηχανισμών που συμβαίνουν στα μεταλλαγμένα μόρια των κυττάρων.
\end{itemize}
Η εκτέλεση του GPUGRID στις GPUs, καινοτομεί στον εθελοντικό υπολογισμό, παραδίδοντας εφαρμογές υπερ-υπολογιστών, σε υποδομές χαμηλού κόστους. Η απόδοση των μονάδων γραφικής επεξεργασίας, καταγράφεται και συγκρίνεται σε σχέση με άλλους χρήστες, ανάλογα με την διάρκεια ολοκλήρωσης των WU (Work Units)\cite{gpugrid-2}.\\
\begin{figure}[h]
\centering
\includegraphics[scale=0.75]{gpugrid-charts}
\caption{Βιολογία και πληροφορική}
\end{figure}

\subsection{Προγράμματα}
\begin{apptable}{Βιοπληροφορική}{bioinformatics}
BarraCUDA & Λογισμικό χαρτογράφησης ακολουθίας & 6-10x & ΝΑΙ \\ \hline
CUDASW++ & Λογισμικό ανοιχτού κώδικα για αναζητήσεις Smith-Waterman σε πρωτεϊνικές βάσεις δεδομένων με χρήση GPUs & 10-50x & ΝΑΙ \\ \hline
CUSHAW & Ευθυγραμμιστής παράλληλων μικρών προσπελάσεων  & 10x & ΝΑΙ \\ \hline
G-BLASTN & Επιταχυνόμενο από GPU εργαλείο ευθυγράμμισης νουκλεοτιδίων βασισμένο στο ευρέως διαδεδομένο NCBI-BLAST & 4-15x & \\ \hline
GPU-BLAST & Τοπική αναζήτηση με γρήγορους ευρετικούς  αλγόριθμους k-tuple & 3-4x & \\ \hline
mCUDA-MEME & Πολύ γρήγορη κλιμακωτή ανακάλυψη μοτίβων βασισμένη στο MEME & 4-10x & ΝΑΙ \\ \hline
MUMmer GPU & Πρόγραμμα υψηλής απόδοσης τοπικής ευθυγράμμισης ακολουθίας & 3-10x & \\ \hline
NVBIO & Βιβλιοθήκη ανοιχτού κώδικα C++ αποτελούμενη από στοιχεία επαναχρησιμοποιήσιμα σχεδιασμένα για να επιταχύνουν εφαρμογές βιοπληροφορικής με χρήση CUDA. & 4-5x & ΝΑΙ \\ \hline
NVBowtie & Μια σε μεγάλο βαθμό ολοκληρωμένη εφαρμογή του ευθυγραμμιστή Bowtie2 πάνω από το NVBIO & 2.75x-8.35x & ΝΑΙ \\ \hline
PEANUT & Καθορισμός ανάγνωσης για ακολουθίες DNA ή RNA σε γνωστή αναφορά γονιδιώματος. & 10x & \\ \hline
REACTA & Ο σκοπός του REACTA είναι ο ποσοτικός προσδιορισμός της συμβολής της γενετικής διακύμανσης στην φαινοτυπική διακύμανση σε πολύπλοκα χαρακτηριστικά. & 2-4x & ΝΑΙ \\ \hline
SeqNFind & Ακολουθία επόμενης γενιάς και συγκρίσεις γονιδιωμάτων & 400x & ΝΑΙ \\ \hline
SOAP3 & Λογισμικό βασισμένο σε GPU για ευθυγράμμιση μικρών προσπελάσεων με αναφορά ακολουθίας. Μπορεί να βρει όλες τις ευθυγραμμίσεις με k ασυμφωνίες, όπου το k είναι ένας αριθμός απο το 0 έως το 3 & 10x & ΝΑΙ \\ \hline
SOAP3-dp & Πολύ γρήγορο εργαλείο βασισμένο σε GPU για ευθυγραμμίσεις μικρών προσπελάσεων μέσω δυναμικού προγραμματισμού υποβοηθούμενου από ευρετήριο & 28-64x & ΝΑΙ \\ \hline
UGENE & Λογισμικό ανοιχτού κώδικα Smith-Waterman για SSE/CUDA, επαναλήψεις βασισμένες σε πίνακα δεικτών & 6-8x & ΝΑΙ \\ \hline
WideLM & Ταιριάζει πολυάριθμα γραμμικά μοντέλα σε μια σταθερή σχεδίαση και απάντηση & 150x & ΝΑΙ \\ \hline
\end{apptable}


%\begin{apptable}{Μοριακή δυναμική}{molecular}
%ACEMD & GPU simulation of molecular mechanics force fields, implicit and explicit solvent & & \\ \hline
%AMBER & Suite of programs to simulate molecular dynamics on biomolecule & & \\ \hline
%CHARMM & MD package to simulate molecular dynamics on biomolecule & & \\ \hline
%DESMOND & High-speed molecular dynamics simulations of biological systems & & \\ \hline
%DL-POLY & Simulate macromolecules, polymers, ionic systems, etc on a distributed memory parallel computer & & \\ \hline
%ESPResSo & Highly versatile software package for performing and analyzing scientific Molecular Dynamics. & & \\ \hline
%Folding@Home & A distributed computing project that studies protein folding, misfolding, aggregation, and related diseases & & \\ \hline
%GPUGrid.net & A distributed computing project that uses GPUs for molecular simulations & & \\ \hline
%GROMACS & Simulation of biochemical molecules with complicated bond interactions & & \\ \hline
%HALMD & Large-scale simulations of simple and complex liquids & & \\ \hline
%HOOMD-Blue & Particle dynamics package written grounds up for GPUs & & \\ \hline
%LAMMPS & Classical molecular dynamics package & & \\ \hline
%NAMD & Designed for high-performance simulation of large molecular systems & & \\ \hline
%OpenMM & Library and application for molecular dynamics for HPC with GPUs & & \\ \hline
%\end{apptable}
\section{Ψυχαγωγία}
\subsection{Εισαγωγή}
Μια μηχανή φυσικής είναι ένα πρόγραμμα υπολογιστή που παρέχει εξομοίωση συγκεκριμένων συστημάτων φυσικής, όπως δυναμική άκαμπτων σωμάτων, ανίχνευση σύγκρουσης, δυναμική υγρών, για χρήση σε πεδία όπως τα γραφικά υπολογιστών, παιχνίδια, κινούμενα σχέδια, ταινίες. Μια από τις κύριες χρήσεις τους είναι στα παιχνίδια υπολογιστών, στην οποία περίπτωση η εξομοίωση γίνεται σε πραγματικό χρόνο. O όρος χρησιμοποιείται γενικότερα για να περιγράψει οποιοδήποτε σύστημα λογισμικού που εξομοιώνει φυσικά φαινόμενα, όπως επιστημονικές εξομοιώσεις υψηλής απόδοσης.

Οι μηχανές φυσικής έχουν χρησιμοποιηθεί αρκετά στους υπερ-υπολογιστές από την δεκαετία του '80 για να εκτελέσουν μοντελοποίηση δυναμικών υγρών, όπου αναθέτουμε διανύσματα ισχύος σε σωματίδια, για να δείξουμε την κυκλοφορία. Λόγω των υψηλών απαιτήσεων σε ταχύτητα και ακρίβεια, ειδικοί επεξεργαστές δημιουργήθηκαν που είναι γνωστοί ως επεξεργαστές διανυσμάτων για να επιταχύνουν τους υπολογισμούς. Οι τεχνικές μπορούν να χρησιμοποιηθούν για να μοντελοποιήσουν πρότυπα καιρού για την πρόβλεψη καιρού, δεδομένα σήραγγας αέρα για σχεδιασμό αεροπλάνων και υποβρυχίων, και ανάλυση θερμικής απόδοσης για καλύτερο σχεδιασμό ψηκτρών για επεξεργαστές. Φυσικά μεγάλο ρόλο παίζει η ακρίβεια των υπολογισμών, αφού μικρές αποκλίσεις μπορούν να αλλάξουν δραστικά τα αποτελέσματα των υπολογισμών. Οι κατασκευαστές ελαστικών χρησιμοποιούν εξομοιώσεις φυσικής για να μελετήσουν πώς οι καινούριοι τύποι ελαστικών θα αποδίδουν σε συνθήκες βρεγμένου και στεγνού οδοστρώματος, χρησιμοποιώντας καινούρια υλικά και κάτω από διαφορετικές συνθήκες βάρους.

Υπάρχουν γενικά δύο τύποι μηχανών φυσικής. Οι πραγματικού χρόνου, και οι υψηλής ακρίβειας. Οι υψηλής ακρίβειας απαιτούν περισσότερη υπολογιστική δύναμη για να υπολογίσουν φυσικά φαινόμενα με ακρίβεια και χρησιμοποιούνται συνήθως από επιστήμονες αλλά και σε κινούμενα σχέδια. Οι πραγματικού χρόνου - χρησιμοποιούνται σε παιχνίδια υπολογιστών και σε άλλες μορφές διαδραστικού υπολογισμού - χρησιμοποιούν απλοποιημένους υπολογισμούς με μειωμένη ακρίβεια ώστε να επιτρέπουν στο παιχνίδι να αντιδράει σε αποδεκτό ρυθμό για την εμπειρία χρήσης.


\subsection{Παιχνίδια}
\subsubsection{Γενικά}
Στα περισσότερα παιχνίδια, η ταχύτητα των επεξεργαστών και η εμπειρία χρήσης είναι πιο σημαντικά από την ακρίβεια της εξομοίωσης. Αυτό μας οδηγεί σε σχεδιασμούς μηχανών φυσικής που παράγουν αποτελέσματα σε πραγματικό χρόνο αλλά αντιγράφουν φυσικά φαινόμενα μόνο για απλές περιπτώσεις. Τις περισσότερες φορές, η εξομοίωση είναι σχεδιασμένη να παρέχει μια φαινομενικά σωστή εκτίμηση, παρά απόλυτη ακρίβεια. Όμως μερικές μηχανές, απαιτούν μεγαλύτερη ακρίβεια σε σκηνές μάχης ή σε παιχνίδια τύπου παζλ. Οι κινήσεις χαρακτήρων στο παρελθόν χρησιμοποιούσαν φυσική άκαμπτων σωμάτων γιατί είναι γρηγορότερο και πιο εύκολο να υπολογιστεί, όμως τα τελευταία χρόνια τα παιχνίδια και οι ταινίες έχουν αρχίσει να χρησιμοποιούν φυσική μαλακών σωμάτων. Αυτού του τύπου η εξομοίωση χρησιμοποιείται επίσης για εφέ σωματιδίων, κίνηση υγρών και υφασμάτων. Μια μορφή εξομοίωσης δυναμικής υγρών χρησιμοποιείται για να εξομοιώσει νερό και άλλα υγρά αλλά και την ροή της φωτιάς και του καπνού στον αέρα.

\begin{figure}[h]
\centering
\includegraphics[width=0.5\linewidth]{havok-logo}
\caption{Λογότυπο της μηχανή φυσικής havok}
\end{figure}

\subsubsection{Ανίχνευση σύγκρουσης}
Η ανίχνευση σύγκρουσης συνήθως αναφέρεται στο υπολογιστικό πρόβλημα της ανίχνευσης της διασταύρωσης δύο η περισσότερων αντικειμένων. Αν και το θέμα έχει σχέση περισσότερο με την χρήση του στα παιχνίδια και σε άλλες εξομοιώσεις φυσικής, έχει και χρήσεις στην ρομποτική. Εκτός από την ανίχνευση του αν δύο αντικείμενα έχουν συγκρουστεί, τα συστήματα ανίχνευσης μπορούν να υπολογίσουν τον χρόνο της σύγκρουσης(Time Of Impact), και να αναφέρουν ένα σύνολο από σημεία διασταύρωσης. Η αντίδραση σύγκρουσης είναι η εξομοίωση του τι συμβαίνει όταν ανιχνευθεί μια σύγκρουση.
\subsection{Υλοποιήσεις}
\subsubsection{Μονάδα Επεξεργασίας Φυσικής}
Η μονάδα επεξεργασίας φυσικής (PPU) είναι ένας μικροεπεξεργαστής αποκλειστικά σχεδιασμένος για να χειρίζεται τους υπολογισμούς φυσικής, ειδικά σε μηχανές φυσικής των παιχνιδιών υπολογιστών. Η ιδέα είναι ότι αυτοί οι ειδικοί επεξεργαστές ελαφρύνουν το φόρτο εργασίας των CPUs, όπως μια κάρτα γραφικών εκτελεί υπολογισμούς γραφικών. Ο όρος αρχικά δημιουργήθηκε από την εταιρία Ageia για να περιγράψει τους επεξεργαστές PhysX στους καταναλωτές.
H NVIDIA απέκτησε την Ageia Technologies το 2008 και συνεχίζει να αναπτύσσει την πλατφόρμα PhysX και στο υλικό αλλά και στο λογισμικό. Απο την έκδοση 2.8.3, η υποστήριξη για κάρτες PPU σταμάτησε, και δεν κατασκευάζονται πλέον.  
\subsubsection{Υπολογισμοί γενικής χρήσης σε GPUs}
Η επιτάχυνση υλικού για υπολογισμούς φυσικής χρησιμοποιείται πλέον από τις GPUs που υποστηρίζουν υπολογισμό γενικής χρήσης. Η εκτέλεση φυσικών υπολογισμών σε GPUs είναι συνήθως αρκετά πιο γρήγορη απο ότι σε μια CPU, έτσι η απόδοση των παιχνιδιών βελτιώνεται και ή ροή εικόνας μπορεί να είναι πολύ πιο γρήγορη. Όμως η χρήση υπολογισμών φυσικής σε ένα παιχνίδι δημιουργεί επιπλέον φόρτο στην GPU. Έτσι, η χρήση ξεχωριστής μονάδας επεξεργασίας γραφικών για εκτελέσεις υπολογισμών φυσικής μπορεί να αποδώσει τα βέλτιστα αποτελέσματα. Το PhysX εκτελείται γρήγορα και αποδίδει μεγαλύτερο ρεαλισμό όταν εκτελείται στην GPU, αποφέροντας 10-20 φορές περισσότερα εφέ και οπτική πιστότητα απο ότι οι υπολογισμοί φυσικής που εκτελούνται σε μια κεντρική μονάδα επεξεργασίας τελευταίας τεχνολογίας. Το PhysX χρησιμοποιεί ετερογενή υπολογισμό για να αποδώσει την καλύτερη εμπειρία χρήσης. Καθώς το παιχνίδι εκτελείται, το σύστημα PhysX εκτελεί μέρη της τεχνολογίας στην CPU αλλά και άλλα μέρη στην GPU. Αυτό γίνεται ώστε να χρησιμοποιείται αποδοτικά το υλικό του υπολογιστή ώστε να παρέχουν την καλύτερη δυνατή εμπειρία στον χρήστη. Το πιο σημαντικό, είναι οτι η τεχνολογία PhysX μπορεί να κλιμακώνεται με την χρήση GPU, σε αντίθεση με άλλες ανταγωνιστικές υλοποιήσεις φυσικής.
\subsubsection{Σύγκριση}
Οι πιο σημαντικές μηχανές γραφικών που χρησιμοποιούνται σήμερα είναι οι παρακάτω:
\begin{itemize}
\item PhysX - Είναι μια μηχανή φυσικής πραγματικού χρόνου, από την NVIDIA. Είναι κλειστού κώδικα και χρησιμοποιείται σε πολλά παιχνίδια υπολογιστών και κονσολών. Υποστηρίζει μεγάλο αριθμό συσκευών.
\item Havok - Είναι μια μηχανή φυσικής που χρησιμοποιείται σε πολλά παιχνίδια υπολογιστών. 
\item ODE - Είναι μια μηχανή φυσικής που υποστηρίζει κυρίως ανίχνευση συγκρούσεων, δυναμικές άκαμπτων σωμάτων. Αποτελεί ανοιχτό και ελεύθερο λογισμικό. Έχει χρησιμοποιηθεί σε πολλά παιχνίδια και εφαρμογές. Αποτελεί δημοφιλής επιλογή για εφαρμογές εξομοίωσης ρομποτικής.
\item Newton Game Dynamics - Είναι μια μηχανή φυσικής ανοιχτού κώδικα που εξομοιώνει άκαμπτα σώματα σε παιχνίδια και άλλες εφαρμογές πραγματικού χρόνου. 
\item Bullet - Είναι μια μηχανή φυσικής που εξομοιώνει ανίχνευση συγκρούσεων, δυναμική άκαμπτων και μαλακών σωμάτων. Χρησιμοποιείται σε παιχνίδια υπολογιστών αλλά και για οπτικά εφέ σε ταινίες. Η βιβλιοθήκη της bullet physics είναι ελεύθερη και ανοιχτού κώδικα κάτω από την άδεια zlib.
\end{itemize}
\section{Μετατροπή}
\subsection{Εισαγωγή}
Η μετατροπή είναι η άμεση μετατροπή ψηφιακής κωδικοποίησης, όπως για παράδειγμα σε αρχεία μιας ταινίας(π.χ .mp4, avi), αρχεία ήχου(.wav .mp3), ή κωδικοποίηση χαρακτήρων. (π.χ UTF-8, ISO/IEC 8859). Αυτή γίνεται σε περιπτώσεις που η συσκευή στόχος δεν υποστηρίζει την μορφή του αρχείου ή χρειαζόμαστε μικρότερο μέγεθος αρχείου, ή για να μετατρέψουμε αρχεία παλαιότερης μορφής σε μια πιο σύγχρονη για καλύτερη υποστήριξη σε μελλοντικές εφαρμογές. 

Η μετατροπή χρησιμοποιείται συχνά στα λογισμικά προβολής βίντεο για να ελαττώσουμε το μέγεθος του αρχείου βίντεο. Μια διαδικασία που γίνεται συχνά είναι η μετατροπή από αρχεία MPEG-2(DVD) σε αρχεία μορφής MPEG-4, που ενσωματώνει σύγχρονους αλγόριθμους για καλύτερη ποιότητα εικόνας σε συνδυασμό με μικρότερο μέγεθος αρχείου.

\begin{figure}[h]
\centering
\includegraphics[width=0.5\linewidth]{h264-logo}
\caption{H264, το πιο δημοφιλές πρότυπο συμπίεσης βίντεο}
\end{figure}
\subsection{Μειονεκτήματα} 
Το μεγαλύτερο μειονέκτημα της μετατροπής σε απωλεστικές μορφές αρχείου είναι η μειωμένη ποιότητα. Τα τεχνουργήματα συμπίεσης συσσωρεύονται, οπότε κάθε διαδικασία μετατροπής δημιουργεί μια βαθμιαία απώλεια ποιότητας, που είναι γνωστή ως ψηφιακή απώλεια. Για αυτόν τον λόγο, η μετατροπή συνήθως δεν συνίσταται εκτός και αν δεν μπορούμε να την αποφύγουμε.

\subsection{Βίντεο}
H συμπίεση βίντεο χρησιμοποιεί σύγχρονες τεχνικές για να μειώσει πλεονασμούς στα δεδομένα βίντεο. Οι περισσότεροι αλγόριθμοι συμπίεσης βίντεο συνδυάζουν συμπίεση εικόνας και προσωρινή αποζημίωση κίνησης. Ο ήχος κωδικοποιείται παράλληλα με διαφορετικούς αλγόριθμους συμπίεσης άλλα συνήθως συνδυάζεται σε ένα πακέτο όταν ολοκληρωθεί η διαδικασία.

Οι περισσότεροι αλγόριθμοι συμπίεσης βίντεο χρησιμοποιούν απωλεστική συμπίεση, καθώς το ασυμπίεστο βίντεο απαιτεί πολύ μεγάλους ρυθμούς μετάδοσης. Αν και οι περισσότεροι αλγόριθμοι έχουν ένα παράγοντα συμπίεσης 3, μια τυπική συμπίεση βίντεο MPEG-4 μπορεί να έχει παράγοντα συμπίεσης από 20 έως και 200. Όπως σε όλες τις απωλεστικές διαδικασίες συμπίεσης, υπάρχει ένα δίλημμα μεταξύ της ποιότητας του βίντεο, το κόστος της επεξεργασίας της συμπίεσης και της αποσυμπίεσης, και των απαιτήσεων του συστήματος. Υπερβολικά συμπιεσμένο βίντεο μπορεί να δημιουργήσει οπτικά τεχνουργήματα.

Συνήθως ο τρόπος με τον οποίο λειτουργεί ένας αλγόριθμος συμπίεσης βίντεο είναι με ομάδες γειτονικών εικονοστοιχείων(pixel), που ονομάζονται blocks. Αυτές οι ομάδες από pixels, συγκρίνονται μεταξύ τους από μια εικόνα στην επόμενη, και ο αλγόριθμος αποστέλλει μόνο τις αλλαγές μεταξύ αυτών των block. Σε περιοχές του βίντεο που υπάρχει μεγαλύτερη κίνηση, ο αλγόριθμος πρέπει να συμπιέσει περισσότερα δεδομένα για να προλάβει τον μεγαλύτερο αριθμό εικονοστοιχείων που αλλάζουν. Συνήθως σε σκηνές με φωτιά, εκρήξεις, καπνούς, το αποτέλεσμα της συμπίεσης έχει μεγαλύτερη απώλεια ποιότητας, η αύξηση του ρυθμού μετάδοσης.

Μερικοί από τους σύγχρονους αλγόριθμους μετατροπής είναι οι παρακάτω
\begin{itemize}
\item Lagarith - Είναι ένας μη απωλεστικός αλγόριθμος ανοιχτού κώδικα που δίνει έμφαση στην ταχύτητα, στην υποστήριξη διαφόρων χώρου χρωμάτων(YV12,RGB,YUY2). Είναι ιδανικός για επεξεργασία και αποθήκευση αρχείων. Αν και υπάρχουν κάποιοι καλύτεροι μη απωλεστικοί αλγόριθμοι συμπίεσης, ο lagarith είναι ο πιο γρήγορος και έτσι έχει κερδίσει την υποστήριξη της κοινότητας.
\item VP9 - Είναι ένα ελεύθερο πρότυπο ανοιχτού κώδικα που αναπτύσσεται από την Google. Είναι ο διάδοχος του VP8. Σκοπός του είναι να μειώσει περισσότερο τον χώρο που απαιτείται από το βίντεο διατηρώντας την ίδια ποιότητα. 
\item H.264 - Είναι ένα πρότυπο συμπίεσης βίντεο, που είναι ίσως το πιο επιτυχημένο για την καταγραφή, συμπίεση, και αναμετάδοση περιεχομένου βίντεο. Χρησιμοποιείται για αναμετάδοση ψηφιακού σήματος τηλεόρασης, δορυφόρων, και από υπηρεσίες αναμετάδοσης στο διαδίκτυο.
\item H.265 - Είναι ένα πρότυπο συμπίεσης βίντεο, διάδοχος του επιτυχημένου H.264. Σκοπός του είναι ο διπλασιασμός της συμπίεσης διατηρώντας την ίδια ποιότητα. Υποστηρίζει αναλύσεις έως και 8192x4320.
\item Daala - Είναι μια τεχνολογία συμπίεσης από το ίδρυμα Xiph.Org. Χρησιμοποιεί περιτύλιξη μεταμορφώσεων για να μειώσει τα οπτικά τεχνουργήματα. Ο στόχος είναι η απόδοση του να ξεπεράσει τις δυνατότητες του VP9 και του H.265. 
\end{itemize}

\subsection{Συμπίεση δεδομένων γενετικής}
Οι γενετικοί αλγόριθμοι συμπίεσης είναι η τελευταία γενιά μη απωλεστικών αλγορίθμων για συμπίεση δεδομένων (συνήθως αλληλουχίες nucleotides) χρησιμοποιώντας συμβατικούς αλγορίθμους συμπίεσης αλλά και γενετικούς αλγόριθμους βελτιστοποιημένους στο συγκεκριμένο τύπο δεδομένων. Το 2012, μια ομάδα επιστημόνων απο το John Kopkins University ανακοίνωσε έναν αλγόριθμο συμπίεσης γενετικής(HapZipper), ο οποίος χρησιμοποιεί HapMap δεδομένα για να πετύχει συμπίεση 95\% μείωση στο μέγεθος αρχείου, πετυχαίνοντας καλύτερη συμπίεση σε πολύ καλύτερο χρόνο από ότι οι γνωστοί μη απωλεστικοί αλγόριθμοι. Άλλοι γενετικοί αλγόριθμοι συμπίεσης π.χ GenomeZip πετυχαίνουν μεγαλύτερη συμπίεση καταφέρνοντας αποθήκευση 6 δισεκατομμυρίων ανθρώπινων ζευγαριών γονιδιώματος σε 2.5 megabyte.
\section{Αστροφυσική}
Οι πρώτες καταγραφές της θεωρητικής αστρονομίας χρονολογούνται το 1550-1292 π.Χ Οι υπολογισμοί που βρέθηκαν σχηματισμένοι σε αιγυπτιακούς τάφους δείχνουν ότι υπήρχαν τεχνικές για την αναγνώριση και την καταγραφή προτύπων στον ουρανό, ένα πεδίο της αστρονομίας που είναι χρήσιμο ακόμα και σήμερα για την καταγραφή και χαρτογράφηση. Μπορεί οι σημερινοί ερευνητές να μην χρησιμοποιούν τα αρχαία εργαλεία, αλλά η ανάπτυξη τους έγινε σταδιακά. \\
Το μέγεθος των τηλεσκοπίων χρειάστηκε 400 χρόνια για να μεγαλώσει από 1 τετραγωνικό μέτρο σε 110 τετραγωνικά μέτρα. Οι ψηφιακοί υπολογιστές εμφανίστηκαν στο τέλος του 1940 με υπολογιστική ταχύτητα περίπου 100 floating point λειτουργιών το δευτερόλεπτο (FLOPS), και εξελίχθηκαν σε περίπου $3*10^{16}$ FLOPS σε λιγότερο από 65 χρόνια. Αυτή η επανάσταση των υπολογιστών συνεχίζεται ακόμα και σήμερα, και έχει οδηγήσει σε ένα καινούριο κομμάτι έρευνας στο οποίο οι εγκαταστάσεις δεν βρίσκονται στο ψηλότερο βουνό του κόσμου, αλλά στο διπλανό δωμάτιο. Οι Αστρονόμοι κατανόησαν γρήγορα ότι μπορούν να χρησιμοποιήσουν τους υπολογιστές για να καταγράψουν, αναλύσουν, αρχειοθετήσουν, τις τεράστιες ποσότητες πληροφοριών που καταγράφονται από τις εκστρατείες παρατηρητών. \\
\begin{figure}[h]
\centering
\includegraphics[width=\linewidth]{astrophysics}
\caption{Αστροφυσική}
\end{figure}
Την μεγαλύτερη όμως επίδραση στον τρόπο με τον οποίο αντιμετωπίζουν οι αστρονόμοι τα αναπάντητα ερωτήματα τους, έχει το πεδίο της εξομοίωσης. Με την χρήση των υπολογιστών, είναι δυνατόν να μελετήσουμε την λειτουργία του Διαγαλαξιακούς κενού, την φυσική των μαύρων τρυπών, κ.α