\chapter{Εφαρμογές}
\section{Εισαγωγή}
Τα τελευταία περίπου 20 χρόνια οι εταιρίες παραγωγής υλικού γραφικών έχουν εστιάσει στην προσπάθεια να παράγουν γρήγορες μονάδες γραφικής επεξεργασίας (GPU), ειδικότερα για την κοινότητα των gamer. Αυτό έχει ως αποτέλεσμα πρόσφατα να δημιουργηθούν συσκευές οι επιδόσεις των οποίων ξεπερνούν τις κεντρικές μονάδες επεξεργασίας (CPU), σε συγκεκριμένες εφαρμογές, ειδικότερα σε μετρήσεις εκατομμυρίων εντολών το δευτερόλεπτο (MIPS). Έτσι, καθιερώθηκε μια κοινότητα για να αξιοποιήσει αυτήν την μεγάλη δύναμη των GPU για υπολογισμούς γενικής χρήσης(GPGPU). Τα τελευταία δύο χρόνια έχουν εξαλειφθεί οι περισσότεροι περιορισμοί που υπήρχαν όσον αφορά το σετ εντολών και την διαχείριση μνήμης, με την ενσωμάτωση ενοποιημένων υπολογιστικών μονάδων στις κάρτες γραφικών, δίνοντας έτσι την δυνατότητα στους προγραμματιστές να δημιουργήσουν ένα πλήθος από προγράμματα με εφαρμογές σε πολλούς τομείς.
\section{Κρυπτογράφηση}
Στο πεδίο της ασύμμετρης κρυπτογράφησης, η ασφάλεια όλων των πρακτικών κρυπτοσυστημάτων βασίζεται στην δυσκολία υπολογισμού προβλημάτων, εξαρτημένη από την επιλογή των παραμέτρων. Με την όποια αύξηση των παραμέτρων όμως (συνήθως στο εύρος 1024-4096 bits), οι υπολογισμοί γίνονται όλο και πιο απαιτητικοί για τον εκάστοτε επεξεργαστή. Σε σύγχρονο υλικό, ο υπολογισμός μιας μονής εντολής κρυπτογράφησης δεν είναι κρίσιμος, όμως σε ένα σύστημα επικοινωνίας πολλών-προς-ένα, για παράδειγμα ένας κεντρικός server στο κέντρο δεδομένων μιας εταιρίας, μπορεί να αντιμετωπίσει ταυτόχρονα εκατοντάδες η και χιλιάδες ταυτόχρονες συνδέσεις και εντολές κρυπτογράφησης.

Ως αποτέλεσμα, η πιο συνήθης λύση για ένα τέτοιο σενάριο είναι η χρήση καρτών επιτάχυνσης κρυπτογράφησης. Λόγω της μικρής αγοράς, η τιμή τους φτάνει συνήθως αρκετά χιλιάδες ευρώ η δολάρια.
Τελευταία, η ερευνητική κοινότητα έχει αρχίσει να εξερευνά τεχνικές για επιτάχυνση των αλγορίθμων κρυπτογράφησης με χρήση της GPU.  

\begin{figure}[h]
\centering
\includegraphics[scale=0.50]{crypto-1}
\caption{Μηχανή κρυπτογράφησης German Lorenz, χρησιμοποιήθηκε στον δεύτερο παγκόσμιο πόλεμο για να κρυπτογραφεί μηνύματα για προσωπικό πολύ υψηλής σημασίας}
\end{figure}

\subsection{Κρυπτογραφία συμμετρικού κλειδιού}

\subsection{Κρυπτογραφία δημοσίου κλειδιού}
Τα κρυπτοσυστήματα συμμετρικών κλειδιών χρησιμοποιούν το ίδιο κλειδί για την κρυπτογράφηση και την αποκρυπτογράφηση ενός μηνύματος, αν και ένα μήνυμα ή ομάδα μηνυμάτων μπορεί να έχουν διαφορετικό κλειδί από τα άλλα. Ένα σημαντικό μειονέκτημα των συμμετρικών κρυπτογραφημάτων είναι η διαχείριση κλειδιών, ώστε αυτά να χρησιμοποιηθούν με ασφάλεια. 

\begin{figure}[h]
\centering
\includegraphics[scale=0.50]{crypto-2}
\caption{Γαλλική μηχανή κρυπτογράφησης σε σχήμα βιβλίου του 16ου-αιώνα}
\end{figure}

\begin{figure}[h]
\centering
\includegraphics[scale=1]{crypto-3}
\caption{Η μηχανή κρυπτογράφησης Enigma, χρησιμοποιήθηκε από τον Γερμανικό στρατό και τις πολιτικές αρχές από τα τέλη του 1920 μέχρι και τον δεύτερο παγκόσμιο πόλεμο, παρείχε ένα πολύπλοκο ηλεκτρο-μηχανικό πολύ-αλφαβητικό κρυπτογράφημα. H αποκρυπτογράφηση του αλγόριθμου αποδείχτηκε μεγάλης σημασίας για την νίκη των συμμάχων.}
\end{figure}

\subsection{Hashcat}
Για λόγους έρευνας της εργασίας επιλέχτηκε το πρόγραμμα oclHashcat ώστε να μελετήσουμε την απόδοση των κρυπτογραφικών προγραμμάτων με χρήση GPGPU. Το Hashcat είναι το πιο γρήγορο πρόγραμμα επαναφοράς κωδικών. Είναι ελεύθερο, αν και κλειστού κώδικα. Υπάρχουν εκδόσεις για Linux,OSX,και Windows, και υλοποιήσεις για CPU ή GPU. Το hashcat υποστηρίζει μεγάλο αριθμό από hashing αλγορίθμους, συμπεριλαμβανομένου MD4,MD5,SHA, Unix Crypt, κ.α
\subsubsection{Υλοποιήσεις}
Το Hashcat διανέμεται σε δύο εκδόσεις
\begin{itemize}
\item Hashcat - Ένα εργαλείο επαναφοράς κωδικών για CPU
\item oclHashcat - Επαναφορά κωδικών με χρήση GPGPU
\end{itemize}

Πολλοί από τους αλγόριθμους που υποστηρίζει το hashcat μπορούν να βρεθούν σε μικρότερο χρόνο με την χρήση της GPU-επιτάχυνσης του oclHashcat(όπως MD5,SHA1, κ.α) Όμως, δεν επιταχύνονται όλοι οι αλγόριθμοι απο την χρήση GPU. Το Bcrypt είναι ένα τέτοιο παράδειγμα. Λόγω παραγόντων όπως διακλαδώσεις δεδομένων, σειριακοποίηση, μνήμη, κ.α, το oclHashcat δεν είναι απόλυτος αντικαταστάτης του Hashcat.

\subsubsection{Επιθέσεις}
Το Hashcat υποστηρίζει πολλούς τύπους επιθέσεων για την επαναφορά πολύπλοκων κωδικών από τον χώρο κλειδιού ενός hash. Αυτοί οι τύποι περιλαμβάνουν:
\begin{itemize}
\item Επίθεση Brute-Force
\item Επίθεση Combinator
\item Επίθεση λεξικού
\item Επίθεση αποτυπώματος
\item Υβριδική επίθεση
\item Επίθεση μάσκας
\item Επίθεση Permutation
\item Επίθεση βασισμένη σε κανόνες
\item Επίθεση με χρήση πίνακα
\item Επίθεση Toggle-Case
\end{itemize}


\subsection{Κρυπτο-νομίσματα}
Τα κρυπτονομίσματα είναι μια μορφή συναλλαγής με χρήση κρυπτογραφίας για ασφάλεια και τον έλεγχο της δημιουργίας καινούριων νομισμάτων. Το πρώτο κρυπτο-νόμισμα που δημιουργήθηκε ήταν το Bitcoin το 2009. Από τότε, πολλά κρυπτονομίσματα έχουν δημιουργηθεί. Ένα χαρακτηριστικό τους είναι ότι δεν υπάρχει κεντρικός έλεγχος, σε αντίθεση με άλλα συστήματα ηλεκτρονικού χρήματος όπως το Paypal. Ακόμα ένα χαρακτηριστικό είναι ότι οι συναλλαγές καταγράφονται δημόσια, για παράδειγμα στο Bitcoin, οι συναλλαγές καταγράφονται στην block αλυσίδα. 

Η δημιουργία κρυπτονομισμάτων γίνεται μέσω ειδικών προγραμμάτων που υπολογίζουν συγκεκριμένα hashes. Τα κρυπτονομίσματα χρησιμοποιούν έναν αλγόριθμο "Proof of work" ώστε να γνωρίζουν ότι ο χρήστης έχει εξορύξει τα νομίσματα που αυτός αναφέρει. (π.χ Scrypt, SHA-256, κ.α). Η εξόρυξη ενός block είναι μια διαδικασία που μοιάζει πολύ με την επαναφορά κωδικών. Για αυτό οι GPUs, είναι πιο αποδοτικές απο τις CPUs - όταν μια τυπική CPU έχει μέχρι 8 πυρήνες, μια GPU μπορεί να έχει εκαντοντάδες, όπου η καθεμία απο αυτές υπολογίζει ένα διαφορετικό "hash".

Υπάρχουν δύο τρόποι για εξόρυξη εικονικών νομισμάτων.
\begin{itemize}
\item Ατομική εξόρυξη - Η ατομική προσπάθεια για εξόρυξη νομισμάτων με χρήση μιας η περισσότερων GPU
\item Εξόρυξη κοινοπραξίας - Η προσπάθεια από πολλούς ανθρώπους να εξορύξουν το ίδιο block, διαιρώντας τα κέρδη.
\end{itemize}
Οι ταχύτητες της εξόρυξης μετρώνται σε KH/s (kilohashes ανα δευτερόλεπτο) και MH/s (megahashes ανα δευτερόλεπτο). Για παράδειγμα, μια Radeon 4870 εξορύσσει ένα νόμισμα SHA-256 με ταχύτητα 80-100 MH/s, ενώ ένα νόμισμα Scrypt με ταχύτητα μόλις 130-140 KH/s.


\section{Βιοπληροφορική}
\subsection{Εισαγωγή}
Η συνεχής αύξηση της ποσότητας βιολογικών δεδομένων, η ανάγκη για ανάλυση τους και το συνεχές ενδιαφέρον από την επιστημονική κοινότητα για την κατανόηση των δομικών λειτουργιών των βιολογικών μορίων, αποτέλεσαν τους κύριους λόγους για την ανάπτυξη της βιοπληροφορικής. Για να κατανοήσουμε τις κυτταρικές και βιομοριακές λειτουργίες, τα βιολογικά δεδομένα πρέπει να συνενωθούν για να σχηματίσουν μια ακριβής εικόνα. Οι ερευνητές της βιοπληροφορικής, έχουν αναπτύξει υπολογιστικές τεχνικές για την επεξεργασία των βιολογικών δεδομένων, όπως νουκλεοτιδικές αλληλουχιών, αλληλουχίες αμινο οξέων, τρισδιάστατων δομών, όπως επίσης βιολογικών σημάτων και εικόνων. Μεγάλες ερευνητικές προσπάθειες του πεδίου συμπεριλαμβάνουν αναγνώριση προτύπων, ευθυγράμμιση αλληλουχιών, ανάλυση πρωτεϊνικών δομών, φυλογενητική ανάλυση, μοριακή δυναμική, ανάλυση γονιδιώματος, σχεδιασμός φαρμάκων και ανάπτυξη φαρμάκων. Επίσης, υπάρχουν προφητικές τεχνικές ειδικές για τις εκφράσεις γονιδίων, και την αλληλεπίδραση πρωτεϊνών.\\
\begin{figure}[h]
\centering
\includegraphics[width=\linewidth]{bioinformatics}
\caption{Βιολογία και πληροφορική}
\end{figure}
Η Βιοπληροφορική παίζει μεγάλο ρόλο σε πολλές πτυχές της βιολογίας. Στην πειραματική μοριακή βιολογία, οι τεχνικές βιοπληροφορικής όπως επεξεργασία εικόνας και σήματος, επιτρέπει την εξόρυξη χρήσιμων αποτελεσμάτων από μεγάλο όγκο δεδομένων. Στο πεδίο της γενετικής και γονιδιωματικής, συμβάλλει στην αλληλουχία και υποσημείωση γονιδιωμάτων και την παρατήρηση των μεταλλάξεων τους.\cite{bioinformatics-1} Παίζει μεγάλο ρόλο στην εξόρυξη τεχνικών όρων και στην κατασκευή βιολογικών και γονιδιακών οντολογιών για την οργάνωση και αναζήτηση βιολογικών δεδομένων. Έχει επίσης μεγάλο ρόλο στην ανάλυση των γονιδίων και στην ρύθμιση πρωτεϊνών. Τα εργαλεία της Βιοπληροφορικής συμβάλουν στην σύγκριση γενετικών και γονιδιακών δεδομένων και γενικότερα στην κατανόηση των αναπτυξιακών πτυχών της μοριακής βιολογίας. Σε πιο εσωτερικό επίπεδο, συμβάλει στην ανάλυση και κατηγοριοποίηση των βιολογικών διαδρόμων και δικτύων τα οποία είναι σημαντικό κομμάτι της συστεμικής βιολογίας. Στην Δομική βιολογία, συμβάλει στην εξομοίωση και μοντελισμό του DNA, RNA, και δομές πρωτεϊνών όπως και μοριακών αλληλεπιδράσεων.\\

\subsection{Μοριακή δυναμική}
Η Βιοπληροφορική είναι ένα επιστημονικό πεδίο που εστιάζει στην εφαρμογή της τεχνολογίας υπολογιστών στην διαχείριση βιολογικών δεδομένων. Με το πέρασμα του χρόνου, οι εφαρμογές βιοπληροφορικής έχουν χρησιμοποιηθεί για να αποθηκεύσουν, αναλύσουν και να ενσωματώσουν βιολογικές και γενετικές πληροφορίες, χρησιμοποιώντας ένα μεγάλο εύρος μεθοδολογιών. Μια από τις πλέον γνωστές τεχνικές για την κατανόηση των φυσικών κινήσεων των ατόμων και των μορίων, είναι η μοριακή δυναμική. Η μοριακή δυναμική είναι μια μέθοδος εξομοίωσης των φυσικών κινήσεων των ατόμων και των μορίων κάτω από συγκεκριμένες συνθήκες. Έχει ρόλο κλειδί σε επιστήμες όπως η βιολογία, η χημεία, η φυσική, ιατρική. Λόγω της πολυπλοκότητας τους, οι υπολογισμοί της μοριακής δυναμικής χρειάζονται μεγάλες ποσότητες μνήμης και υπολογιστικής δύναμης, και για αυτό η εκτέλεση τους είναι συχνά μεγάλο πρόβλημα.\cite{bioinformatics-3} 

Οι εξομοιώσεις της μοριακής δυναμικής χρησιμοποιούν πολύπλοκους αριθμητικούς υπολογισμούς, που πολλές φορές οδηγούν σε αριθμητικά λάθη. Πριν την ανακάλυψη του προγραμματισμού γενικής χρήσης, οι GPU χρησιμοποιούνταν μόνο για διαδικασίες απεικόνισης των μοριακών δομών, και η εκτέλεση των αλγορίθμων μοριακής δυναμικής μπορούσε να διαρκέσει από ώρες, έως και μέρες. Η λύση προήλθε από το GPGPU, καθώς οι GPU έχουν πολλές αριθμητικές μονάδες που μπορούν να εκτελεστούν παράλληλα.

Στο πεδίο της μοριακής δυναμικής, έχουν αναπτυχθεί πολλές εφαρμογές εφαρμογές βασισμένα στο GPGPU, που υποστηρίζουν εξομοιώσεις σε πολλαπλές μονάδες. Αυτή η καινοτομία δημιουργεί ευκαιρίες για το μέλλον, ειδικά για μικρότερες ερευνητικές ομάδες. Μειώνει τον χρόνο που απαιτείται για διαδικασίες και τα απαραίτητα κονδύλια για έρευνα-ανάπτυξη, προάγει την ανάπτυξη καινούριων εφαρμογών και την επιστημονική πρόοδο.\cite{bioinformatics-4}
\subsubsection{Μετάβαση από CPU σε GPU}
Η διαφορά στην αρχιτεκτονική μεταξύ CPU και GPU, είναι ότι στην τελευταία είναι δυνατή η εκτέλεση πολλαπλών παράλληλων διεργασιών, κάτι που επιτρέπει την καλύτερη εκτέλεση πολύπλοκων αλγορίθμων και καλύτερη διαχείριση μεγάλου όγκου δεδομένων. Επίσης, μια μονάδα επεξεργασίας γραφικών έχει λιγότερες ενεργειακές απαιτήσεις, έτσι η δημιουργία υπερ-υπολογιστών με χρήση GPU εξαλείφει την ανάγκη για τεράστιους χώρους γεμάτους με υπολογιστές. Η εγκατάσταση μιας επιπλέον μονάδας, αντιγράφει τον παραλληλισμό του προγραμματισμού, χωρίς καμία επιπλέον ενέργεια. Επιπλέον, η GPU έχει εντυπωσιακές δυνατότητες υπολογισμού floating point και μεγάλο εύρος ζώνης μνήμης, δίνοντας την δυνατότητα για βελτιστοποιημένη πρόσβαση στην μνήμη, ελεγχόμενη εκτέλεση επιλογών, και διαχείριση πόρων, με χρήση λίγων γραμμών κώδικα. Συγκεκριμένα, οι εφαρμογές μοριακής δυναμικής, κβαντικής χημείας, η απεικόνιση των αποτελεσμάτων τους, τρέχουν μέχρι και 5 φορές πιο γρήγορα. \\
Από την αρχή του GPU προγραμματισμού μέχρι και σήμερα, η προγραμματιστική ανάπτυξη συνεχίζεται αδιάκοπα. Ο αριθμός των εφαρμογών βασισμένων σε αρχιτεκτονικές GPU, φτάνουν τις 200, το οποίο είναι αύξηση της τάξεως πάνω από 60\% μέσα σε δύο χρόνια. Οι καλύτερες εφαρμογές βασισμένες σε GPGPU έχουν σχεδιαστεί για την μοριακή δυναμική, τον σχεδιασμό φαρμάκων, κβαντική χημεία, το κλίμα, την φυσική σύμφωνα με την Nvidia\cite{bioinformatics-2}

\begin{apptable}{Μοριακή δυναμική}{molecular}
ACEMD & Εξομοίωση μοριακών μηχανικών πεδίων ισχύος με χρήση GPU & & \\ \hline
AMBER & Σουίτα προγραμμάτων για εξομοίωση μοριακής δυναμικής σε βιομόρια & & \\ \hline
CHARMM & Πακέτο μοριακής δυναμικής για εξομοίωση σε βιομόρια & & \\ \hline
DESMOND & Δυναμικές εξομοιώσεις βιολογικών συστημάτων υψηλών ταχυτήτων & & \\ \hline
DL-POLY & Εξομοίωση μακρομορίων, πολυμερών, ιονικών συστημάτων, κ.α σε υπολογιστή κατανεμημένης παράλληλης μνήμης & & \\ \hline
ESPResSo & Πακέτο λογισμικού υψηλής ευελιξίας για ανάλυση και απόδοση επιστημονικών μοριακών δυναμικών. & & \\ \hline
Folding@Home & Ένα πολύ γνωστό κατανεμημένο υπολογιστικό σύστημα που μελετάει διπλώσεις πρωτεϊνών και σχετικές αρρώστιες & & \\ \hline
GPUGrid.net & Ένα κατανεμημένο υπολογιστικό σύστημα που χρησιμοποιεί GPUs για εξομοιώσεις μορίων & & \\ \hline
GROMACS & Εξομοίωση βιοχημικών μορίων με διαδραστικότητα περίπλοκων δεσμών & & \\ \hline
HALMD & Εξομοιώσεις μεγάλης κλίμακας απλών και πολύπλοκων υγρών & & \\ \hline
HOOMD-Blue & Πακέτο λογισμικού εξομοίωσης σωματιδίων για GPUs & & \\ \hline
LAMMPS & Πακέτο λογισμικού κλασσικής μοριακής δυναμικής & & \\ \hline
NAMD & Σχεδιασμένο για εξομοίωση υψηλής απόδοσης σε μεγάλα μοριακά συστήματα & & \\ \hline
OpenMM & Βιβλιοθήκη και εφαρμογή για μοριακή δυναμική υψηλής υπολογιστικής απόδοσης με χρήση GPUs & & \\ \hline
\end{apptable}

\newpage

\subsection{GPUGRID.net}
\epigraph{"I hope mankind will acknowledge people like you, its real heroes."}{Grzegorz Granowski, Volunteer \& Donor}
Το GPUGRID είναι ένα εθελοντικό κατανεμημένο σύστημα, το οποίο στοχεύει στην βιοϊατρική έρευνα από το πανεπιστήμιο Universitat Pompeu Fabra της Ισπανίας. Το GPUGRID αποτελείται από πολλές μονάδες επεξεργασίας γραφικών, που συνεργάζονται μεταξύ τους για να παραδώσουν υψηλών επιδόσεων εξομοιώσεις βιομορίων. Οι μοριακές εξομοιώσεις πού εκτελούνται από τους εθελοντές του, αποτελούν μερικούς απο τους πιο συνήθης τύπους εξομοιώσεων που εκτελούνται απο τους επιστήμονες του πεδίου, αλλά ταυτόχρονα είναι από τους πιο απαιτητικούς σε υπολογιστική δύναμη και συνήθως απαιτούν υπερ-υπολογιστές.\\
\begin{figure}[h]
\centering
\includegraphics[scale=0.75]{gpugrid}
\caption{Βιολογία και πληροφορική}
\end{figure}\\
Το σύστημα ερευνά μεταξύ άλλων τα παρακάτω προβλήματα
\begin{itemize}
\item Εξομοίωση της ωρίμανσης πρωτεολυτικών του HIV - Μια απο τις πιο σημαντικές πτυχές της ωρίμανσης του HIV είναι το πώς η πρωτεΐνη "ψαλιδιών", δημιουργείται. Η απάντηση σε αυτό το ερώτημα χρειάζεται εξομοιώσεις μοριακής δυναμικής στο όριο των μοντέρνων υπολογιστικών δυνατοτήτων. Το GPUGRID μας επιτρέπει να λύσουμε αυτο το πρόβλημα και έχουμε καταφέρει να δείξουμε οτι τα πρώτα "ψαλίδια" κόβονται απο το "σκοινί" που είναι δεμένα. Αυτό το γεγονός συμβαίνει στην αρχή της ωρίμανσης, και αν σταματήσουμε την ωρίμανση των πρωτεολυτικών, τότε θα σταματήσουμε και την ωρίμανση του HIV σαν σύνολο.\cite{gpugrid-1}
\item Ανακάλυψη του ρόλου των μεμβρανών λιπιδίων στην δραστηριότητα ενζύμων.
\item Μοριακή εξομοίωση αισθητήρων ντοπαμίνης κάτω απο φυσιολογικές ιονικές δυνάμεις.
\item Αποκάλυψη των μηχανισμών αντίδρασης φαρμάκων καρκίνου παχέος εντέρου - Ο καρκίνος είναι βασικά ή ανεξέλεγκτη ανάπτυξη ιστών και εισβολή από μεταλλαγμένα κύτταρα σε έναν οργανισμό. Σε αντίθεση με τις παραδοσιακές χημειοθεραπείες ή ραδιοθεραπείες, οι νεότερες θεραπείες στοχεύουν σε συγκεκριμένους στόχους κακοήθων κυττάρων. Αυτό επιτυγχάνεται με τον εντοπισμό ορισμένων πρωτεϊνών που εκφράζονται διαφορικά σε ογκογεννητικά κύτταρα. Με την βοήθεια του GPUGRID, επιτυγχάνεται η επεξήγηση των μοριακών μηχανισμών που συμβαίνουν στα μεταλλαγμένα μόρια των κυττάρων.
\end{itemize}
Η εκτέλεση του GPUGRID στις GPUs, καινοτομεί στον εθελοντικό υπολογισμό, παραδίδοντας εφαρμογές υπερ-υπολογιστών, σε υποδομές χαμηλού κόστους. Η απόδοση των μονάδων γραφικής επεξεργασίας, καταγράφεται και συγκρίνεται σε σχέση με άλλους χρήστες, ανάλογα με την διάρκεια ολοκλήρωσης των WU (Work Units)\cite{gpugrid-2}.\\
\begin{figure}[h]
\centering
\includegraphics[scale=0.75]{gpugrid-charts}
\caption{Βιολογία και πληροφορική}
\end{figure}

\subsection{Προγράμματα}
\begin{apptable}{Βιοπληροφορική}{bioinformatics}
BarraCUDA & Λογισμικό χαρτογράφησης ακολουθίας & 6-10x & ΝΑΙ \\ \hline
CUDASW++ & Λογισμικό ανοιχτού κώδικα για αναζητήσεις Smith-Waterman σε πρωτεϊνικές βάσεις δεδομένων με χρήση GPUs & 10-50x & ΝΑΙ \\ \hline
CUSHAW & Ευθυγραμμιστής παράλληλων μικρών προσπελάσεων  & 10x & ΝΑΙ \\ \hline
G-BLASTN & Επιταχυνόμενο από GPU εργαλείο ευθυγράμμισης νουκλεοτιδίων βασισμένο στο ευρέως διαδεδομένο NCBI-BLAST & 4-15x & \\ \hline
GPU-BLAST & Τοπική αναζήτηση με γρήγορους ευρετικούς  αλγόριθμους k-tuple & 3-4x & \\ \hline
mCUDA-MEME & Πολύ γρήγορη κλιμακωτή ανακάλυψη μοτίβων βασισμένη στο MEME & 4-10x & ΝΑΙ \\ \hline
MUMmer GPU & Πρόγραμμα υψηλής απόδοσης τοπικής ευθυγράμμισης ακολουθίας & 3-10x & \\ \hline
NVBIO & Βιβλιοθήκη ανοιχτού κώδικα C++ αποτελούμενη από στοιχεία επαναχρησιμοποιήσιμα σχεδιασμένα για να επιταχύνουν εφαρμογές βιοπληροφορικής με χρήση CUDA. & 4-5x & ΝΑΙ \\ \hline
NVBowtie & Μια σε μεγάλο βαθμό ολοκληρωμένη εφαρμογή του ευθυγραμμιστή Bowtie2 πάνω από το NVBIO & 2.75x-8.35x & ΝΑΙ \\ \hline
PEANUT & Καθορισμός ανάγνωσης για ακολουθίες DNA ή RNA σε γνωστή αναφορά γονιδιώματος. & 10x & \\ \hline
REACTA & Ο σκοπός του REACTA είναι ο ποσοτικός προσδιορισμός της συμβολής της γενετικής διακύμανσης στην φαινοτυπική διακύμανση σε πολύπλοκα χαρακτηριστικά. & 2-4x & ΝΑΙ \\ \hline
SeqNFind & Ακολουθία επόμενης γενιάς και συγκρίσεις γονιδιωμάτων & 400x & ΝΑΙ \\ \hline
SOAP3 & Λογισμικό βασισμένο σε GPU για ευθυγράμμιση μικρών προσπελάσεων με αναφορά ακολουθίας. Μπορεί να βρει όλες τις ευθυγραμμίσεις με k ασυμφωνίες, όπου το k είναι ένας αριθμός απο το 0 έως το 3 & 10x & ΝΑΙ \\ \hline
SOAP3-dp & Πολύ γρήγορο εργαλείο βασισμένο σε GPU για ευθυγραμμίσεις μικρών προσπελάσεων μέσω δυναμικού προγραμματισμού υποβοηθούμενου από ευρετήριο & 28-64x & ΝΑΙ \\ \hline
UGENE & Λογισμικό ανοιχτού κώδικα Smith-Waterman για SSE/CUDA, επαναλήψεις βασισμένες σε πίνακα δεικτών & 6-8x & ΝΑΙ \\ \hline
WideLM & Ταιριάζει πολυάριθμα γραμμικά μοντέλα σε μια σταθερή σχεδίαση και απάντηση & 150x & ΝΑΙ \\ \hline
\end{apptable}

\newpage

\section{Κβαντική χημεία}


\subsection{Προγράμματα}

\begin{apptable}{Κβαντική Χημεία}{chemistry}
Abinit & Allows to find total energy, charge density and electronic structure of systems made of electrons and nuclei within DFT  & 1.3-2.7x & ΝΑΙ \\ \hline
ADF & Density Functional Theory (DFT) software package that enables first-principles electronic structure calculations & 1.5-2x & ΝΑΙ \\ \hline
BigDFT & Implements density functional theory by solving the Kohn-Sham equations describing the electrons in a material & 2-25x & ΝΑΙ \\ \hline
CP2K & Program to perform atomistic and molecular simulations of solid state, liquid, molecular and biological systems & 2-7x & ΝΑΙ \\ \hline
GAMESS-UK & The general purpose ab initio molecular electronic structure program for performing SCF-, DFT- and MCSCF-gradient calculations & 8x & ΝΑΙ \\ \hline
GAMESS-US & Computational chemistry suite used to simulate atomic and molecular electronic structure & 1.3-2.9x & ΝΑΙ \\ \hline
Gaussian (In Development) & Predicts energies, molecular structures,and vibrational frequencies of molecular systems & - & ΝΑΙ \\ \hline 
GPAW & Real-space grid DFT code written in C and Python & 8x  & ΝΑΙ \\ \hline
LATTE & Density matrix computations & - & ΝΑΙ \\ \hline
MOLCAS & Methods for calculating general electronic structures in molecular systems in both ground and excited states & 1.1x & ΝΑΙ \\ \hline
MOPAC2013 & Semiempirical Quantum Chemistry & 2x & ΟΧΙ \\ \hline
NWChem & Calculations & 3-10x & ΝΑΙ \\ \hline
Octopus & Used for ab initio virtual experimentation and quantum chemistry calculations & 1.5-8x & - \\ \hline
Q-CHEM & Computational chemistry package designed for HPC clusters & 8-14x & - \\ \hline
QUICK & QUICK is a GPU-enabled ab intio quantum chemistry software package & 10-100x  & ΝΑΙ \\ \hline
TeraChem & Quantum chemistry software designed to run on NVIDIA GPU & 44-650x & ΝΑΙ \\ \hline
\end{apptable}

\section{Ψυχαγωγία}
\subsection{Εισαγωγή}
Μια μηχανή φυσικής είναι ένα πρόγραμμα υπολογιστή που παρέχει εξομοίωση συγκεκριμένων συστημάτων φυσικής, όπως δυναμική άκαμπτων σωμάτων, ανίχνευση σύγκρουσης, δυναμική υγρών, για χρήση σε πεδία όπως τα γραφικά υπολογιστών, παιχνίδια, κινούμενα σχέδια, ταινίες. Μια από τις κύριες χρήσεις τους είναι στα παιχνίδια υπολογιστών, στην οποία περίπτωση η εξομοίωση γίνεται σε πραγματικό χρόνο. O όρος χρησιμοποιείται γενικότερα για να περιγράψει οποιοδήποτε σύστημα λογισμικού που εξομοιώνει φυσικά φαινόμενα, όπως επιστημονικές εξομοιώσεις υψηλής απόδοσης.

Οι μηχανές φυσικής έχουν χρησιμοποιηθεί αρκετά στους υπερ-υπολογιστές από την δεκαετία του '80 για να εκτελέσουν μοντελοποίηση δυναμικών υγρών, όπου αναθέτουμε διανύσματα ισχύος σε σωματίδια, για να δείξουμε την κυκλοφορία. Λόγω των υψηλών απαιτήσεων σε ταχύτητα και ακρίβεια, ειδικοί επεξεργαστές δημιουργήθηκαν που είναι γνωστοί ως επεξεργαστές διανυσμάτων για να επιταχύνουν τους υπολογισμούς. Οι τεχνικές μπορούν να χρησιμοποιηθούν για να μοντελοποιήσουν πρότυπα καιρού για την πρόβλεψη καιρού, δεδομένα σήραγγας αέρα για σχεδιασμό αεροπλάνων και υποβρυχίων, και ανάλυση θερμικής απόδοσης για καλύτερο σχεδιασμό ψηκτρών για επεξεργαστές. Φυσικά μεγάλο ρόλο παίζει η ακρίβεια των υπολογισμών, αφού μικρές αποκλίσεις μπορούν να αλλάξουν δραστικά τα αποτελέσματα των υπολογισμών. Οι κατασκευαστές ελαστικών χρησιμοποιούν εξομοιώσεις φυσικής για να μελετήσουν πώς οι καινούριοι τύποι ελαστικών θα αποδίδουν σε συνθήκες βρεγμένου και στεγνού οδοστρώματος, χρησιμοποιώντας καινούρια υλικά και κάτω από διαφορετικές συνθήκες βάρους.

Υπάρχουν γενικά δύο τύποι μηχανών φυσικής. Οι πραγματικού χρόνου, και οι υψηλής ακρίβειας. Οι υψηλής ακρίβειας απαιτούν περισσότερη υπολογιστική δύναμη για να υπολογίσουν φυσικά φαινόμενα με ακρίβεια και χρησιμοποιούνται συνήθως από επιστήμονες αλλά και σε κινούμενα σχέδια. Οι πραγματικού χρόνου - χρησιμοποιούνται σε παιχνίδια υπολογιστών και σε άλλες μορφές διαδραστικού υπολογισμού - χρησιμοποιούν απλοποιημένους υπολογισμούς με μειωμένη ακρίβεια ώστε να επιτρέπουν στο παιχνίδι να αντιδράει σε αποδεκτό ρυθμό για την εμπειρία χρήσης.

\subsection{Παιχνίδια}
\subsubsection{Γενικά}
Στα περισσότερα παιχνίδια, η ταχύτητα των επεξεργαστών και η εμπειρία χρήσης είναι πιο σημαντικά από την ακρίβεια της εξομοίωσης. Αυτό μας οδηγεί σε σχεδιασμούς μηχανών φυσικής που παράγουν αποτελέσματα σε πραγματικό χρόνο αλλά αντιγράφουν φυσικά φαινόμενα μόνο για απλές περιπτώσεις. Τις περισσότερες φορές, η εξομοίωση είναι σχεδιασμένη να παρέχει μια φαινομενικά σωστή εκτίμηση, παρά απόλυτη ακρίβεια. Όμως μερικές μηχανές, απαιτούν μεγαλύτερη ακρίβεια σε σκηνές μάχης ή σε παιχνίδια τύπου παζλ. Οι κινήσεις χαρακτήρων στο παρελθόν χρησιμοποιούσαν φυσική άκαμπτων σωμάτων γιατί είναι γρηγορότερο και πιο εύκολο να υπολογιστεί, όμως τα τελευταία χρόνια τα παιχνίδια και οι ταινίες έχουν αρχίσει να χρησιμοποιούν φυσική μαλακών σωμάτων. Αυτού του τύπου η εξομοίωση χρησιμοποιείται επίσης για εφέ σωματιδίων, κίνηση υγρών και υφασμάτων. Μια μορφή εξομοίωσης δυναμικής υγρών χρησιμοποιείται για να εξομοιώσει νερό και άλλα υγρά αλλά και την ροή της φωτιάς και του καπνού στον αέρα.

\begin{figure}[h]
\centering
\includegraphics[width=0.5\linewidth]{havok-logo}
\caption{Λογότυπο της μηχανή φυσικής havok}
\end{figure}

\subsubsection{Ανίχνευση σύγκρουσης}
Η ανίχνευση σύγκρουσης συνήθως αναφέρεται στο υπολογιστικό πρόβλημα της ανίχνευσης της διασταύρωσης δύο η περισσότερων αντικειμένων. Αν και το θέμα έχει σχέση περισσότερο με την χρήση του στα παιχνίδια και σε άλλες εξομοιώσεις φυσικής, έχει και χρήσεις στην ρομποτική. Εκτός από την ανίχνευση του αν δύο αντικείμενα έχουν συγκρουστεί, τα συστήματα ανίχνευσης μπορούν να υπολογίσουν τον χρόνο της σύγκρουσης(Time Of Impact), και να αναφέρουν ένα σύνολο από σημεία διασταύρωσης. Η αντίδραση σύγκρουσης είναι η εξομοίωση του τι συμβαίνει όταν ανιχνευθεί μια σύγκρουση.
\subsection{Υλοποιήσεις}
\subsubsection{Μονάδα Επεξεργασίας Φυσικής}
Η μονάδα επεξεργασίας φυσικής (PPU) είναι ένας μικροεπεξεργαστής αποκλειστικά σχεδιασμένος για να χειρίζεται τους υπολογισμούς φυσικής, ειδικά σε μηχανές φυσικής των παιχνιδιών υπολογιστών. Η ιδέα είναι ότι αυτοί οι ειδικοί επεξεργαστές ελαφρύνουν το φόρτο εργασίας των CPUs, όπως μια κάρτα γραφικών εκτελεί υπολογισμούς γραφικών. Ο όρος αρχικά δημιουργήθηκε από την εταιρία Ageia για να περιγράψει τους επεξεργαστές PhysX στους καταναλωτές.
H NVIDIA απέκτησε την Ageia Technologies το 2008 και συνεχίζει να αναπτύσσει την πλατφόρμα PhysX και στο υλικό αλλά και στο λογισμικό. Απο την έκδοση 2.8.3, η υποστήριξη για κάρτες PPU σταμάτησε, και δεν κατασκευάζονται πλέον.  
\subsubsection{Υπολογισμοί γενικής χρήσης σε GPUs}
Η επιτάχυνση υλικού για υπολογισμούς φυσικής χρησιμοποιείται πλέον από τις GPUs που υποστηρίζουν υπολογισμό γενικής χρήσης. Η εκτέλεση φυσικών υπολογισμών σε GPUs είναι συνήθως αρκετά πιο γρήγορη απο ότι σε μια CPU, έτσι η απόδοση των παιχνιδιών βελτιώνεται και ή ροή εικόνας μπορεί να είναι πολύ πιο γρήγορη. Όμως η χρήση υπολογισμών φυσικής σε ένα παιχνίδι δημιουργεί επιπλέον φόρτο στην GPU. Έτσι, η χρήση ξεχωριστής μονάδας επεξεργασίας γραφικών για εκτελέσεις υπολογισμών φυσικής μπορεί να αποδώσει τα βέλτιστα αποτελέσματα. Το PhysX εκτελείται γρήγορα και αποδίδει μεγαλύτερο ρεαλισμό όταν εκτελείται στην GPU, αποφέροντας 10-20 φορές περισσότερα εφέ και οπτική πιστότητα απο ότι οι υπολογισμοί φυσικής που εκτελούνται σε μια κεντρική μονάδα επεξεργασίας τελευταίας τεχνολογίας. Το PhysX χρησιμοποιεί ετερογενή υπολογισμό για να αποδώσει την καλύτερη εμπειρία χρήσης. Καθώς το παιχνίδι εκτελείται, το σύστημα PhysX εκτελεί μέρη της τεχνολογίας στην CPU αλλά και άλλα μέρη στην GPU. Αυτό γίνεται ώστε να χρησιμοποιείται αποδοτικά το υλικό του υπολογιστή ώστε να παρέχουν την καλύτερη δυνατή εμπειρία στον χρήστη. Το πιο σημαντικό, είναι οτι η τεχνολογία PhysX μπορεί να κλιμακώνεται με την χρήση GPU, σε αντίθεση με άλλες ανταγωνιστικές υλοποιήσεις φυσικής.
\subsubsection{Σύγκριση}
Οι πιο σημαντικές μηχανές γραφικών που χρησιμοποιούνται σήμερα είναι οι παρακάτω:
\begin{itemize}
\item PhysX - Είναι μια μηχανή φυσικής πραγματικού χρόνου, από την NVIDIA. Είναι κλειστού κώδικα και χρησιμοποιείται σε πολλά παιχνίδια υπολογιστών και κονσολών. Υποστηρίζει μεγάλο αριθμό συσκευών.
\item Havok - Είναι μια μηχανή φυσικής που χρησιμοποιείται σε πολλά παιχνίδια υπολογιστών. 
\item ODE - Είναι μια μηχανή φυσικής που υποστηρίζει κυρίως ανίχνευση συγκρούσεων, δυναμικές άκαμπτων σωμάτων. Αποτελεί ανοιχτό και ελεύθερο λογισμικό. Έχει χρησιμοποιηθεί σε πολλά παιχνίδια και εφαρμογές. Αποτελεί δημοφιλής επιλογή για εφαρμογές εξομοίωσης ρομποτικής.
\item Newton Game Dynamics - Είναι μια μηχανή φυσικής ανοιχτού κώδικα που εξομοιώνει άκαμπτα σώματα σε παιχνίδια και άλλες εφαρμογές πραγματικού χρόνου. 
\item Bullet - Είναι μια μηχανή φυσικής που εξομοιώνει ανίχνευση συγκρούσεων, δυναμική άκαμπτων και μαλακών σωμάτων. Χρησιμοποιείται σε παιχνίδια υπολογιστών αλλά και για οπτικά εφέ σε ταινίες. Η βιβλιοθήκη της bullet physics είναι ελεύθερη και ανοιχτού κώδικα κάτω από την άδεια zlib.
\end{itemize}
\section{Εικόνα και Βίντεο}

\subsection{Ιχνογράφηση ακτίνας}
\subsubsection{Πραγματικότητα}
Στην φύση, μια πηγή φωτός εκπέμπει μια ακτίνα φωτός όταν ταξιδεύει, προς μια επιφάνεια που εμποδίζει την πρόοδο της. Μπορούμε να σκεφτούμε την "ακτίνα" σαν μια ροή φωτονίων που ταξιδεύουν προς το ίδιο μονοπάτι. Στο απόλυτο κενό, αυτή η ακτίνα θα είναι μια ευθεία γραμμή, αν αγνοήσουμε την δράση της σχετικότητας. Ένας συνδυασμός της απορρόφησης, ανάκλασης, διάθλασης και φθορισμού μπορεί να σχηματιστεί από αυτήν την ακτίνα. Μια επιφάνεια μπορεί να απορροφήσει μέρος της ακτίνας, που θα έχει ως αποτέλεσμα την μείωση της έντασης της ανακλώμενης ακτίνας. Μπορεί επίσης να δημιουργήσει αντανάκλαση όλης η μέρος της ακτίνας, προς μία ή παραπάνω κατευθύνσεις. Αν η επιφάνεια έχει διάφανες ιδιότητες, θα διαθλάσει την ακτίνα στην ίδια και σε άλλη μια κατεύθυνση, και θα απορροφήσει μόνο μέρος της ακτίνας, ίσως αλλάζοντας και το χρώμα της. Έπειτα, οι ακτίνες μπορεί να φτάσουν σε επιπλέον επιφάνειες, όπου οι ιδιότητες τους θα επηρεάσουν ξανά την πρόοδο τους. Πολλές από αυτές τις ακτίνες, ταξιδεύουν με τρόπο που φτάνουν στο μάτι μας, επιτρέποντας μας να δούμε την σκηνή και να συνεισφέρουν στην τελική εικόνα. 


\subsubsection{Γραφικά υπολογιστών}
Στα γραφικά των υπολογιστών, η ιχνογράφηση ακτίνας είναι μια τεχνική για παραγωγή μιας εικόνας που χρησιμοποιεί ιχνογράφηση στο μονοπάτι του φωτός μέσα στον χώρο μιας εικόνας και εξομοιώνει τα εφέ τις επίδρασης του στα εικονικά αντικείμενα. Η τεχνική είναι δυνατόν να παράγει πολύ υψηλό βαθμό εικονικού ρεαλισμού, συνήθως μεγαλύτερο από τις τυπικές μεθόδους, αλλά με πολύ μεγαλύτερο υπολογιστικό κόστος. Αυτό κάνει την ιχνογράφηση ακτίνας πιο χρήσιμη για εφαρμογές όπου η εικόνα μπορεί να δημιουργηθεί σε βάθος του χρόνου, όπως οι σταθερές εικόνες και τα εφέ ταινιών, ενώ είναι συνήθως λιγότερο επιθυμητή σε εφαρμογές πραγματικού χρόνου, όπως τα παιχνίδια όπου η ταχύτητα είναι σημαντική. Η ιχνογράφηση ακτίνας μπορεί να εξομοιώσει ένα μεγάλο αριθμό από οπτικά εφέ, όπως η αντανάκλαση και η διάθλαση, διασκόρπιση και φαινόμενα διασποράς.

\begin{figure}[h]
\centering
\includegraphics[width=\linewidth]{ray-tracing-1}
\caption{Κάθε ακτίνα χρησιμοποιεί αντανάκλαση έως και 16 φορές}
\end{figure}

\subsubsection{Ιστορία}
Ο πρώτος αλγόριθμος ιχνογράφησης ακτίνας παρουσιάστηκε απο τον Arthur Appel το 1968. Ο αλγόριθμος ονομάστηκε από τότε διάχυση ακτίνας. Ένα σημαντικό πλεονέκτημα της ιχνογράφησης ακτίνας προς τους παραδοσιακούς scanline αλγορίθμους είναι η δυνατότητα να αντιμετωπίζει επιφάνειες όπως οι κώνοι και οι σφαίρες. Η επόμενη μεγάλη ανακάλυψη ήρθε απο τον Turner Whitted το 1979. Ενώ οι παλιοί αλγόριθμοι υπολόγιζαν μόνο την ακτίνα έως την στιγμή που θα έφτανε στο πρώτο εμπόδιο, ο Whittman συνέχισε την διαδικασία. Όταν μια ακτίνα φτάνει σε μια επιφάνεια, μπορεί να παράγει ως και τρεις τύπους ακτίνων: αντανάκλαση, διάθλαση, σκίαση. Αυτή η επαναληπτική ιχνογράφηση, έδωσε περισσότερο ρεαλισμό στις εικόνες.

\subsubsection{Μειονεκτήματα}
Ένα σοβαρό μειονέκτημα της ιχνογράφησης ακτίνας είναι η απόδοση. Οι τυπικοί αλγόριθμοι (scanline,κ.α) χρησιμοποιούν στοιχεία συνοχής για να διαμοιράσουν τους υπολογισμούς μεταξύ των εικονοστοιχείων, ενώ η ιχνογράφηση ακτίνας συνήθως ξεκινάει μια καινούρια διαδικασία, θεωρώντας την κάθε ακτίνα σαν ξεχωριστή οντότητα. Αυτός ο διαχωρισμός όμως δίνει και πλεονεκτήματα, όπως η ικανότητα να έχουμε περισσότερες ακτίνες για να δημιουργήσουμε anti-aliasing διαστήματος και να βελτιώσουμε την ποιότητα της εικόνας όπου χρειάζεται.

Αν και χειρίζεται τα οπτικά εφέ όπως η διασκόρπιση με ακρίβεια, η παραδοσιακή ιχνογράφηση ακτίνας δεν είναι πάντα φωτορεαλιστική. Ο πραγματικός φωτορεαλισμός επιτυγχάνεται όταν η εξίσωση ανταπόδοσης προσεγγίζεται πολύ κοντά ή εκτελείται ολοκληρωτικά. Η εκτέλεση της εξίσωσης ανταπόδοσης δίνει αληθινό φωτορεαλισμό, καθώς η εξίσωση περιγράφει κάθε φυσικό εφέ ή ροή φωτός. Συνήθως όμως, λόγω υπολογιστικών περιορισμών δεν είναι δυνατόν να συμβεί.

Ο ρεαλισμός των μεθόδων ανταπόδοσης μπορεί να αξιολογηθεί ως προσέγγιση της εξίσωσης. Η ιχνογράφηση ακτίνας, αν περιοριστεί στον αλγόριθμο του Whitted\cite{image-1} δεν είναι απαραίτητα η πιο ρεαλιστική. Οι μέθοδοι που ιχνογραφούν ακτίνες, αλλά περιέχουν επιπλέον τεχνικές (καταγραφή φωτονίων, ιχνογράφηση μονοπατιού), δίνουν μεγαλύτερη ακρίβεια εξομοίωσης του φωτισμού στον πραγματικό κόσμο.

Είναι επίσης δυνατό να προσεγγίσουμε την εξίσωση χρησιμοποιώντας διάχυση ακτίνας με διαφορετικό τρόπο από ότι χρησιμοποιείται συνήθως στην ιχνογράφηση ακτίνας. Για λόγους απόδοσης, οι ακτίνες μπορούν να συμπλεχθούν σε σχέση με την κατεύθυνση τους, με ειδικό υλικό να υπολογίζει τις ακτίνες.

\subsubsection{Προγράμματα}
Η μαζική παράλληλη υπολογιστική δύναμη των GPUs ταιριάζει τέλεια με την παράλληλη φύση της ιχνογράφησης ακτίνας, και βελτιώνει αισθητά τον χρόνο ανταπόδοσης για διάφορες βιομηχανίες, χρησιμοποιώντας διάφορους τύπους τεχνικών. Οι GPUs είναι ένα απαραίτητο εργαλείο για τις εταιρίες ανταπόδοσης. Παρακάτω βλέπουμε μια λίστα από προγράμματα που χρησιμοποιούν αυτήν την υπολογιστική δύναμη για να επιταχύνουν αισθητά τις υλοποιήσεις τους και να ορίσουν νέες διαδραστικές δυνατότητες.

\begin{figure}[h]
\centering
\includegraphics[width=\linewidth]{ray-tracing-2}
\caption{Ανταπόδοση φωτορεαλιστικής εικόνας σε 3,5 λεπτά με την χρήση του Furryball}
\end{figure}


\begin{itemize}
\item Nvidia Optix - Είναι ένα εργαλείο ιχνογράφησης ακτίνας για προγραμματιστές υπολογιστών που δημιουργούν εφαρμογές που δίνουν γρήγορα αποτελέσματα. Σε αντίθεση με υλοποιήσεις με υπαγορευμένη αισθητική, ή που περιορίζουν τις δομές δεδομένων τους ή και της γλώσσας ανταπόδοσης τους, η μηχανή Optix είναι υπερβολικά γενικευμένη, δίνοντας την δυνατότητα στους προγραμματιστές να επιταχύνουν ότι διεργασία ιχνογράφησης ακτίνας θέλουν και να την εκτελέσουν σε μονάδες επεξεργασίας γραφικών
\item Arion -  Eίναι ένα προϊόν επόμενης γενιάς που χρησιμοποιείται για εξομοίωση φωτός με τεχνολογία CUDA για να επιταχύνει την ανταπόδοση σε μια ή πολλές μονάδες επεξεργασίας γραφικών. Το Arion είναι ένας εξομοιωτής φωτός βασισμένος σε φυσικά φαινόμενα, ενώ μπορεί να χρησιμοποιήσει CPUs και GPUs για παραγωγή ανταπόδοσης υψηλών απαιτήσεων.
\item Furryball - Είναι ένα πρόγραμμα πραγματικού χρόνου ανταπόδοσης τελικής εικόνας με ανεπτυγμένες δυνατότητες ανταπόδοσης, το οποίο εκτελείται σαν πρόσθετο στις υλοποιήσεις Maya και Autodesk 3ds Max, με ανταπόδοση μέσω δικτύου και υποστήριξη πολλαπλών GPUs. Το Furryball είναι βασισμένο στο Nvidia Optix, ενώ συνδυάζει την ταχύτητα των GPU με την ποιότητα και τις δυνατότητες της ανταπόδοσης CPU. Η δύναμη του έχει αποδειχθεί σε πραγματικές ταινίες και παραγωγές παιχνιδιών από πολλές εταιρίες.
\item Lightworks - Εκμεταλλεύεται το Nvidia Optix για να δημιουργήσει μια καινούρια γενιά πολύ γρήγορων μηχανών ιχνογράφησης ακτίνας για αρχιτεκτονικές, βιομηχανικές ανάγκες, και ανάγκες εσωτερικής διακόσμησης χώρων.
\item Octane Renderer - Είναι μια διαδραστική μηχανή ανταπόδοσης βασισμένη σε GPU, που παράγει φωτορεαλιστικές ανταποδόσεις με μεγάλη ταχύτητα. Αυτό επιτρέπει στους χρήστες να δημιουργήσουν έργα υψηλής αισθητικής, σε ένα κλάσμα του χρόνου που χρειάζεται από τις μηχανές ανταπόδοσης σε CPU. Η εταιρία που δημιουργεί το Octane Renderer είναι μια απο τις πρώτες που ασχολήθηκαν με τις μηχανές φωτορεαλιστικής ανταπόδοσης και έπαιξε μεγάλο ρόλο στην εξέλιξη τους.
\item V-ray - Είναι μια μηχανή ανταπόδοσης εικόνας που χρησιμοποιεί τις διαθέσιμες GPUs που βρίσκονται στο σύστημα, αντί για την CPU. Χρησιμοποιεί OpenCL, και βρίσκεται ως πρόσθετο στο πρόγραμμα Autodesk 3ds Max.
\end{itemize}



\begin{apptable}{Μοντελοποίηση}{modeling}
Autodesk 3ds Max + NVIDIA iray & 3D modeling, animation, and rendering  & & ΝΑΙ \\ \hline
Autodesk Maya & 3D modeling, animation, and rendering  & & ΝΑΙ \\ \hline
Autodesk Motion Builder & Character animation and motion capture & & ΝΑΙ \\ \hline
Autodesk Mudbox & 3D sculpting  & & ΝΑΙ \\ \hline
Cebas finalRender & GPU Renderer  & & ΝΑΙ \\ \hline
CentiLeo GPU Render & GPU Renderer & & ΝΑΙ \\ \hline
Chaos V-Ray RT  & GPU Renderer  & & ΝΑΙ \\ \hline
Jawset TurbulenceFD & Physics-based simulation plug in  & & ΝΑΙ \\ \hline
Maxon Cinema 4D & 3D modeling, animation, and rendering & & ΝΑΙ \\ \hline
NewTek Lightwave & 3D modeling, animation, and rendering & & ΝΑΙ \\ \hline
Otoy Octane Render & GPU Renderer  & & ΝΑΙ \\ \hline
Pixologic Sculptris & 3D sculpting  & & ΝΑΙ \\ \hline
Redshift Renderer & GPU-accelerated, biased renderer  & & ΝΑΙ \\ \hline
Side Effects Houdini & 3D modeling, animation, and rendering   & & ΝΑΙ \\ \hline
The Foundry Mari & 3D Paint & & ΝΑΙ \\ \hline
\end{apptable}

\subsection{Μετατροπή}
Η μετατροπή είναι η άμεση μετατροπή ψηφιακής κωδικοποίησης, όπως για παράδειγμα σε αρχεία μιας ταινίας(π.χ .mp4, avi), αρχεία ήχου(.wav .mp3), ή κωδικοποίηση χαρακτήρων. (π.χ UTF-8, ISO/IEC 8859). Αυτή γίνεται σε περιπτώσεις που η συσκευή στόχος δεν υποστηρίζει την μορφή του αρχείου ή χρειαζόμαστε μικρότερο μέγεθος αρχείου, ή για να μετατρέψουμε αρχεία παλαιότερης μορφής σε μια πιο σύγχρονη για καλύτερη υποστήριξη σε μελλοντικές εφαρμογές. 

Η μετατροπή χρησιμοποιείται συχνά στα λογισμικά προβολής βίντεο για να ελαττώσουμε το μέγεθος του αρχείου βίντεο. Μια διαδικασία που γίνεται συχνά είναι η μετατροπή από αρχεία MPEG-2(DVD) σε αρχεία μορφής MPEG-4, που ενσωματώνει σύγχρονους αλγόριθμους για καλύτερη ποιότητα εικόνας σε συνδυασμό με μικρότερο μέγεθος αρχείου.

\begin{figure}[h]
\centering
\includegraphics[width=0.5\linewidth]{h264-logo}
\caption{H264, το πιο δημοφιλές πρότυπο συμπίεσης βίντεο}
\end{figure}
\subsection{Μειονεκτήματα} 
Το μεγαλύτερο μειονέκτημα της μετατροπής σε απωλεστικές μορφές αρχείου είναι η μειωμένη ποιότητα. Τα τεχνουργήματα συμπίεσης συσσωρεύονται, οπότε κάθε διαδικασία μετατροπής δημιουργεί μια βαθμιαία απώλεια ποιότητας, που είναι γνωστή ως ψηφιακή απώλεια. Για αυτόν τον λόγο, η μετατροπή συνήθως δεν συνίσταται εκτός και αν δεν μπορούμε να την αποφύγουμε.

\subsection{Βίντεο}
H συμπίεση βίντεο χρησιμοποιεί σύγχρονες τεχνικές για να μειώσει πλεονασμούς στα δεδομένα βίντεο. Οι περισσότεροι αλγόριθμοι συμπίεσης βίντεο συνδυάζουν συμπίεση εικόνας και προσωρινή αποζημίωση κίνησης. Ο ήχος κωδικοποιείται παράλληλα με διαφορετικούς αλγόριθμους συμπίεσης άλλα συνήθως συνδυάζεται σε ένα πακέτο όταν ολοκληρωθεί η διαδικασία.

Οι περισσότεροι αλγόριθμοι συμπίεσης βίντεο χρησιμοποιούν απωλεστική συμπίεση, καθώς το ασυμπίεστο βίντεο απαιτεί πολύ μεγάλους ρυθμούς μετάδοσης. Αν και οι περισσότεροι αλγόριθμοι έχουν ένα παράγοντα συμπίεσης 3, μια τυπική συμπίεση βίντεο MPEG-4 μπορεί να έχει παράγοντα συμπίεσης από 20 έως και 200. Όπως σε όλες τις απωλεστικές διαδικασίες συμπίεσης, υπάρχει ένα δίλημμα μεταξύ της ποιότητας του βίντεο, το κόστος της επεξεργασίας της συμπίεσης και της αποσυμπίεσης, και των απαιτήσεων του συστήματος. Υπερβολικά συμπιεσμένο βίντεο μπορεί να δημιουργήσει οπτικά τεχνουργήματα.

Συνήθως ο τρόπος με τον οποίο λειτουργεί ένας αλγόριθμος συμπίεσης βίντεο είναι με ομάδες γειτονικών εικονοστοιχείων(pixel), που ονομάζονται blocks. Αυτές οι ομάδες από pixels, συγκρίνονται μεταξύ τους από μια εικόνα στην επόμενη, και ο αλγόριθμος αποστέλλει μόνο τις αλλαγές μεταξύ αυτών των block. Σε περιοχές του βίντεο που υπάρχει μεγαλύτερη κίνηση, ο αλγόριθμος πρέπει να συμπιέσει περισσότερα δεδομένα για να προλάβει τον μεγαλύτερο αριθμό εικονοστοιχείων που αλλάζουν. Συνήθως σε σκηνές με φωτιά, εκρήξεις, καπνούς, το αποτέλεσμα της συμπίεσης έχει μεγαλύτερη απώλεια ποιότητας, η αύξηση του ρυθμού μετάδοσης.

Μερικοί από τους σύγχρονους αλγόριθμους μετατροπής είναι οι παρακάτω
\begin{itemize}
\item Lagarith - Είναι ένας μη απωλεστικός αλγόριθμος ανοιχτού κώδικα που δίνει έμφαση στην ταχύτητα, στην υποστήριξη διαφόρων χώρου χρωμάτων(YV12,RGB,YUY2). Είναι ιδανικός για επεξεργασία και αποθήκευση αρχείων. Αν και υπάρχουν κάποιοι καλύτεροι μη απωλεστικοί αλγόριθμοι συμπίεσης, ο lagarith είναι ο πιο γρήγορος και έτσι έχει κερδίσει την υποστήριξη της κοινότητας.
\item VP9 - Είναι ένα ελεύθερο πρότυπο ανοιχτού κώδικα που αναπτύσσεται από την Google. Είναι ο διάδοχος του VP8. Σκοπός του είναι να μειώσει περισσότερο τον χώρο που απαιτείται από το βίντεο διατηρώντας την ίδια ποιότητα. 
\item H.264 - Είναι ένα πρότυπο συμπίεσης βίντεο, που είναι ίσως το πιο επιτυχημένο για την καταγραφή, συμπίεση, και αναμετάδοση περιεχομένου βίντεο. Χρησιμοποιείται για αναμετάδοση ψηφιακού σήματος τηλεόρασης, δορυφόρων, και από υπηρεσίες αναμετάδοσης στο διαδίκτυο.
\item H.265 - Είναι ένα πρότυπο συμπίεσης βίντεο, διάδοχος του επιτυχημένου H.264. Σκοπός του είναι ο διπλασιασμός της συμπίεσης διατηρώντας την ίδια ποιότητα. Υποστηρίζει αναλύσεις έως και 8192x4320.
\item Daala - Είναι μια τεχνολογία συμπίεσης από το ίδρυμα Xiph.Org. Χρησιμοποιεί περιτύλιξη μεταμορφώσεων για να μειώσει τα οπτικά τεχνουργήματα. Ο στόχος είναι η απόδοση του να ξεπεράσει τις δυνατότητες του VP9 και του H.265. 
\end{itemize}

\subsection{Συμπίεση δεδομένων γενετικής}
Οι γενετικοί αλγόριθμοι συμπίεσης είναι η τελευταία γενιά μη απωλεστικών αλγορίθμων για συμπίεση δεδομένων (συνήθως αλληλουχίες nucleotides) χρησιμοποιώντας συμβατικούς αλγορίθμους συμπίεσης αλλά και γενετικούς αλγόριθμους βελτιστοποιημένους στο συγκεκριμένο τύπο δεδομένων. Το 2012, μια ομάδα επιστημόνων απο το John Kopkins University ανακοίνωσε έναν αλγόριθμο συμπίεσης γενετικής(HapZipper), ο οποίος χρησιμοποιεί HapMap δεδομένα για να πετύχει συμπίεση 95\% μείωση στο μέγεθος αρχείου, πετυχαίνοντας καλύτερη συμπίεση σε πολύ καλύτερο χρόνο από ότι οι γνωστοί μη απωλεστικοί αλγόριθμοι. Άλλοι γενετικοί αλγόριθμοι συμπίεσης π.χ GenomeZip πετυχαίνουν μεγαλύτερη συμπίεση καταφέρνοντας αποθήκευση 6 δισεκατομμυρίων ανθρώπινων ζευγαριών γονιδιώματος σε 2.5 megabyte.

\subsection{Προγράμματα}
\begin{apptable}{Κωδικοποίηση βίντεο}{videoencode}
ArcVideo Core & Εξαιρετικά γρήγορο και υψηλής απόδοσης σύστημα επεξεργασίας βίντεο και κωδικοποίησης & & ΝΑΙ \\ \hline
ArcVideo Live & Υψηλής πυκνότητας σύστημα επεξεργασίας βίντεο πραγματικού χρόνου και κωδικοποίησης & & ΝΑΙ \\ \hline
Cinnafilm Tachyon & Μετατροπή προτύπων & & ΝΑΙ \\ \hline
Digimetrics Aurora & Δοκιμές αυτοματοποιημένου βίντεο και μετρήσεις ήχου & & ΝΑΙ \\ \hline
Elemental Live & Ζωντανή μετάδοση επεξεργασίας βίντεο και κωδικοποίησης & & ΝΑΙ \\ \hline
Elemental Server & Επεξεργασία και κωδικοποίηση βίντεο βασισμένο σε αρχεία & & ΝΑΙ \\ \hline
isovideo Viarte  & Μετατροπή προτύπων βίντεο & & ΝΑΙ \\ \hline
MainConcept CUDA H.264/AVC Encoder SDK & Κωδικοποίηση βίντεο με χρήση αλγορίθμου H.264 & & ΝΑΙ \\ \hline
Snell Alchemist on Demand & Μετατροπή προτύπων βίντεο & & ΝΑΙ \\ \hline
Sorenson Squeeze & Εφαρμογή μετατροπής κωδικοποίησης βίντεο και plug-In & & ΝΑΙ \\ \hline
Telestream Vantage & Μετατροπή κωδικοποίησης βίντεο και επεξεργασία  & & ΝΑΙ \\ \hline
\end{apptable}

\begin{apptable}{Επεξεργασία Βίντεο}{videoediting}

\end{apptable}


\subsection{Επεξεργασία εικόνας}

\subsubsection{Προγράμματα}


\begin{apptable}{Βελτίωση εικόνας}{imageimprovement}
Adobe SpeedGrade CC & Color grading  & & ΝΑΙ \\ \hline
Assimilate Scratch & Color grading and finishing & & ΝΑΙ \\ \hline
Blackmagic DaVinci & Color grading  & & ΝΑΙ \\ \hline
Resolve & & & ΝΑΙ \\ \hline
Cinnafilm Dark Energy & Application and plug-in for image enhancement & & ΝΑΙ \\ \hline
Digital Vision Nucoda  & Color grading  & & ΝΑΙ \\ \hline
HS-ART DIAMANT-Film & Image restoration \& enhance & & ΝΑΙ \\ \hline
Restoration & & & ΝΑΙ \\ \hline
Marquise Technologies Rain & Color grading  & & ΝΑΙ \\ \hline
Quantel Pablo Rio & & & ΝΑΙ \\ \hline
Red Digital Cinema & Color grading  & & ΝΑΙ \\ \hline
REDCINE-X & & & ΝΑΙ \\ \hline
SGO Mistika & Color grading and finishing & & ΝΑΙ \\ \hline
The Pixel Farm PFClean & Image restoration and remastering  & & ΝΑΙ \\ \hline
\end{apptable}

\begin{apptable}{Επεξεργασία εικόνας}{imageediting}
Adobe Photoshop CC & Image editing & & ΝΑΙ \\ \hline
Adobe Premiere Pro CC & Video editing  & & ΝΑΙ \\ \hline
Apple Final Cut Pro & Video editing  & & ΝΑΙ \\ \hline
Avid Media Composer & Video editing  & & ΝΑΙ \\ \hline
Grass Valley Edius & Video editing  & & ΝΑΙ \\ \hline
Harris Velocity & Video editing  & & ΝΑΙ \\ \hline
Quantel Qube & Broadcast video editing & & ΝΑΙ \\ \hline
Sony Vegas Pro & Video editing  & & ΝΑΙ \\ \hline
VidiCert & Video essence quality checking for production and preservation & & ΝΑΙ \\ \hline
\end{apptable}

%\begin{apptable}{Σύνθεση, Τελειοποίηση, εφέ}{imageeffects}
%ABSoft Neat Video  & Video noise reduction plug-in & & ΝΑΙ \\ \hline
%Adobe After Effects CC  & Motion graphics and effects & & ΝΑΙ \\ \hline
%Autodesk Flame Premium & Finishing and color grading  & & ΝΑΙ \\ \hline
%Autodesk Smoke  & Finishing and editing & & ΝΑΙ \\ \hline
%Boris FX Continuum Complete & Visual effects plug-in  & & ΝΑΙ \\ \hline
%Cinnafilm Dark Energy Plug-in & Color management  & & ΝΑΙ \\ \hline
%CoreMelt complete & Visual effects plug-in  & & ΝΑΙ \\ \hline
%eyeon Fusion  & Effects and compositing & & ΝΑΙ \\ \hline
%GenArts Monsters GT & Visual effects plug-in  & & ΝΑΙ \\ \hline
%GenArts Sapphire & Visual effects plug-in  & & ΝΑΙ \\ \hline
%HS-ART DIAMANT-Film Restoration & Object removal and retouching  & & ΝΑΙ \\ \hline
%Neat Video Open FX  & Video noise reduction plug-in  & & ΝΑΙ \\ \hline
%NewBlueFX Video Essentials & Video effects plug-in & & ΝΑΙ \\ \hline
%NewBlue Titler Pro  & Video titling plug-in  & & ΝΑΙ \\ \hline
%Pixelan AnyFX  & Video effects plug-in  & & ΝΑΙ \\ \hline
%Re:Vision Effects  & Visual effects plug-in & & ΝΑΙ \\ \hline
%Red Giant Effects Suite  & Visual effects plug-ins  & & ΝΑΙ \\ \hline
%Red Giant Magic Bullet Looks & Color and finishing tools  & & ΝΑΙ \\ \hline
%ROBUSKEY  & Chroma keyer plug-in  & & ΝΑΙ \\ \hline
%The Foundry HIERO  & Shot management, conform and review timeline & & ΝΑΙ \\ \hline
%The Foundry NUKE and NUKEX & Compositing tools with 3D tracker  & & ΝΑΙ \\ \hline
%Video Copilot Software Element 3D & 3D object based particle system  & & ΝΑΙ \\ \hline
%Video Copilot Optical Flares & Lens flares plug-in for After Effects  & & ΝΑΙ \\ \hline
%Video Copilot Twitch  & Video effects plug-in for After Effects  & & ΝΑΙ \\ \hline
%\end{apptable}
\section{Γεωπληροφορική}
\subsection{Γεωγραφικά πληροφοριακά συστήματα}



\subsection{Ωκεανοί}

\subsection{Σεισμική δραστηριότητα}
Η ενσωματωμένη εξομοίωση σεισμικής δραστηριότητας είναι μια ανώτερη τεχνολογία για προσδιορισμό και απεικόνιση της δομικής βλάβης που δημιουργείται σε ένα σενάριο σεισμικής δραστηριότητας. Στο IES, όλα τα κτήρια που βρίσκονται στην περιοχή δοκιμής απεικονίζονται σαν δομικά μοντέλα, και η δομική ζημιά μπορεί να προσδιοριστεί με γραμμική και μη-γραμμική δυναμική δομική ανάλυση. Επειδή υπάρχουν πολλά κτήρια σε μια περιοχή δοκιμής, και κάθε κτήριο έχει διαφορετικά χαρακτηριστικά, ο υπολογισμός υψηλής ανάλυσης είναι απαραίτητος ώστε να εκτελεστεί η ανάλυση σε σχετικά μικρό χρόνο.

\begin{figure}[h]
\centering
\includegraphics[scale=0.50]{ies}
\caption{Ενσωματωμένη εξομοίωση σεισμικής δραστηριότητας}
\end{figure}

Ένας καλός υποψήφιος για τον IES, είναι το OBASAN (Object Based Structural Analysis). Σε μια εξομοίωση σεισμικής δραστηριότητας, κάθε κτήριο σε μια περιοχή δοκιμών μπορεί να παρασταθεί ως δομικό μοντέλο για ανάλυση του κάθε αντικειμένου. Το γεωγραφικό πληροφοριακό σύστημα παρέχει δεδομένα δομικών σχημάτων στην μορφή των στοιχείων και πολυγώνων, και άλλα συστήματα βάσεων δεδομένων παρέχουν τύπους κτηρίων (π.χ δομή οπλισμένου σκυροδέματος, δομή χάλυβα, ξύλινη δομή, κ.α) Βασισμένο στον τρόπο σχεδιασμού της Ιαπωνίας, η ανάλυση μοντέλων των κτηρίων στον IES δημιουργούνται αυτόματα από τις διαστάσεις και τον τύπο του κάθε κτηρίου, και τότε το OBASAN μπορεί να χρησιμοποιηθεί για να αναλύσει αυτα τα μοντέλα δοκιμής ανάλυσης. Η δοκιμή ζημιά σε όλα τα κτήρια μιας περιοχής δοκιμής μπορεί να προβλεφθεί και η ολική ζημιά σε μια περιοχή πόλης μπορεί να απεικονιστεί από αυτήν την εξομοίωση σεισμού.

Τα πιο πολύπλοκα δομικά μοντέλα μπορούν να αυξήσουν την αξιοπιστία του προσδιορισμού δομικής βλάβης, π.χ τα μοντέλα μπορούν να χρησιμοποιηθούν για δυναμική ανάλυση δομικής απόκρισης. Μοντέλα δομικών αντικειμένων, μπορούν να χρησιμοποιηθούν για να δώσουν ένα πιο αξιόπιστο αποτέλεσμα. 

To OBASAN, είναι μια τεχνολογία που δημιουργήθηκε για την ανάλυση των δομικών μοντέλων ώστε να δέχεται τα αποτελέσματα των αντικειμένων (αξονική δύναμη, παραμόρφωση, καταπόνηση) και τα αποτελέσματα των κόμβων (π.χ ταχύτητα, επιτάχυνση, ισχύς) κάθε κτηρίου σε μια εξομοίωση σεισμικής δραστηριότητας, και έτσι το OBASAN μπορεί να παρέχει διάφορους τύπους αποτελεσμάτων δομικής ανάλυσης. Υπάρχουν δυνατότητες για χρήση υπολογισμού υψηλής απόδοσης ώστε να αυξηθεί η αποδοτικότητα και η δυνατότητα του συστήματος εξομοίωσης σεισμών (IES). 

\begin{figure}[h]
\centering
\includegraphics[scale=0.50]{obasan}
\caption{OBASAN}
\end{figure}

Το OBASAN αναπτύχθηκε με αντικειμενοστραφή προγραμματισμό σε γλώσσα C++. Η γλώσσα C++ μπορεί να επεκταθεί ώστε να χρησιμοποιεί παράλληλο προγραμματισμό στην GPU. Όπως φαίνεται στο διάγραμμα, η αρχιτεκτονική του OBASAN έχει οργανωθεί συστηματικά και είναι εύκολο να κατανοηθεί. Λόγω του ότι υπάρχουν πολλά κτήρια σε μια εξομοίωση σεισμικής δραστηριότητας σε μια περιοχή δοκιμών, ο IES χρησιμοποιεί την αρχιτεκτονική OpenMPI για να ενεργοποιήσει παράλληλο υπολογισμό στην κεντρική μονάδα επεξεργασίας (CPU). Στην CPU, ένα υποσύνολο (π.χ 10) κτηρίων μπορεί να εκτελεστεί με τον μέγιστο αριθμό νημάτων σε έναν παράλληλο υπολογισμό. Όπως και στον IES όπου η αρχιτεκτονική OpenMPI χρησιμοποιείται για τον παράλληλο υπολογισμό των κτηρίων σε μια περιοχή δοκιμών, στο OBASAN χρησιμοποιείται η αρχιτεκτονική CUDA για την παράλληλη επεξεργασία της δυναμικής ανάλυσης δομικής απόκρισης(DSRA).

Λόγω του ότι το DSRA έχει να κάνει με πιο πολύπλοκα μοντέλα, η δομική ανάλυση των μοντέλων πρέπει να παρέχει μεγάλο αριθμό παραμέτρων, που εκφράζονται με μαθηματικές εξισώσεις που περιέχουν μεγάλους πίνακες. Για ένα τόσο προχωρημένο υπολογιστικό έργο, η τεχνική HPC είναι η συνηθέστερη προσέγγιση για την επίλυση των μαθηματικών εξισώσεων σε μικρό χρόνο. Στο HPC, η τεχνολογία GPGPU είναι σχεδιασμένη ώστε να αντιμετωπίζει μεγάλο αριθμό δεδομένων και μεγάλο αριθμό υπολογισμών σε μαθηματικές εξισώσεις. Στο GPGPU, η αρχιτεκτονική CUDA μπορεί να επιλεχτεί για να λύσει αυτές τις μαθηματικές εξισώσεις του DSRA. Το κάθε βήμα της ανάλυσης του DSRA, απαιτεί πολλαπλασιασμούς πινάκων για την δομική ανάλυση.

Στο παρακάτω σχήμα μπορούμε να δούμε ένα παράδειγμα χρόνου εκτέλεσης πολλαπλασιασμού πινάκων. Οι πίνακες Α,Β και C υποθέτουμε ότι είναι τετράγωνοι πίνακες και το μέγεθος τους είναι Ν * Ν. Ο κώδικας C++ του πολλαπλασιασμού πινάκων εκτελείται σε μια CPU (Intel Xeon), ενώ οι εντολές CUDA σε μια GPU(Tesla M2050). 

\begin{figure}[h]
\centering
\includegraphics[scale=0.50]{matrix-multi}
\caption{OBASAN}
\end{figure}


\section{Αστροφυσική}
Οι πρώτες καταγραφές της θεωρητικής αστρονομίας χρονολογούνται το 1550-1292 π.Χ Οι υπολογισμοί που βρέθηκαν σχηματισμένοι σε αιγυπτιακούς τάφους δείχνουν ότι υπήρχαν τεχνικές για την αναγνώριση και την καταγραφή προτύπων στον ουρανό, ένα πεδίο της αστρονομίας που είναι χρήσιμο ακόμα και σήμερα για την καταγραφή και χαρτογράφηση. Μπορεί οι σημερινοί ερευνητές να μην χρησιμοποιούν τα αρχαία εργαλεία, αλλά η ανάπτυξη τους έγινε σταδιακά. \\
Το μέγεθος των τηλεσκοπίων χρειάστηκε 400 χρόνια για να μεγαλώσει από 1 τετραγωνικό μέτρο σε 110 τετραγωνικά μέτρα. Οι ψηφιακοί υπολογιστές εμφανίστηκαν στο τέλος του 1940 με υπολογιστική ταχύτητα περίπου 100 floating point λειτουργιών το δευτερόλεπτο (FLOPS), και εξελίχθηκαν σε περίπου $3*10^{16}$ FLOPS σε λιγότερο από 65 χρόνια. Αυτή η επανάσταση των υπολογιστών συνεχίζεται ακόμα και σήμερα, και έχει οδηγήσει σε ένα καινούριο κομμάτι έρευνας στο οποίο οι εγκαταστάσεις δεν βρίσκονται στο ψηλότερο βουνό του κόσμου, αλλά στο διπλανό δωμάτιο. Οι Αστρονόμοι κατανόησαν γρήγορα ότι μπορούν να χρησιμοποιήσουν τους υπολογιστές για να καταγράψουν, αναλύσουν, αρχειοθετήσουν, τις τεράστιες ποσότητες πληροφοριών που καταγράφονται από τις εκστρατείες παρατηρητών. \\
\begin{figure}[h]
\centering
\includegraphics[width=\linewidth]{astrophysics}
\caption{Αστροφυσική}
\end{figure}
Την μεγαλύτερη όμως επίδραση στον τρόπο με τον οποίο αντιμετωπίζουν οι αστρονόμοι τα αναπάντητα ερωτήματα τους, έχει το πεδίο της εξομοίωσης. Με την χρήση των υπολογιστών, είναι δυνατόν να μελετήσουμε την λειτουργία του Διαγαλαξιακούς κενού, την φυσική των μαύρων τρυπών, κ.α


\begin{apptable}{Φυσική}{physics}
Chemora & Chemora is a system for performing simulations of systems described by differential equations running on accelerated computational clusters & & ΝΑΙ \\ \hline
Chroma & Lattice Quantum Chromodynamics (LQCD) & & ΝΑΙ \\ \hline
ENZO & 3D block-structured AMR code for cosmological structure formation & & ΝΑΙ \\ \hline
GTC & Simulates microturbulence and transport in magnetically confined fusion plasma & & ΝΑΙ \\ \hline
GTS & Simulates microturbulence and the motion of charged particles and interactions in fusion plasma & & ΝΑΙ \\ \hline
MILC & Lattice Quantum Chromodynamics (LQCD) codes simulate how elemental particles are formed and bound by the “strong force” to create larger particles like protons and neutrons & & ΝΑΙ \\ \hline
PIConGPU & A relativistic Particle-in-Cell code that describes the dynamics of a plasma by computing the motion of electrons and ions subject to the Maxwell-Vlasov equation. & & ΝΑΙ \\ \hline
QUDA & Library for Lattice QCD calculations using GPUs & & ΝΑΙ \\ \hline
RAMSES  & Simulates astrophysical problems on different scales (e.g. star formation, galaxy dynamics, cosmological structure formation) & & ΝΑΙ \\ \hline
\end{apptable}