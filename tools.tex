\section{Εργαλεία που χρησιμοποιήθηκαν}

Για την εκπόνηση της πτυχιακής εργασίας, χρησιμοποιήθηκαν αρκετά εργαλεία για λόγους ευχρηστίας αλλά και για εκπαιδευτικούς σκοπούς, σε μια προσπάθεια να αποκτηθούν γνώσεις για τεχνολογίες που πιστεύω πως χρειάζονται σε έναν απόφοιτο πληροφορικής.\\
1) \hologo{XeTeX} - Texmaker\\
Το \hologo{XeTeX} είναι μια μηχανή τυπογραφίας τύπου TeX η οποία χρησιμοποιεί κωδικοποίηση Unicode και υποστηρίζει σύγχρονες τεχνολογίες γραμματοσειρών όπως οι Opentype, Graphite και Apple Advanced Typography. Έχει σχεδιαστεί από τον Jonathan Kew και διανέμεται κάτω από την ελεύθερη άδεια λογισμικού X11. Ενώ δημιουργήθηκε αποκλειστικά για το Mac OS X, πλέον είναι διαθέσιμο για όλες τις γνωστές πλατφόρμες. Υποστηρίζει κωδικοποίηση Unicode και τα αρχεία κειμένου είναι εξαρχής σε μορφή UTF-8. Το \hologo{XeTeX} μπορεί να χρησιμοποιήσει τις γραμματοσειρές που είναι εγκατεστημένες στο σύστημα, και να κάνει χρήση των ανεπτυγμένων τυπογραφικών δυνατοτήτων τους. Υποστηρίζει και μικρο-τυπογραφία, δηλαδή μια σειρά από μεθόδους που βελτιώνουν την αισθητική του κειμένου και το καθιστούν πιο ευανάγνωστο. Οι μέθοδοι συμπεριλαμβάνουν την μείωση μεγάλων κενών μεταξύ των λέξεων(expansion), την επέκταση των γραμμών όταν τελειώνουν με κάποιο μικρό σύμβολο, όπως η τελεία η ένα στρογγυλό γράμμα όπως το "ο" (protrusion), και ο χωρισμός γραμμών (hyphenation). Αν και το \hologo{XeTeX} είναι υποδεέστερο του \hologo{LaTeX} στην μικρο-τυπογραφία, επιλέχτηκε για αυτήν την πτυχιακή εργασία λόγω των πολλών άλλων πλεονεκτημάτων, όπως η χρήση unicode χαρακτήρων αλλά και η ευκολία διαχείρισης γραμματοσειρών. Το \hologo{XeTeX} 

2) Git - TortoiseGit
3) 