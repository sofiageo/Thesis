\chapter{Εργαλεία που χρησιμοποιήθηκαν}
\section{Εισαγωγή}
Για την εκπόνηση της πτυχιακής εργασίας, χρησιμοποιήθηκαν αρκετά εργαλεία για λόγους ευχρηστίας αλλά και για εκπαιδευτικούς σκοπούς, σε μια προσπάθεια να αποκτηθούν γνώσεις για τεχνολογίες που πιστεύω πως χρειάζονται σε έναν απόφοιτο πληροφορικής.
\subsection{\hologo{XeTeX}}
Το \hologo{XeTeX} είναι μια μηχανή τυπογραφίας τύπου \hologo{TeX} η οποία χρησιμοποιεί κωδικοποίηση Unicode και υποστηρίζει σύγχρονες τεχνολογίες γραμματοσειρών όπως οι Opentype, Graphite και Apple Advanced Typography. Έχει σχεδιαστεί από τον Jonathan Kew και διανέμεται κάτω από την ελεύθερη άδεια λογισμικού X11. Ενώ δημιουργήθηκε αποκλειστικά για το Mac OS X, πλέον είναι διαθέσιμο για όλες τις γνωστές πλατφόρμες. Υποστηρίζει κωδικοποίηση Unicode και τα αρχεία κειμένου είναι εξαρχής σε μορφή UTF-8.

Το \hologo{XeTeX} μπορεί να χρησιμοποιήσει τις γραμματοσειρές που είναι εγκατεστημένες στο σύστημα, και να κάνει χρήση των ανεπτυγμένων τυπογραφικών δυνατοτήτων τους. Υποστηρίζει και μικρο-τυπογραφία, δηλαδή μια σειρά από μεθόδους που βελτιώνουν την αισθητική του κειμένου και το καθιστούν πιο ευανάγνωστο. Οι μέθοδοι συμπεριλαμβάνουν την μείωση μεγάλων κενών μεταξύ των λέξεων(expansion), την επέκταση των γραμμών όταν τελειώνουν με κάποιο μικρό σύμβολο, όπως η τελεία η ένα στρογγυλό γράμμα όπως το "ο" (protrusion), και ο χωρισμός γραμμών (hyphenation). Αν και το \hologo{XeTeX} είναι υποδεέστερο του \hologo{LaTeX} στην μικρο-τυπογραφία, επιλέχτηκε για αυτήν την πτυχιακή εργασία λόγω των πολλών άλλων πλεονεκτημάτων, όπως η χρήση unicode χαρακτήρων αλλά και η ευκολία διαχείρισης γραμματοσειρών. Το \hologo{XeTeX} 

\subsection{Texmaker}
Το Texmaker είναι ένα επεξεργαστής κειμένου για \hologo{LaTeX} και \hologo{XeLaTeX}. Είναι cross-platform ανοιχτού κώδικα με ενσωματωμένο PDF Viewer. Το Texmaker είναι μια εξολοκλήρου εφαρμογή Qt.
O επεξεργαστής κειμένου έχει άρτια υποστήριξη unicode, λειτουργία ελέγχου ορθογραφίας, αυτόματη συμπλήρωση κειμένου, κ.α. Επίσης υποστηρίζει regular expressions για την εύρεση και αντικατάσταση κειμένου.

Το Texmaker περιέχει λειτουργίες για αυτόματη χρήση των παρακάτω λειτουργιών:
\begin{itemize}
\item Παραγωγή νέου εγγράφου, γράμματος, βιβλίου, κ.α
\item Παραγωγή πινάκων, περιβάλλοντος σχημάτων, κ.α
\item Εξαγωγή κειμένου σε HTML ή ODT διάταξης
\end{itemize} 

Μερικές από τις ετικέτες \hologo{LaTeX} και μαθηματικά σύμβολα μπορούν να χρησιμοποιηθούν με ένα click. Ο Texmaker βρίσκει αυτόματα τα λάθη και προειδοποιήσεις και τα εμφανίζει στο αρχείο καταγραφής.

\subsection{TortoiseGit}
Το TortoiseGIT είναι μια εφαρμογή που υλοποιεί ένα σύστημα διαχείρισης εκδόσεων λογισμικού. Το σύστημα αυτό, είναι ένα υποσύνολο του ελέγχου αναθεωρήσεων. Ο έλεγχος αναθεωρήσεων χρησιμοποιείται για την διαχείριση των αλλαγών σε κείμενα, προγράμματα υπολογιστών, ιστοσελίδες, και άλλες συλλογές απο πληροφορίες. Οι αλλαγές συνήθως αναγνωρίζονται απο έναν αριθμό η απο ένα γράμμα, το οποίο ονομάζεται αριθμός αναθεώρησης.\\

Για παράδειγμα, αν τα αρχεία είχαν αρχικό αριθμό το "αναθεώρηση 1", μετά τις αλλαγές θα αποκτήσει τον αριθμό "αναθεώρηση 2", κτλ. Κάθε αναθεώρηση σχετίζεται με μια χρονοσήμανση, και με το άτομο το οποίο πραγματοποίησε την αλλαγή. Οι αναθεωρήσεις μπορούν να συγκριθούν, να αποκαθιστούν, και με κάποιους τύπους αρχείων, να συγχωνευτούν.

Η ανάγκη για ένα λογικό τρόπο οργάνωσης και ελέγχου αναθεωρήσεων υπάρχει σχεδόν απο τότε που εφευρέθηκε η γραφή, αλλά ο έλεγχος αναθεωρήσεων έγινε πιο σημαντικός και πολύπλοκος όταν ξεκίνησε η εποχή των υπολογιστών. Η αρίθμηση των εκδόσεων βιβλίων είναι ένα απο τα παραδείγματα της προηγούμενης εποχής. Σήμερα, τα πιο πολύπλοκα εργαλεία ελέγχου αναθεωρήσεων είναι αυτά που χρησιμοποιούνται στην δημιουργία λογισμικού, στα οποία μια ομάδα απο ανθρώπους μπορεί να αλλάξει τα ίδια αρχεία.

Μπορεί το σύστημα ελέγχου να είναι ένα ξεχωριστό πρόγραμμα, αλλα ο έλεγχος αναθεωρήσεων βρίσκεται ενσωματωμένος και μέσα σε διάφορους τύπους λογισμικού όπως οι επεξεργαστές κειμένου, αλλα και συστήματα διαχείρισης περιεχομένου
π.χ Wikipedia, Wordpress, κ.α.

Ο έλεγχος αναθεωρήσεων δίνει την δυνατότητα να επαναφέρουμε ένα κείμενο σε μια προηγούμενη αναθεώρηση, το οποίο είναι πολύ σημαντικό για να υπάρχει έλεγχος στις αλλαγές, να διορθώνονται τα λάθη, και να προστατεύεται το περιεχόμενο από βανδαλισμό και ανεπιθύμητα μηνύματα.
\subsection{Github}
Το Github είναι μια διαδικτυακή υπηρεσία αποθηκευτικού χώρου Git, που επιτρέπει πλήρη κατανεμημένο ελέγχο αναθεωρήσεων και διαχείριση πηγαίου κώδικα μέσω Git, ενώ προσθέτει επιπλέον λειτουργίες σε αυτό. Το Github προσφέρει ένα web-based γραφικό περιβάλλον, εφαρμογές σταθερών υπολογιστών, αλλά και εφαρμογές κινητών συσκευών. Επίσης παρέχει έλεγχο πρόσβασης και αρκετά χαρακτηριστικά συνεργασίας, όπως wiki, διαχείριση έργων, παρακολούθηση λαθών, και αιτήματα χαρακτηριστικών για κάθε σχέδιο.

Η ανάπτυξη του Github ξεκίνησε το 2007. Η γλώσσα προγραμματισμού που χρησιμοποιήθηκε είναι η Ruby on Rails και η Erlang. Η ιστοσελίδα ανακοινώθηκε τον Απρίλιο του 2008 απο τον Tom Preston-Werner, τον Chris Wanstrath, και τον Pj Hyett.

Το Github παρέχει δωρεάν λογαριασμούς, που συνηθίζεται να χρησιμοποιούνται για έργα λογισμικού ανοιχτού κώδικα. Ως το 2014, οι χρήστες του φτάνουν τα 3.4 εκατομμύρια, καθιστώντας το το μεγαλύτερο αποθηκευτικό χώρο πηγαίου κώδικα στον πλανήτη.\\
\begin{figure}[h]
\centering
\includegraphics[scale=0.15]{Octocat}
\caption{Λογότυπο Github}
\end{figure}